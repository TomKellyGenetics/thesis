%%
%% Now we have to get the source code in as a set of Appendices.
%% Source code will be Appendix A, with each file numbered X.y
%%
%\appendix

%%
%% -> \Chapter will cause the next bit to be labelled Appendix A
%% -> \section will give us A.1, \subsection A.1.1 etc.
%%
%% I suggest a section for each program and a subsection for each file
%% in the program.  Alternatively, a Chapter for each program, a
%% section for each library and a subsection for each file.
%%

\FloatBarrier

\chapter{Synthetic Lethal Genes in Pathways}
\label{appendix:brca_networks}

\FloatBarrier

\begin{figure*}[!htp]
\begin{mdframed}
  \begin{center}
  \resizebox{0.85 \textwidth}{!}{
    %\input{{{"SL_Model.pdf_tex"}}
    \fbox{
    \includegraphics{{"/home/tomkelly/Downloads/Pathway_Structure/graph_plot_Pi3K_exprSL_2".pdf}}
   }
   }
   \end{center}
   \caption[Synthetic Lethality in the PI3K Cascade]{\small \textbf{Synthetic Lethality in the PI3K Cascade.} The Reactome PI3K Cascade pathway with synthetic lethal candidates detected, coloured as shown in the Legend.
}
\label{fig:SL_Pathway_Pi3K}
\end{mdframed}
\end{figure*}

\begin{figure*}[!p]
\begin{mdframed}
  \begin{center}
  \resizebox{0.95 \textwidth}{!}{
    %\input{{{"SL_Model.pdf_tex"}}
    \fbox{
    \includegraphics{{"/home/tomkelly/Downloads/Pathway_Structure/graph_plot_Pi3kAkt_exprSL"".pdf}}
   }
   }
   \end{center}
   \caption[Synthetic Lethality in the PI3K/AKT Pathway]{\small \textbf{Synthetic Lethality in the PI3K/AKT Pathway.} The Reactome PI3K/AKT Pathway pathway with synthetic lethal candidates detected, coloured as shown in the Legend.
}
\label{fig:SL_Pathway_Pi3KAkt}
\end{mdframed}
\end{figure*}

\begin{figure*}[!p]
\begin{mdframed}
  \begin{center}
  \resizebox{0.95 \textwidth}{!}{
    %\input{{{"SL_Model.pdf_tex"}}
    \fbox{
    \includegraphics{{"/home/tomkelly/Downloads/Pathway_Structure/graph_plot_Pi3kAktCancer_exprSL".pdf}}
   }
   }
   \end{center}
   \caption[Synthetic Lethality in the PI3K/AKT Pathway in Cancer]{\small \textbf{Synthetic Lethality in the PI3K/AKT Pathway in Cancer.} The Reactome PI3K/AKT Pathway in Cancer pathway with synthetic lethal candidates detected, coloured as shown in the Legend.
}
\label{fig:SL_Pathway_Pi3KAktCancer}
\end{mdframed}
\end{figure*}

\begin{figure*}[!p]
\begin{mdframed}
  \begin{center}
  \resizebox{0.95 \textwidth}{!}{
    %\input{{{"SL_Model.pdf_tex"}}
    \fbox{
    \includegraphics{{"/home/tomkelly/Downloads/Pathway_Structure/graph_plot_ElasticFibre_exprSL".pdf}}
   }
   }
   \end{center}
   \caption[Synthetic Lethality in the Elastic Fibre Formation Pathway]{\small \textbf{Synthetic Lethality in the Elastic Fibre Formation Pathway.} The Reactome Elastic Fibre Formation Pathway pathway with synthetic lethal candidates detected, coloured as shown in the Legend.
}
\label{fig:SL_Pathway_ElasticFibre}
\end{mdframed}
\end{figure*}

\begin{figure*}[!p]
\begin{mdframed}
  \begin{center}
  \resizebox{0.95 \textwidth}{!}{
    %\input{{{"SL_Model.pdf_tex"}}
    \fbox{
    \includegraphics{{"/home/tomkelly/Downloads/Pathway_Structure/graph_plot_FibrinFormation_exprSL".pdf}}
   }
   }
   \end{center}
   \caption[Synthetic Lethality in the Fibrin Clot Formation]{\small \textbf{Synthetic Lethality in the Fibrin Clot Formation.} The Reactome Fibrin Clot Formation pathway with synthetic lethal candidates detected, coloured as shown in the Legend.
}
\label{fig:SL_Pathway_FibrinFormation}
\end{mdframed}
\end{figure*}


\begin{figure*}[!p]
\begin{mdframed}
  \begin{center}
  \resizebox{0.95 \textwidth}{!}{
    %\input{{{"SL_Model.pdf_tex"}}
    \fbox{
    \includegraphics{{"/home/tomkelly/Downloads/Pathway_Structure/graph_plot_ExtracellularMatrix_exprSL".pdf}}
   }
   }
   \end{center}
   \caption[Synthetic Lethality in the Extracellular Matrix]{\small \textbf{Synthetic Lethality in the Extracellular Matrix.} The Reactome Extracellular Matrix pathway with synthetic lethal candidates detected, coloured as shown in the Legend.
}
\label{fig:SL_Pathway_ExtracellularMatrix}
\end{mdframed}
\end{figure*}


\begin{figure*}[!p]
\begin{mdframed}
  \begin{center}
  \resizebox{0.95 \textwidth}{!}{
    %\input{{{"SL_Model.pdf_tex"}}
    \fbox{
    \includegraphics{{"/home/tomkelly/Downloads/Pathway_Structure/graph_plot_GPCR_exprSL".pdf}}
   }
   }
   \end{center}
   \caption[Synthetic Lethality in the GPCRs]{\small \textbf{Synthetic Lethality in the GPCRs.} The Reactome GPCRs pathway with synthetic lethal candidates detected, coloured as shown in the Legend.
}
\label{fig:SL_Pathway_GPCR}
\end{mdframed}
\end{figure*}

\begin{figure*}[!p]
\begin{mdframed}
  \begin{center}
  \resizebox{0.95 \textwidth}{!}{
    %\input{{{"SL_Model.pdf_tex"}}
    \fbox{
    \includegraphics{{"/home/tomkelly/Downloads/Pathway_Structure/graph_plot_GPCR_Downstream_exprSL".pdf}}
   }
   }
   \end{center}
   \caption[Synthetic Lethality in the GPCR Downstream]{\small \textbf{Synthetic Lethality in the GPCR Downstream.} The Reactome GPCR Downstream pathway with synthetic lethal candidates detected, coloured as shown in the Legend.
}
\label{fig:SL_Pathway_GPCR_Downstream}
\end{mdframed}
\end{figure*}

\begin{figure*}[!p]
\begin{mdframed}
  \begin{center}
  \resizebox{0.95 \textwidth}{!}{
    %\input{{{"SL_Model.pdf_tex"}}
    \fbox{
    \includegraphics{{"/home/tomkelly/Downloads/Pathway_Structure/graph_plot_TranslationElongation_exprSL".pdf}}
   }
   }
   \end{center}
   \caption[Synthetic Lethality in the Translation Elongation]{\small \textbf{Synthetic Lethality in the Translation Elongation.} The Reactome Translation Elongation pathway with synthetic lethal candidates detected, coloured as shown in the Legend.
}
\label{fig:SL_Pathway_TranslationElongation}
\end{mdframed}
\end{figure*}


\begin{figure*}[!p]
\begin{mdframed}
  \begin{center}
  \resizebox{0.95 \textwidth}{!}{
    %\input{{{"SL_Model.pdf_tex"}}
    \fbox{
    \includegraphics{{"/home/tomkelly/Downloads/Pathway_Structure/graph_plot_NMD_exprSL".pdf}}
   }
   }
   \end{center}
   \caption[Synthetic Lethality in the Nonsense-mediated Decay]{\small \textbf{Synthetic Lethality in the Nonsense-mediated Decay.} The Reactome Nonsense-mediated Decay pathway with synthetic lethal candidates detected, coloured as shown in the Legend.
}
\label{fig:SL_Pathway_NMD}
\end{mdframed}
\end{figure*}


\begin{figure*}[!p]
\begin{mdframed}
  \begin{center}
  \resizebox{0.95 \textwidth}{!}{
    %\input{{{"SL_Model.pdf_tex"}}
    \fbox{
    \includegraphics{{"/home/tomkelly/Downloads/Pathway_Structure/graph_plot_3primeUTR_exprSL".pdf}}
   }
   }
   \end{center}
   \caption[Synthetic Lethality in the 3$\prime$ UTR]{\small \textbf{Synthetic Lethality in the 3$\prime$ UTR.} The Reactome 3$\prime$ UTR pathway with synthetic lethal candidates detected, coloured as shown in the Legend.
}
\label{fig:SL_Pathway_3primeUTR}
\end{mdframed}
\end{figure*}


\begin{figure*}[!p]
\begin{mdframed}
  \begin{center}
  \resizebox{0.95 \textwidth}{!}{
    %\input{{{"SL_Model.pdf_tex"}}
    \fbox{
    \includegraphics{{"/home/tomkelly/Downloads/Pathway_Structure/graph_plot_Three_prime_UTR_exprSL".png}}
   }
   }
   \end{center}
   \caption[Synthetic Lethality in the 3$\prime$ UTR]{\small \textbf{Synthetic Lethality in the 3$\prime$ UTR.} The Reactome 3$\prime$ UTR pathway with synthetic lethal candidates detected, coloured as shown in the Legend.
}
\label{fig:SL_Pathway_Three_prime_UTR}
\end{mdframed}
\end{figure*}

\FloatBarrier

\chapter{Pathway Connectivity for Mutation SLIPT}
\label{appendix:connectivity_mtSL}

\begin{figure*}[!htp]
\begin{mdframed}
  \begin{center}
  \resizebox{0.95 \textwidth}{!}{
    %\input{{{"SL_Model.pdf_tex"}}
    \fbox{
    \includegraphics{{"/home/tomkelly/Downloads/Pathway_Structure/Centrality_mtSL/Pi3K_network_vertex_degree".png}}
   }
   }
   \end{center}
   \caption[Synthetic Lethality and Vertex Degree]{\small \textbf{Synthetic Lethality and Vertex Degree.} Synthetic Lethality and Vertex Degree.
}
\label{fig:mtSL_Pathway_PI3K_Vertex_Degree}
\end{mdframed}
\end{figure*}

\begin{figure*}[!htp]
\begin{mdframed}
  \begin{center}
  \resizebox{0.95 \textwidth}{!}{
    %\input{{{"SL_Model.pdf_tex"}}
    \fbox{
    \includegraphics{{"/home/tomkelly/Downloads/Pathway_Structure/Centrality_mtSL/Pi3K_network_Info_Centrality(Log)".png}}
   }
   }
   \end{center}
   \caption[Synthetic Lethality and Centrality]{\small \textbf{Synthetic Lethality and Centrality.} Synthetic Lethality and Information Centrality (log-scale).
}
\label{fig:mtSL_Pathway_PI3K_InfoCent}
\end{mdframed}
\end{figure*}

\begin{figure*}[!htp]
\begin{mdframed}
  \begin{center}
  \resizebox{0.95 \textwidth}{!}{
    %\input{{{"SL_Model.pdf_tex"}}
    \fbox{
    \includegraphics{{"/home/tomkelly/Downloads/Pathway_Structure/Centrality_mtSL/Pi3K_network_pagerank".png}}
   }
   }
   \end{center}
   \caption[Synthetic Lethality and PageRank]{\small \textbf{Synthetic Lethality and PageRank.} Synthetic Lethality and PageRank.
}
\label{fig:mtSL_Pathway_PI3K_PageRank}
\end{mdframed}
\end{figure*}

\chapter{Pathway Structure for Mutation SLIPT}
\label{appendix:Pathway_Structure_mtSL}

\resizebox{1 \textwidth}{!}{
	\begin{tabular}{lllllllllll}
                                          & \multicolumn{2}{l}{Graph:} & \multicolumn{2}{l}{States:} & \multicolumn{4}{l}{Observed:}        & \multicolumn{2}{l}{Permutation p-value:} \\
\hline
pathway                                   & nodes & edges  & mtSL & siRNA & up   & down & up-down & up/down           & up abs & down abs \\
\hline
PI3K Cascade                              & 138   & 1495   & 42   & 25    & 131  & 123  & 8       & 1.0650406504065   & 0.4473 & 0.5466   \\
PI3K/AKT Signaling in Cancer              & 275   & 12882  & 56   & 44    & 478  & 440  & 38      & 1.08636363636364  & 0.4163 & 0.581    \\
G$_{\alpha i}$                  & 292   & 22003  & 57   & 58    & 543  & 866  & -323    & 0.6270207852194   & 0.9507 & 0.0488   \\
GPCR downstream                           & 1270  & 142071 & 218  & 160   & 7632 & 6500 & 1132    & 1.17415384615385  & 0.1707 & 0.8291   \\
Elastic fibre formation                   & 42    & 175    & 16   & 7     & 6    & 7    & -1      & 0.857142857142857 & 0.5512 & 0.3681   \\
Extracellular matrix                      & 299   & 3677   & 81   & 29    & 313  & 347  & -34     & 0.902017291066282 & 0.5762 & 0.4215   \\
Formation of Fibrin                       & 52    & 243    & 11   & 5     & 8    & 19   & -11     & 0.421052631578947 & 0.7993 & 0.18     \\
Nonsense-Mediated Decay                   & 103   & 102    & 56   & 2     & 0    & 0    & 0       & NaN               & 0.197  & 0.1373   \\
3' -UTR-mediated translational regulation & 107   & 2860   & 56   & 1     & 52   & 1    & 51      & 52                & 0.121  & 0.8751   \\
Eukaryotic Translation Elongation         & 92    & 3746   & 57   & 0     & 0    & 0    & 0       & NaN               & 0.4952 & 0.4892   \\ 
\hline
\end{tabular}
}

\chapter{Information Centrality for Gene Essentiality}
\label{appendix:infocent_essential}

%Reactome Network structure and Information Centrality as a measure of gene essentiality

Network structure is another useful strategy to analyse gene function and this has been used to investigate network properties of a network constructed from of Reactome pathways imported with the paxtoolsr R package (Demir et al. 2010). Most notably, information centrality which has been proposed as a measure of gene essentiality was calculated as performed by Kranthi et al. (2013) using the efficiency and shortest path between each pair or nodes in the network before and after a node of interest is removed to test the importance of a node to network connectivity. Reactome contains substrates and cofactors in addition to genes or proteins, supporting the idea of centrality as a measure of essentiality, a number nodes with the highest centrality were essential nutrients including Mg\textsuperscript{2$+$}, Ca\textsuperscript{2$+$}, Zn\textsuperscript{2$+$},  and Fe\textsuperscript{3$+$}.