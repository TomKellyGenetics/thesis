%%
%% Now we have to get the source code in as a set of Appendices.
%% Source code will be Appendix A, with each file numbered X.y
%%
%\appendix

%%

%% -> \section will give us A.1, \subsection A.1.1 etc.
%%
%% I suggest a section for each program and a subsection for each file

%% section for each library and a subsection for each file.
%%

\FloatBarrier

\chapter{Synthetic Lethal Genes in Pathways}
\label{appendix:brca_networks}

\FloatBarrier

\iffalse
\begin{figure*}[!htp]
%\begin{mdframed}
  \begin{center}
  \resizebox{1 \textwidth}{!}{
    %\input{{{"SL_Model.pdf_tex"}}
    \fbox{
    \includegraphics{{"/home/tomkelly/Downloads/Pathway_Structure/graph_plot_Pi3K_exprSL2".pdf}}
   }
   }
   \end{center}
   \caption[Synthetic lethality in the PI3K Cascade]{\small \textbf{Synthetic lethality in the PI3K Cascade.} The Reactome PI3K Cascade pathway with synthetic lethal candidates, coloured as shown in the legend.
}
\label{fig:SL_Pathway_Pi3K}
%\end{mdframed}
\end{figure*}
\fi

\begin{figure*}[!htp]
%\begin{mdframed}
  \begin{center}
  \resizebox{0.8 \textwidth}{!}{
    %\input{{{"SL_Model.pdf_tex"}}
    %\fbox{
    \includegraphics{{"/home/tomkelly/Downloads/Pathway_Structure/graph_plot_Pi3kAkt_exprSL2".png}}
   %}
   }
   \end{center}
   \caption[Synthetic lethality in the PI3K/AKT pathway]{\small \textbf{Synthetic lethality in the PI3K/AKT pathway.} The Reactome PI3K/AKT pathway with synthetic lethal candidates, coloured as shown in the legend.
}
\label{fig:SL_Pathway_Pi3KAkt}
%\end{mdframed}
\end{figure*}

\begin{figure*}[!htp]
%\begin{mdframed}
  \begin{center}
  \resizebox{1 \textwidth}{!}{
    %\input{{{"SL_Model.pdf_tex"}}
    %\fbox{
    \includegraphics{{"/home/tomkelly/Downloads/Pathway_Structure/graph_plot_Pi3kAktCancer_exprSL2".png}}
   %}
   }
   \end{center}
   \caption[Synthetic lethality in the PI3K/AKT pathway in cancer]{\small \textbf{Synthetic lethality in the PI3K/AKT pathway in cancer.} The Reactome PI3K/AKT  in cancer pathway with synthetic lethal candidates, coloured as shown in the legend.
}
\label{fig:SL_Pathway_Pi3KAktCancer}
%\end{mdframed}
\end{figure*}

\iffalse

\begin{figure*}[!htp]
%\begin{mdframed}
  \begin{center}
  \resizebox{1 \textwidth}{!}{
    %\input{{{"SL_Model.pdf_tex"}}
    \fbox{
    \includegraphics{{"/home/tomkelly/Downloads/Pathway_Structure/graph_plot_ElasticFibre_exprSL2".pdf}}
   }
   }
   \end{center}
   \caption[Synthetic lethality in the Elastic Fibre Formation Pathway]{\small \textbf{Synthetic lethality in the Elastic Fibre Formation Pathway.} The Reactome Elastic Fibre Formation pathway with synthetic lethal candidates, coloured as shown in the legend.
}
\label{fig:SL_Pathway_ElasticFibre}
%\end{mdframed}
\end{figure*}

\begin{figure*}[!htp]
%\begin{mdframed}
  \begin{center}
  \resizebox{1 \textwidth}{!}{
    %\input{{{"SL_Model.pdf_tex"}}
    \fbox{
    \includegraphics{{"/home/tomkelly/Downloads/Pathway_Structure/graph_plot_FibrinFormation_exprSL2".pdf}}
   }
   }
   \end{center}
   \caption[Synthetic lethality in the Fibrin Clot Formation]{\small \textbf{Synthetic lethality in the Fibrin Clot Formation.} The Reactome Fibrin Clot Formation pathway with synthetic lethal candidates, coloured as shown in the legend.
}
\label{fig:SL_Pathway_FibrinFormation}
%\end{mdframed}
\end{figure*}

\fi


\begin{figure*}[!htp]
%\begin{mdframed}
  \begin{center}
  \resizebox{1 \textwidth}{!}{
    %\input{{{"SL_Model.pdf_tex"}}
    %\fbox{
    \includegraphics{{"/home/tomkelly/Downloads/Pathway_Structure/graph_plot_ExtracellularMatrix_exprSL2".pdf}}
   %}
   }
   \end{center}
   \caption[Synthetic lethality in the Extracellular Matrix]{\small \textbf{Synthetic lethality in the Extracellular Matrix.} The Reactome Extracellular Matrix pathway with synthetic lethal candidates, coloured as shown in the legend.
}
\label{fig:SL_Pathway_ExtracellularMatrix}
%\end{mdframed}
\end{figure*}


\begin{figure*}[!htp]
%\begin{mdframed}
  \begin{center}
  \resizebox{1 \textwidth}{!}{
    %\input{{{"SL_Model.pdf_tex"}}
    %\fbox{
    \includegraphics{{"/home/tomkelly/Downloads/Pathway_Structure/graph_plot_GPCR_exprSL2".png}}
   %}
   }
   \end{center}
   \caption[Synthetic lethality in the GPCRs]{\small \textbf{Synthetic lethality in the GPCRs.} The Reactome G$_{\alpha i}$ pathway with synthetic lethal candidates, coloured as shown in the legend.
}
\label{fig:SL_Pathway_GPCR}
%\end{mdframed}
\end{figure*}

\begin{figure*}[!htp]
%\begin{mdframed}
  \begin{center}
  \resizebox{1 \textwidth}{!}{
    %\input{{{"SL_Model.pdf_tex"}}
    %\fbox{
    \includegraphics{{"/home/tomkelly/Downloads/Pathway_Structure/graph_plot_GPCR_Downstream_exprSL2".png}}
   %}
   }
   \end{center}
   \caption[Synthetic lethality in the GPCR Downstream]{\small \textbf{Synthetic lethality in the GPCR Downstream.} The Reactome \gls{GPCR} Downstream pathway with synthetic lethal candidates, coloured as shown in the legend.
}
\label{fig:SL_Pathway_GPCR_Downstream}
%\end{mdframed}
\end{figure*}

\begin{figure*}[!htp]
%\begin{mdframed}
  \begin{center}
  \resizebox{1 \textwidth}{!}{
    %\input{{{"SL_Model.pdf_tex"}}
    %\fbox{
    \includegraphics{{"/home/tomkelly/Downloads/Pathway_Structure/graph_plot_TranslationElongation_exprSL2".png}}
   %}
   }
   \end{center}
   \caption[Synthetic lethality in the Translation Elongation]{\small \textbf{Synthetic lethality in the Translation Elongation.} The Reactome Translation Elongation pathway with synthetic lethal candidates, coloured as shown in the legend.
}
\label{fig:SL_Pathway_TranslationElongation}
%\end{mdframed}
\end{figure*}


\begin{figure*}[!htp]
%\begin{mdframed}
  \begin{center}
  \resizebox{1 \textwidth}{!}{
    %\input{{{"SL_Model.pdf_tex"}}
    %\fbox{
    \includegraphics{{"/home/tomkelly/Downloads/Pathway_Structure/graph_plot_NMD_exprSL2".pdf}}
   %}
   }
   \end{center}
   \caption[Synthetic lethality in the Nonsense-mediated Decay]{\small \textbf{Synthetic lethality in the Nonsense-mediated Decay.} The Reactome \gls{NMD} pathway with synthetic lethal candidates, coloured as shown in the legend.
}
\label{fig:SL_Pathway_NMD}
%\end{mdframed}
\end{figure*}

\begin{figure*}[!htp]
%\begin{mdframed}
  \begin{center}
  \resizebox{1 \textwidth}{!}{
    %\input{{{"SL_Model.pdf_tex"}}
    %\fbox{
    \includegraphics{{"/home/tomkelly/Downloads/Pathway_Structure/graph_plot_Three_prime_UTR_exprSL2".png}}
   %}
   }
   \end{center}
   \caption[Synthetic lethality in the 3$^\prime$ UTR]{\small \textbf{Synthetic lethality in the 3$^\prime$ UTR.} The Reactome 3$^\prime$ \gls{UTR} pathway with synthetic lethal candidates, coloured as shown in the legend.
}
\label{fig:SL_Pathway_Three_prime_UTR}
%\end{mdframed}
\end{figure*}

\FloatBarrier

\chapter{Pathway Connectivity for Mutation SLIPT}
\label{appendix:connectivity_mtSL}

\begin{figure*}[!htp]
%\begin{mdframed}
  \begin{center}
  \resizebox{0.95 \textwidth}{!}{
    %\input{{{"SL_Model.pdf_tex"}}
    \fbox{
    \includegraphics{{"/home/tomkelly/Downloads/Pathway_Structure/Centrality_mtSL/Pi3K_network_vertex_degree_stripchart2".pdf}}
   }
   }
   \end{center}
   \caption[Synthetic lethality and vertex degree]{\small \textbf{Synthetic lethality and vertex degree.} The number of connected genes (\gls{vertex degree}) was compared (on a log-scale across genes deteced by \acrshort{mtSLIPT} and \gls{siRNA} screening in the Reactome \gls{PI3K} cascade pathway. There were very few differences in vertex degree between the groups, although genes detected by \gls{siRNA} included those with the fewest connections.
}
\label{fig:mtSL_Pathway_PI3K_Vertex_Degree}
%\end{mdframed}
\end{figure*}

%\FloatBarrier

\begin{figure*}[!htp]
%\begin{mdframed}
  \begin{center}
  \resizebox{0.95 \textwidth}{!}{
    %\input{{{"SL_Model.pdf_tex"}}
    \fbox{
    \includegraphics{{"/home/tomkelly/Downloads/Pathway_Structure/Centrality_mtSL/Pi3K_network_Info_Centrality(Log)_stripchart2".pdf}}
   }
   }
   \end{center}
   \caption[Synthetic lethality and centrality]{\small \textbf{Synthetic lethality and centrality.}  The \gls{information centrality} was compared (on a log-scale across genes deteced by \acrshort{mtSLIPT} and \gls{siRNA} screening in the Reactome \gls{PI3K} cascade pathway. Genes detected by \acrshort{mtSLIPT} or \gls{siRNA} did not have higher connectivity than genes not detected by either approach. The gene with the highest centrality was detected by \gls{siRNA}.
}
\label{fig:mtSL_Pathway_PI3K_InfoCent}
%\end{mdframed}
\end{figure*}

%info cent pathway 1.338433


%\FloatBarrier

\begin{figure*}[!htp]
%\begin{mdframed}
  \begin{center}
  \resizebox{0.95 \textwidth}{!}{
    %\input{{{"SL_Model.pdf_tex"}}
    \fbox{
    \includegraphics{{"/home/tomkelly/Downloads/Pathway_Structure/Centrality_mtSL/Pi3K_network_pagerank_stripchart2".pdf}}
   }
   }
   \end{center}
   \caption[Synthetic lethality and PageRank]{\small \textbf{Synthetic lethality and PageRank.}  The \gls{PageRank centrality} was compared (on a log-scale across genes deteced by \acrshort{mtSLIPT} and \gls{siRNA} screening in the Reactome PI3K cascade pathway. Genes detected by \gls{siRNA} had a more restricted range of centrality values than other genes not detected by either approach, although these groups also had fewer genes.
}
\label{fig:mtSL_Pathway_PI3K_PageRank}
%\end{mdframed}
\end{figure*}


\begin{table*}[!htb]
\caption{\acrshort{ANOVA} for synthetic lethality and vertex degree}
\label{tab:mtSL_Pathway_PI3K_Vertex_Degree}
\noindent\makebox[\textwidth][c]{%               %centering
\resizebox{0.8 \textwidth}{!}{
\begin{threeparttable}
\begin{tabular}{lccccc}
\hline
                 & DF & Sum Squares & Mean Squares & F-value & p-value \\
\hline
\rowcolor{black!10}
siRNA              &     1    &    15  & 15.50 & 0.0134 & 0.9084 \\
\rowcolor{black!5}
mtSLIPT              &     1    &    196 & 195.94 & 0.1689 & 0.6825 \\
\rowcolor{black!10}
siRNA$\times$mtSLIPT     &     1    &    9 &  9.17 & 0.0079 & 0.9294 \\
\hline
\end{tabular}
\begin{tablenotes}
\raggedright \small
Analysis of variance for \gls{vertex degree} against \gls{synthetic lethal} detection approaches (with an interaction term)
\end{tablenotes}
\end{threeparttable}
}
}
\end{table*}


\begin{table*}[!htb]
\caption{\acrshort{ANOVA} for synthetic lethality and information centrality}
\label{tab:mtSL_Pathway_PI3K_InfoCent}
\noindent\makebox[\textwidth][c]{%               %centering
\resizebox{0.8 \textwidth}{!}{
\begin{threeparttable}
\begin{tabular}{lccccc}
\hline
                 & DF & Sum Squares & Mean Squares & F-value & p-value \\
\hline
\rowcolor{black!10}
siRNA              &     1    &    0.000256 & 0.0002561 & 0.1851 & 0.6685 \\
\rowcolor{black!5}
mtSLIPT              &     1    &    0.003225 & 0.0032247 & 2.3308 & 0.1318 \\
\rowcolor{black!10}
siRNA$\times$mtSLIPT     &     1    &    0.001238 & 0.0012385 & 0.8952 & 0.3476 \\
\hline
\end{tabular}
\begin{tablenotes}
\raggedright \small
Analysis of variance for \gls{information centrality} against \gls{synthetic lethal} detection approaches (with an interaction term)
\end{tablenotes}
\end{threeparttable}
}
}
\end{table*}



\begin{table*}[!htb]
\caption{\acrshort{ANOVA} for synthetic lethality and PageRank centrality}
\label{tab:mtSL_Pathway_PI3K_PageRank}
\noindent\makebox[\textwidth][c]{%               %centering
\resizebox{0.8 \textwidth}{!}{
\begin{threeparttable}
\begin{tabular}{lccccc}
\hline
                 & DF & Sum Squares & Mean Squares & F-value & p-value \\
\hline
\rowcolor{black!10}
siRNA              &     1    &     0.0002038 & $2.0385 \times 10^{-4}$ & 1.1423 & 0.2892 \\
\rowcolor{black!5}
mtSLIPT              &     1    &    0.0000208 & $2.0752 \times 10^{-5}$ & 0.1163 & 0.7342 \\
\rowcolor{black!10}
siRNA$\times$mtSLIPT     &     1    &    0.0000137 & $1.3743 \times 10^{-5}$ & 0.0770 & 0.7823 \\
\hline
\end{tabular}
\begin{tablenotes}
\raggedright \small
Analysis of variance for \gls{PageRank centrality} against \gls{synthetic lethal} detection approaches (with an interaction term)
\end{tablenotes}
\end{threeparttable}
}
}
\end{table*}

\FloatBarrier

\chapter{Information Centrality for Gene Essentiality}
\label{appendix:infocent_essential}

\FloatBarrier

%Reactome Network structure and Information Centrality as a measure of gene essentiality

Network structure could be used to analyse gene function. This has been performed to investigate network properties of a network constructed from Reactome pathways \citep{Reactome} imported via Pathway Commons with Paxtools \citep{PathwayCommons, paxtools}. \Gls{information centrality}, which has been proposed as a measure of gene \glslink{essential}{essentiality}, was calculated as performed by \citet{Kranthi2013} using the efficiency and shortest path between each pair or nodes in the network before and after a node of interest is removed to test the importance of a node to network connectivity. Reactome contains substrates and cofactors in addition to genes or proteins. In support of centrality as a measure of essentiality, a number nodes with the highest centrality (shown in Table~\ref{tab:infocent_reactome}) were essential nutrients including Mg\textsuperscript{2$+$}, Ca\textsuperscript{2$+$}, Zn\textsuperscript{2$+$},  and Fe. In addition, there were genes important in development of epithelial tissues and breast cancer such as \textit{IL8}, \textit{GATA3}, and \textit{CTNNB1} detected with relatively high \gls{information centrality}. 
%Fe\textsuperscript{3$+$}??

\begin{table*}[!htb]
\caption{Information centrality for genes and molecules in the Reactome network}
\label{tab:infocent_reactome}
\centering
\resizebox{0.5 \textwidth}{!}{
\begin{threeparttable}
\begin{tabular}{>{\em}lc}
  \hline
  \em{\textbf{Node}} & \textbf{Centrality} \\
  \hline
  \rowcolor{black!10}
  ZNF473 & 0.0510 \\ 
  \rowcolor{black!5}
  \em{Magnesium (Mg\textsuperscript{2$+$})} & 0.0082 \\ 
  \rowcolor{black!10}
  XBP1 & 0.0053 \\ 
  \rowcolor{black!5}
  \em{Calcium (Ca\textsuperscript{2$+$})} & 0.0050 \\ 
  \rowcolor{black!10}
  \em{Zinc (Zn\textsuperscript{2$+$})} & 0.0048 \\ 
  \rowcolor{black!5}
  \em{Iron atom (Fe)} & 0.0041 \\ 
  \rowcolor{black!10}
  FMN & 0.0040 \\ 
  \rowcolor{black!5}
  AGT & 0.0037 \\ 
  \rowcolor{black!10}
  HSP90AA1 & 0.0029 \\ 
  \rowcolor{black!5}
  \em{Phosphatidyl-L-serine} & 0.0029 \\ 
  \rowcolor{black!10}
  P2RX7 & 0.0026 \\ 
  \rowcolor{black!5}
  PANX1 & 0.0024 \\ 
  \rowcolor{black!10}
  NCAM1 & 0.0022 \\ 
  \rowcolor{black!5}
  NUDT1 & 0.0021 \\ 
  \rowcolor{black!10}
  PLAUR & 0.0020 \\ 
  \rowcolor{black!5}
  IL8 & 0.0020 \\ 
  \rowcolor{black!10}
  HSPA8 & 0.0019 \\ 
  \rowcolor{black!5}
  TYROBP & 0.0019 \\ 
  \rowcolor{black!10}
  CASP3 & 0.0017 \\ 
  \rowcolor{black!5}
  GNAL & 0.0015 \\ 
  \rowcolor{black!10}
  CBLB & 0.0015 \\ 
  \rowcolor{black!5}
  HBB & 0.0014 \\ 
  \rowcolor{black!10}
  GATA4 & 0.0013 \\ 
  \rowcolor{black!5}
  TGS1 & 0.0013 \\ 
  \rowcolor{black!10}
  CTNNB1 & 0.0012 \\ 
\hline
\end{tabular}
\begin{tablenotes}
\raggedright \small
Highest \gls{information centrality} for genes (proteins), cofactors, and minerals in the Reactome network 
\end{tablenotes}
\end{threeparttable}
}
\end{table*}


\begin{figure*}[!htp]
%\begin{mdframed}
  \begin{center}
  \resizebox{0.95 \textwidth}{!}{
    %\input{{{"SL_Model.pdf_tex"}}
    \fbox{
    \includegraphics{{"/home/tomkelly/Downloads/Networks/infocent_genes".pdf}}
   }
   }
   \end{center}
   \caption[Information centrality distribution]{\small \textbf{Information centrality distribution.} \Gls{information centrality} in the Reactome network for nodes, including genes/proteins and other biomolecules.
}
\label{fig:infocent_reactome}
%\end{mdframed}
\end{figure*}


\chapter{Pathway Structure for Mutation SLIPT}
\label{appendix:Pathway_Structure_mtSL}

\FloatBarrier

\begin{figure*}[!htp]
%\begin{mdframed}
  \begin{center}
  \resizebox{0.75 \textwidth}{!}{
    \fbox{
    \includegraphics{{"/home/tomkelly/Downloads/Pathway_Structure/Discrete_Pi3k/SL_distance_counts_vioplot_mtSL".pdf}}
   }
   }
   \end{center}
   \caption[Synthetic lethality and heirarchy score in PI3K]{\small \textbf{Synthetic lethality and heirarchy score in PI3K.} The hierarchical distance scores were similarly distributed across \acrshort{mtSLIPT} and \gls{siRNA} genes. Genes detected by both methods had a higher (downstream) median than either group.
}
\label{fig:mtSL_Pathway_PI3K_Distance_Vioplot_Counts}
%\end{mdframed}
\end{figure*}


\begin{table*}[!htb]
\caption{\acrshort{ANOVA} for synthetic lethality and PI3K hierarchy}
\label{tab:mtSL_Pathway_PI3K_Distance_Counts}
\noindent\makebox[\textwidth][c]{%               %centering
\resizebox{0.8 \textwidth}{!}{
\begin{threeparttable}
\begin{tabular}{lccccc}
\hline
                 & DF & Sum Squares & Mean Squares & F-value & p-value \\
\hline
\rowcolor{black!10}
siRNA              &     1    &     0.001 & 0.00070 & 0.0004 & 0.9841 \\
\rowcolor{black!5}
mtSLIPT              &     1    &    0.007 & 0.0066 & 0.0040 & 0.9496 \\
\rowcolor{black!10}
siRNA$\times$mtSLIPT     &     1    &   3.906 & 3.9056 & 2.3829 & 0.1250 \\
\hline
\end{tabular}
\begin{tablenotes}
\raggedright \small
Analysis of variance for \gls{PI3K} hierarchy score against \gls{synthetic lethal} detection approaches (with an interaction term)
\end{tablenotes}
\end{threeparttable}
}
}
\end{table*}

\iffalse
\begin{figure*}[!h]
%\begin{mdframed}
 \begin{center}
%
        \subcaptionbox{Hierarchical Distance Score \label{fig:mtSL_Pathway_PI3K_Distance_Vioplot_Counts}}{
	  %\fbox{
	  \includegraphics[width=0.75 \textwidth]{{"/home/tomkelly/Downloads/Pathway_Structure/Discrete_Pi3k/SL_distance_counts_vioplot_mtSL".pdf}}
	%}
        }%

        \subcaptionbox{Proportion of Genes \label{fig:mtSL_Pathway_PI3K_Distance_Barplot_Counts}}{%
	  %\fbox{
	  \includegraphics[width=0.75 \textwidth]{{"/home/tomkelly/Downloads/Pathway_Structure/Discrete_Pi3k/SL_distance_counts_barplot_prop_mtSL".pdf}}
	%}
        }%
      \end{center}
   \caption[Hierarchy Score in PI3K against Synthetic lethality in PI3K]{\small \textbf{Hierarchy Score in PI3K against Synthetic lethality in PI3K.}  The number of \acrshort{mtSLIPT} and \gls{siRNA} genes against the hierarchical distance scores showing no significant tendency for either method to either of the pathway upstream or downstream extremities. The number of \acrshort{mtSLIPT} and \gls{siRNA} genes upstream or downstream of each gene in the Reactome PI3K pathway were tested (by the $\chi^2$-test). These were plotted as a split violin plot against the hierarchical distance scores showing no significant tendency for either method to either of the pathway upstream or downstream extremities.
}
%\end{mdframed}
\end{figure*}
\fi

\begin{figure*}[!htp]
%\begin{mdframed}
  \begin{center}
  \resizebox{0.75 \textwidth}{!}{
    \fbox{
    \includegraphics{{"/home/tomkelly/Downloads/Pathway_Structure/Discrete_Pi3k/SL_distance_counts_barplot_prop_mtSL".pdf}}
   }
   }
   \end{center}
   \caption[Heirarchy score in PI3K against synthetic lethality in PI3K]{\small \textbf{Heirarchy score in PI3K against synthetic lethality in PI3K.} The number of \acrshort{mtSLIPT} and \gls{siRNA} genes against the hierarchical distance scores showing no significant tendency for either method to either of the pathway upstream or downstream extremities.
}
\label{fig:mtSL_Pathway_PI3K_Distance_Barplot_Counts}
%\end{mdframed}
\end{figure*}


\begin{figure*}[!htp]
%\begin{mdframed}
  \begin{center}
  \resizebox{0.75 \textwidth}{!}{
    %\input{{{"SL_Model.pdf_tex"}}
    \fbox{
    %\includegraphics{{"/home/tomkelly/Downloads/Pathway_Structure/Discrete_Pi3k/SL_distance_vioplot_mtSL".pdf}}
    \includegraphics{{"/home/tomkelly/Downloads/Pathway_Structure/Discrete_Pi3k/SL_distance_stripchart_mtSL".pdf}}
   }
   }
   \end{center}
   \caption[Structure of synthetic lethality in PI3K]{\small \textbf{Structure of synthetic lethality in PI3K.} The number of \acrshort{mtSLIPT} and \gls{siRNA} genes against the hierarchical distance scores showing no significant tendency for either method to either of the pathway upstream or downstream extremities. The number of \acrshort{mtSLIPT} and \gls{siRNA} genes upstream or downstream of each gene in the Reactome \gls{PI3K} pathway were tested (by the $\chi^2$-test). These were plotted as a split jitter stripchart against the hierarchical distance scores showing no significant tendency for either method to either of the pathway upstream or downstream extremities.
}
\label{fig:mtSL_Pathway_PI3K_Distance_Vioplot}
%\end{mdframed}
\end{figure*}

\begin{figure*}[!htp]
%\begin{mdframed}
  \begin{center}
  \resizebox{0.75 \textwidth}{!}{
    %\input{{{"SL_Model.pdf_tex"}}
   %\fbox{
    \includegraphics{{"/home/tomkelly/Downloads/Pathway_Structure/test_PI3K_mtSL".pdf}}
   %}
   }
   \end{center}
   \caption[Structure of synthetic lethality resampling]{\small \textbf{Structure of synthetic lethality resampling.} A null distribution (10,000 iterations) of the \gls{siRNA} genes upstream or downstream of \acrshort{mtSLIPT} genes (shown by the difference) in the PI3K pathway. The observed events (red) were compared to the the distribution (violet) and were not significant. Genes detected by both methods were fixed for the distribution (blue). The genes detected by both approaches were used. 
}
\label{fig:mtSL_Pathway_PI3K_Perm}
%\end{mdframed}
\end{figure*}


\begin{table*}[!htb]
\caption{Resampling for pathway structure of \gls{synthetic lethal} detection methods}
\label{tab:pathway_str_mtSL}
\noindent\makebox[\textwidth][c]{%               %centering
\resizebox{1.2 \textwidth}{!}{
\begin{threeparttable}
\begin{tabular}{l|cc|cc|cccc|cc}
\cline{2-11}
                                          & \multicolumn{2}{c|}{\textbf{Graph}} & \multicolumn{2}{c|}{\textbf{States}} & \multicolumn{4}{c|}{\textbf{Observed}}        & \multicolumn{2}{c}{\textbf{Permutation p-value}}  \\
\hline
Pathway                                   & Nodes & Edges  & mtSL & \gls{siRNA} & Up   & Down & Up$-$Down & Up$/$Down           & Up$-$Down & Down$-$Up \\
\hline
  \rowcolor{black!10}
PI3K Cascade                              & 138   & 1495   & 42   & 25    & 131  & 123  & 8       & 1.065             & 0.4473 & 0.5466   \\
  \rowcolor{black!5}
PI3K/AKT Signalling in Cancer              & 275   & 12882  & 56   & 44    & 478  & 440  & 38      & 1.086             & 0.4163 & 0.5810   \\
  \rowcolor{black!10}
\textbf{G$_{\alpha i}$ Signalling}                  & 292   & 22003  & 57   & 58    & 543  & 866  & -323    & 0.627             & 0.9507 & 0.0488             \\  
  \rowcolor{black!5}
GPCR downstream                           & 1270  & 142071 & 218  & 160   & 7632 & 6500 & 1132    & 1.174             & 0.1707 & 0.8291   \\
  \rowcolor{black!10}
Elastic fibre formation                   & 42    & 175    & 16   & 7     & 6    & 7    & -1      & 0.857             & 0.5512 & 0.3681   \\
  \rowcolor{black!5}
Extracellular matrix                      & 299   & 3677   & 81   & 29    & 313  & 347  & -34     & 0.902             & 0.5762 & 0.4215   \\
  \rowcolor{black!10}
Formation of Fibrin                       & 52    & 243    & 11   & 5     & 8    & 19   & -11     & 0.421             & 0.7993 & 0.1800   \\
  \rowcolor{black!5}
Nonsense-Mediated Decay                   & 103   & 102    & 56   & 2     & 0    & 0    & 0       &                   & 0.197  & 0.1373   \\
  \rowcolor{black!10}
3$^\prime$-UTR-mediated translational regulation & 107   & 2860   & 56   & 1     & 52   & 1    & 51      & 52                & 0.1210  & 0.8751   \\
  \rowcolor{black!5}
Eukaryotic Translation Elongation         & 92    & 3746   & 57   & 0     & 0    & 0    & 0       &                   & 0.4952 & 0.4892   \\ 
\hline
\end{tabular}
\begin{tablenotes}
\raggedright \small
Pathways in the Reactome network tested for structural relationships between \acrshort{mtSLIPT} and \gls{siRNA} genes by resampling. The raw p-value (computed without adjusting for multiple comparisons over pathways) is given for the difference in upstream and downstream paths from \acrshort{mtSLIPT} to \gls{siRNA} gene candidate partners of \texti{CDH1} with significant pathways highlighted in bold. Sampling was performed only in the target pathway and shortest paths were computed within it. Loops or paths in either direction that could not be resolved were excluded from the analysis. The gene detected by both \acrshort{mtSLIPT} and \gls{siRNA} (or resampling for them) were includued in the analysis and the number of these were fixed to the number observed.
\end{tablenotes}
\end{threeparttable}
}
}
\end{table*}


\FloatBarrier
