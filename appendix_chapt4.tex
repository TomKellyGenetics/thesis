%%
%% Now we have to get the source code in as a set of Appendices.
%% Source code will be Appendix A, with each file numbered X.y
%%
%\appendix

%%
%% -> \Chapter will cause the next bit to be labelled Appendix A
%% -> \section will give us A.1, \subsection A.1.1 etc.
%%
%% I suggest a section for each program and a subsection for each file
%% in the program.  Alternatively, a Chapter for each program, a
%% section for each library and a subsection for each file.
%%

\FloatBarrier

\chapter{Synthetic Lethal Genes in Pathways}
\label{appendix:brca_networks}

\FloatBarrier

\iffalse
\begin{figure*}[!htp]
\begin{mdframed}
  \begin{center}
  \resizebox{0.85 \textwidth}{!}{
    %\input{{{"SL_Model.pdf_tex"}}
    \fbox{
    \includegraphics{{"/home/tomkelly/Downloads/Pathway_Structure/graph_plot_Pi3K_exprSL_2".pdf}}
   }
   }
   \end{center}
   \caption[Synthetic Lethality in the PI3K Cascade]{\small \textbf{Synthetic Lethality in the PI3K Cascade.} The Reactome PI3K Cascade pathway with synthetic lethal candidates coloured as shown in the Legend.
}
\label{fig:SL_Pathway_Pi3K}
\end{mdframed}
\end{figure*}
\fi

\begin{figure*}[!htp]
\begin{mdframed}
  \begin{center}
  \resizebox{0.85 \textwidth}{!}{
    %\input{{{"SL_Model.pdf_tex"}}
    \fbox{
    \includegraphics{{"/home/tomkelly/Downloads/Pathway_Structure/graph_plot_Pi3kAkt_exprSL_2".png}}
   }
   }
   \end{center}
   \caption[Synthetic Lethality in the PI3K/AKT Pathway]{\small \textbf{Synthetic Lethality in the PI3K/AKT Pathway.} The Reactome PI3K/AKT pathway with synthetic lethal candidates coloured as shown in the Legend.
}
\label{fig:SL_Pathway_Pi3KAkt}
\end{mdframed}
\end{figure*}

\begin{figure*}[!htp]
\begin{mdframed}
  \begin{center}
  \resizebox{0.85 \textwidth}{!}{
    %\input{{{"SL_Model.pdf_tex"}}
    \fbox{
    \includegraphics{{"/home/tomkelly/Downloads/Pathway_Structure/graph_plot_Pi3kAktCancer_exprSL_2".png}}
   }
   }
   \end{center}
   \caption[Synthetic Lethality in the PI3K/AKT Pathway in Cancer]{\small \textbf{Synthetic Lethality in the PI3K/AKT Pathway in Cancer.} The Reactome PI3K/AKT Pathway in Cancer pathway with synthetic lethal candidates coloured as shown in the Legend.
}
\label{fig:SL_Pathway_Pi3KAktCancer}
\end{mdframed}
\end{figure*}

\iffalse

\begin{figure*}[!htp]
\begin{mdframed}
  \begin{center}
  \resizebox{0.85 \textwidth}{!}{
    %\input{{{"SL_Model.pdf_tex"}}
    \fbox{
    \includegraphics{{"/home/tomkelly/Downloads/Pathway_Structure/graph_plot_ElasticFibre_exprSL_2".pdf}}
   }
   }
   \end{center}
   \caption[Synthetic Lethality in the Elastic Fibre Formation Pathway]{\small \textbf{Synthetic Lethality in the Elastic Fibre Formation Pathway.} The Reactome Elastic Fibre Formation pathway with synthetic lethal candidates coloured as shown in the Legend.
}
\label{fig:SL_Pathway_ElasticFibre}
\end{mdframed}
\end{figure*}

\begin{figure*}[!htp]
\begin{mdframed}
  \begin{center}
  \resizebox{0.85 \textwidth}{!}{
    %\input{{{"SL_Model.pdf_tex"}}
    \fbox{
    \includegraphics{{"/home/tomkelly/Downloads/Pathway_Structure/graph_plot_FibrinFormation_exprSL_2".pdf}}
   }
   }
   \end{center}
   \caption[Synthetic Lethality in the Fibrin Clot Formation]{\small \textbf{Synthetic Lethality in the Fibrin Clot Formation.} The Reactome Fibrin Clot Formation pathway with synthetic lethal candidates coloured as shown in the Legend.
}
\label{fig:SL_Pathway_FibrinFormation}
\end{mdframed}
\end{figure*}

\fi


\begin{figure*}[!htp]
\begin{mdframed}
  \begin{center}
  \resizebox{0.85 \textwidth}{!}{
    %\input{{{"SL_Model.pdf_tex"}}
    \fbox{
    \includegraphics{{"/home/tomkelly/Downloads/Pathway_Structure/graph_plot_ExtracellularMatrix_exprSL_2".pdf}}
   }
   }
   \end{center}
   \caption[Synthetic Lethality in the Extracellular Matrix]{\small \textbf{Synthetic Lethality in the Extracellular Matrix.} The Reactome Extracellular Matrix pathway with synthetic lethal candidates coloured as shown in the Legend.
}
\label{fig:SL_Pathway_ExtracellularMatrix}
\end{mdframed}
\end{figure*}


\begin{figure*}[!htp]
\begin{mdframed}
  \begin{center}
  \resizebox{0.85 \textwidth}{!}{
    %\input{{{"SL_Model.pdf_tex"}}
    \fbox{
    \includegraphics{{"/home/tomkelly/Downloads/Pathway_Structure/graph_plot_GPCR_exprSL_2".png}}
   }
   }
   \end{center}
   \caption[Synthetic Lethality in the GPCRs]{\small \textbf{Synthetic Lethality in the GPCRs.} The Reactome G$_{\alpha i}$ pathway with synthetic lethal candidates coloured as shown in the Legend.
}
\label{fig:SL_Pathway_GPCR}
\end{mdframed}
\end{figure*}

\begin{figure*}[!htp]
\begin{mdframed}
  \begin{center}
  \resizebox{0.85 \textwidth}{!}{
    %\input{{{"SL_Model.pdf_tex"}}
    \fbox{
    \includegraphics{{"/home/tomkelly/Downloads/Pathway_Structure/graph_plot_GPCR_Downstream_exprSL_2".png}}
   }
   }
   \end{center}
   \caption[Synthetic Lethality in the GPCR Downstream]{\small \textbf{Synthetic Lethality in the GPCR Downstream.} The Reactome GPCR Downstream pathway with synthetic lethal candidates coloured as shown in the Legend.
}
\label{fig:SL_Pathway_GPCR_Downstream}
\end{mdframed}
\end{figure*}

\begin{figure*}[!htp]
\begin{mdframed}
  \begin{center}
  \resizebox{0.85 \textwidth}{!}{
    %\input{{{"SL_Model.pdf_tex"}}
    \fbox{
    \includegraphics{{"/home/tomkelly/Downloads/Pathway_Structure/graph_plot_TranslationElongation_exprSL_2".png}}
   }
   }
   \end{center}
   \caption[Synthetic Lethality in the Translation Elongation]{\small \textbf{Synthetic Lethality in the Translation Elongation.} The Reactome Translation Elongation pathway with synthetic lethal candidates coloured as shown in the Legend.
}
\label{fig:SL_Pathway_TranslationElongation}
\end{mdframed}
\end{figure*}


\begin{figure*}[!htp]
\begin{mdframed}
  \begin{center}
  \resizebox{0.85 \textwidth}{!}{
    %\input{{{"SL_Model.pdf_tex"}}
    \fbox{
    \includegraphics{{"/home/tomkelly/Downloads/Pathway_Structure/graph_plot_NMD_exprSL_2".pdf}}
   }
   }
   \end{center}
   \caption[Synthetic Lethality in the Nonsense-mediated Decay]{\small \textbf{Synthetic Lethality in the Nonsense-mediated Decay.} The Reactome Nonsense-mediated Decay pathway with synthetic lethal candidates coloured as shown in the Legend.
}
\label{fig:SL_Pathway_NMD}
\end{mdframed}
\end{figure*}

\begin{figure*}[!htp]
\begin{mdframed}
  \begin{center}
  \resizebox{0.85 \textwidth}{!}{
    %\input{{{"SL_Model.pdf_tex"}}
    \fbox{
    \includegraphics{{"/home/tomkelly/Downloads/Pathway_Structure/graph_plot_Three_prime_UTR_exprSL_2".png}}
   }
   }
   \end{center}
   \caption[Synthetic Lethality in the 3$^\prime$ UTR]{\small \textbf{Synthetic Lethality in the 3$^\prime$ UTR.} The Reactome 3$^\prime$ UTR pathway with synthetic lethal candidates coloured as shown in the Legend.
}
\label{fig:SL_Pathway_Three_prime_UTR}
\end{mdframed}
\end{figure*}

\FloatBarrier

\chapter{Pathway Connectivity for Mutation SLIPT}
\label{appendix:connectivity_mtSL}

\begin{figure*}[!htp]
\begin{mdframed}
  \begin{center}
  \resizebox{0.95 \textwidth}{!}{
    %\input{{{"SL_Model.pdf_tex"}}
    \fbox{
    \includegraphics{{"/home/tomkelly/Downloads/Pathway_Structure/Centrality_mtSL/Pi3K_network_vertex_degree".png}}
   }
   }
   \end{center}
   \caption[Synthetic Lethality and Vertex Degree]{\small \textbf{Synthetic Lethality and Vertex Degree.} Synthetic Lethality and Vertex Degree.
}
\label{fig:mtSL_Pathway_PI3K_Vertex_Degree}
\end{mdframed}
\end{figure*}

\begin{figure*}[!htp]
\begin{mdframed}
  \begin{center}
  \resizebox{0.95 \textwidth}{!}{
    %\input{{{"SL_Model.pdf_tex"}}
    \fbox{
    \includegraphics{{"/home/tomkelly/Downloads/Pathway_Structure/Centrality_mtSL/Pi3K_network_Info_Centrality(Log)".png}}
   }
   }
   \end{center}
   \caption[Synthetic Lethality and Centrality]{\small \textbf{Synthetic Lethality and Centrality.} Synthetic Lethality and Information Centrality (log-scale).
}
\label{fig:mtSL_Pathway_PI3K_InfoCent}
\end{mdframed}
\end{figure*}

\begin{figure*}[!htp]
\begin{mdframed}
  \begin{center}
  \resizebox{0.95 \textwidth}{!}{
    %\input{{{"SL_Model.pdf_tex"}}
    \fbox{
    \includegraphics{{"/home/tomkelly/Downloads/Pathway_Structure/Centrality_mtSL/Pi3K_network_pagerank".png}}
   }
   }
   \end{center}
   \caption[Synthetic Lethality and PageRank]{\small \textbf{Synthetic Lethality and PageRank.} Synthetic Lethality and PageRank.
}
\label{fig:mtSL_Pathway_PI3K_PageRank}
\end{mdframed}
\end{figure*}


\FloatBarrier

\chapter{Pathway Structure for Mutation SLIPT}
\label{appendix:Pathway_Structure_mtSL}

\FloatBarrier

\begin{table*}[!htb]
\caption{Information centrality for genes and molecules in the Reactome network}
\label{tab:pathway_str_mtSL}
\noindent\makebox[\textwidth][c]{%               %centering
\resizebox{1.1 \textwidth}{!}{
\begin{threeparttable}
\begin{tabular}{lccccccccccc}
                                          & \multicolumn{2}{l}{Graph:} & \multicolumn{2}{l}{States:} & \multicolumn{4}{l}{Observed:}        & \multicolumn{2}{l}{Permutation p-value:} \\
\hline
Pathway                                   & Nodes & Edges  & mtSL & siRNA & Up   & Down & Up$-$Down & Up$/$Down           & Up$-$Down & Down$-$Up \\
\hline
  \rowcolor{black!10}
PI3K Cascade                              & 138   & 1495   & 42   & 25    & 131  & 123  & 8       & 1.065             & 0.4473 & 0.5466   \\
  \rowcolor{black!5}
PI3K/AKT Signaling in Cancer              & 275   & 12882  & 56   & 44    & 478  & 440  & 38      & 1.086             & 0.4163 & 0.5810   \\
  \rowcolor{black!10}
G$_{\alpha i}$ Signaling                  & 292   & 22003  & 57   & 58    & 543  & 866  & -323    & 0.627             & 0.9507 & \textbf{0.0488}             \\  
  \rowcolor{black!5}
GPCR downstream                           & 1270  & 142071 & 218  & 160   & 7632 & 6500 & 1132    & 1.174             & 0.1707 & 0.8291   \\
  \rowcolor{black!10}
Elastic fibre formation                   & 42    & 175    & 16   & 7     & 6    & 7    & -1      & 0.857             & 0.5512 & 0.3681   \\
  \rowcolor{black!5}
Extracellular matrix                      & 299   & 3677   & 81   & 29    & 313  & 347  & -34     & 0.902             & 0.5762 & 0.4215   \\
  \rowcolor{black!10}
Formation of Fibrin                       & 52    & 243    & 11   & 5     & 8    & 19   & -11     & 0.421             & 0.7993 & 0.1800   \\
  \rowcolor{black!5}
Nonsense-Mediated Decay                   & 103   & 102    & 56   & 2     & 0    & 0    & 0       &                   & 0.197  & 0.1373   \\
  \rowcolor{black!10}
3$^\prime$-UTR-mediated translational regulation & 107   & 2860   & 56   & 1     & 52   & 1    & 51      & 52                & 0.1210  & 0.8751   \\
  \rowcolor{black!5}
Eukaryotic Translation Elongation         & 92    & 3746   & 57   & 0     & 0    & 0    & 0       &                   & 0.4952 & 0.4892   \\ 
\hline
\end{tabular}
\begin{tablenotes}
\raggedright \small
Pathways in the Reactome network tested for structural relationships between mtSLIPT and siRNA genes by resampling (raw p-value)

Significant resampling in bold

Sampling only within target pathway

Number of siRNA+SLIPT matched to observed

siRNA+SLIPT kept for up/down evaluation 
\end{tablenotes}
\end{threeparttable}
}
}
\end{table*}


\FloatBarrier

\chapter{Information Centrality for Gene Essentiality}
\label{appendix:infocent_essential}

\FloatBarrier

%Reactome Network structure and Information Centrality as a measure of gene essentiality

Network structure is another useful strategy to analyse gene function and this has been used to investigate network properties of a network constructed from of Reactome pathways imported with the paxtoolsr R package (Demir et al. 2010). Most notably, information centrality which has been proposed as a measure of gene essentiality was calculated as performed by Kranthi et al. (2013) using the efficiency and shortest path between each pair or nodes in the network before and after a node of interest is removed to test the importance of a node to network connectivity. Reactome contains substrates and cofactors in addition to genes or proteins, supporting the idea of centrality as a measure of essentiality, a number nodes with the highest centrality were essential nutrients including Mg\textsuperscript{2$+$}, Ca\textsuperscript{2$+$}, Zn\textsuperscript{2$+$},  and Fe\textsuperscript{3$+$}.


\begin{table*}[!htb]
\caption{Information centrality for genes and molecules in the Reactome network}
\label{tab:pathway_ccle_brca_exprSL}
\centering
\resizebox{0.5 \textwidth}{!}{
\begin{threeparttable}
\begin{tabular}{>{\em}lc}
  \hline
  \em{\textbf{Node}} & \textbf{Centrality} \\
  \hline
  \rowcolor{black!10}
  ZNF473 & 0.0510 \\ 
  \rowcolor{black!5}
  \em{magnesium(2+)} & 0.0082 \\ 
  \rowcolor{black!10}
  XBP1 & 0.0053 \\ 
  \rowcolor{black!5}
  \em{calcium(2+)} & 0.0050 \\ 
  \rowcolor{black!10}
  \em{zinc(2+)} & 0.0048 \\ 
  \rowcolor{black!5}
  \em{iron atom} & 0.0041 \\ 
  \rowcolor{black!10}
  FMN & 0.0040 \\ 
  \rowcolor{black!5}
  AGT & 0.0037 \\ 
  \rowcolor{black!10}
  HSP90AA1 & 0.0029 \\ 
  \rowcolor{black!5}
  \em{phosphatidyl-L-serine} & 0.0029 \\ 
  \rowcolor{black!10}
  P2RX7 & 0.0026 \\ 
  \rowcolor{black!5}
  PANX1 & 0.0024 \\ 
  \rowcolor{black!10}
  NCAM1 & 0.0022 \\ 
  \rowcolor{black!5}
  NUDT1 & 0.0021 \\ 
  \rowcolor{black!10}
  PLAUR & 0.0020 \\ 
  \rowcolor{black!5}
  IL8 & 0.0020 \\ 
  \rowcolor{black!10}
  HSPA8 & 0.0019 \\ 
  \rowcolor{black!5}
  TYROBP & 0.0019 \\ 
  \rowcolor{black!10}
  CASP3 & 0.0017 \\ 
  \rowcolor{black!5}
  GNAL & 0.0015 \\ 
  \rowcolor{black!10}
  CBLB & 0.0015 \\ 
  \rowcolor{black!5}
  HBB & 0.0014 \\ 
  \rowcolor{black!10}
  GATA4 & 0.0013 \\ 
  \rowcolor{black!5}
  TGS1 & 0.0013 \\ 
  \rowcolor{black!10}
  CTNNB1 & 0.0012 \\ 
\hline
\end{tabular}
\begin{tablenotes}
\raggedright \small
Highest information centrality for genes (proteins), cofactors, and minerals in the Reactome network 
\end{tablenotes}
\end{threeparttable}
}
\end{table*}


\begin{figure*}[!htp]
\begin{mdframed}
  \begin{center}
  \resizebox{0.95 \textwidth}{!}{
    %\input{{{"SL_Model.pdf_tex"}}
    \fbox{
    \includegraphics{{"/home/tomkelly/Downloads/Networks/infocent_genes".pdf}}
   }
   }
   \end{center}
   \caption[Information centrality distribution]{\small \textbf{Information centrality distribution.} Information centrality in the Reactome network for nodes, including genes/proteins and other biomolecules.
}
\label{fig:mtSL_Pathway_PI3K_PageRank}
\end{mdframed}
\end{figure*}