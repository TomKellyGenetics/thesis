\chapter{Methods, Techniques, and Resources}
\label{chap:methods}

\section{Overview/meta-text}
\section{Bioinformatics to Enable Genomics Research}
\subsection{Public Data and Software Packages}
Bioinformatics resources, such as databases and methods, have become an integral part of genetics and genomics research. Reference genomes are used routinely to facilitate more effective experiments and for mapping reads from later genomics and transcriptomics studies. These include studies of large-scale genetic variation, including risk factors for disease, and gene expression studies, including those in cancers.

Similarly, various gene expression databases have been developed for sharing gene expression data from previous studies, including Gene Expression Omnibus (GEO), caArray, and arrayExpress. These were originally developed to share microarray gene expression data but now many support RNA-Seq data (with the benefits discussed previously) and have set a precedent for data sharing, data mining, and the wider benefits of publicly available data for enabling the scientific community further utilise the data compared to a single research group or consortium. Such practices of integrating findings from publicly available genomics data with the research questions and experimental results of individual research groups has carried over into RNA-Seq datasets including the large-scale cancer genomics projects (such as TCGA). The thesis is one such example of an investigation enabled by this wider movement and tools developed in various disciplines to generate, disseminate, and process genomic-scale data.
 
Along with databases, it is also becoming common practice for Bioinformatics researcher to release their code open-source or provide a software package to enable replication of the findings or further applications of the methods. This is part of a wider movement in software and data analysis with many tools to facilitate such work being released for use in Linux or the R programming environment. In addition to the R packages hosted on CRAN, the Bioconductor repositories also contain many packages specifically for applications in Bioinformatics, and the GitHub site hosts many packages in various stages of development and early release. Packages from these various sources have been used throughout this project and cited where possible. Several R packages have been developed during this thesis project and either publicly released on GitHub or prepared to accompany a publication.


\subsubsection{[more detail e.g., TCGA, Reactome]}
Gene expression data from normal and tumour samples was analysed from publicly available sources, using the TCGA \cite{TCGA2012} and the International Cancer Genome Consortium (ICGC) \cite{Zhang2011} web portals.
%%The Cancer Genome Altas project supports a wide range of cancers including gene expression for breast invasive adenocarcinoma with 600 samples (63 normal, 534 primary tumour, and 3 metastases) available the AligentG4502 244K cDNA microarray mapping 17811 genes \cite{TCGA2012}. TCGA provides microarray expression data processed with loess normalisation and integrated to give an expression call for each gene. %% Array data removed
Raw count and RSEM normalised TCGA expression data from Illumina RNA-Seq protocols were available from 1,177 samples (113 normal, 1,057 primary tumour, and 7 metastases) for 20,501 genes. TCGA somatic mutation data for 981 samples (976 primary tumours and 5 metastases) across 25,836 genes were available including 969 samples (964 primary tumours and 5 metastases) with corresponding RNA-Seq expression data and 19,166 genes mapped from Ensembl identifiers to gene symbols, of which 16,156 had corresponding gene expression information.


\subsection{Computational Tools and Enabling Biological Research (remove/state assumptions only)}
%\subsubsection{Computational Tools for Biological Research}
%In addition to hosting data repositories on the web, tools developed with computational expertise have had wider benefit in  genetics research. One of the main impacts of a techniques developed from computer science is the alignment of reads, to either assemble a genome \textit{de novo} from it's reads or map reads to a pre-existing reference genome. Mapping reads is commonplace to call variants between samples, this is useful for studies of human disease interested in risks of these variants or wider application such as comparing populations or species in an evolution phylogeny. Mapping reads has further utility in functional genetics to identify which regions or a genome are expressed or have DNA methylation. Similarly, mapping is used to map RNA-Seq or Bisulfite-Seq reads to measure gene expression or DNA methylation across the genome in a cohort or sample. While mapping is not performed in this thesis, it has performed an important role in the adoption of genomics in genetics research.

There remains debate regarding the optimal methods to perform an alignment. However, the statistical and biological aspects of Bioinformatics are the focus of this thesis, comparing alignment methods is outside the scope of this investigations. The TCGA project used the widely adopted ''Bowtie`` tools for alignment, with ''mapslice`` to detect splice sites, and the Reads Per Kilobase per Million mapped reads (RPKM) approach to qualify reads per transcript as a measure of gene expression. These are widely acceptable tools for processing RNA-Seq data and this is the raw data publicly available from TCGA.

\paragraph{High Performance and Parallel Computing}
Another significant development in computer science for bioinformatics is parallel computing, performing independent operations in separate cores, such ''multithreading`` is widely used to increase the time to compute results. Bioinformatics is particularly amenable to this since performing multiple iterations of a simulation or testing separate genes is often ''embarrassingly parallel``, being completely independent of the results of each other. As such parallel computing is offered by many high-performance ''supercomputers`` including national research infrastructure. The New Zealand eScience Infrastructure (NeSI) is once such computational resource providing the Intel Pan cluster (hosted by the University of Auckland) used throughout this thesis project to optimise and perform computations which would have otherwise been infeasible in the timeframe of thesis. This is another example of how technological developments and infrastructure has enabled research including this project.  

\subsection{Gene Expression Analysis and Statistical Challenges}
\subsubsection{Hypothesis Testing and Multiple Comparisons Procedures}
\subsubsection{Candidate Triage and Integration with Experimental Data}
\subsection{Mathematical Challenges in Bioinformatics}
\subsubsection{Graph Theory, Systems, and Network Biology}
\subsubsection{Matrix Operations and Pathway “Metagenes”}
\section{Data Handling}

The TCGA RNA-Seq data were assessed for batch effects using a correlation matrix of the log-transformed raw counts for which a heatmap (Euclidean distance, complete linkage) is shown in Figure \ref{fig:corr_map}. While no major batch effects were detectable between the samples, 9 samples were excluded due to poor correlation with the remaining samples as detailed in Table \ref{tab:qc}. These samples show unusual density plots compared to the rest of the dataset and low mean read count in Figures \ref{fig:density} and \ref{fig:boxplot}. A heatmap showing key clinical properties of these excluded samples and their correlation with other rest of the samples is shown in Figure \ref{fig:corr_map_part}.
%In both of these cases, a shared tissue source site or patient donor indicates that variations in sample preparation are likely behind the outlying expression profiles. The Christiana Healthcare patients were sequenced in triplicate when replicate samples were rare in this dataset suggesting there were suspected errors in these samples during data generation which have lower mean read count than most of the dataset. %tangent
%Clinical characteristics over-represented in removed samples were ER+, ductal, state 2, luminal A or basal type tumours but these are the most common in the dataset. %relevance
After removal of these samples, the TCGA dataset used for analysis consisted of the remaining 1168 samples (from 1040 patients): 1049 tumour samples, 112 normal tissue for matched samples, and 7 metastases.

\subsection{Normalisation (voom)}

Apart from the PAM50 subtyping procedure \cite{Parker2009} which required RSEM normalised data as (J.S. Parker personal communication), the analysis of the RNA-Seq data presented here was based on raw read count data. Raw read counts were log-scaled, samples were trimmed as described above based on a correlation matrix (Euclidean distance), and the final dataset was TMM normalised \cite{Robinson2010} and then processed using the \texttt{voom} function \cite{Law2014} in the \texttt{limma} R package \cite{limma}. Standardised z-scores of the data were used to plot heatmaps on an appropriate scale.


\subsection{Sample triage}
\subsection{SVD/mg}
\section{Techniques}
\subsection{Clustering}
\subsection{Heatmap}
\subsection{Modeling and Simulations}
\subsubsection{(AU)ROC}
\subsection{Permutation / Resampling}
\subsection{Network Metrics / Techniques}

Network theory is an interdisciplinary field which combines the approaches of in computer science with the metrics and fundamental principles of graph theory, an area of pure mathematics dealing with relationships between sets of discrete elements. \ The first large networks were generated randomly and exhibited interesting small-world and scale-free properties. \ Application of network theory in the life sciences has, until recently, been largely restricted to small networks in sociology or ecology. \ The vast amounts of molecular and cellular data from high-throughput technologies have raised the need for systems-level, network-based, and genome-wide bioinformatics analysis to capture the complexity of a cell at the molecular level and understand aberrations in cancer. 

Graph theory is a branch of pure mathematics which deals with the properties of sets of discrete objects (referred to as a {\textquoteleft}node{\textquoteright} or {\textquoteleft}vertex{\textquoteright}) with some pairs are joined (by a {\textquoteleft}link{\textquoteright} or an {\textquoteleft}edge{\textquoteright}). \ Originally conceived as a reductionist abstraction to solve problems in mathematics and more complex problems later in computer science, graph theory serves as the fundamental basis for a wide range of studies including material physics, traffic analysis, computer architecture, and phylogenetic trees. \ Applications vary depending on the situation modelled, particularly in how the edges between vertices are defined, whether they are directed or weighted, and whether multiple redundant edges between a pair of vertices (referred to as {\textquoteleft}parallel edges{\textquoteright}) or edges connecting a vertex to itself (referred to as {\textquoteleft}loops{\textquoteright}) are permitted in the model. \ Networks are defined such that the edges represent a relationship between the vertices and may be directed, weighted, or contain parallel edges or loops depending on the application. \ 

Network theory is the sub-discipline of graph theory which deals with networks which has become popular due to the vast potential for applications of networks. \ The properties of large networks were studied by constructing random networks by randomly linking a fixed number of nodes (Erd\H{o}s \& R\'enyi 1959; Erd\H{o}s \& R\'enyi 1960). \ Despite the random nature of these networks, properties such as their connectivity were well characterised. \ The vertex degree of random network follows a Poisson distribution, however this property does not hold in nature, suggesting that natural networks are non-random or not formed in this way (Barab\'asi \& Oltvai 2004). \ 


This work formed the foundation for studying complex networks which model features of real world networks not found in Erd\H{o}s and R\'enyi{\textquoteright}s random networks. \ The small world property, made popular by findings in social networks (Milgram 1967; Travers \& Milgram 1969), is the remarkably short path lengths between any nodes in a small world network. \ A small world network is well-connected with a characteristic path length proportional to the logarithm of thenumber of nodes (Watts \& Strogatz 1998). \ \hyperlink{ENREF112}{Watts and Strogatz (1998)} developed a model of random rewiring of a regular network to construct random networks with the small world property and a high clustering coefficient. \ While these properties are more representative of networks occurring in nature, their model is limited by the degree distribution which converges to a Poisson distribution as it is rewired (Barrat \& Weigt 2000). \ 


The degree distribution of naturally occurring networks often follows a power law distribution with the majority of nodes having far fewer connections than average and a small subset of highly connected network {\textquoteleft}hubs{\textquoteright}. \ \hyperlink{ENREF7}{Barab\'asi and Albert (1999)} constructed a network model in an entirely different way to randomly generate scale-free networks which have a power law degree distribution. \ They constructed random networks by preferential attachment, modelling growth of a network by sequentially adding nodes with links to existing nodes. \ The scale-free nature of the random networks was ensured by adding new nodes with an increasing probability of attachment to an existing node if it has higher degree. \ These networks successfully capture the scale-free nature of many real world networks with short characteristic path length and low eccentricity resulting in super small worlds. \ The \hyperlink{ENREF7}{Barab\'asi and Albert (1999)} scale-free networks are limited by a low clustering coefficient and lack of modular structure; however, they have enabled the study of scale-free network topology and served as a basis for modified scale-free models (Dorogovtsev \& Mendes 2003; Holme \& Kim 2002). \ 


\hyperlink{ENREF47}{Han}\hyperlink{ENREF47}{\textit{ et al.}}\hyperlink{ENREF47}{ (2004)} observed dynamic modularity in biological networks and suggested the network structure may underpin genetic robustness and plasticity. \ They focus on network hubs which are more likely to be essential genes and define the subgroups of hubs based on correlation of gene expression with protein-protein interaction partners: {\textquoteleft}party{\textquoteright} hubs (which interact simultaneously with many partners) and {\textquoteleft}date{\textquoteright} hubs (which interact with different partners in different conditions). \ Party and date hubs occurred most frequently within and between network modules respectively. \ Party hubs were considered local regulators, whereas date hubs were considered important to network connectivity as global regulators. \ This distinction between classes of network hubs was supported by differences in tissue specificity and clinical relevance as a proposed predictor of clinical outcome in breast cancer with an AUROC of 0.784 (Taylor\textit{ et al.} 2009). \ However, correlation between expression and protein interactions were not robustly reproduced. \ The importance of date hubs has been criticised for assuming a bimodal distribution and basing the global importance of data hubs on a small subset (Agarwal\textit{ et al.} 2010). \ As an alternative interpretation, \hyperlink{ENREF2}{Agarwal}\hyperlink{ENREF2}{\textit{ et al.}}\hyperlink{ENREF2}{ (2010)} suggest the importance of interactions rather than network hubs as interactions important to the network were between functionally similar proteins. \ Network hubs can also be classed as associative or dissociative depending on whether they tend toward or away from connecting directly to other network hubs (van Steen 2010). \ The associative and dissociative properties can also be used to test whether nodes of a particular subgroup (e.g., gene function) associate with each other. \ 

Applications of network theory are diverse, including uses in social sciences, engineering, and computer science. \ Due to their complexity and difficulty of gathering sufficient empirical data, biological applications of network theory are relatively unexplored. \ High-throughput technologies such as siRNA screens, two-hybrid screens, microarrays and massively parallel sequencing have made generating genome-scale molecular data feasible and enabled analysis of biological networks at the molecular level. \ Many types of inter-related molecular networks can be constructed and analysed, depending on the biological application, the way interactions between molecular components are defined and the data used to generate them as shown in Table 1. \ 

\textbf{Table 1. \ }Types of biological network based on molecular data
\begin{flushleft}
\tablehead{}
\begin{supertabular}{m{4.0490003cm}|m{2.55cm}|m{4.301cm}|m{4.201cm}}
\multicolumn{1}{m{4.0490003cm}}{\cellcolor{white}\bfseries\color{black}
{\fontsize{10pt}{12.0pt}\selectfont \textcolor{black}{Network Type}}} &
\multicolumn{1}{m{2.55cm}}{\cellcolor{white}\bfseries\color{black}
{\fontsize{10pt}{12.0pt}\selectfont \textcolor{black}{Node}}} &
\multicolumn{1}{m{4.301cm}}{\cellcolor{white}\bfseries\color{black}
{\fontsize{10pt}{12.0pt}\selectfont \textcolor{black}{Interaction}}} &
\cellcolor{white}\bfseries\color{black}
{\fontsize{10pt}{12.0pt}\selectfont \textcolor{black}{Data
Generation}}\\\hline
\cellcolor[rgb]{0.87058824,0.91764706,0.9647059}\bfseries\color{black}
{\fontsize{10pt}{12.0pt}\selectfont \textcolor{black}{Co-expression}} &
\cellcolor[rgb]{0.87058824,0.91764706,0.9647059}\color{black}
{\fontsize{10pt}{12.0pt}\selectfont \textcolor{black}{Gene}} &
\cellcolor[rgb]{0.87058824,0.91764706,0.9647059}\color{black}
{\fontsize{10pt}{12.0pt}\selectfont \textcolor{black}{Correlation of
expression}} &
\cellcolor[rgb]{0.87058824,0.91764706,0.9647059}\color{black}
{\fontsize{10pt}{12.0pt}\selectfont \textcolor{black}{Array,
RNA-Seq}}\\\hline
\bfseries {\fontsize{10pt}{12.0pt}\selectfont Protein (Physical)} &
{\fontsize{10pt}{12.0pt}\selectfont Protein} &
{\fontsize{10pt}{12.0pt}\selectfont Protein-protein binding} &
{\fontsize{10pt}{12.0pt}\selectfont Two-Hybrid}\\\hline
\cellcolor[rgb]{0.87058824,0.91764706,0.9647059}\bfseries\color{black}
{\fontsize{10pt}{12.0pt}\selectfont \textcolor{black}{Signal
Transduction}} &
\cellcolor[rgb]{0.87058824,0.91764706,0.9647059}\color{black}
{\fontsize{10pt}{12.0pt}\selectfont \textcolor{black}{Protein}} &
\cellcolor[rgb]{0.87058824,0.91764706,0.9647059}\color{black}
{\fontsize{10pt}{12.0pt}\selectfont \textcolor{black}{Protein mediated
signalling}} &
\cellcolor[rgb]{0.87058824,0.91764706,0.9647059}\color{black}
{\fontsize{10pt}{12.0pt}\selectfont \textcolor{black}{Curate known
pathways}}\\\hline
\bfseries {\fontsize{10pt}{12.0pt}\selectfont Metabolic} &
{\fontsize{10pt}{12.0pt}\selectfont Metabolite or cofactor} &
{\fontsize{10pt}{12.0pt}\selectfont Involved in same reaction (links by
reactions/enzymes)} &
{\fontsize{10pt}{12.0pt}\selectfont Curate known pathways}\\\hline
\cellcolor[rgb]{0.87058824,0.91764706,0.9647059}\bfseries\color{black}
{\fontsize{10pt}{12.0pt}\selectfont \textcolor{black}{Chemical or
Drug}} &
\cellcolor[rgb]{0.87058824,0.91764706,0.9647059}\color{black}
{\fontsize{10pt}{12.0pt}\selectfont \textcolor{black}{Protein}} &
\cellcolor[rgb]{0.87058824,0.91764706,0.9647059}\color{black}
{\fontsize{10pt}{12.0pt}\selectfont \textcolor{black}{Targets of the
same drug}} &
\cellcolor[rgb]{0.87058824,0.91764706,0.9647059}\color{black}
{\fontsize{10pt}{12.0pt}\selectfont \textcolor{black}{Curate known drug
targets}}\\\hline
\bfseries {\fontsize{10pt}{12.0pt}\selectfont Regulation} &
{\fontsize{10pt}{12.0pt}\selectfont Gene} &
{\fontsize{10pt}{12.0pt}\selectfont Regulate each other by encoded
proteins} &
{\fontsize{10pt}{12.0pt}\selectfont Array, RNA-Seq, ChIP}\\\hline
\cellcolor[rgb]{0.87058824,0.91764706,0.9647059}\bfseries\color{black}
{\fontsize{10pt}{12.0pt}\selectfont \textcolor{black}{RNA or DNA
binding}} &
\cellcolor[rgb]{0.87058824,0.91764706,0.9647059}\color{black}
{\fontsize{10pt}{12.0pt}\selectfont \textcolor{black}{Gene or miRNA}} &
\cellcolor[rgb]{0.87058824,0.91764706,0.9647059}\color{black}
{\fontsize{10pt}{12.0pt}\selectfont \textcolor{black}{Binding of
encoded RNA or protein or DNA or mRNA}} &
\cellcolor[rgb]{0.87058824,0.91764706,0.9647059}\color{black}
{\fontsize{10pt}{12.0pt}\selectfont \textcolor{black}{ChIP,
RIP}}\\\hline
\bfseries {\fontsize{10pt}{12.0pt}\selectfont Functional} &
{\fontsize{10pt}{12.0pt}\selectfont Gene} &
{\fontsize{10pt}{12.0pt}\selectfont Shared gene function} &
{\fontsize{10pt}{12.0pt}\selectfont Curate known pathways}\\\hline
\cellcolor[rgb]{0.87058824,0.91764706,0.9647059}\bfseries\color{black}
{\fontsize{10pt}{12.0pt}\selectfont \textcolor{black}{Genetic
Interaction}} &
\cellcolor[rgb]{0.87058824,0.91764706,0.9647059}\color{black}
{\fontsize{10pt}{12.0pt}\selectfont \textcolor{black}{Gene}} &
\cellcolor[rgb]{0.87058824,0.91764706,0.9647059}\color{black}
{\fontsize{10pt}{12.0pt}\selectfont \textcolor{black}{Unexpected
phenotype with combined loss of function}} &
\cellcolor[rgb]{0.87058824,0.91764706,0.9647059}\color{black}
{\fontsize{10pt}{12.0pt}\selectfont \textcolor{black}{SGA, EMAP, DAmP,
siRNA}}\\\hline
\end{supertabular}
\end{flushleft}

Genetic interaction networks will be the focus of this project because they are relatively unexplored compared to other molecular networks, have potential for applications in drug discovery (particularly cancer treatment), and may lead to better understanding of the role of genetics in cellular function and disease. \ Genetic interactions are usually studied at a high-throughput scale in simple model organisms such as bacteria, yeasts or the nematode worm; studies in humans, mammals, and non-model organisms (where applications would have the most societal impact) are limited by cost, time and labour constraints. \ Computational approaches with effective predictive models are the only feasible approach to study the connectivity of a biological network in a complex metazoan cell at the genome-scale.


\section{Pathway Structure Methods}
\subsection{Sourcing graph structure data}
\subsection{Constructing pathway subgraphs}
\subsection{Centrality Measures}


\subsection[Data Sources]{Data Sources}

As summarised in Table 5, there is are a vast array of publicly available resources for model organism, human, and cancer genomics analysis. These will be used as a resource for bioinformatics analysis of genetic interactions and networks in addition to modelling and simulation approaches.  

\textbf{Table 5. }Data sources for Research
\begin{flushleft}
\tablehead{}
\begin{supertabular}{m{2.299cm}|m{3.8009999cm}|m{3.552cm}|m{6.5490003cm}}
\multicolumn{1}{m{2.299cm}}{\cellcolor{white}\bfseries\color{black}
Database} &
\multicolumn{1}{m{3.8009999cm}}{\cellcolor{white}~
} &
\multicolumn{1}{m{3.552cm}}{\cellcolor{white}\bfseries\color{black}
Study Type} &
\cellcolor{white}\bfseries\color{black} Data Types Supported\\\hline
\cellcolor[rgb]{0.8509804,0.8862745,0.9529412}\color{black}
\textcolor{black}{TCGA} &
\cellcolor[rgb]{0.8509804,0.8862745,0.9529412}\color{black} Cancer
Genome Atlas &
\cellcolor[rgb]{0.8509804,0.8862745,0.9529412}\color{black} Cancer &
\cellcolor[rgb]{0.8509804,0.8862745,0.9529412}\color{black} Sequence,
Mutation, CNV, SNP, expression, DNA Methylation, Clinical,
RNA-Seq\\\hline
ICGC &
International Cancer Genome Consortium &
Cancer &
Sequence, Mutation, CNV, SNP, expression, DNA Methylation, Clinical,
RNA-Seq\\\hline
\cellcolor[rgb]{0.8509804,0.8862745,0.9529412}\color{black}
\textcolor{black}{ENCODE} &
\cellcolor[rgb]{0.8509804,0.8862745,0.9529412}\color{black}
Encyclopaedia of DNA Elements &
\cellcolor[rgb]{0.8509804,0.8862745,0.9529412}{\color{black} Human
Normal Tissue}

\color{black} Cancer Cell Line &
\cellcolor[rgb]{0.8509804,0.8862745,0.9529412}{\color{black} Sequence,
expression, DNA/RNA binding}

\color{black} RNA-Seq, ChIP-Seq, RIP-Seq\\\hline
CCLE &
Cell Line Encyclopaedia &
Cancer Cell Line &
DNA CNV, expression, mutation, drug sensitivity\\\hline
\cellcolor[rgb]{0.8509804,0.8862745,0.9529412}\color{black}
\textcolor{black}{GEO} &
\cellcolor[rgb]{0.8509804,0.8862745,0.9529412}\color{black} Gene
Expression Omnibus &
\cellcolor[rgb]{0.8509804,0.8862745,0.9529412}\color{black} Various &
\cellcolor[rgb]{0.8509804,0.8862745,0.9529412}\color{black} Gene
Expression, RNA-Seq\\\hline
GENT &
Gene Expression Atlas &
Human Normal Tissue and Cancer &
Gene Expression\\\hline
\cellcolor[rgb]{0.8509804,0.8862745,0.9529412}\color{black}
\textcolor{black}{BIND} &
\cellcolor[rgb]{0.8509804,0.8862745,0.9529412}\color{black} Biomolecular
interaction network database &
\cellcolor[rgb]{0.8509804,0.8862745,0.9529412}{\color{black} Model
Organisms}

\color{black} Human &
\cellcolor[rgb]{0.8509804,0.8862745,0.9529412}{\color{black} DNA, RNA,
ligand, or protein binding}

\color{black} Two-hybrid data, ChIP, CLIP, RIP\\\hline
STRING &
Search Tool for retrieving interacting genes/proteins &
Various &
Protein-protein interactions: curated and predicted from MINT, PPRD,
BIND, DIP, BioGRID, KEGG, Reactome, IntAct, EcoCyc, NCI, and GO\\\hline
\cellcolor[rgb]{0.8509804,0.8862745,0.9529412}\color{black}
\textcolor{black}{BioGRID} &
\cellcolor[rgb]{0.8509804,0.8862745,0.9529412}\color{black} Biological
General Repository for Interaction Datasets  &
\cellcolor[rgb]{0.8509804,0.8862745,0.9529412}{\color{black}
\textit{\textcolor{black}{S. cerevisiae}},}

{\color{black} \textit{\textcolor{black}{S. pombe}},}

\color{black} \textit{\textcolor{black}{A. thaliana}}, etc &
\cellcolor[rgb]{0.8509804,0.8862745,0.9529412}\color{black} Protein and
Genetic Interaction\\\hline
Human Interactome &
Human Interactome &
Human &
Protein Interactions\\\hline
\cellcolor[rgb]{0.8509804,0.8862745,0.9529412}\color{black}
\textcolor{black}{DRYGIN} &
\cellcolor[rgb]{0.8509804,0.8862745,0.9529412}\color{black} Data
Repository of Yeast Genetic Interactions &
\cellcolor[rgb]{0.8509804,0.8862745,0.9529412}\color{black} Yeast &
\cellcolor[rgb]{0.8509804,0.8862745,0.9529412}\color{black} Genetic
Interactions\\\hline
SGD &
Saccharomyces Genome Database &
Yeast &
Sequence, Expression, Protein and Genetic Interactions\\\hline
\cellcolor[rgb]{0.8509804,0.8862745,0.9529412}\color{black}
\textcolor{black}{GO} &
\cellcolor[rgb]{0.8509804,0.8862745,0.9529412}\color{black} Gene
Ontology &
\cellcolor[rgb]{0.8509804,0.8862745,0.9529412}\color{black} Various &
\cellcolor[rgb]{0.8509804,0.8862745,0.9529412}\color{black} Gene
Function\\\hline
KEGG &
Kyoto Encyclopedia of Genes and Genomes &
Various &
Gene Function and Pathway\\\hline
\cellcolor[rgb]{0.8509804,0.8862745,0.9529412}\color{black}
\textcolor{black}{Reactome} &
\cellcolor[rgb]{0.8509804,0.8862745,0.9529412}~
 &
\cellcolor[rgb]{0.8509804,0.8862745,0.9529412}\color{black} Various &
\cellcolor[rgb]{0.8509804,0.8862745,0.9529412}\color{black} Gene
Function and Pathway\\\hline
DrugBank &
~
 &
Human &
Chemical and drug target sequence/structure\\\hline
\cellcolor[rgb]{0.8509804,0.8862745,0.9529412}\color{black}
\textcolor{black}{TTD} &
\cellcolor[rgb]{0.8509804,0.8862745,0.9529412}\color{black} Therapeutic
Target &
\cellcolor[rgb]{0.8509804,0.8862745,0.9529412}\color{black} Human &
\cellcolor[rgb]{0.8509804,0.8862745,0.9529412}\color{black} Protein and
Nucleic Acid drug targets\\\hline
TDR &
Targets database &
Human &
Chemical Genomics (Tropical Disease)\\\hline
\cellcolor[rgb]{0.8509804,0.8862745,0.9529412}\color{black}
\textcolor{black}{BindingDB} &
\cellcolor[rgb]{0.8509804,0.8862745,0.9529412}~
 &
\cellcolor[rgb]{0.8509804,0.8862745,0.9529412}~
 &
\cellcolor[rgb]{0.8509804,0.8862745,0.9529412}\color{black} Protein-drug
binding affinity\\\hline
\end{supertabular}
\end{flushleft}


\chapter{Methods Developed During Thesis}
\label{chap:methods_dev}
\section{Overview/meta-text}
\section{Developing a Synthetic Lethal Detection Methodology}
\subsection{Rationale and Design of Test}
\subsection{Synthetic Lethal Detection Method}
The SLIPT methodology identifies gene expression patterns consistent with synthetic lethal interactions between a query gene and a panel of candidate interacting partners. Gene expression is called low, medium, or high by separating samples into tertiles (3-quantiles) for each gene. Genes with insufficient expression across all samples were excluded by requiring that the first tertile of raw counts is above zero. Then a $\chi^2$ test is performed between the query gene and each candidate partner, the p-values for the $\chi^2$ test were corrected for multiple testing using false discovery rate (FDR) error control to reduce false positives for large candidate gene panels \cite{fdr1995}. Significance was called only if FDR adjusted p-values were below the threshold $p < 0.05$. A synthetic lethal interaction is predicted  (as shown in Figure \ref{fig:SLIPT_Method}) when (i) the $\chi^2$ test is significant; (ii) observed low-query, low-candidate samples are less frequent than expected; and (iii) observed low-query, high-candidate and high-query, low-candidate samples are more frequent than expected.
%The query and candidate genes are swapped to replicate the directional condition. %redundant
%Where synthetic lethality is scored SL-Q if it is predicted in query-low samples and SL-C if it is predicted in candidate-low samples (as shown in Figure \ref{fig:SLIPT_Method}). Synthetic lethality is only reported in this text if it meets both of these conditions and a significant p-value where it is scored SL-2. %too detailed


%This methodology was adapted to used pathway metagene quantiles rather than gene expression as an input for pathway synthetic lethality testing. The p-values for $\chi^2$ tests were also corrected for multiple testing with the false discovery rate \cite{fdr1995} across all pathways tested from the same database and with significance defined as a FDR adjusted p-values $p < 0.05$ as above.

%mtSLIPT method
%A similar methodology was developed in both cases to test for synthetic lethality where the query gene has an inactivating mutation in some patients. Since most mutations, particularly in tumour suppressor genes, are deleterious all somatic non-synonymous mutations were counted as mutant and synthetic lethality was tested with the query gene changed accordingly (as shown in Figure \ref{fig:mtSLIPT_Method}. To distinguish these methods they are abbreviated to exprSLIPT and mtSLIPT respectively depending on the molecular property used to define low gene activity of the query gene.


\section{Simulations and Modelling Synthetic Lethality in Expression Data}
\subsection{Synthetic Lethal Modeling}
\subsection{Simulation Procedure}
\section{Assessing the Synthetic Lethal detection methodology (simulation results part 1)}
\subsection{Binomial Simulation of Synthetic lethality[?]}
\subsection{Multivariate Normal Simulation of Synthetic lethality[+dir?]}
\subsubsection{Receiver Operating Characteristic Curves}
\subsubsection{Simulated Expression Heatmaps}
\subsection{Replication Simulation Heatmap}
\section{Graph Structure Methods}
\subsection{Upstream and downstream gene detection}
\subsection{Permutation analysis}
\subsection{Simulating gene expression from graph structure}
\section{Customised methods and R packages developed}
\subsection{slipt}
\subsection{plotting}
\subsection{simulation from graph structures}
\subsection{igraph methods}


%\subsection*{Comparison Subsetting}
%\subsubsection*{Comparison to Experimental Screen}
% Comparison to screen validation data utilised primary screen data from a cell line experiment by Telford \textit{et al.} \cite{Telford2015} using the synthetic lethal selection criteria defined by the authors based on gene knockout viability in isogenic MCF10A breast cell lines. A direct comparison was implemented using Venn diagrams from the \texttt{limma} package in R \cite{limma}. This comparison used a `universe' of all 16,298 genes tested by both methods, which had sufficient non-zero variation to define 3-quantiles in the TCGA expression dataset \cite{TCGA2012} and had siRNA targets in the primary screen \cite{Telford2015}, excluding any genes not tested by one of these approaches. A $\chi^2$ test was performed on this reduced universe to test for association between each synthetic lethal identification approach.

%\subsubsection*{Gene Set Analysis}
%Gene set over-represent\-ation was tested for genes within each sector of the Venn diagram to compare pathways specific to each approach to those identified by both. This over-represent\-ation analysis was performed using the hypergeometric test implemented in R as described below \cite{HyperGeo, R_core}. Further analysis of the intersection of these approaches was performed using the resampling procedure as described below.

%\subsection*{Pathway Analysis}
%\subsubsection*{Pathway Over-representation Analysis}
%Pathway over-represent\-ation analysis was performed using the \texttt{phyper} function of the R stats package unless specified otherwise \cite{HyperGeo, R_core}. This performs a hypergeometric test for over-represent\-ation of members of a pathway in a given group of genes as suggested by Rivals \textit{et al.} \cite{Rivals2007}. 1,652 pathways were defined using the Reactome database \cite{Reactome} (version 52). Pathway over-represent\-ation used a `universe' of genes tested by both approaches for the intersection, as in the Venn analysis above, and otherwise used all of the genes tested by the respective approach. 1,030 Reactome pathways were used with a size filtering criteria of containing at least 10 genes and no more than 500 genes, as recommended by Araki \textit{et al.} \cite{genesetdb}.

%\subsubsection*{Pathway Resampling Analysis}
%Resamping was performed to test whether over-represented pathways were to be expected between the two methods by chance. The number of predicted genes from SLIPT (4,629) was sampled randomly without replacement from the gene `universe' as defined by the Venn diagram in Figure \ref{fig:Venn_allgenes}, and as described above.  This sample was compared to the 2,203 experimental synthetic lethal candidates from screen data \cite{Telford2015} that were tested by SLIPT to resample the intersection between computationally and experimentally identified synthetic lethal candidates. For both the sample from universe gene set and the intersection with experimental screen candidates, a $\chi^2$ test was performed for association with each of the Reactome pathways for a computationally feasible measure of pathway over-represent\-ation.

%This procedure was repeated in parallel over 1 million replicates using the \texttt{snow} and \texttt{Rmpi} R packages \cite{snow, Rmpi} to generate a null distribution of $\chi^2$ values for each Reactome pathway. An estimate of significance for each pathway was generated as a empirical p-value from the proportion of 1 million $\chi^2$ expected values from random samples that exceeded the $\chi^2$ value observed performing the same over-represent\-ation procedure on the predicted synthetic lethal candidates for \textit{CDH1} from SLIPT. Note that the above procedure does not assume a fixed intersection size and the expected distribution also can be compared to the observed number of intersection genes in a similar manner. 

%\subsubsection*{Pathway Metagene Analysis}
%insert text

%\subsection*{Heatmap Procedure}
%\subsubsection*{Parameters}
%Heatmaps were generated in R using modifications to the \texttt{heatmap.2} from the \texttt{gplots} package \cite{gplots} (with annotation modifications documented in the \texttt{heatmap.2x} R package on GitHub: \url{https://github.com/TomKellyGenetics/heatmap.2x}.

%\subsubsection*{Correlation matrix}
%Pairwise Pearson correlation between samples (log-scale) were plotted with Euclidean distance and complete linkage on a $[-1,1]$ blue-red scale.
%\subsubsection*{Expression (gene) matrix}
%Raw (log-scale) read counts or voom normalised counts per gene (as specified) were plotted with correlation (Pearson) distance and complete linkage as normalised z-scores on a $[-3,+3]$ blue-red scale.
%\subsubsection*{Expression (metagene) matrix}
%raw (log-scale) read counts or voom normalised counts per gene (as specified) were as normalised z-scores on a $[-3,+3]$ scale and used for generating pathway metagenes which were plotted with Correlation (Pearson) distance and complete linkage on a ranked $[0,1]$ blue-red scale.
%\subsubsection*{Annotation}
%TCGA clinical and mutation data is plotted as colour bars above heatmaps on a continuous red-blue spectrum, or as discrete colours as described in legends or figure captions. All clinical and molecular data comes from TCGA \cite{TCGA2012} or ICGC \cite{Zhang2011} sources apart from PAM50 intrinsic subtype which was %<downloaded from the University of California Santa Cruz website from microarray data in 2012> OR 
%calculated using the PAM50 methodology as described by Parker \textit{et al.} \cite{Parker2009} from RSEM normalised RNA-Seq data using centroids provided by J.S. Parker (personal communication). %in 2015
%Row annotation bars contain similar colours, which correspond to genes in expression data, factors including genes in pathway datasets, experimental, and computational results again on a red-blue colour spectrum or discrete colours as described accompanying each figure. 

%\subsection*{Implementation}
%All analyses were performed via R \cite{R_core} and custom shell scripts using local and national computing infrastructure. Novel analysis approaches for synthetic lethality are available in the \texttt{slipt} R package on GitHub (\url{http://github.com/TomKellyGenetics/slipt}) while other analyses used various Bioconductor packages to annotate and process genetic data \cite{Gentleman2004} (\url{http://bioconductor.org}).
