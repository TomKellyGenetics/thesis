\chapter{Literature Review: Bioinformatics Analysis of Empirical Biological and
Molecular Networks}
\label{chap:lit_review}

\section{Abstract}
Genomics technologies, including various techniques involving
microarrays, massively parallel sequencing of DNA or RNA, have
stimulated genetic interaction research and screens in mammalian or
yeast cells. A vast amount of data on genotype, mutations, gene
expression, genetic interactions, and protein abundance or binding for
normal and cancerous cells is publicly available from databases such as
The Cancer Genome Atlas (TCGA), The International of Cancer Genome
Consortium (ICGC), Gene Expression Omnibus (GEO), ArrayExpress,
Encyclopaedia of DNA Elements (ENCODE), the Cancer Cell Line
Encyclopaedia (CCLE), BioGrid, and the Human Interactome database.
\ Saccharomyces Genome Database (SGD) and Data Repository of Yeast
Genetic Interactions (DRYGIN) provide genomic and genetic interaction
data for species comparison. 


Examples of gene interactions include the synthetic lethal interaction
of BRCA1 or BRCA2 with PARP1 in breast cancer and the unexpected
synergistic effect of BRAF\textsuperscript{V600E} and EGFR inhibitors
in colon cancer. PARP inhibitors are one of the first targeted
therapeutics against a tumour suppressor mutation with success in
clinical trials. BRAF\textsuperscript{V600E} and EGFR inhibitors
block a feedback regulation loop and enable use of BRAF inhibitors
beyond melanomas. Thus we can develop targeted anticancer
therapeutics which exploit complex interactions to distinguish normal
and cancerous cells. Investigation of such interactions at a network
level could enable deeper understanding of their impact on cancers and
prediction of novel drug targets.


Genetic and chemical high-throughput screens have already identified
unexpected, functionally informative, and clinically relevant synthetic
lethal interactions; including synthetic lethal partners of genes
recurrently mutated in cancer or attributed to familial early-onset
cancer syndromes. While screening presents an appealing strategy for
synthetic lethal discovery, computational approaches are becoming
popular as an alternative or complement to experimental methods to
overcome inherent bias and limitations of experimental screens. An
array of novel computational methods show the need for synthetic lethal
discovery in the fundamental genetics and translational cancer research
community. However, existing computational methods are complex,
error-prone, and difficult to understand, interpret or adopt for
biologically trained researchers. A robust prediction of gene
interactions is an effective and practical approach at a scale of the
entire genome for ideal translational applications, analysis of
biological systems, and constructing functional gene networks.


Network theory is an interdisciplinary field which combines the
approaches of in computer science with the metrics and fundamental
principles of graph theory, an area of pure mathematics dealing with
relationships between sets of discrete elements. The first large
networks were generated randomly and exhibited interesting small-world
and scale-free properties. Application of network theory in the life
sciences has, until recently, been largely restricted to small networks
in sociology or ecology. The vast amounts of molecular and cellular
data from high-throughput technologies have raised the need for
systems-level, network-based, and genome-wide bioinformatics analysis
to capture the complexity of a cell at the molecular level and
understand aberrations in cancer. 



\section{Literature Review}
\subsection{Network Theory}

Graph theory is a branch of pure
mathematics which deals with the properties of sets of discrete objects
(referred to as a {\textquoteleft}node{\textquoteright} or
{\textquoteleft}vertex{\textquoteright}) with some pairs are joined (by
a {\textquoteleft}link{\textquoteright} or an
{\textquoteleft}edge{\textquoteright}). Originally conceived as a
reductionist abstraction to solve problems in mathematics and more
complex problems later in computer science, graph theory serves as the
fundamental basis for a wide range of studies including material
physics, traffic analysis, computer architecture, and phylogenetic
trees. Applications vary depending on the situation modelled,
particularly in how the edges between vertices are defined, whether
they are directed or weighted, and whether multiple redundant edges
between a pair of vertices (referred to as {\textquoteleft}parallel
edges{\textquoteright}) or edges connecting a vertex to itself
(referred to as {\textquoteleft}loops{\textquoteright}) are permitted
in the model. Networks are defined such that the edges represent a
relationship between the vertices and may be directed, weighted, or
contain parallel edges or loops depending on the application. 


Network theory is the sub-discipline of graph theory which deals with
networks which has become popular due to the vast potential for
applications of networks. The properties of large networks were
studied by constructing random networks by randomly linking a fixed
number of nodes (Erd\H{o}s \& R\'enyi 1959; Erd\H{o}s \& R\'enyi 1960).
\ Despite the random nature of these networks, properties such as their
connectivity were well characterised. The vertex degree of random
network follows a Poisson distribution, however this property does not
hold in nature, suggesting that natural networks are non-random or not
formed in this way (Barab\'asi \& Oltvai 2004). 


This work formed the foundation for studying complex networks which
model features of real world networks not found in Erd\H{o}s and
R\'enyi{\textquoteright}s random networks. The small world property,
made popular by findings in social networks (Milgram 1967; Travers \&
Milgram 1969), is the remarkably short path lengths between any nodes
in a small world network. A small world network is well-connected
with a characteristic path length proportional to the logarithm of the
number of nodes (Watts \& Strogatz 1998). \hyperlink{ENREF112}{Watts
and Strogatz (1998)} developed a model of random rewiring of a regular
network to construct random networks with the small world property and
a high clustering coefficient. While these properties are more
representative of networks occurring in nature, their model is limited
by the degree distribution which converges to a Poisson distribution as
it is rewired (Barrat \& Weigt 2000). 


The degree distribution of naturally occurring networks often follows a
power law distribution with the majority of nodes having far fewer
connections than average and a small subset of highly connected network
{\textquoteleft}hubs{\textquoteright}. \hyperlink{ENREF7}{Barab\'asi
and Albert (1999)} constructed a network model in an entirely different
way to randomly generate scale-free networks which have a power law
degree distribution. They constructed random networks by preferential
attachment, modelling growth of a network by sequentially adding nodes
with links to existing nodes. The scale-free nature of the random
networks was ensured by adding new nodes with an increasing probability
of attachment to an existing node if it has higher degree. These
networks successfully capture the scale-free nature of many real world
networks with short characteristic path length and low eccentricity
resulting in super small worlds. The \hyperlink{ENREF7}{Barab\'asi
and Albert (1999)} scale-free networks are limited by a low clustering
coefficient and lack of modular structure; however, they have enabled
the study of scale-free network topology and served as a basis for
modified scale-free models (Dorogovtsev \& Mendes 2003; Holme \& Kim
2002). 


\hyperlink{ENREF47}{Han}\hyperlink{ENREF47}{\textit{ et
al.}}\hyperlink{ENREF47}{ (2004)} observed dynamic modularity in
biological networks and suggested the network structure may underpin
genetic robustness and plasticity. They focus on network hubs which
are more likely to be essential genes and define the subgroups of hubs
based on correlation of gene expression with protein-protein
interaction partners: {\textquoteleft}party{\textquoteright} hubs
(which interact simultaneously with many partners) and
{\textquoteleft}date{\textquoteright} hubs (which interact with
different partners in different conditions). Party and date hubs
occurred most frequently within and between network modules
respectively. Party hubs were considered local regulators, whereas
date hubs were considered important to network connectivity as global
regulators. This distinction between classes of network hubs was
supported by differences in tissue specificity and clinical relevance
as a proposed predictor of clinical outcome in breast cancer with an
AUROC of 0.784 (Taylor\textit{ et al.} 2009). However, correlation
between expression and protein interactions were not robustly
reproduced. The importance of date hubs has been criticised for
assuming a bimodal distribution and basing the global importance of
data hubs on a small subset (Agarwal\textit{ et al.} 2010). As an
alternative interpretation,
\hyperlink{ENREF2}{Agarwal}\hyperlink{ENREF2}{\textit{ et
al.}}\hyperlink{ENREF2}{ (2010)} suggest the importance of interactions
rather than network hubs as interactions important to the network were
between functionally similar proteins. Network hubs can also be
classed as associative or dissociative depending on whether they tend
toward or away from connecting directly to other network hubs (van
Steen 2010). The associative and dissociative properties can also be
used to test whether nodes of a particular subgroup (e.g., gene
function) associate with each other. 

Applications of network theory are diverse, including uses in social
sciences, engineering, and computer science. Due to their complexity
and difficulty of gathering sufficient empirical data, biological
applications of network theory are relatively unexplored.
\ High-throughput technologies such as siRNA screens, two-hybrid
screens, microarrays and massively parallel sequencing have made
generating genome-scale molecular data feasible and enabled analysis of
biological networks at the molecular level. Many types of
inter-related molecular networks can be constructed and analysed,
depending on the biological application, the way interactions between
molecular components are defined and the data used to generate them as
shown in Table 1. 

\textbf{Table 1. }Types of biological network based on molecular data
\begin{flushleft}
\tablehead{}
\begin{supertabular}{m{4.0490003cm}|m{2.55cm}|m{4.301cm}|m{4.201cm}}
\multicolumn{1}{m{4.0490003cm}}{\cellcolor{white}\bfseries\color{black}
{\fontsize{10pt}{12.0pt}\selectfont \textcolor{black}{Network Type}}} &
\multicolumn{1}{m{2.55cm}}{\cellcolor{white}\bfseries\color{black}
{\fontsize{10pt}{12.0pt}\selectfont \textcolor{black}{Node}}} &
\multicolumn{1}{m{4.301cm}}{\cellcolor{white}\bfseries\color{black}
{\fontsize{10pt}{12.0pt}\selectfont \textcolor{black}{Interaction}}} &
\cellcolor{white}\bfseries\color{black}
{\fontsize{10pt}{12.0pt}\selectfont \textcolor{black}{Data
Generation}}\\\hline
\cellcolor[rgb]{0.87058824,0.91764706,0.9647059}\bfseries\color{black}
{\fontsize{10pt}{12.0pt}\selectfont \textcolor{black}{Co-expression}} &
\cellcolor[rgb]{0.87058824,0.91764706,0.9647059}\color{black}
{\fontsize{10pt}{12.0pt}\selectfont \textcolor{black}{Gene}} &
\cellcolor[rgb]{0.87058824,0.91764706,0.9647059}\color{black}
{\fontsize{10pt}{12.0pt}\selectfont \textcolor{black}{Correlation of
expression}} &
\cellcolor[rgb]{0.87058824,0.91764706,0.9647059}\color{black}
{\fontsize{10pt}{12.0pt}\selectfont \textcolor{black}{Array,
RNA-Seq}}\\\hline
\bfseries {\fontsize{10pt}{12.0pt}\selectfont Protein (Physical)} &
{\fontsize{10pt}{12.0pt}\selectfont Protein} &
{\fontsize{10pt}{12.0pt}\selectfont Protein-protein binding} &
{\fontsize{10pt}{12.0pt}\selectfont Two-Hybrid}\\\hline
\cellcolor[rgb]{0.87058824,0.91764706,0.9647059}\bfseries\color{black}
{\fontsize{10pt}{12.0pt}\selectfont \textcolor{black}{Signal
Transduction}} &
\cellcolor[rgb]{0.87058824,0.91764706,0.9647059}\color{black}
{\fontsize{10pt}{12.0pt}\selectfont \textcolor{black}{Protein}} &
\cellcolor[rgb]{0.87058824,0.91764706,0.9647059}\color{black}
{\fontsize{10pt}{12.0pt}\selectfont \textcolor{black}{Protein mediated
signalling}} &
\cellcolor[rgb]{0.87058824,0.91764706,0.9647059}\color{black}
{\fontsize{10pt}{12.0pt}\selectfont \textcolor{black}{Curate known
pathways}}\\\hline
\bfseries {\fontsize{10pt}{12.0pt}\selectfont Metabolic} &
{\fontsize{10pt}{12.0pt}\selectfont Metabolite or cofactor} &
{\fontsize{10pt}{12.0pt}\selectfont Involved in same reaction (links by
reactions/enzymes)} &
{\fontsize{10pt}{12.0pt}\selectfont Curate known pathways}\\\hline
\cellcolor[rgb]{0.87058824,0.91764706,0.9647059}\bfseries\color{black}
{\fontsize{10pt}{12.0pt}\selectfont \textcolor{black}{Chemical or
Drug}} &
\cellcolor[rgb]{0.87058824,0.91764706,0.9647059}\color{black}
{\fontsize{10pt}{12.0pt}\selectfont \textcolor{black}{Protein}} &
\cellcolor[rgb]{0.87058824,0.91764706,0.9647059}\color{black}
{\fontsize{10pt}{12.0pt}\selectfont \textcolor{black}{Targets of the
same drug}} &
\cellcolor[rgb]{0.87058824,0.91764706,0.9647059}\color{black}
{\fontsize{10pt}{12.0pt}\selectfont \textcolor{black}{Curate known drug
targets}}\\\hline
\bfseries {\fontsize{10pt}{12.0pt}\selectfont Regulation} &
{\fontsize{10pt}{12.0pt}\selectfont Gene} &
{\fontsize{10pt}{12.0pt}\selectfont Regulate each other by encoded
proteins} &
{\fontsize{10pt}{12.0pt}\selectfont Array, RNA-Seq, ChIP}\\\hline
\cellcolor[rgb]{0.87058824,0.91764706,0.9647059}\bfseries\color{black}
{\fontsize{10pt}{12.0pt}\selectfont \textcolor{black}{RNA or DNA
binding}} &
\cellcolor[rgb]{0.87058824,0.91764706,0.9647059}\color{black}
{\fontsize{10pt}{12.0pt}\selectfont \textcolor{black}{Gene or miRNA}} &
\cellcolor[rgb]{0.87058824,0.91764706,0.9647059}\color{black}
{\fontsize{10pt}{12.0pt}\selectfont \textcolor{black}{Binding of
encoded RNA or protein or DNA or mRNA}} &
\cellcolor[rgb]{0.87058824,0.91764706,0.9647059}\color{black}
{\fontsize{10pt}{12.0pt}\selectfont \textcolor{black}{ChIP,
RIP}}\\\hline
\bfseries {\fontsize{10pt}{12.0pt}\selectfont Functional} &
{\fontsize{10pt}{12.0pt}\selectfont Gene} &
{\fontsize{10pt}{12.0pt}\selectfont Shared gene function} &
{\fontsize{10pt}{12.0pt}\selectfont Curate known pathways}\\\hline
\cellcolor[rgb]{0.87058824,0.91764706,0.9647059}\bfseries\color{black}
{\fontsize{10pt}{12.0pt}\selectfont \textcolor{black}{Genetic
Interaction}} &
\cellcolor[rgb]{0.87058824,0.91764706,0.9647059}\color{black}
{\fontsize{10pt}{12.0pt}\selectfont \textcolor{black}{Gene}} &
\cellcolor[rgb]{0.87058824,0.91764706,0.9647059}\color{black}
{\fontsize{10pt}{12.0pt}\selectfont \textcolor{black}{Unexpected
phenotype with combined loss of function}} &
\cellcolor[rgb]{0.87058824,0.91764706,0.9647059}\color{black}
{\fontsize{10pt}{12.0pt}\selectfont \textcolor{black}{SGA, EMAP, DAmP,
siRNA}}\\\hline
\end{supertabular}
\end{flushleft}

\subsection[Biological Networks]{Biological Networks}

Genetic interaction networks will be the focus of this project because
they are relatively unexplored compared to other molecular networks,
have potential for applications in drug discovery (particularly cancer
treatment), and may lead to better understanding of the role of
genetics in cellular function and disease. Genetic interactions are
usually studied at a high-throughput scale in simple model organisms
such as bacteria, yeasts or the nematode worm; studies in humans,
mammals, and non-model organisms (where applications would have the
most societal impact) are limited by cost, time and labour constraints.
\ Computational approaches with effective predictive models are the
only feasible approach to study the connectivity of a biological
network in a complex metazoan cell at the genome-scale.


Genetic interactions are a core concept of molecular biology, discovered
in some of the earliest experiments investigating Mendelian genetics.
\ Epistasis was biologically defined as the effect of an allele at one
locus may mask the phenotype of another (Bateson \& Mendel 1909).
\ Statistically, epistasis is defined by significant disparity between
the observed and expected phenotype of a double mutant, compared to the
respective phenotypes of single mutants and the wild-type (Fisher
1919). Fisher{\textquoteright}s broader definition lends itself well
to quantitative traits and also encompasses Synthetic genetic
interactions (SGIs) which have become popular for studies in yeast
genetics and cancer drug design (Boone\textit{ et al.} 2007; Kaelin
2005). 


Synthetic genetic interactions are substantial deviations from the
expected null mutant phenotype (usually measured by organism or
cellular growth or viability) rather than the expected additive effects
of the single mutants. Unlike biological epistasis, this deviation
does not necessarily constitute a single mutant phenotype (as shown in
Figure 1). SGIs of interest tend to be more viable than either single
mutant or less viable than the expected double mutant. In negative
SGIs, mutations are {\textquoteleft}synergistic{\textquoteright},
resulting in more deviation from the wild-type than expected. For
yeast viability phenotypes, the terms {\textquoteleft}synthetic
sick{\textquoteright} (SSL) and {\textquoteleft}synthetic
lethal{\textquoteright} (SL) refer to negative SGIs giving growth
inhibition and inviability respectively. In cancer research synthetic
lethality is used to describe any negative synthetic genetic
interaction with specific killing or growth inhibition of a mutant
cell, even though these may be formally classed as SSL interactions.
\ In positive SGIs, mutations are
{\textquoteleft}alleviating{\textquoteright}, resulting in less
deviation from the wild-type than expected. For yeast viability
phenotypes, the terms {\textquoteleft}synthetic
suppression{\textquoteright} (SS) and {\textquoteleft}synthetic
rescue{\textquoteright} (SR) are used to refer to positive SGIs giving
partial or complete restoration of the wild-type growth rate from
single mutants with growth impairment and lethal phenotypes
respectively. Yeast studies have found that negative SGIs are
markedly more common than positive SGIs (Tong\textit{ et al.} 2004);
this has since been replicated in a number of model systems (Boucher \&
Jenna 2013). 


\includegraphics[width=12.912cm,height=10.922cm]{TomKellyLiteratureReview2015v7notrack-img1.png}
\textbf{Figure 1. }Impact of various negative (a) and positive (b, c)
SGIs on growth viability fitness in yeast. Adapted from
(Costanzo\textit{ et al.} 2011). 


\subsection[Experimental Studies of Synthetic Lethality]{Experimental
Studies of Synthetic Lethality}

A high-throughput method, the synthetic gene array (SGA), has been
developed specifically to test for SGIs in the budding yeast
\textit{Saccharomyces cerevisiae} (Tong\textit{ et al.} 2001).
\ Automated mating experiments are used to generate and measure the
viability of haploid double null mutants for a query gene and a library
of \~{}4700 deletion mutants. A proof of concept experiment with 8
query genes found SGIs within and between biological pathways
(Tong\textit{ et al.} 2001). This approach was later extended to 132
query genes to experimentally map \~{}4100 SGIs between \~{}1000 genes
with each query gene involved in an average of 34 SGIs (0.6\% of the
genome) with high variation: up to 146 partners (Tong\textit{ et al.}
2004). They noted that this was 4-fold higher than interactions
discovered with yeast-2-hybrid studies of protein-protein interactions.
\ The network is scale-free with power-law vertex degree distribution
and low average shortest path length (3.3). 


These observations are consistent with a larger network (\~{}170,000
SGIs out of 5.4 million gene pairs) constructed from SGA analysis of
1712 query genes in \textit{S. cerevisiae} which showed high functional
organisation, genetic redundancy, and pleiotropy (Baryshnikova\textit{
et al.} 2010b; Costanzo\textit{ et al.} 2010). Hub genes were
functionally important with many negative SGI hubs involved in cell
cycle regulation and many positive SGI hubs involved in translation.
\ Negative SGIs were far more common than positive SGIs. The
modularity of the \textit{S. cerevisiae} SGI network is pronounced
enough to predict function of novel genes (Costanzo\textit{ et al.}
2011). The supplementary data for this network has been updated to
support SGA data for 1897 query genes. 


\hyperlink{ENREF15}{Boone}\hyperlink{ENREF15}{\textit{ et
al.}}\hyperlink{ENREF15}{ (2007)} predicted \~{}200,000 SGIs in the
yeast genome of \~{}6200 genes; the number of genes individually
essential for life is vastly outnumbered by the number of genes
essential in combination due to synthetic lethality. This is
consistent with the chemical genomic screens finding 97\% of yeast
genes are conditionally essential for optimal growth in some
environmental stress condition (Hillenmeyer 2008). Together this
gives an evolutionary rationale for the abundance of SGIs and
surprisingly low number of essential genes in a genome due to the
genetic redundancy and network robustness of a cell. 


The SGA methodology has been adapted to screen deletion or gene
knockdown libraries for SGIs using SpSGA with mating in fission yeast,
\textit{Schizosaccharomyces pombe }\textit{(Dixon et al. 2008)}; using
eSGA or GIANT-coli with conjugation in bacteria, \textit{Escherichia
coli} (Babu\textit{ et al.} 2014; Butland\textit{ et al.} 2008;
Typas\textit{ et al.} 2008); or using RNA interference (RNAi) in the
nematode worm, \textit{Caenorhabdiits elegans} (Lehner\textit{ et al.}
2006; Tischler\textit{ et al.} 2008). The SGA methodology has also
been modified for the analysis of essential genes in \textit{S.
cerevisiae} by using hypomorphic promoter replacement alleles rather
than (lethal) null mutants (Davierwala\textit{ et al.} 2005). The SGA
method has been combined with microarray technologies for the
high-throughput diploid based synthetic lethality analysis of
microarrays (dSLAM) and epistatic miniarray profiles (EMAP) approaches
which enable pooled mating experiments and increased sensitivity in
gene subsets respectively in \textit{S. cerevisiae} (Collins\textit{ et
al.} 2006; Ooi\textit{ et al.} 2003; Schuldiner\textit{ et al.} 2005;
Schuldiner\textit{ et al.} 2006). The EMAP approach has also been
successfully adapted for comparative analysis of the distantly related
\textit{S. pombe} (Roguev\textit{ et al.} 2008; Roguev\textit{ et al.}
2007). \hyperlink{ENREF28}{Davierwala}\hyperlink{ENREF28}{\textit{ et
al.}}\hyperlink{ENREF28}{ (2005)} used conditional (tet regulated) or
temperature sensitive alleles to study essential genes with SGA,
whereas \hyperlink{ENREF87}{Schuldiner}\hyperlink{ENREF87}{\textit{ et
al.}}\hyperlink{ENREF87}{ (2005)} developed the decreased abundance of
mRNA perturbation (DAmP) method for high-throughput generation of
partial disruption in essential genes. This enables SGI screening in
various species at the genomic-scale, including essential genes, by
gene knockdown or non-lethal mutation. 


\hyperlink{ENREF28}{Davierwala}\hyperlink{ENREF28}{\textit{ et
al.}}\hyperlink{ENREF28}{ (2005)} found that the SGI network (30 query
genes against 575 essential genes) for essential genes was 5x denser
than the non-essential network in \textit{S. cerevisiae
}\textit{(Costanzo et al. 2010; Tong et al. 2004)}. Therefore
essential genes are SGI network hubs and are involved in the majority
of SGIs, despite comprising only \~{}18\% of the \textit{S. cerevisiae}
genome. Essential pathways are therefore highly buffered consistent
with strong selection for survival with partial loss of essential gene
function. The essential gene network also presented with scale-free
topology and rarely contained interactions found in a protein-protein
interaction network. While interacting essential genes were likely to
have related functions, this was less evident than in non-essential
networks. Around the same proportion of essential genes are found in
other Eukaryotes indicating that mammalian SGI networks may have
similar levels of SGIs. 


While distantly related, by millions of years of evolution, the budding
yeast \textit{S. cerevisiae} and the fission yeast \textit{S. pombe}
SGI networks are comparable (Dixon\textit{ et al.} 2009b). These
yeasts are both single-celled, were analysed by similar SGA methodology
with colony growth phenotypes, and used null or hypomorphic mutation to
inactivate genes. The RNAi based SGI screens in \textit{C. elegans
}or \textit{Drosophila melanogaster} are less comparable to the yeast
models due to differences from phenotyping diploid multicellular
organisms and the potential residual wild-type gene activity in an RNAi
experiment. Relatively poor conservation of specific SGIs in
\textit{S. cerevisiae} were replicated in \textit{S. pombe} (222
queries against a library of 2663 genes); however, around \~{}29\% of
the interactions tested in both species form a conserved SGI network
(Dixon\textit{ et al.} 2008). The rest of the interactions reveal
species specific differences expected in distantly related species,
however many of the species specific interactions were still conserved
between biological pathways, protein complexes, or protein-protein
interaction modules. Negative SGIs were likely to be conserved
between biological pathways, whereas positive SGIs were more likely to
be conserved within a pathway or protein complex (Roguev\textit{ et
al.} 2008). Conservation of pathway redundancy was also found between
\textit{S. cerevisiae} and prokaryotes with \textit{E. coli}
experiments (Butland\textit{ et al.} 2008). 


\subsection[RNAi Studies in Mammals]{RNAi Studies in Mammals}

Despite reasons difficulties comparing yeast SGA data and metazoan RNAi
screens for SGIs, \~{}5\% of interactions in \textit{C. elegans} were
conserved in \textit{S. cerevisiae,} and the nematode SGI network
showed similar scale-free topology and modularity (Bussey\textit{ et
al.} 2006). The nematode SGI screen also identified network hubs and
implicated their importance via interaction with orthologues of known
human disease genes (Lehner\textit{ et al.} 2006). However, genetic
redundancy at the gene or pathway level fails to account for the lack
of direct conservation of SGIs between yeasts and nematode worms,
consistent with an induced essentiality model of SGIs which allows for
conservation of gene function with network restructuring over
evolutionary time (Tischler\textit{ et al.} 2008). While Eukaryotic
models are more closely related to human cells, cancer cells can
present growth and viability phenotypes more comparable to yeast
models. Therefore findings from both SGA and RNAi models are relevant
to understanding cellular network structure and in healthy and
cancerous human cells. RNAi has also been applied to human and mouse
cancer cells in cell culture and genetic screening experiments. 


\subsection[Examples of Clinical Impact]{Examples of Clinical Impact}

Many genes are lost in cancer and yet few interventions target these
tumour suppressor mutations compared to targeted therapies for gain of
function mutation in oncogenes. Synthetic lethality, also known as
{\textquoteleft}non-oncogene addiction{\textquoteright} or
{\textquoteleft}induced essentiality{\textquoteright} in the context of
cancer therapy, is a powerful design strategy for therapies selective
against loss of gene function with potential for application against a
range of genes and diseases. There are several examples of clinically
relevant applications of genetic interactions including specific
targeting of mutations in BRCA tumour suppressor genes with PARP
inhibitors by inducing synthetic lethality in breast cancer
(Farmer\textit{ et al.} 2005) and the synergy between inhibitors of the
oncogenes BRAF\textsuperscript{V600E} and EGFR in colorectal cancers
(Prahallad\textit{ et al.} 2012).


BRCA and PARP genes demonstrate the application of the synthetic lethal
approach to cancer therapy (Ashworth 2008; Kaelin 2005). BRCA1 and
BRCA1 are homologous DNA repair genes which have been popularised for
their role as tumour suppressors; mutation carriers have high risk of
breast and ovarian cancers. The BRCA genes, which usually repair DNA
or destroy the cell if it cannot be repaired, have inactivating somatic
mutations in some familial and sporadic cancers. Poly-ADP-ribose
polymerase (PARP) genes are tumour suppressor genes involved in base
excision DNA repair. Loss of PARP activity results in single-stranded
DNA breaks. However, PARP1\textsuperscript{{}-/-}\textsubscript{
}knockout mice are viable and healthy indicating low toxicity from PARP
inhibition (Bryant\textit{ et al.} 2005). 


\hyperlink{ENREF19}{Bryant}\hyperlink{ENREF19}{\textit{ et
al.}}\hyperlink{ENREF19}{ (2005)} showed that
BRCA2\textsuperscript{{}-/-} cells were sensitive to PARP inhibition by
siRNA of PARP1 or drug inhibition (which targets PARP1 and PARP2) using
Chinese hamster ovary cells, MCF7 and MDA-MB-231 breast cell lines.
\ This effect was sufficient to kill mouse tumour xenografts and showed
high specificity to BRCA2 deficient cells in culture and xenografts.
\ \hyperlink{ENREF39}{Farmer}\hyperlink{ENREF39}{\textit{ et
al.}}\hyperlink{ENREF39}{ (2005)} replicated these results in embryonic
stem cells and showed that BRCA1\textsuperscript{{}-/-} cells were also
sensitive to PARP inhibition relative to the wild-type with siRNA and
drug experiments in cell culture and drug activity against BRCA
deficient embryonic stem cell mouse xenografts. They found evidence
that PARP inhibition causes DNA lesions, usually repaired in wild-type
cells, which lead to chromosomal instability, cell cycle arrest, and
induction of apoptosis in BRCA deficient cells. Therefore, the
pathways cooperate to repair DNA giving a plausible mechanism for
combined loss as an effective anti-cancer treatment. 


Thus PARP inhibitors have potential to clinical uses against BRCA
mutations in hereditary and sporadic cancers (Ashworth 2008; Kaelin
2005). PARP inhibition has been found to be effective in cancer
patients carrying BRCA mutations and some non-BRCA mutant ovarian
cancers, suggesting synthetic lethality between PARP and other DNA
repair pathways (Str\"om \& Helleday 2012). This supports the
potential for PARP inhibition as a chemo-preventative alternative to
prophylactic surgery for high risk individuals with BRCA mutations
(Str\"om \& Helleday 2012). Hormone-based therapy has also been
suggested as a chemo-preventative in such high risk individuals and
aromatase inhibitors have completed phase I clinical trials for this
purpose (Bozovic-Spasojevic\textit{ et al.} 2012).
\ \hyperlink{ENREF91}{Str\"om and Helleday (2012)} also postulate
increased efficacy of PARP inhibitors in the hypoxic DNA-damaging
tumour micro-environment. 


A PARP inhibitor, olaparib, showed fewer adverse effects than cytotoxic
chemotherapy and anti-tumour activity in phase I trials against BRCA
deficient familial breast, ovarian, and prostate cancers (Fong\textit{
et al.} 2009) and sporadic ovarian cancer (Fong\textit{ et al.} 2010).
\ AstraZeneca has reported phase II trials showing the treatment is
effective in BRCA deficient breast (Tutt\textit{ et al.} 2010) and
ovarian cancers (Audeh\textit{ et al.} 2010) with a favourable
therapeutic window and similar toxicity between carriers of BRCA
mutations and sporadic cases. AstraZeneca announced that olaparib has
begun phase III trials for breast and ovarian cancers in 2013. Mixed
results in phase II trials in ovarian cancer are behind the delays
addressed by retrospective analysis of the cohort subgroup with
confirmed mutation of BRCA genes in the tumour; unsurprisingly, these
patients, benefit most from the PARP inhibitor treatment and have
increased platinum sensitivity in combination treatment. This
demonstrates the clinical impact of a well characterised system of
synthetic lethality with known cancer risk genes. Synthetic lethality
has the benefit of being effective against inactivation of tumour
suppressor genes by any means, broader than targeting a particular
oncogenic mutation (Kaelin 2005). The targeted therapy is effective
in both sporadic and hereditary BRCA deficient tumours acting against
an oncogenic molecular aberration across several tissues. 


Oncogene targeted therapies have also been developed but have problems
with resistance, recurrence, tissue specificity, and design of a drug
which selectively inhibits the oncogenic variant rather than the
normally functional proto-oncogene. BRAF is a serine/threonine kinase
gene in the MAP kinase pathway with several oncogenic mutations
including the V600E mutation implicated as a driver in some colorectal,
hairy cell leukaemia, lung, melanoma, non-Hodgkin{\textquoteright}s
lymphoma, and thyroid cancers (Entrez Gene). EGFR is the epidermal
growth factor receptor gene which detects extracellular signal ligands
including EGF and TGF$\alpha $. Mutations resulting in overexpression
of EGFR are oncogenic in brain, colorectal, and lung cancers (Entrez
Gene). Both oncogenes have been explored in some detail as separate
drug targets with successful development of drugs acting against
BRAF\textsuperscript{V600E} in melanoma and drugs against EGFR in lungs
cancers, however they have limited efficacy separately in colorectal
cancer. 


BRAF V600 mutations are a feasible target for anticancer therapy,
occurring in 8\% of all solid tumours including substantial numbers of
melanomas, thyroid, and 10-15\% of colorectal cancers (Davies\textit{
et al.} 2002). BRAF mutant tumours use the MAPK/ERK (extracellular
signal-regulated kinase signalling cascade) pathway to induce tumour
growth (Dienstmann \& Tabernero 2011). Specific inhibitors, such as
the drug vemurafenib, have shown efficacy against melanoma and slowed
cell growth in culture. Specific BRAF inhibition is more effective
than non-selective inhibition of the RAF kinase family of the MEK
downstream targets (Dienstmann \& Tabernero 2011). Vermurafenib is
effective for treatment of BRAF\textsuperscript{V600E} mutant
melanomas, shown by success in phase I-III clinical trials and FDA
approval (Ravnan \& Matalka 2012). BRAF inhibitors are effective with
regard to efficacy and toxicity; however, acquired drug resistance is a
serious problem for this treatment in melanoma, the mechanisms of which
need to be explored to ensure optimal clinical outcomes (Dienstmann \&
Tabernero 2011). \hyperlink{ENREF82}{Ravnan and Matalka (2012)}
suggest combination therapy as a solution to vemurafenib resistance in
melanoma. 


\hyperlink{ENREF115}{Yang}\hyperlink{ENREF115}{\textit{ et
al.}}\hyperlink{ENREF115}{ (2012)} tested vemurafenib as a single-agent
BRAF\textsuperscript{V600E} inhibitor in preclinical models of
colorectal cancer cell lines and mouse xenografts. Vemurafenib
showeddose-dependent inhibition of the MEK and MAPK/ERK signalling
pathways which arrested cell proliferation; however, clinical activity
was limited in single-agent experiments. Enhanced anti-tumour
activity and xenograft survival were found when combined with
clinically approved drugs, including EGFR inhibitors, suggesting
combination with standard or other targeted treatments is most
effective approach for clinical treatment of advanced
BRAF\textsuperscript{V600E} mutant colorectal cancers. This is
consistent with known association of BRAF mutations with resistance to
EGFR inhibitors in colorectal cancer (Di Nicolantonio\textit{ et al.}
2008; Yuan\textit{ et al.} 2013). Understanding the mechanism of EGFR
inhibitor resistance in colorectal cancer is needed for improved
application of the drug as a personalised medicine to refine patient
selection for single-agent treatment and to develop novel drug
combinations; notably, markers of EGFR inhibitor resistance notably
include KRAS and BRAF mutations (Loupakis\textit{ et al.} 2009;
Shaib\textit{ et al.} 2013; Siena\textit{ et al.} 2009). 


Despite successful application of vemurafenib against
BRAF\textsuperscript{V600E} in melanomas, colorectal cancers with
BRAF\textsuperscript{V600E} muations have poor prognosis and lack drug
response. \hyperlink{ENREF80}{Prahallad}\hyperlink{ENREF80}{\textit{
et al.}}\hyperlink{ENREF80}{ (2012)} approached this with an RNAi
screen of the kinome (a library of 518 genes) in the WiDr colon cell
line. Blocking EGFR with shRNA or EGFR inhibiting drugs has shown
strong synergy with vemurafenib in cell line experiments; drug synergy
was replicated in xenografts. This synergy arises mechanistically
from feedback activation of EGFR in response to BRAF inhibition,
consistent with higher EGFR expression in colorectal and thyroid cancer
cell lines compared to melanoma cell lines and vemurafenib resistance
induced from ectopic EGFR expression in melanomas.
\ \hyperlink{ENREF24}{Corcoran}\hyperlink{ENREF24}{\textit{ et
al.}}\hyperlink{ENREF24}{ (2012)} supported this mechanism with
transient inhibition of BRAF\textsuperscript{V600E} by vemurafenib
which induced rapid reactivation of MAPK/ERK signalling via EGFR in
colorectal cell lines. This did not occur in melanoma cell lines
suggesting colorectal tissue-specific regulation of EGFR and supporting
the use of combined BRAF\textsuperscript{V600E} and EGFR inhibitors
which effectively block MAPK/ERK signalling in cell lines and
xenografts. Across these studies, synergy between vemurafenib and
EGFR inhibitors was replicated for multiple BRAF mutant colorectal cell
lines with both antibody (cetuximab) and small-molecule drugs
(gefitinib and erlotinib) giving a mechanistically derived means to
overcome vemurafenib resistance. 


\hyperlink{ENREF92}{Sun}\hyperlink{ENREF92}{\textit{ et
al.}}\hyperlink{ENREF92}{ (2014)} applied these findings to understand
vemurafenib resistance in BRAF\textsuperscript{V600E} melanomas. EGFR
expression was found to be an adaptive response to BRAF or MEK
inhibition in BRAF mutant biopsies and cell lines. An RNAi screen of
chromatin regulators (a library of 661 genes) was performed on the A375
melanoma cell line with low initial EGFR expression and selection for
vemurafenib resistance. This showed that both gene silencing and drug
selection were required for high EGFR expression to emerge. SOX10 was
identified as the only gene for which multiple shRNAs could induce EGFR
expression and was sufficient for increased EGFR expression and
TGF$\beta $ signalling by transcriptome analysis. EGFR overexpression
or TGF$\beta $ treatment also cause a slow growth phenotype, showing
oncogene-induced senescence and so are only beneficial to melanomas
under drug selection. Therefore, Heterogeneous SOX10 expression
levels in melanoma cell lines account for variation in selection and
vemurafenib resistance. If BRAF inhibition is removed, SOX10
expression and drug sensitivity are restored giving support to a drug
holiday and retreatment approach to counter drug resistance. 


Combination treatment of vemurafenib (as a BRAF inhibitor) and various
FDA approved EGFR inhibitors entered phase I clinical trials for
treatment of BRAF\textsuperscript{V600E} mutant tumours in 2014.
\ Vemurafenib and panitumamab are being trialled in colorectal cancers
by the Memorial Sloan Kettering Cancer Center, New York. Combination
treatment of vemurafenib, cetuximab (an EGFR inhibitor) and irinotecan
(a topoisomerase inhibitor) are being trialled in advanced solid
cancers by the MD Anderson Cancer Centre, Houston. Combination
vemurafenib and erlotinib are being trialled in BRAF mutant cancers
including colorectal and lung by the Peter MacCallum Cancer Centre,
Melbourne. 


\subsection[Network Drug Design]{Network Drug Design}

While targeted therapeutics have shown some levels of success for drug
discovery, particularly in anticancer applications, exploiting the
connectivity of molecular networks or designing combination therapies
are alternatives that may yield more effective results using a network
pharmacology framework (Hopkins 2008). Rational design of drugs
selective to a single target has largely failed to deliver clinical
efficacy. In fact, many existing effective drugs modulate multiple
proteins and were selected for biological effects or clinical outcome
rather than molecular targets. Network biology and polypharmacology
(specific binding to multiple targets) are a possible way to develop
drugs with a desired target profile designed for the target topology.
\ Multi-target treatments aim to achieve a clinical outcome through
modulation of molecular networks since the genetic robustness of a cell
often compensates for loss of a single molecular target. 


[Warning: Draw object ignored]While multi-target drugs may be more
difficult to design, they are faster to test clinically than drug
combinations which are usually required to be tested separately first
(Hopkins 2008). Networks have already been exploited to design
synthetic lethal treatments for cancer, drug combinations and
multi-target drugs to combat resistance to chemotherapy and
antibiotics. Nevertheless, further optimisation of timing and dosing
of drug combinations may increase efficacy and needs to be explored for
combination effects with low efficacy as separate treatments. Low
doses and drug holidays are other counter intuitive approaches which
may increase clinical efficacy, reduce adverse effects, and reduce drug
resistance (Sun\textit{ et al.} 2014; Tsai\textit{ et al.} 2012). 


Thus a network understanding of a cell has the potential to impact upon
drug design and clinical practice, particularly in treatment of cancer
and infectious disease. Characterisation of the target system and
impact of existing treatments, such as PARP inhibitors or
BRAF\textsuperscript{V600E} and EGFR inhibitor combination therapy,
could enable the understanding of the mechanisms of such interventions
which exploit genetic interactions, protein interactions, cell
signalling or gene regulation pathways. This could lead to
development of more effective treatment interventions for these systems
and prediction of similar molecular systems for development of novel
drug targets and combinations. 


\subsection[High{}-throughput Screening for Synthetic
Lethality]{High-throughput Screening for Synthetic Lethality}

The function of signalling pathways and combinations of interacting
genes are important in cancer research but classical genetics
approaches have been limited to non-redundant pathways (Fraser 2004).
\ The emerging RNAi technologies have vastly expanded the potential for
studying genetic redundancy in mammalian experimental models including
testing experimentally for synthetic lethality (Fraser 2004).
\ Identifying synthetic lethality is crucial to study gene function,
drug mechanisms, and design novel therapies (Lum\textit{ et al.} 2004).
\ Candidate selection of synthetic lethal gene pairs relevant to cancer
has shown some success but is limited because interactions are
difficult to predict; they can occur between seemingly unrelated
pathways in model organisms (Costanzo\textit{ et al.} 2011). While
biologically informed hypotheses have had some success in synthetic
lethal discovery (Bitler\textit{ et al.} 2015; Bryant\textit{ et al.}
2005; Farmer\textit{ et al.} 2005), interactions occurring indirectly
between distinct pathways would be missed (Boone\textit{ et al.} 2007;
Costanzo\textit{ et al.} 2011). Scanning the entire genome for
interactions against a clinically relevant gene is an emerging strategy
being explored with high-throughput screens (Fece de la Cruz\textit{ et
al.} 2015) and computational approaches (Boucher \& Jenna 2013; van
Steen 2011). 


Experimental screening for synthetic lethality is an appealing strategy
for wider discovery of functional interactions \textit{in vivo} despite
many potential sources of error which must be considered. The
synthetic lethal concept has both genetic and pharmacological screening
applications to cancer research. Genetic screens, with RNAi to
discover the specific genes involved, inform development of targeted
therapies with a known mode of action, anticipated mechanisms of
resistance, and biomarkers for treatment response. RNAi is a
transient knockdown of gene expression more similar to the effect of
drugs than complete gene loss and makes comparison to screens in model
organisms difficult (Bussey\textit{ et al.} 2006). The RNAi gene
knockdown process has inherent toxicity to some cells, potential
off-target effects, and issues with a high false positive rate.
\ Therefore, it is important to validate any candidates in a secondary
screen and replicate knockdown experiments with a number of independent
shRNAs. Alternative gene knockout procedures have also been proposed
for synthetic lethal screening including a genome-wide application of
the CRIPR/Cas9/sgRNA genome editing technology (Sander \& Joung 2014),
episomal gene transfer (Vargas\textit{ et al.} 2004), or RNAi with
lentiviral transfection for delivery of shRNA (Telford\textit{ et al.}
2015). Genetic screens have potential for quantitative gene
disruption experiments to selectively target overexpressed genes in
cancer via synthetic dosage lethality. While powerful for
understanding fundamental cellular function, analysis of isogenic cell
lines is inherently limited by assuming only a single mutation differs
between them despite susceptibility to {\textquoteleft}genetic
drift{\textquoteright} and cannot account for diverse genetic
backgrounds or tumour heterogeneity (Fece de la Cruz\textit{ et al.}
2015). Genetic screens thus identify targets to develop or repurpose
targeted therapies for disease but alone will not directly identify a
lead compound to develop for the market or clinical translation. 


Chemical screens are immediately applicable to the clinic by directly
screening for \ selective lead compounds with suitable pharmacological
properties. However chemical screens lack a known mode of action, may
affect many targets, and screen a narrow range of genes with existing
drugs. With either approach there are many challenges translating
candidates into the clinic such as finding targets relevant to a range
of patients, validation of targets, accounting for a range of genetic
(and epigenetic) contexts or tumour micro-environment, identifying
effective synergistic combinations, enhancers of existing radiation or
cytotoxic treatments, avoiding inherent or acquired drug resistance,
and developing biomarkers for patients which will respond to synthetic
lethal treatment, including integrating these into clinical trials and
clinical practice. Identifying specific target genes is an effective
way to anticipate such challenges, which can be approached with genetic
screens, so we will focus on these and computational alternatives.
\ Screening methods have proven a fruitful area of research, despite
being costly, laborious, and having many different sources of error.
\ These limitations suggest a need for complementary computational
approaches to synthetic lethal discovery. 


\subsection[Examples of High{}-throughput Synthetic Lethal
Screens]{Examples of High-throughput Synthetic Lethal Screens}

Overexpression of genes is another suitable application for synthetic
lethality since overexpressed genes cannot be distinguished from the
wild-type by direct sequence specific targeted therapy.
\ Overexpression of oncogenes, such as EGFR, MYC, and PIM1, has been
found to drive many cancers. PIM1 is a candidate for synthetic lethal
drug design in lymphomas and prostate cancers, where it interacts with
MYC to drive cancer growth. \hyperlink{ENREF103}{van der
Meer}\hyperlink{ENREF103}{\textit{ et al.}}\hyperlink{ENREF103}{
(2014)} performed an RNAi screen to for synthetic lethality between
PIM1 overexpression and gene knockdown in RWPE prostate cancer cell
lines. They recognise RNAi screens are valuable for finding
therapeutic targets and biomarkers for therapeutic response. PLK1
gene knockdown and drug inhibition was an effective as a specific
inhibitor of PIM1 overexpressing prostate cells in cell culture and
mouse tumour xenografts. PLK1 inhibition reduced MYC expression in
pre-clinical models, consistent with expression in human tumours which
PIM1 and PLK1 are co-expressed and correlated with tumour grade.
\ Therefore PLK1 is justified as a candidate for drug target against
prostate cancer progression. 


Hereditary leiomyomatosis and renal cell carcinoma (HLRCC) is a cancer
syndrome of predisposition to benign tumours in the uterus and risk of
malignant cancer of the kidney attributed to inherited mutations in
fumarate hydratase (FH).
\ \hyperlink{ENREF14}{Boettcher}\hyperlink{ENREF14}{\textit{ et
al.}}\hyperlink{ENREF14}{ (2014)} performed an RNAi screen on HEK293T
renal cells for synthetic lethality with FH. They found enrichment of
haem metabolism (consistent with the literature) and adenylate cyclase
pathways (consistent with cAMP dysregulation in FH mutant cells).
\ Synthetic lethality between FH mutation and adenylate cyclases was
validated with gene knockdown, drug experiments, and replicated across
both HEK293T renal cells and VOK262 cells derived from a HLRCC patient,
suggesting new potential treatments against the disease. Therefore,
synthetic lethality is applicable to metabolic dysregulation in cancer,
consistent with the Warburg hypothesis (Warburg 1956), and successfully
identifies specific anti-cancer drugs, even when the mechanism is
unclear. \


Similarly, hereditary diffuse gastric cancer (HDGC) is a cancer syndrome
of predisposition to early-onset malignant stomach and breast cancers
attributed to mutations in E-Cadherin (CDH1).
\ \hyperlink{ENREF94}{Telford}\hyperlink{ENREF94}{\textit{ et
al.}}\hyperlink{ENREF94}{ (2015)} performed an RNAi screen on MCF10A
breast cells for synthetic lethality with CDH1. They found enrichment
of G-protein coupled receptors (GPCRs) and cytoskeletal gene functions.
\ The results were consistent with a concurrent drug compound screen
with a number of candidates validated by lentiviral shRNA gene
knockdown and drug testing including inhibitors of Janus kinase,
histone deacetylases, phosphoinositide 3-kinase, aurora kinase, and
tyrosine kinases. Therefore the synthetic lethal strategy has
potential for clinical impact against HDGC, with particular interest in
interventions with low adverse effects for chemo-prevention, including
repurposing existing approved drugs for activity against CDH1 deficient
cancers. 


RNAi screening for synthetic lethality is also useful for functional
genetics to understand drug sensitivity.
\ \hyperlink{ENREF1}{Aarts}\hyperlink{ENREF1}{\textit{ et
al.}}\hyperlink{ENREF1}{ (2015)} screened WiDr colorectal cells for
synthetic lethality between WEE1 inhibitor treatment and an RNAi
library of 1206 genes with functions known to be amenable to drug
treatment or important in cancer such as kinases, phosphatases, tumour
suppressors, and DNA repair (a pathway WEE1 regulates). Screening
identified a number of synthetic lethal candidates including genes
involved in cell cycle regulation, DNA replication, repair, homologous
recombination, and Fanconi anaemia. Synthetic lethality with
cell-cycle and DNA repair genes was consistent with the literature and
validation in a panel of breast and colorectal cell lines supported
checkpoint kinases, Fanconi anaemia, and homologous recombination as
synthetic lethal partners of WEE1. These results show that synthetic
lethality can be used to improve drug sensitivity as a combination
treatment, especially to exploit genomic instability and DNA repair,
which are known to be clinically applicable from previous results with
BRCA genes and PARP inhibitors (Lord\textit{ et al.} 2014).
\ Therefore, WEE1 inhibitors are an example of treatment which could be
repurposed with the synthetic lethal strategy and similar findings
would be valuable to clinicians as a source of biomarkers and novel
treatments. While using a panel of cell lines to replicate findings
across genetic background is a promising approach to ensure wide
clinical application of validated synthetic lethal partners, a
computational approach may be more effective as it could account for
wider patient variation than scaling up intensive experiments on a wide
array of cell lines and could screen beyond limited candidates from an
RNAi library. 


Chemical genetic screens are also a viable strategy to identify
therapeutically relevant synthetic lethal interactions.
\ \hyperlink{ENREF13}{Bitler}\hyperlink{ENREF13}{\textit{ et
al.}}\hyperlink{ENREF13}{ (2015)} investigated ARID1A mutations,
aberrations in chromatin remodelling known to be common in ovarian
cancers, for drug response. Ovarian RMG1 cells were screened for drug
response specific to ARID1A knockdown cells. They used ARID1A gene
knockdown for consistent genetic background, with control experiments
and 3D cell culture to ensure relevance to drug activity in the tumour
micro-environment. Screening a panel of commercially available drugs
targeting epigenetic regulators found EZH2 methyltransferase inhibitors
effective and specific against ARID1A mutation with validation in a
panel of ovarian cell lines. Synthetic lethality between ARID1A and
EZH2 was supported by decreases in H3K27Me3 epigenetic marks and
markers of apoptosis in response to EZH2 inhibitors. This was
mechanistically supported with differential expression of PIK3IP1 and
association of both synthetic lethal genes with the PIK3IP1 promoter
identifying the P1I3K-AKT signalling pathway as disrupted when both
genes are inhibited. This successfully demonstrates the importance of
synthetic lethality in epigenetic regulators, identifies a
therapeutically relevant synthetic lethal interaction, and shows that
chemical genetic screens could model drug response and combination
therapy in cancer cells. However this approach is limited to finding
synthetic lethal interactions between genes with known similar
function, which may not be the most suitable for treatment. Further
limiting experiments to genes with existing targeted drugs reduces the
number of synthetic lethal interactions detected, assumes on their drug
specificity to a particular target, and many of these drugs are not
clinically available yet anyway as they are still in clinical trials
for other diseases or are not supported by healthcare systems in many
countries. 


\hyperlink{ENREF53}{Jerby-Arnon}\hyperlink{ENREF53}{\textit{ et
al.}}\hyperlink{ENREF53}{ (2014)} combined a computational approach to
triage candidates with a conventional RNAi screen to validate synthetic
lethal partners. They screened a selection of computationally
predicted candidates and randomly selected genes with RNAi against VHL
loss of function mutation in RCC4 renal cell lines. The computational
method had a high AUROC of 0.779 and predictions were enriched 4x for
validated RNAi hits over randomly selected genes. This approach
detected known synthetic lethal pairs such as BRCA genes with PARP1 and
MSH2 with DHFR. The synthetic lethal candidates identified with both
RNAi screening and computational prediction formed an extensive network
of 2077 genes with 2816 synthetic lethal interactions and similar
network of 3158 genes with 3635 synthetic dosage lethal interactions
(for synthetic lethality with over-expression). Each network was
scale-free as expected of a biological network and was enriched for
known cancer genes, essential genes in mice, and could be harnessed for
predicting prognosis and drug response. While demonstrating the
feasibility of combining experimental and computational approaches to
synthetic lethality in cancer, there remain challenges in predicting
synthetic lethal genes, novel drug targets, and translation into the
clinic. 


The examples above show that high-throughput screens are an effective
approach to discover synthetic lethality in cancer with a wide range of
applications. Screens are more comprehensive than hypothesis-driven
candidate gene approaches and successfully find known and novel
synthetic lethal interactions with potential for rapid clinical
application. They have the power to test mode of action of drugs,
find unexpected synthetic lethal interactions between pathways, or
identify effective treatment strategies without needing a clear
mechanism. However, synthetic lethal screens are costly,
labour-intensive, error-prone, and biased towards genes with effective
RNAi knockdown libraries. Limited genetic background, lethality to
wild-type cell during gene knockdown, off-target effects, and
difficultly replicating synthetic lethality across different cell
lines, tissues, laboratories, or conditions stems from a high false
positive rate and a lack of standardised thresholds to identify
synthetic lethality in a high-throughput screen. Therefore there is a
need for replication, validation, and alternative approaches to
identify synthetic lethal candidates. Varied conditions between
experimental screens and differences between RNAi or drug screens makes
meta-analysis difficult. Thus genome-scale synthetic lethal experiments
are not feasible, even in model organisms, so a computational approach
would be more suitable for this task. 


\subsection[Computational Prediction of Synthetic
Lethality]{Computational Prediction of Synthetic Lethality}

Prediction of gene interaction networks is a feasible alternative to
high-throughput screening with biological importance and clinical
relevance. There are many existing methods to predict gene networks,
as reviewed by \hyperlink{ENREF104}{van Steen (2011)} and
\ \hyperlink{ENREF16}{Boucher and Jenna (2013)}, summarised in Table 2
below. However, many of these methods have limitations including the
requirement for existing SGI data, several data inputs, and reliability
of gene function annotation. Many of the existing methods also assume
conservation of individual interactions between species, which has been
found not to hold in yeast studies (Dixon\textit{ et al.} 2008).
\ Tissue specificity is important in gene regulation and gene
expression, which are used as predictors of genetic interaction.
\ However, tissue specificity of genetic interactions cannot be
explored in yeast studies and has not been considered in any of the
following studies of multicellular model organisms, human networks, or
cancers. Similarly, investigation into tissue specificity of
protein-protein interactions (PPIs) , an important predictor of genetic
interactions, is difficult given the high-throughput two-hybrid screens
occur out of cellular context for multicellular organisms. 


\textbf{Table 2. }Existing prediction methods for Genetic Interaction
Networks in Model Organisms
\begin{flushleft}
\tablehead{}
\begin{supertabular}{m{4.421cm}|m{2.342cm}|m{2.388cm}|m{4.552cm}|m{2.299cm}}
\multicolumn{1}{m{4.421cm}}{\cellcolor{white}\bfseries\color{black}
Method} &
\multicolumn{1}{m{2.342cm}}{\cellcolor{white}\color{black}
\textcolor{black}{Input Data}} &
\multicolumn{1}{m{2.388cm}}{\cellcolor{white}\color{black}
\textcolor{black}{Species}} &
\multicolumn{1}{m{4.552cm}}{\cellcolor{white}\bfseries\color{black}
Source} &
\cellcolor{white}\bfseries\color{black} Tool Offered\\\hline
\cellcolor[rgb]{0.8509804,0.8862745,0.9529412}\color{black}
\textcolor{black}{Between Pathways Model} &
\cellcolor[rgb]{0.8509804,0.8862745,0.9529412}\color{black} PPI, SGI &
\cellcolor[rgb]{0.8509804,0.8862745,0.9529412}\color{black}
\textit{\textcolor{black}{S. cerevisiae}} &
\cellcolor[rgb]{0.8509804,0.8862745,0.9529412}\color{black}
\hyperlink{ENREF56}{Kelley and Ideker (2005)} &
\cellcolor[rgb]{0.8509804,0.8862745,0.9529412}~
\\\hline
Within Pathways Model &
PPI, SGI &
\textit{S. cerevisiae} &
\hyperlink{ENREF56}{Kelley and Ideker (2005)} &
~
\\\hline
\cellcolor[rgb]{0.8509804,0.8862745,0.9529412}\color{black}
\textcolor{black}{Decision Tree} &
\cellcolor[rgb]{0.8509804,0.8862745,0.9529412}\color{black} PPI,
expression, phenotype &
\cellcolor[rgb]{0.8509804,0.8862745,0.9529412}\color{black}
\textit{\textcolor{black}{S. cerevisiae}} &
\cellcolor[rgb]{0.8509804,0.8862745,0.9529412}\color{black}
\hyperlink{ENREF113}{Wong}\hyperlink{ENREF113}{\textit{\textcolor{black}{
et al.}}}\hyperlink{ENREF113}{ (2004)} &
\cellcolor[rgb]{0.8509804,0.8862745,0.9529412}\color{black} 2
Hop\\\hline
Logistic Regression &
SGI, PPI, co-expression, phenotype &
\textit{C. elegans} &
\hyperlink{ENREF118}{Zhong and Sternberg (2006)} &
Gene Orienteer\\\hline
\cellcolor[rgb]{0.8509804,0.8862745,0.9529412}\color{black}
\textcolor{black}{Network Sampling} &
\cellcolor[rgb]{0.8509804,0.8862745,0.9529412}\color{black} SGI, PPI, GO
&
\cellcolor[rgb]{0.8509804,0.8862745,0.9529412}\color{black}
\textit{\textcolor{black}{S. cerevisiae}} &
\cellcolor[rgb]{0.8509804,0.8862745,0.9529412}{\color{black}
\hyperlink{ENREF61}{Le Meur and Gentleman (2008)}}

\color{black}
\hyperlink{ENREF67}{LeMeur}\hyperlink{ENREF67}{\textit{\textcolor{black}{
et al.}}}\hyperlink{ENREF67}{ (2014)} &
\cellcolor[rgb]{0.8509804,0.8862745,0.9529412}\color{black}
SLGI(R)\\\hline
Random Walk &
GO, PPI, expression &
\textit{S. cerevisiae}

\textit{C. elegans} &
\hyperlink{ENREF22}{Chipman and Singh (2009)} &
~
\\\hline
\cellcolor[rgb]{0.8509804,0.8862745,0.9529412}\color{black}
\textcolor{black}{Shared Function} &
\cellcolor[rgb]{0.8509804,0.8862745,0.9529412}\color{black}
Co-expression, PPI, text mining, phylogeny &
\cellcolor[rgb]{0.8509804,0.8862745,0.9529412}\color{black}
\textit{\textcolor{black}{C. elegans}} &
\cellcolor[rgb]{0.8509804,0.8862745,0.9529412}\color{black}
\hyperlink{ENREF63}{Lee}\hyperlink{ENREF63}{\textit{\textcolor{black}{
et al.}}}\hyperlink{ENREF63}{ (2010b)} &
\cellcolor[rgb]{0.8509804,0.8862745,0.9529412}\color{black}
WormNet\\\hline
Logistic Regression &
Co-expression, PPI, phenotype &
\textit{C. elegans} &
\hyperlink{ENREF62}{Lee}\hyperlink{ENREF62}{\textit{ et
al.}}\hyperlink{ENREF62}{ (2010a)} &
GI Finder\\\hline
\cellcolor[rgb]{0.8509804,0.8862745,0.9529412}\color{black}
\textcolor{black}{Jaccard Index} &
\cellcolor[rgb]{0.8509804,0.8862745,0.9529412}\color{black} GO, SGI,
PPI, phenotype &
\cellcolor[rgb]{0.8509804,0.8862745,0.9529412}\color{black} Eukarya &
\cellcolor[rgb]{0.8509804,0.8862745,0.9529412}\color{black}
\hyperlink{ENREF50}{Hoehndorf}\hyperlink{ENREF50}{\textit{\textcolor{black}{
et al.}}}\hyperlink{ENREF50}{ (2013)} &
\cellcolor[rgb]{0.8509804,0.8862745,0.9529412}~
\\\hline
Bimodal Statistics &
~
 &
~
 &
\hyperlink{ENREF110}{Wappett (2014)} &
BiSEp(R)\\\hline
\cellcolor[rgb]{0.8509804,0.8862745,0.9529412}\color{black}
\textcolor{black}{Machine Learning} &
\cellcolor[rgb]{0.8509804,0.8862745,0.9529412}~
 &
\cellcolor[rgb]{0.8509804,0.8862745,0.9529412}~
 &
\cellcolor[rgb]{0.8509804,0.8862745,0.9529412}\color{black} Discussed by
\hyperlink{ENREF6}{Babyak (2004)} and \hyperlink{ENREF64}{Lee and
Marcotte (2009)} &
\cellcolor[rgb]{0.8509804,0.8862745,0.9529412}~
\\\hline
\textbf{Machine Learning as discussed by
}\hyperlink{ENREF114}{\textbf{Wu}}\hyperlink{ENREF114}{\textbf{\textit{
et al.}}}\hyperlink{ENREF114}{\textbf{ (2014)}} &
~
 &
~
 &
\hyperlink{ENREF81}{Qi}\hyperlink{ENREF81}{\textit{ et
al.}}\hyperlink{ENREF81}{ (2008)}

\hyperlink{ENREF78}{Paladugu}\hyperlink{ENREF78}{\textit{ et
al.}}\hyperlink{ENREF78}{ (2008)}

\hyperlink{ENREF68}{Li}\hyperlink{ENREF68}{\textit{ et
al.}}\hyperlink{ENREF68}{ (2011)} &
~
\\\hline
\cellcolor[rgb]{0.8509804,0.8862745,0.9529412}\color{black}
\textcolor{black}{Machine Learning} &
\cellcolor[rgb]{0.8509804,0.8862745,0.9529412}~
 &
\cellcolor[rgb]{0.8509804,0.8862745,0.9529412}~
 &
\cellcolor[rgb]{0.8509804,0.8862745,0.9529412}\color{black}
\hyperlink{ENREF79}{Pandey}\hyperlink{ENREF79}{\textit{\textcolor{black}{
et al.}}}\hyperlink{ENREF79}{ (2010)} &
\cellcolor[rgb]{0.8509804,0.8862745,0.9529412}\color{black} MNMC\\\hline
Machine Learning Meta-Analysis &
~
 &
~
 &
\hyperlink{ENREF114}{Wu}\hyperlink{ENREF114}{\textit{ et
al.}}\hyperlink{ENREF114}{ (2014)} &
MetaSL\\\hline
\cellcolor[rgb]{0.8509804,0.8862745,0.9529412}{\color{black}
\textcolor{black}{Flux Variability Analysis}}

{\color{black} \textcolor{black}{Flux Balance Analysis}}

\color{black} \textcolor{black}{Network Simulation} &
\cellcolor[rgb]{0.8509804,0.8862745,0.9529412}\color{black} Metabolism &
\cellcolor[rgb]{0.8509804,0.8862745,0.9529412}{\color{black}
\textit{\textcolor{black}{E. coli}}}

\color{black} \textit{\textcolor{black}{Mycoplasma pneumoniae}} &
\cellcolor[rgb]{0.8509804,0.8862745,0.9529412}\color{black}
\hyperlink{ENREF45}{G\"uell}\hyperlink{ENREF45}{\textit{\textcolor{black}{
et al.}}}\hyperlink{ENREF45}{ (2014)} &
\cellcolor[rgb]{0.8509804,0.8862745,0.9529412}~
\\\hline
\end{supertabular}
\end{flushleft}




\textbf{Table 3. }Existing prediction methods for Synthetic Lethality
in Cancer
\begin{flushleft}
\tablehead{}
\begin{supertabular}{m{4.421cm}|m{2.342cm}|m{2.388cm}|m{4.552cm}|m{2.299cm}}
\multicolumn{1}{m{4.421cm}}{\cellcolor{white}\bfseries\color{black}
Method} &
\multicolumn{1}{m{2.342cm}}{\cellcolor{white}\color{black}
\textcolor{black}{Input Data}} &
\multicolumn{1}{m{2.388cm}}{\cellcolor{white}\color{black}
\textcolor{black}{Species}} &
\multicolumn{1}{m{4.552cm}}{\cellcolor{white}\bfseries\color{black}
Source} &
\cellcolor{white}\bfseries\color{black} Tool Offered\\\hline
\cellcolor[rgb]{0.8509804,0.8862745,0.9529412}\color{black}
\textcolor{black}{Network Centrality} &
\cellcolor[rgb]{0.8509804,0.8862745,0.9529412}\color{black} PPI &
\cellcolor[rgb]{0.8509804,0.8862745,0.9529412}\color{black}
\textit{\textcolor{black}{H. sapiens}} &
\cellcolor[rgb]{0.8509804,0.8862745,0.9529412}\color{black}
\hyperlink{ENREF59}{Kranthi}\hyperlink{ENREF59}{\textit{\textcolor{black}{
et al.}}}\hyperlink{ENREF59}{ (2013)} &
\cellcolor[rgb]{0.8509804,0.8862745,0.9529412}~
\\\hline
Differential Expression &
Expression, Mutation &
\textit{H. sapiens} &
\hyperlink{ENREF109}{Wang and Simon (2013)} &
~
\\\hline
\cellcolor[rgb]{0.8509804,0.8862745,0.9529412}{\color{black}
\textcolor{black}{Comparative Genomics}}

\color{black} \textcolor{black}{Chemical-Genomics} &
\cellcolor[rgb]{0.8509804,0.8862745,0.9529412}\color{black} Yeast SGI,
Homology &
\cellcolor[rgb]{0.8509804,0.8862745,0.9529412}\color{black}
\textit{\textcolor{black}{H. sapiens}} &
\cellcolor[rgb]{0.8509804,0.8862745,0.9529412}\color{black}
\hyperlink{ENREF48}{Heiskanen and Aittokallio (2012)} &
\cellcolor[rgb]{0.8509804,0.8862745,0.9529412}~
\\\hline
Comparative Genomics &
Yeast SGI, Homology &
\textit{H. sapiens} &
\hyperlink{ENREF30}{Deshpande}\hyperlink{ENREF30}{\textit{ et
al.}}\hyperlink{ENREF30}{ (2013)} &
~
\\\hline
\cellcolor[rgb]{0.8509804,0.8862745,0.9529412}\color{black}
\textcolor{black}{Genome Evolution} &
\cellcolor[rgb]{0.8509804,0.8862745,0.9529412}~
 &
\cellcolor[rgb]{0.8509804,0.8862745,0.9529412}~
 &
\cellcolor[rgb]{0.8509804,0.8862745,0.9529412}\color{black}
\hyperlink{ENREF73}{Lu}\hyperlink{ENREF73}{\textit{\textcolor{black}{
et al.}}}\hyperlink{ENREF73}{ (2013)} &
\cellcolor[rgb]{0.8509804,0.8862745,0.9529412}~
\\\hline
Machine Learning &
~
 &
~
 &
Discussed by \hyperlink{ENREF6}{Babyak (2004)} and
\hyperlink{ENREF64}{Lee and Marcotte (2009)} &
~
\\\hline
\cellcolor[rgb]{0.8509804,0.8862745,0.9529412}\color{black}
\textcolor{black}{Differential Expression} &
\cellcolor[rgb]{0.8509804,0.8862745,0.9529412}~
 &
\cellcolor[rgb]{0.8509804,0.8862745,0.9529412}~
 &
\cellcolor[rgb]{0.8509804,0.8862745,0.9529412}\color{black}
\hyperlink{ENREF95}{Tiong}\hyperlink{ENREF95}{\textit{\textcolor{black}{
et al.}}}\hyperlink{ENREF95}{ (2014)} &
\cellcolor[rgb]{0.8509804,0.8862745,0.9529412}~
\\\hline
Literature Database &
~
 &
~
 &
\hyperlink{ENREF69}{Li}\hyperlink{ENREF69}{\textit{ et
al.}}\hyperlink{ENREF69}{ (2014)} &
Syn-Lethality\\\hline
\cellcolor[rgb]{0.8509804,0.8862745,0.9529412}\color{black}
\textcolor{black}{Meta-Analysis} &
\cellcolor[rgb]{0.8509804,0.8862745,0.9529412}\color{black}
Meta-Analysis Machine Learning &
\cellcolor[rgb]{0.8509804,0.8862745,0.9529412}~
 &
\cellcolor[rgb]{0.8509804,0.8862745,0.9529412}\color{black}
\hyperlink{ENREF114}{Wu}\hyperlink{ENREF114}{\textit{\textcolor{black}{
et al.}}}\hyperlink{ENREF114}{ (2014)} &
\cellcolor[rgb]{0.8509804,0.8862745,0.9529412}\color{black}
MetaSL\\\hline
Pathway Analysis &
~
 &
~
 &
\hyperlink{ENREF117}{Zhang}\hyperlink{ENREF117}{\textit{ et
al.}}\hyperlink{ENREF117}{ (2015)} &
~
\\\hline
\cellcolor[rgb]{0.8509804,0.8862745,0.9529412}\color{black}
\textcolor{black}{Protein Domains} &
\cellcolor[rgb]{0.8509804,0.8862745,0.9529412}\color{black} Homology &
\cellcolor[rgb]{0.8509804,0.8862745,0.9529412}~
 &
\cellcolor[rgb]{0.8509804,0.8862745,0.9529412}\color{black}
\hyperlink{ENREF58}{Kozlov}\hyperlink{ENREF58}{\textit{\textcolor{black}{
et al.}}}\hyperlink{ENREF58}{ (2015)} &
\cellcolor[rgb]{0.8509804,0.8862745,0.9529412}~
\\\hline
Data-Mining &
~
 &
~
 &
\hyperlink{ENREF53}{Jerby-Arnon}\hyperlink{ENREF53}{\textit{ et
al.}}\hyperlink{ENREF53}{ (2014)}

\hyperlink{ENREF85}{Ryan}\hyperlink{ENREF85}{\textit{ et
al.}}\hyperlink{ENREF85}{ (2014)}

\hyperlink{ENREF27}{Crunkhorn (2014)}

\hyperlink{ENREF70}{Lokody (2014)} &
DAISY (method)\\\hline
\cellcolor[rgb]{0.8509804,0.8862745,0.9529412}{\color{black}
\textcolor{black}{Cancer Genome Evolution}}

{\color{black} \textcolor{black}{Hypothesis Test}}

\color{black} \textcolor{black}{Machine Learning} &
\cellcolor[rgb]{0.8509804,0.8862745,0.9529412}{\color{black} Expression,
DNA CNV, }

\color{black} Known SL &
\cellcolor[rgb]{0.8509804,0.8862745,0.9529412}\color{black}
\textit{\textcolor{black}{H. sapiens}} &
\cellcolor[rgb]{0.8509804,0.8862745,0.9529412}\color{black}
\hyperlink{ENREF74}{Lu}\hyperlink{ENREF74}{\textit{\textcolor{black}{
et al.}}}\hyperlink{ENREF74}{ (2015)} &
\cellcolor[rgb]{0.8509804,0.8862745,0.9529412}~
\\\hline
Chi-Squared Test &
Expression, DNA CNV, Methylation, or Mutation &
\textit{H. sapiens} &
Tom Kelly, Parry Guilford, and Mik Black

Dissertation (Kelly 2013)

Manuscript in Preparation &
SLIPT\\\hline
\cellcolor[rgb]{0.8509804,0.8862745,0.9529412}\color{black}
\textcolor{black}{Survival Analysis} &
\cellcolor[rgb]{0.8509804,0.8862745,0.9529412}\color{black} Expression,
Clinical &
\cellcolor[rgb]{0.8509804,0.8862745,0.9529412}\color{black}
\textit{\textcolor{black}{H. sapiens}} &
\cellcolor[rgb]{0.8509804,0.8862745,0.9529412}\color{black} Mik Black
(personal communication) &
\cellcolor[rgb]{0.8509804,0.8862745,0.9529412}~
\\\hline
\end{supertabular}
\end{flushleft}


There are a number of existing computational methods for predicting
synthetic lethal gene pairs. While these demonstrate the power and
need for predictions of synthetic lethality in human and cancer
contexts, limitations of previous methods \ could be met with a
different approach. For instance, computational approaches to
synthetic lethal prediction are often difficult to interpret, replicate
for new genes, or reliant on data types not available for genes that
other cancer researchers work on. 


\hyperlink{ENREF59}{Kranthi}\hyperlink{ENREF59}{\textit{ et
al.}}\hyperlink{ENREF59}{ (2013)} took a network approach to discovery
of synthetic lethal candidate selection applying the concept to
{\textquoteleft}centrality{\textquoteright} to a human PPI network
involving interacting partners of known cancer genes. The effect of
removing pairs of genes on connectivity of the network was used as a
surrogate for viability which is supported by observations that the PPI
and synthetic lethal networks are orthogonal in \textit{S. cerevisiae}
studies (Tong\textit{ et al.} 2004). While they showed the power law
distribution expected of a scale-free synthetic lethal network with
centrality measures, their approach was limited to known cancer genes
and is not applicable to genes without PPI data. Other nucleotide
sequencing data types are more commonly available for cancer studies at
a genomic scale. Of further concern is that the results were enriched
for p53 synthetic lethal partners which is relevant to many cancer
researchers but makes using this approach for other cancer genes
difficult with respect to multiple testing. This enrichment may be
due to the known drastic effect of removing p53 itself from the network
as a master regulator, cancer driving tumour suppressor gene, and
highly connected network {\textquoteleft}hub{\textquoteright}. The
focus on cancer genes is useful for translation into therapeutics but
does not account for variable genetic backgrounds or effect of protein
removal on the whole cellular network. 


A comparative genomics approach by
\hyperlink{ENREF30}{Deshpande}\hyperlink{ENREF30}{\textit{ et
al.}}\hyperlink{ENREF30}{ (2013)} used the results of well
characterised high-throughput mutation screens in \textit{S. cerevisiae
}as candidates for synthetic lethality in humans (Baryshnikova\textit{
et al.} 2010a; Boone\textit{ et al.} 2007; Costanzo\textit{ et al.}
2010; Costanzo\textit{ et al.} 2011; Tong\textit{ et al.} 2001;
Tong\textit{ et al.} 2004). Yeast synthetic lethal partners were
compared to human orthologues to find cancer relevant synthetic lethal
candidate pairs with direct therapeutic potential. Proposed as a
complementary approach to siRNA screens, several synthetic lethal
candidates were successfully validated in cell culture; however, this
methodology is limited to application on human genes with known yeast
orthologues. Synthetic lethal interactions themselves may not be
conserved between species (Dixon\textit{ et al.} 2009a), although
synthetic lethal interactions between pathways may are more comparable.
\ There have been many gene duplications in the separate evolutionary
histories of humans and yeast which may lead to differences in genetic
redundancy. Yeast are further not an ideal human cancer model because
they are do not have tissue specificity, multicellular gene regulation,
or orthologues to a number of known cancer genes such as p53. 


Differential gene expression has also been explored to predict synthetic
lethal pairs in cancer which would be widely applicable due to the
availability of public gene expression data for a large number of
samples and cancer types. \hyperlink{ENREF109}{Wang and Simon (2013)}
found differentially expressed genes between tumours with or without
functional p53 mutations in Cancer Genome Atlas (TCGA) and Cell Line
Encyclopaedia (CCLE) RNA-Seq gene expression data as candidate
synthetic lethal partner pathways of p53. Some of these pathways were
consistent with the literature and drug sensitivity cell-line screens
demonstrating the potential of gene expression as a surrogate for gene
function and use of public genomic data to predict synthetic lethal
gene pairs in cancer. However, the analyses were limited to kinase
genes and focused on currently druggable genes, lacking wider
application of synthetic lethal prediction methodology. This approach
may not be feasible or applicable in cancer genes with a lower mutation
rate than p53. 


\hyperlink{ENREF95}{Tiong}\hyperlink{ENREF95}{\textit{ et
al.}}\hyperlink{ENREF95}{ (2014)} also investigated gene expression as
a predictor of synthetic lethal pairs with colorectal cancer
microarrays. Simultaneously differentially expressed
{\textquotedblleft}tumour dependent{\textquotedblright} gene pairs
between cancer and normal tissue were used as candidate synthetic
lethal interactions. The top 20 genes were tested for differential
expression at the protein level with immunohistochemistry staining and
correlation with clinical characteristics. Some of the predicted
synthetic lethal pairs were consistent with the literature and 2 novel
synthetic lethal interactions with p53 were validated in pre-clinical
models. While a valuable proof-of-concept for integration of
\textit{in silico} approaches to synthetic lethal discovery in cancer,
the results again focus on p53 rather than the wider application of
synthetic lethal prediction. The gene expression analyses were
conducted in a Han Chinese population with a small sample size (70
tumour, 12 normal) and may not be applicable to other populations. 


Another approach to systematic synthetic lethality discovery specific to
human cancer (in contrast to the plethora of yeast synthetic lethality
data) was to build a database as done by
\hyperlink{ENREF69}{Li}\hyperlink{ENREF69}{\textit{ et
al.}}\hyperlink{ENREF69}{ (2014)}. In their relational database,
called {\textquotedblleft}Syn-lethality{\textquotedblright}, they have
curated both known experimentally discovered synthetic lethal pairs in
humans (113 pairs) from the literature and those predicted from
synthetic lethality between orthologous genes in \textit{S. cerevisiae}
yeast (1114 pairs). This knowledge-based database is the first known
dedicated to human cancer synthetic lethal interactions and integrates
gene functional, annotation, pathway and molecular mechanism data with
experimental and predicted synthetic lethal gene pairs. This
combination of data sources is intended to tackle the trade-off between
more conclusive synthetic lethal experiments in yeast and more
clinically relevant synthetic lethal experiments in human cancer
models, such as RNAi, especially when high-throughput screens are
costly and prone to false positives in either system and difficult to
replicate across gene backgrounds. This database centralises a wealth
of knowledge scattered in the literature including cancer relevant
genes (BRCA, PARP, PTEN, VHL, MYC, EGFR, MSH2, KRAS, and TP53) and is
publicly available as a Java App. However, the methodology was not
released to replicate or add to the findings with new datasets.
\ Suggested future directions were promising, such as constructing
networks of known synthetic lethality, applying known synthetic
lethality to cancer treatment, data mining, replicating the approach
for synthetic lethality in model organisms, signalling pathways, and
develop a complete global network in human cancer or yeast (both of
which are still incomplete with experimental data). 


From the same group,
\hyperlink{ENREF114}{Wu}\hyperlink{ENREF114}{\textit{ et
al.}}\hyperlink{ENREF114}{ (2014)} developed a meta-analysis method
(based on the machine learning methods in Table 4) for synthetic lethal
gene pairs relevant to developing selective drugs against human cancer.
\ They note that computational approaches scale-up across the genome at
lower cost than experimental screens but existing methods are limited
by noise and overfitting to a particular predictive feature. Their
{\textquotedblleft}metaSL{\textquotedblright} approach performs well
with an AUROC of 0.871 with the claimed strengths of existing machine
learning methods and the results are shared on the web. However, once
again, the method is not available for analysis of other genes studied
by the cancer research community and the method lacks mechanisms,
reproducibility, and interpretation by researchers. While machine
learning has great potential as a predictor, it is difficult to
interpret which features are being used for prediction and their
mechanistic significance, particularly for biologists with limited
exposure to computational concepts. 


Focusing on the potential for synthetic lethality to be an effective
anti-cancer drug target,
\hyperlink{ENREF117}{Zhang}\hyperlink{ENREF117}{\textit{ et
al.}}\hyperlink{ENREF117}{ (2015)} used modelling signalling pathways
to identify synthetic lethal interactions between known drug targets
and cancer genes. A computational approach was again used here to
tackle the limitations of experimental RNAi screens such as scale,
instability of knockdown, and off-target effects. Strangely, they
seemed more concerned with the needs of the pharmaceutical companies
than those of the patients. However, their
{\textquotesingle}hybrid{\textquotesingle} method of a data-driven
model and known signalling pathways showed potential as a means to
predict cell death in single and combination gene knockouts. They
used time series gene expression data (Lee\textit{ et al.} 2012) and
pathways (the Gene Ontology system). This approach successfully
detected many known essential genes in the human gene essentiality
database, known synthetic lethal partners in the Syn-Lethality
database, and predicted novel synthetic lethal gene pairs. Novel
results were enriched for TP53 and AKT synthetic lethal partners, genes
known to be important in many cancers but also predicted to be
essential by single gene disruption having a large impact on the
signalling pathways. Notably, they claim to be able to detect all 3
types of functionally related pathways or protein complexes. The
results are consistent with the experimental results in the literature
but the group has not shown validation for novel synthetic lethal
interactions. 


\textbf{Table 4. }Existing Computational Methods used for
meta-analysis by \hyperlink{ENREF114}{Wu}\hyperlink{ENREF114}{\textit{
et al.}}\hyperlink{ENREF114}{ (2014)}
\begin{flushleft}
\tablehead{}
\begin{supertabular}{m{4.421cm}|m{2.342cm}|m{2.388cm}|m{4.552cm}|m{2.299cm}}
\multicolumn{1}{m{4.421cm}}{\cellcolor{white}\bfseries\color{black}
Method} &
\multicolumn{1}{m{2.342cm}}{\cellcolor{white}\color{black}
\textcolor{black}{Input Data}} &
\multicolumn{1}{m{2.388cm}}{\cellcolor{white}\color{black}
\textcolor{black}{Species}} &
\multicolumn{1}{m{4.552cm}}{\cellcolor{white}\bfseries\color{black}
Source} &
\cellcolor{white}\bfseries\color{black} Tool Offered\\\hline
\cellcolor[rgb]{0.8509804,0.8862745,0.9529412}\color{black}
\textcolor{black}{Random Forest} &
\cellcolor[rgb]{0.8509804,0.8862745,0.9529412}\color{black} Machine
Learning &
\cellcolor[rgb]{0.8509804,0.8862745,0.9529412}\color{black}
\textit{\textcolor{black}{H. sapiens}} &
\cellcolor[rgb]{0.8509804,0.8862745,0.9529412}{\color{black}
\hyperlink{ENREF46}{Hall}\hyperlink{ENREF46}{\textit{\textcolor{black}{
et al.}}}\hyperlink{ENREF46}{ (2009)}}

\color{black} \hyperlink{ENREF18}{Breiman (2001)} &
\cellcolor[rgb]{0.8509804,0.8862745,0.9529412}\color{black} WEKA\\\hline
J48 (decision tree) &
Machine Learning &
\textit{H. sapiens} &
\hyperlink{ENREF46}{Hall}\hyperlink{ENREF46}{\textit{ et
al.}}\hyperlink{ENREF46}{ (2009)}

~
 &
WEKA\\\hline
\cellcolor[rgb]{0.8509804,0.8862745,0.9529412}\color{black}
\textcolor{black}{Bayes (Log Regression)} &
\cellcolor[rgb]{0.8509804,0.8862745,0.9529412}\color{black} Machine
Learning &
\cellcolor[rgb]{0.8509804,0.8862745,0.9529412}\color{black}
\textit{\textcolor{black}{H. sapiens}} &
\cellcolor[rgb]{0.8509804,0.8862745,0.9529412}{\color{black}
\hyperlink{ENREF46}{Hall}\hyperlink{ENREF46}{\textit{\textcolor{black}{
et al.}}}\hyperlink{ENREF46}{ (2009)}}

~
 &
\cellcolor[rgb]{0.8509804,0.8862745,0.9529412}\color{black} WEKA\\\hline
Bayes (Network) &
Machine Learning &
\textit{H. sapiens} &
\hyperlink{ENREF46}{Hall}\hyperlink{ENREF46}{\textit{ et
al.}}\hyperlink{ENREF46}{ (2009)}

~
 &
WEKA\\\hline
\cellcolor[rgb]{0.8509804,0.8862745,0.9529412}\color{black}
\textcolor{black}{PART (Rule-based)} &
\cellcolor[rgb]{0.8509804,0.8862745,0.9529412}\color{black} Machine
Learning &
\cellcolor[rgb]{0.8509804,0.8862745,0.9529412}\color{black}
\textit{\textcolor{black}{H. sapiens}} &
\cellcolor[rgb]{0.8509804,0.8862745,0.9529412}{\color{black}
\hyperlink{ENREF46}{Hall}\hyperlink{ENREF46}{\textit{\textcolor{black}{
et al.}}}\hyperlink{ENREF46}{ (2009)}}

~
 &
\cellcolor[rgb]{0.8509804,0.8862745,0.9529412}\color{black} WEKA\\\hline
RBF Network &
Machine Learning &
\textit{H. sapiens} &
\hyperlink{ENREF46}{Hall}\hyperlink{ENREF46}{\textit{ et
al.}}\hyperlink{ENREF46}{ (2009)}

~
 &
WEKA\\\hline
\cellcolor[rgb]{0.8509804,0.8862745,0.9529412}\color{black}
\textcolor{black}{Bagging / Bootstrap} &
\cellcolor[rgb]{0.8509804,0.8862745,0.9529412}\color{black} Machine
Learning &
\cellcolor[rgb]{0.8509804,0.8862745,0.9529412}\color{black}
\textit{\textcolor{black}{H. sapiens}} &
\cellcolor[rgb]{0.8509804,0.8862745,0.9529412}{\color{black}
\hyperlink{ENREF46}{Hall}\hyperlink{ENREF46}{\textit{\textcolor{black}{
et al.}}}\hyperlink{ENREF46}{ (2009)}}

~
 &
\cellcolor[rgb]{0.8509804,0.8862745,0.9529412}\color{black} WEKA\\\hline
Classification via Regression &
Machine Learning &
\textit{H. sapiens} &
\hyperlink{ENREF46}{Hall}\hyperlink{ENREF46}{\textit{ et
al.}}\hyperlink{ENREF46}{ (2009)}

~
 &
WEKA\\\hline
\cellcolor[rgb]{0.8509804,0.8862745,0.9529412}\color{black}
\textcolor{black}{Support Vector Machine (Linear)} &
\cellcolor[rgb]{0.8509804,0.8862745,0.9529412}\color{black} Machine
Learning &
\cellcolor[rgb]{0.8509804,0.8862745,0.9529412}\color{black}
\textit{\textcolor{black}{H. sapiens}} &
\cellcolor[rgb]{0.8509804,0.8862745,0.9529412}\color{black}
\hyperlink{ENREF106}{Vapnik (1995)} &
\cellcolor[rgb]{0.8509804,0.8862745,0.9529412}~
\\\hline
Support Vector Machine (RBF -- Gaussian) &
Machine Learning &
\textit{H. sapiens} &
\hyperlink{ENREF54}{Joachims (1999)} &
~
\\\hline
\cellcolor[rgb]{0.8509804,0.8862745,0.9529412}\color{black}
\textcolor{black}{Multi-Network Multi-Class (MNMC)} &
\cellcolor[rgb]{0.8509804,0.8862745,0.9529412}\color{black} Machine
Learning &
\cellcolor[rgb]{0.8509804,0.8862745,0.9529412}\color{black}
\textit{\textcolor{black}{H. sapiens}} &
\cellcolor[rgb]{0.8509804,0.8862745,0.9529412}\color{black}
\hyperlink{ENREF79}{Pandey}\hyperlink{ENREF79}{\textit{\textcolor{black}{
et al.}}}\hyperlink{ENREF79}{ (2010)} &
\cellcolor[rgb]{0.8509804,0.8862745,0.9529412}~
\\\hline
MetaSL (Meta-Analysis) &
Machine Learning &
\textit{H. sapiens} &
\hyperlink{ENREF114}{Wu}\hyperlink{ENREF114}{\textit{ et
al.}}\hyperlink{ENREF114}{ (2014)} &
MetaSL\\\hline
\cellcolor[rgb]{0.8509804,0.8862745,0.9529412}\color{black}
\textcolor{black}{Pathway Analysis} &
\cellcolor[rgb]{0.8509804,0.8862745,0.9529412}\color{black} Pathway
Model &
\cellcolor[rgb]{0.8509804,0.8862745,0.9529412}\color{black}
\textit{\textcolor{black}{H. sapiens}} &
\cellcolor[rgb]{0.8509804,0.8862745,0.9529412}\color{black}
\hyperlink{ENREF117}{Zhang}\hyperlink{ENREF117}{\textit{\textcolor{black}{
et al.}}}\hyperlink{ENREF117}{ (2015)} &
\cellcolor[rgb]{0.8509804,0.8862745,0.9529412}~
\\\hline
\end{supertabular}
\end{flushleft}

While the mathematical reasoning and algorithms are given, code was not
released and it is unlikely that the wider biologically trained
research community will be able to reproduce or apply the findings
beyond the signalling pathways discussed by
\hyperlink{ENREF117}{Zhang}\hyperlink{ENREF117}{\textit{ et
al.}}\hyperlink{ENREF117}{ (2015)}. The authors note limitations as
directions for further research including the potential of their method
to detect mechanisms, types of interactions, impact of activation or
inhibition of proteins, and improve performance with a Boolean network
or differential equation approach, all of which have been claimed but
not shown. Further, this approach is limited by existing pathway data
with limited scale, scope, and reliability coming from a range of
sources. So far, modelling has been restricted to signalling pathways
which are immediately applicable to cancer; while important, the
approach lacks broader application to other diseases and pathway types.
\ Zhang et al. (2015) also lack validation, replication, or
application of findings and are heavily reliant on existing literature
for testing their predictions. 


Recognising the utility of synthetic lethality to drug inhibition and
specificity of anti-cancer treatments,
\hyperlink{ENREF53}{Jerby-Arnon}\hyperlink{ENREF53}{\textit{ et
al.}}\hyperlink{ENREF53}{ (2014)} also saw the need for effective
prediction of gene essentiality and synthetic lethality to augment
experimental studies of SL. They have developed a data-driven
pipeline called DAISY (data mining synthetic lethality identification
pipeline) and tested for genome-wide analysis of synthetic lethality in
public cancer genomics data from TCGA and CCLE. DAISY is intended to
predict the candidate synthetic lethal partners of a query gene such as
genes recurrently mutated in cancer. 


DAISY compares the results of analysis of several data types to predict
synthetic lethality, namely: DNA copy number, mutation and gene
expression profiles for clinical samples and cell lines. The cell
lines data also analysed gene essentiality profiles from shRNA screens.
\ Genes are classed as inactivated by copy number deletion, somatic
loss of function mutation, or low expression and tested for synthetic
lethal gene partners which are either essential in screens or not
deleted with copy number variants. Co-expression is also used for
synthetic lethality prediction based on studies in yeast
(Costanzo\textit{ et al.} 2010; Kelley \& Ideker 2005). Copy number,
gene expression and, essentiality analyses are stringently compared by
adjusting each for multiple tests with Bonferroni correction and only
taking hits which occur in all analyses. This methodology was also
adapted for synthetic dosage lethality by testing for partner genes
where genes are overactive with high copy number or expression. As
discussed above, the predictions performed well and an RNAi screen for
the example of VHL in renal cancer validated predicted synthetic lethal
partners of VHL demonstrating the feasibility of combining approaches
to synthetic lethal discovery in cancer and using computational
predictions to enable more efficient high-throughput screening.
\ However, this methodology is very stringent, missing potentially
valuable synthetic lethal candidates, may not be applicable to genes of
interest to other groups and the software for the procedure is not
publicly released for replication. 


Although the DAISY procedure performs well and has been well received by
the scientific community (Crunkhorn 2014; Lokody 2014; Ryan\textit{ et
al.} 2014), showing a need for such methodology, there is no indication
of adoption of the methodology in the community yet. The
co-expression analysis may not be the most effective way to test gene
expression for directional synthetic lethal interactions (where inverse
correlation would be expected). Presumably in the interests of a
large sample size, little care is taken to test tissue types separately
for tissue specific synthetic lethality (of interest since expression,
isoforms, gene function, and clinical characteristics of cancers are
tissue-dependent). Some data forms and analyses used, such as gene
essentiality, may not be available for all cancers, genes, or tissues,
and may not be reproduced. 


\hyperlink{ENREF74}{Lu}\hyperlink{ENREF74}{\textit{ et
al.}}\hyperlink{ENREF74}{ (2015)} critique the reliance of DAISY on
co-expression and propose an alternative computational prediction of
synthetic lethality based on machine learning methods and a cancer
genome evolution hypothesis. Using both DNA copy number and gene
expression data from TCGA, a cancer genome evolution model assumes that
synthetic lethal gene pairs behave in 2 distinct ways in response to an
inactive synthetic lethal partner gene, either a
{\textquoteleft}compensation{\textquoteright} pattern where the other
synthetic lethal partner is overactive or a {\textquoteleft}co-loss
underrepresentation{\textquoteright} pattern where the other synthetic
lethal partner is less likely to be lost, since loss of both genes
would cause death of the cancer cell. During the cancer genome
evolution as the cell becomes addicted to the remaining synthetic
lethal partner due to induced gene essentiality. These patterns would
explain why DAISY detects only a small number of synthetic lethal
pairs, compared to the large number expected based on model organism
studies (Boone\textit{ et al.} 2007), and the disparity between
screening and computationally predicted synthetic lethal candidates due
to testing different classes of synthetic lethal gene pairs.


\hyperlink{ENREF74}{Lu}\hyperlink{ENREF74}{\textit{ et
al.}}\hyperlink{ENREF74}{ (2015)} compared a genome-wide computational
model of genome evolution and gene expression patterns to the
experimental data of
\hyperlink{ENREF108}{Vizeacoumar}\hyperlink{ENREF108}{\textit{ et
al.}}\hyperlink{ENREF108}{ (2013)} and
\hyperlink{ENREF60}{Laufer}\hyperlink{ENREF60}{\textit{ et
al.}}\hyperlink{ENREF60}{ (2013)}. The model had an AUROC of 0.751,
performing well for a simpler method than DAISY. They predict a
larger comprehensive list of 591,000 human synthetic lethal partners
with a probability score threshold of 0.81, giving a precision of 67\%
and 14x enrichment of synthetic lethal true positives compared to
randomly selected gene pairs. \ Discovery of such a vast number of
cancer-relevant synthetic lethal interactions in humans would not be
feasible experimentally and is a valuable resource for research and
clinical applications. These predictions are not limited by assuming
co-expression of synthetic lethal partners or evolutionary conservation
with model organisms enabling wider synthetic lethal discovery.
\ However, there remains a lack of basis for an expectation of how many
synthetic lethal partners a particular gene will have, how many pairs
there are in the human genome, and whether pathways or correlation
structure would influence predicted synthetic lethal partners.


Large scale, computational approaches have yet to determine whether
synthetic lethal interactions are tissue-specific since
\hyperlink{ENREF74}{Lu}\hyperlink{ENREF74}{\textit{ et
al.}}\hyperlink{ENREF74}{ (2015)} used pan-cancer data for 14136
patients with 31 cancer types. Experimental data used for comparison
was a small training dataset specific to colorectal cancer, and based
on screens for other phenotypes, which may limit performance of the
model or application to other cancers. Proposed expansion of the
computational approach to mutation, microRNA, or epigenetic modulation
of gene function and tumour micro-environment or heterogeneity suggests
that synthetic lethal discovery could be widely applied to the current
challenges in cancer genomics. This approach was also based on
machine learning methodology and not supported by a software released
for the community to develop, contribute to, or reproduce beyond the
gene pairs given in the supplementary results.


To address these needs and concerns raised by recent computational
approaches to synthetic lethal discovery in cancer (Jerby-Arnon\textit{
et al.} 2014; Lu\textit{ et al.} 2015), we propose similar analysis
using solely gene expression data which is widely available for a large
number of samples in many different cancers. To firmly understand the
limitations and implications of synthetic lethal predictions, we
propose modelling and simulation of the statistical behaviour of
synthetic lethal gene pairs in genomics data. Comparison of synthetic
lethal gene candidates from public data analysis, predictions, and
networks across datasets will address tissue-specificity concerns.
\ Release of R codes used for simulation, prediction, and analysis will
enable adoption of the methodology in the cancer research community and
comparison to existing methods.


\clearpage\subsection[Data Sources]{Data Sources}

As summarised in Table 5, there is are a vast array of publicly
available resources for model organism, human, and cancer genomics
analysis. These will be used as a resource for bioinformatics analysis
of genetic interactions and networks in addition to modelling and
simulation approaches. 


\textbf{Table 5. }Data sources for Research
\begin{flushleft}
\tablehead{}
\begin{supertabular}{m{2.299cm}|m{3.8009999cm}|m{3.552cm}|m{6.5490003cm}}
\multicolumn{1}{m{2.299cm}}{\cellcolor{white}\bfseries\color{black}
Database} &
\multicolumn{1}{m{3.8009999cm}}{\cellcolor{white}~
} &
\multicolumn{1}{m{3.552cm}}{\cellcolor{white}\bfseries\color{black}
Study Type} &
\cellcolor{white}\bfseries\color{black} Data Types Supported\\\hline
\cellcolor[rgb]{0.8509804,0.8862745,0.9529412}\color{black}
\textcolor{black}{TCGA} &
\cellcolor[rgb]{0.8509804,0.8862745,0.9529412}\color{black} Cancer
Genome Atlas &
\cellcolor[rgb]{0.8509804,0.8862745,0.9529412}\color{black} Cancer &
\cellcolor[rgb]{0.8509804,0.8862745,0.9529412}\color{black} Sequence,
Mutation, CNV, SNP, expression, DNA Methylation, Clinical,
RNA-Seq\\\hline
ICGC &
International Cancer Genome Consortium &
Cancer &
Sequence, Mutation, CNV, SNP, expression, DNA Methylation, Clinical,
RNA-Seq\\\hline
\cellcolor[rgb]{0.8509804,0.8862745,0.9529412}\color{black}
\textcolor{black}{ENCODE} &
\cellcolor[rgb]{0.8509804,0.8862745,0.9529412}\color{black}
Encyclopaedia of DNA Elements &
\cellcolor[rgb]{0.8509804,0.8862745,0.9529412}{\color{black} Human
Normal Tissue}

\color{black} Cancer Cell Line &
\cellcolor[rgb]{0.8509804,0.8862745,0.9529412}{\color{black} Sequence,
expression, DNA/RNA binding}

\color{black} RNA-Seq, ChIP-Seq, RIP-Seq\\\hline
CCLE &
Cell Line Encyclopaedia &
Cancer Cell Line &
DNA CNV, expression, mutation, drug sensitivity\\\hline
\cellcolor[rgb]{0.8509804,0.8862745,0.9529412}\color{black}
\textcolor{black}{GEO} &
\cellcolor[rgb]{0.8509804,0.8862745,0.9529412}\color{black} Gene
Expression Omnibus &
\cellcolor[rgb]{0.8509804,0.8862745,0.9529412}\color{black} Various &
\cellcolor[rgb]{0.8509804,0.8862745,0.9529412}\color{black} Gene
Expression, RNA-Seq\\\hline
GENT &
Gene Expression Atlas &
Human Normal Tissue and Cancer &
Gene Expression\\\hline
\cellcolor[rgb]{0.8509804,0.8862745,0.9529412}\color{black}
\textcolor{black}{BIND} &
\cellcolor[rgb]{0.8509804,0.8862745,0.9529412}\color{black} Biomolecular
interaction network database &
\cellcolor[rgb]{0.8509804,0.8862745,0.9529412}{\color{black} Model
Organisms}

\color{black} Human &
\cellcolor[rgb]{0.8509804,0.8862745,0.9529412}{\color{black} DNA, RNA,
ligand, or protein binding}

\color{black} Two-hybrid data, ChIP, CLIP, RIP\\\hline
STRING &
Search Tool for retrieving interacting genes/proteins &
Various &
Protein-protein interactions: curated and predicted from MINT, PPRD,
BIND, DIP, BioGRID, KEGG, Reactome, IntAct, EcoCyc, NCI, and GO\\\hline
\cellcolor[rgb]{0.8509804,0.8862745,0.9529412}\color{black}
\textcolor{black}{BioGRID} &
\cellcolor[rgb]{0.8509804,0.8862745,0.9529412}\color{black} Biological
General Repository for Interaction Datasets  &
\cellcolor[rgb]{0.8509804,0.8862745,0.9529412}{\color{black}
\textit{\textcolor{black}{S. cerevisiae}},}

{\color{black} \textit{\textcolor{black}{S. pombe}},}

\color{black} \textit{\textcolor{black}{A. thaliana}}, etc &
\cellcolor[rgb]{0.8509804,0.8862745,0.9529412}\color{black} Protein and
Genetic Interaction\\\hline
Human Interactome &
Human Interactome &
Human &
Protein Interactions\\\hline
\cellcolor[rgb]{0.8509804,0.8862745,0.9529412}\color{black}
\textcolor{black}{DRYGIN} &
\cellcolor[rgb]{0.8509804,0.8862745,0.9529412}\color{black} Data
Repository of Yeast Genetic Interactions &
\cellcolor[rgb]{0.8509804,0.8862745,0.9529412}\color{black} Yeast &
\cellcolor[rgb]{0.8509804,0.8862745,0.9529412}\color{black} Genetic
Interactions\\\hline
SGD &
Saccharomyces Genome Database &
Yeast &
Sequence, Expression, Protein and Genetic Interactions\\\hline
\cellcolor[rgb]{0.8509804,0.8862745,0.9529412}\color{black}
\textcolor{black}{GO} &
\cellcolor[rgb]{0.8509804,0.8862745,0.9529412}\color{black} Gene
Ontology &
\cellcolor[rgb]{0.8509804,0.8862745,0.9529412}\color{black} Various &
\cellcolor[rgb]{0.8509804,0.8862745,0.9529412}\color{black} Gene
Function\\\hline
KEGG &
Kyoto Encyclopedia of Genes and Genomes &
Various &
Gene Function and Pathway\\\hline
\cellcolor[rgb]{0.8509804,0.8862745,0.9529412}\color{black}
\textcolor{black}{Reactome} &
\cellcolor[rgb]{0.8509804,0.8862745,0.9529412}~
 &
\cellcolor[rgb]{0.8509804,0.8862745,0.9529412}\color{black} Various &
\cellcolor[rgb]{0.8509804,0.8862745,0.9529412}\color{black} Gene
Function and Pathway\\\hline
DrugBank &
~
 &
Human &
Chemical and drug target sequence/structure\\\hline
\cellcolor[rgb]{0.8509804,0.8862745,0.9529412}\color{black}
\textcolor{black}{TTD} &
\cellcolor[rgb]{0.8509804,0.8862745,0.9529412}\color{black} Therapeutic
Target &
\cellcolor[rgb]{0.8509804,0.8862745,0.9529412}\color{black} Human &
\cellcolor[rgb]{0.8509804,0.8862745,0.9529412}\color{black} Protein and
Nucleic Acid drug targets\\\hline
TDR &
Targets database &
Human &
Chemical Genomics (Tropical Disease)\\\hline
\cellcolor[rgb]{0.8509804,0.8862745,0.9529412}\color{black}
\textcolor{black}{BindingDB} &
\cellcolor[rgb]{0.8509804,0.8862745,0.9529412}~
 &
\cellcolor[rgb]{0.8509804,0.8862745,0.9529412}~
 &
\cellcolor[rgb]{0.8509804,0.8862745,0.9529412}\color{black} Protein-drug
binding affinity\\\hline
\end{supertabular}
\end{flushleft}


\clearpage\section{Conclusion and Research Question}

Synthetic lethality is an important genetic interaction to study
fundamental cellular functions and exploit them for biomarkers and
cancer treatment. While there are a wide range of experimental and
computational approaches to synthetic lethal discovery, many are
limited to particular applications, prone to false positives, and
inconsistent across independent approaches to different genes of
interest. Therefore synthetic lethal interactions are difficult to
replicate or apply in the clinic. Computational approaches to
synthetic lethality are not widely adopted by the cancer research
community and experimental approaches cannot be combined to study
synthetic lethality at a genome-wide scale. However, these show
interest in synthetic lethal discovery in the community and the need
for robust predictions of synthetic lethal interactions in cancer and
human tissues. My thesis aims to develop such predictions with a
focus on the example of E-Cadherin to compare to the findings of
\hyperlink{ENREF94}{Telford}\hyperlink{ENREF94}{\textit{ et
al.}}\hyperlink{ENREF94}{ (2015)} and development of network approaches
to tissue specificity with the following bioinformatics and
computational biology investigations:

\begin{itemize}

\item 

Simulate gene expression data, construct a statistical model for
synthetic lethality, and measure performance of testing for synthetic
lethal genes. 

\item 

Apply Synthetic lethal prediction to public genomics data


%\setcounter{listWWNumviilevelii}{0}

\item 

Gene expression

\item 

DNA copy number

\item 

DNA methylation

\item 

Somatic Mutation

\item 

Select candidates for synthetic lethality with CDH1 in breast cancer,
compare to RNAi data for validation, and triage candidates for drug
development against HDGC and sporadic breast cancers

\item 

Release a synthetic lethal prediction methodology to the research
community for wider application 

\item 

Construct and analyse genome-scale synthetic lethal networks, tissue
specificity, or drug response using synthetic lethal predictions

\end{itemize}

\clearpage\section{References}
A\hypertarget{ENREF1}{}arts M, Bajrami I, Herrera-Abreu MT\textit{, et
al.} (2015) Functional Genetic Screen Identifies Increased Sensitivity
to WEE1 Inhibition in Cells with Defects in Fanconi Anemia and HR
Pathways. \textit{Mol Cancer Ther} \textbf{14}, 865-876.



\hypertarget{ENREF2}{}Agarwal S, Deane CM, Porter MA, Jones NS (2010)
Revisiting Date and Party Hubs: Novel Approaches to Role Assignment in
Protein Interaction Networks. \textit{PLoS Comput Biol} \textbf{6},
e1000817.



Ashworth A (2008) A synthetic lethal therapeutic approach: poly(ADP)
ribose polymerase inhibitors for the treatment of cancers deficient in
DNA double-strand break repair. \textit{J Clin Oncol} \textbf{26},
3785-3790.



Audeh MW, Carmichael J, Penson RT\textit{, et al.} (2010) Oral
poly(ADP-ribose) polymerase inhibitor olaparib in patients with BRCA1
or BRCA2 mutations and recurrent ovarian cancer: a proof-of-concept
trial. \textit{Lancet} \textbf{376}, 245-251.



Babu M, Arnold R, Bundalovic-Torma C\textit{, et al.} (2014)
Quantitative genome-wide genetic interaction screens reveal global
epistatic relationships of protein complexes in\textit{ Escherichia
coli}. \textit{PLoS Genet} \textbf{10}, e1004120.



\hypertarget{ENREF6}{}Babyak MA (2004) What you see may not be what you
get: a brief, nontechnical introduction to overfitting in
regression-type models. \textit{Psychosom Med} \textbf{66}, 411-421.



\hypertarget{ENREF7}{}Barab\'asi AL, Albert R (1999) Emergence of
scaling in random networks. \textit{Science} \textbf{286}, 509-512.



Barab\'asi AL, Oltvai ZN (2004) Network biology: understanding the
cell{\textquotesingle}s functional organization. \textit{Nat Rev Genet}
\textbf{5}, 101-113.



Barrat A, Weigt M (2000) On the properties of small-world network
models. \textit{The European Physical Journal B - Condensed Matter and
Complex Systems} \textbf{13}, 547-560.



Baryshnikova A, Costanzo M, Dixon S\textit{, et al.} (2010a) Synthetic
genetic array (SGA) analysis in \textit{Saccharomyces cerevisiae} and
\textit{Schizosaccharomyces pombe}. \textit{Methods Enzymol}
\textbf{470}, 145-179.



Baryshnikova A, Costanzo M, Kim Y\textit{, et al.} (2010b) Quantitative
analysis of fitness and genetic interactions in yeast on a genome
scale. \textit{Nat Meth} \textbf{7}, 1017-1024.



Bateson W, Mendel G (1909) \textit{Mendel{\textquotesingle}s principles
of heredity, by W. Bateson} University Press, Cambridge [Eng.].



\hypertarget{ENREF13}{}Bitler BG, Aird KM, Garipov A\textit{, et al.}
(2015) Synthetic lethality by targeting EZH2 methyltransferase activity
in ARID1A-mutated cancers. \textit{Nat Med} \textbf{21}, 231-238.



\hypertarget{ENREF14}{}Boettcher M, Lawson A, Ladenburger V\textit{, et
al.} (2014) High throughput synthetic lethality screen reveals a
tumorigenic role of adenylate cyclase in fumarate hydratase-deficient
cancer cells. \textit{BMC Genomics} \textbf{15}, 158.



\hypertarget{ENREF15}{}Boone C, Bussey H, Andrews BJ (2007) Exploring
genetic interactions and networks with yeast. \textit{Nat Rev Genet}
\textbf{8}, 437-449.



\hypertarget{ENREF16}{}Boucher B, Jenna S (2013) Genetic interaction
networks: better understand to better predict. \textit{Front Genet}
\textbf{4}, 290.



Bozovic-Spasojevic I, Azambuja E, McCaskill-Stevens W, Dinh P, Cardoso F
(2012) Chemoprevention for breast cancer. \textit{Cancer treatment
reviews} \textbf{38}, 329-339.



\hypertarget{ENREF18}{}Breiman L (2001) Random Forests. \textit{Machine
Learning} \textbf{45}, 5-32.



\hypertarget{ENREF19}{}Bryant HE, Schultz N, Thomas HD\textit{, et al.}
(2005) Specific killing of BRCA2-deficient tumours with inhibitors of
poly(ADP-ribose) polymerase. \textit{Nature} \textbf{434}, 913-917.



Bussey H, Andrews B, Boone C (2006) From worm genetic networks to
complex human diseases. \textit{Nat Genet} \textbf{38}, 862-863.



Butland G, Babu M, Diaz-Mejia JJ\textit{, et al.} (2008) eSGA:
\textit{E. coli }synthetic genetic array analysis. \textit{Nat Methods}
\textbf{5}, 789-795.



\hypertarget{ENREF22}{}Chipman K, Singh A (2009) Predicting genetic
interactions with random walks on biological networks. \textit{BMC
Bioinformatics} \textbf{10}, 17.



Collins S, Schuldiner M, Krogan N, Weissman J (2006) A strategy for
extracting and analyzing large-scale quantitative epistatic interaction
data. \textit{Genome Biol} \textbf{7}, R63.



\hypertarget{ENREF24}{}Corcoran RB, Ebi H, Turke AB\textit{, et al.}
(2012) EGFR-Mediated Reactivation of MAPK Signalling Contributes to
Insensitivity of BRAF-Mutant Colorectal Cancers to RAF Inhibition with
Vemurafenib. \textit{Cancer Discovery} \textbf{2}, 227-235.



Costanzo M, Baryshnikova A, Bellay J\textit{, et al.} (2010) The genetic
landscape of a cell. \textit{Science} \textbf{327}, 425-431.



Costanzo M, Baryshnikova A, Myers CL, Andrews B, Boone C (2011) Charting
the genetic interaction map of a cell. \textit{Curr Opin Biotechnol}
\textbf{22}, 66-74.



\hypertarget{ENREF27}{}Crunkhorn S (2014) Cancer: Predicting synthetic
lethal interactions. \textit{Nat Rev Drug Discov} \textbf{13}, 812.



\hypertarget{ENREF28}{}Davierwala AP, Haynes J, Li Z\textit{, et al.}
(2005) The synthetic genetic interaction spectrum of essential genes.
\textit{Nat Genet} \textbf{37}, 1147-1152.



Davies H, Bignell GR, Cox C\textit{, et al.} (2002) Mutations of the
BRAF gene in human cancer. \textit{Nature} \textbf{417}, 949-954.



\hypertarget{ENREF30}{}Deshpande R, Asiedu MK, Klebig M\textit{, et al.}
(2013) A comparative genomic approach for identifying synthetic lethal
interactions in human cancer. \textit{Cancer Res} \textbf{73},
6128-6136.



Di Nicolantonio F, Martini M, Molinari F\textit{, et al.} (2008)
Wild-type BRAF is required for response to panitumumab or cetuximab in
metastatic colorectal cancer. \textit{J Clin Oncol} \textbf{26},
5705-5712.



Dienstmann R, Tabernero J (2011) BRAF as a target for cancer therapy.
\textit{Anticancer Agents Med Chem} \textbf{11}, 285-295.



Dixon SJ, Andrews BJ, Boone C (2009a) Exploring the conservation of
synthetic lethal genetic interaction networks. \textit{Commun Integr
Biol} \textbf{2}, 78-81.



Dixon SJ, Costanzo M, Baryshnikova A, Andrews B, Boone C (2009b)
Systematic mapping of genetic interaction networks. \textit{Annu Rev
Genet} \textbf{43}, 601-625.



Dixon SJ, Fedyshyn Y, Koh JL\textit{, et al.} (2008) Significant
conservation of synthetic lethal genetic interaction networks between
distantly related eukaryotes. \textit{Proc Natl Acad Sci U S A}
\textbf{105}, 16653-16658.



Dorogovtsev SN, Mendes JF (2003) \textit{Evolution of networks: From
biological nets to the Internet and WWW} Oxford University Press, USA.



Erd\H{o}s P, R\'enyi A (1959) On Random Graphs I. \textit{Publ. Math.
\ Debrecen} \textbf{6}, 290-297.



Erd\H{o}s P, R\'enyi A (1960) On the Evolution of Random Graphs, 17-61.



\hypertarget{ENREF39}{}Farmer H, McCabe N, Lord CJ\textit{, et al.}
(2005) Targeting the DNA repair defect in BRCA mutant cells as a
therapeutic strategy. \textit{Nature} \textbf{434}, 917-921.



Fece de la Cruz F, Gapp BV, Nijman SM (2015) Synthetic Lethal
Vulnerabilities of Cancer. \textit{Annu Rev Pharmacol Toxicol}
\textbf{55}, 513-531.



Fisher RA (1919) XV.---The Correlation between Relatives on the
Supposition of Mendelian Inheritance. \textit{Earth and Environmental
Science Transactions of the Royal Society of Edinburgh} \textbf{52},
399-433.



Fong PC, Boss DS, Yap TA\textit{, et al.} (2009) Inhibition of
poly(ADP-ribose) polymerase in tumors from BRCA mutation carriers.
\textit{N Engl J Med} \textbf{361}, 123-134.



Fong PC, Yap TA, Boss DS\textit{, et al.} (2010) Poly(ADP)-ribose
polymerase inhibition: frequent durable responses in BRCA carrier
ovarian cancer correlating with platinum-free interval. \textit{J Clin
Oncol} \textbf{28}, 2512-2519.



Fraser A (2004) Towards full employment: using RNAi to find roles for
the redundant. \textit{Oncogene} \textbf{23}, 8346-8352.



\hypertarget{ENREF45}{}G\"uell O, Sagu\'es F, Serrano M\'A (2014)
Essential Plasticity and Redundancy of Metabolism Unveiled by Synthetic
Lethality Analysis. \textit{PLoS Comput Biol} \textbf{10}, e1003637.



\hypertarget{ENREF46}{}Hall M, Frank E, Holmes G\textit{, et al.} (2009)
The WEKA data mining software: an update. \textit{SIGKDD Explor.
Newsl.} \textbf{11}, 10-18.



\hypertarget{ENREF47}{}Han J-DJ, Bertin N, Hao T\textit{, et al.} (2004)
Evidence for dynamically organized modularity in the yeast
protein-protein interaction network. \textit{Nature} \textbf{430},
88-93.



\hypertarget{ENREF48}{}Heiskanen MA, Aittokallio T (2012) Mining
high-throughput screens for cancer drug targets-lessons from yeast
chemical-genomic profiling and synthetic lethality. \textit{Wiley
Interdisciplinary Reviews: Data Mining and Knowledge Discovery}
\textbf{2}, 263-272.



Hillenmeyer ME (2008) The chemical genomic portrait of yeast: uncovering
a phenotype for all genes. \textit{Science} \textbf{320}, 362-365.



\hypertarget{ENREF50}{}Hoehndorf R, Hardy NW, Osumi-Sutherland
D\textit{, et al.} (2013) Systematic Analysis of Experimental Phenotype
Data Reveals Gene Functions. \textit{PLoS One} \textbf{8}, e60847.



Holme P, Kim BJ (2002) Growing scale-free networks with tunable
clustering. \textit{Physical Review E} \textbf{65}, 026107.



\hypertarget{ENREF52}{}Hopkins AL (2008) Network pharmacology: the next
paradigm in drug discovery. \textit{Nat Chem Biol} \textbf{4}, 682-690.



\hypertarget{ENREF53}{}Jerby-Arnon L, Pfetzer N, Waldman
Yedael~Y\textit{, et al.} (2014) Predicting Cancer-Specific
Vulnerability via Data-Driven Detection of Synthetic Lethality.
\textit{Cell} \textbf{158}, 1199-1209.



\hypertarget{ENREF54}{}Joachims T (1999) Making large-scale support
vector machine learning practical. In: \textit{Advances in kernel
methods} (eds. Bernhard S, lkopf, Christopher JCB, Alexander JS), pp.
169-184. MIT Press.



\hypertarget{ENREF55}{}Kaelin WG, Jr. (2005) The concept of synthetic
lethality in the context of anticancer therapy. \textit{Nat Rev Cancer}
\textbf{5}, 689-698.



\hypertarget{ENREF56}{}Kelley R, Ideker T (2005) Systematic
interpretation of genetic interactions using protein networks.
\textit{Nat Biotech} \textbf{23}, 561-566.



Kelly ST (2013) \textit{Statistical Predictions of Synthetic Lethal
Interactions in Cancer} Dissertation, University of Otago.



\hypertarget{ENREF58}{}Kozlov KN, Gursky VV, Kulakovskiy IV, Samsonova
MG (2015) Sequence-based model of gap gene regulation network.
\textit{BMC Genomics} \textbf{15}, S6.



\hypertarget{ENREF59}{}Kranthi T, Rao SB, Manimaran P (2013)
Identification of synthetic lethal pairs in biological systems through
network information centrality. \textit{Mol Biosyst} \textbf{9},
2163-2167.



\hypertarget{ENREF60}{}Laufer C, Fischer B, Billmann M, Huber W, Boutros
M (2013) Mapping genetic interactions in human cancer cells with RNAi
and multiparametric phenotyping. \textit{Nat Methods} \textbf{10},
427-431.



\hypertarget{ENREF61}{}Le Meur N, Gentleman R (2008) Modeling synthetic
lethality. \textit{Genome Biol} \textbf{9}, R135.



\hypertarget{ENREF62}{}Lee AY, Perreault R, Harel S\textit{, et al.}
(2010a) Searching for Signalling Balance through the Identification of
Genetic Interactors of the Rab Guanine-Nucleotide Dissociation
Inhibitor \textit{gdi-1}. \textit{PLoS One} \textbf{5}, e10624.



\hypertarget{ENREF63}{}Lee I, Lehner B, Vavouri T\textit{, et al.}
(2010b) Predicting genetic modifier loci using functional gene
networks. \textit{Genome Res} \textbf{20}, 1143-1153.



\hypertarget{ENREF64}{}Lee I, Marcotte EM (2009) Effects of functional
bias on supervised learning of a gene network model. \textit{Methods
Mol Biol} \textbf{541}, 463-475.



Lee MJ, Ye AS, Gardino AK\textit{, et al.} (2012) Sequential application
of anticancer drugs enhances cell death by rewiring apoptotic signalling
networks. \textit{Cell} \textbf{149}, 780-794.



Lehner B, Crombie C, Tischler J, Fortunato A, Fraser AG (2006)
Systematic mapping of genetic interactions in \textit{Caenorhabditis
elegan}s identifies common modifiers of diverse signalling pathways.
\textit{Nat Genet} \textbf{38}, 896-903.



\hypertarget{ENREF67}{}LeMeur N, Jiang Z, Liu T, Mar J, Gentleman RC
(2014) SLGI: Synthetic Lethal Genetic Interaction. R package version
1.26.0.



\hypertarget{ENREF68}{}Li B, Cao W, Zhou J, Luo F (2011) Understanding
and predicting synthetic lethal genetic interactions in Saccharomyces
cerevisiae using domain genetic interactions. \textit{BMC Syst Biol}
\textbf{5}, 73.



\hypertarget{ENREF69}{}Li XJ, Mishra SK, Wu M, Zhang F, Zheng J (2014)
Syn-Lethality: An Integrative Knowledge Base of Synthetic Lethality
towards Discovery of Selective Anticancer Therapies. \textit{Biomed Res
Int} \textbf{2014}, 196034.



\hypertarget{ENREF70}{}Lokody I (2014) Computational modelling: A
computational crystal ball. \textit{Nature Reviews Cancer} \textbf{14},
649-649.



Lord CJ, Tutt AN, Ashworth A (2014) Synthetic Lethality and Cancer
Therapy: Lessons Learned from the Development of PARP Inhibitors.
\textit{Annu Rev Med}.



Loupakis F, Ruzzo A, Cremolini C\textit{, et al.} (2009) KRAS codon 61,
146 and BRAF mutations predict resistance to cetuximab plus irinotecan
in KRAS codon 12 and 13 wild-type metastatic colorectal cancer.
\textit{Br J Cancer} \textbf{101}, 715-721.



\hypertarget{ENREF73}{}Lu X, Kensche PR, Huynen MA, Notebaart RA (2013)
Genome evolution predicts genetic interactions in protein complexes and
reveals cancer drug targets. \textit{Nat Commun} \textbf{4}.



\hypertarget{ENREF74}{}Lu X, Megchelenbrink W, Notebaart RA, Huynen MA
(2015) Predicting human genetic interactions from cancer genome
evolution. \textit{PLoS One} \textbf{10}, e0125795.



Lum P, Armour C, Stepaniants S\textit{, et al.} (2004) Discovering modes
of action for therapeutic compounds using a genome-wide screen of yeast
heterozygotes. \textbf{116}, 121-137.



Milgram S (1967) The Small World Problem. \textit{Psychology Today}
\textbf{2}, 60-67.



Ooi SL, Shoemaker DD, Boeke JD (2003) DNA helicase gene interaction
network defined using synthetic lethality analyzed by microarray.
\textit{Nat Genet} \textbf{35}, 277-286.



\hypertarget{ENREF78}{}Paladugu S, Zhao S, Ray A, Raval A (2008) Mining
protein networks for synthetic genetic interactions. \textit{BMC
Bioinformatics} \textbf{9}, 426.



\hypertarget{ENREF79}{}Pandey G, Zhang B, Chang AN\textit{, et al.}
(2010) An integrative multi-network and multi-classifier approach to
predict genetic interactions. \textit{PLoS Comput Biol} \textbf{6}.



\hypertarget{ENREF80}{}Prahallad A, Sun C, Huang S\textit{, et al.}
(2012) Unresponsiveness of colon cancer to BRAF(V600E) inhibition
through feedback activation of EGFR. \textit{Nature} \textbf{483},
100-103.



\hypertarget{ENREF81}{}Qi Y, Suhail Y, Lin Y-y, Boeke JD, Bader JS
(2008) Finding friends and enemies in an enemies-only network: A graph
diffusion kernel for predicting novel genetic interactions and
co-complex membership from yeast genetic interactions. \textit{Genome
Res} \textbf{18}, 1991-2004.



\hypertarget{ENREF82}{}Ravnan MC, Matalka MS (2012) Vemurafenib in
patients with BRAF V600E mutation-positive advanced melanoma.
\textit{Clin Ther} \textbf{34}, 1474-1486.



Roguev A, Bandyopadhyay S, Zofall M\textit{, et al.} (2008) Conservation
and rewiring of functional modules revealed by an epistasis map in
fission yeast. \textit{Science} \textbf{322}, 405-410.



Roguev A, Wiren M, Weissman JS, Krogan NJ (2007) High-throughput genetic
interaction mapping in the fission yeast \textit{Schizosaccharomyces
pombe}. \textit{Nat Meth} \textbf{4}, 861-866.



\hypertarget{ENREF85}{}Ryan Colm~J, Lord Christopher~J, Ashworth A
(2014) DAISY: Picking Synthetic Lethals from Cancer Genomes.
\textit{Cancer Cell} \textbf{26}, 306-308.



Sander JD, Joung JK (2014) CRISPR-Cas systems for editing, regulating
and targeting genomes. \textit{Nat Biotechnol} \textbf{32}, 347-355.



\hypertarget{ENREF87}{}Schuldiner M, Collins SR, Thompson NJ\textit{, et
al.} (2005) Exploration of the function and organization of the yeast
early secretory pathway through an epistatic miniarray profile.
\textit{Cell} \textbf{123}, 507-519.



Schuldiner M, Collins SR, Weissman JS, Krogan NJ (2006) Quantitative
genetic analysis in\textit{ Saccharomyces cerevisiae }using epistatic
miniarray profiles (E-MAPs) and its application to chromatin functions.
\textit{Methods} \textbf{40}, 344-352.



Shaib W, Mahajan R, El-Rayes B (2013) Markers of resistance to anti-EGFR
therapy in colorectal cancer. \textit{J Gastrointest Oncol} \textbf{4},
308-318.



Siena S, Sartore-Bianchi A, Di Nicolantonio F, Balfour J, Bardelli A
(2009) Biomarkers predicting clinical outcome of epidermal growth
factor receptor-targeted therapy in metastatic colorectal cancer.
\textit{J Natl Cancer Inst} \textbf{101}, 1308-1324.



\hypertarget{ENREF91}{}Str\"om C, Helleday T (2012) Strategies for the
Use of Poly(adenosine diphosphate ribose) Polymerase (PARP) Inhibitors
in Cancer Therapy. \textit{Biomolecules} \textbf{2}, 635-649.



\hypertarget{ENREF92}{}Sun C, Wang L, Huang S\textit{, et al.} (2014)
Reversible and adaptive resistance to BRAF(V600E) inhibition in
melanoma. \textit{Nature} \textbf{508}, 118-122.



Taylor IW, Linding R, Warde-Farley D\textit{, et al.} (2009) Dynamic
modularity in protein interaction networks predicts breast cancer
outcome. \textit{Nat Biotechnol} \textbf{27}, 199-204.



\hypertarget{ENREF94}{}Telford BJ, Chen A, Beetham H\textit{, et al.}
(2015) Synthetic lethal screens identify vulnerabilities in GPCR
signalling and cytoskeletal organization in E-cadherin-deficient cells.
\textit{Mol Cancer Ther}.



\hypertarget{ENREF95}{}Tiong KL, Chang KC, Yeh KT\textit{, et al.}
(2014) CSNK1E/CTNNB1 are synthetic lethal to TP53 in colorectal cancer
and are markers for prognosis. \textit{Neoplasia} \textbf{16}, 441-450.



Tischler J, Lehner B, Fraser AG (2008) Evolutionary plasticity of
genetic interaction networks. \textit{Nat Genet} \textbf{40}, 390-391.



Tong AH, Evangelista M, Parsons AB\textit{, et al.} (2001) Systematic
genetic analysis with ordered arrays of yeast deletion mutants.
\textit{Science} \textbf{294}, 2364-2368.



Tong AH, Lesage G, Bader GD\textit{, et al.} (2004) Global mapping of
the yeast genetic interaction network. \textit{Science} \textbf{303},
808-813.



Travers J, Milgram S (1969) An Experimental Study of the Small World
Problem. \textit{Sociometry} \textbf{32}, 425-{}-443.



Tsai HC, Li H, Van Neste L\textit{, et al.} (2012) Transient low doses
of DNA-demethylating agents exert durable antitumor effects on
hematological and epithelial tumor cells. \textit{Cancer Cell}
\textbf{21}, 430-446.



Tutt A, Robson M, Garber JE\textit{, et al.} (2010) Oral
poly(ADP-ribose) polymerase inhibitor olaparib in patients with BRCA1
or BRCA2 mutations and advanced breast cancer: a proof-of-concept
trial. \textit{Lancet} \textbf{376}, 235-244.



Typas A, Nichols RJ, Siegele DA\textit{, et al.} (2008) High-throughput,
quantitative analyses of genetic interactions in \textit{E. coli}.
\textit{Nat Methods} \textbf{5}, 781-787.



\hypertarget{ENREF103}{}van der Meer R, Song HY, Park S-H, Abdulkadir
SA, Roh M (2014) RNAi screen identifies a synthetic lethal interaction
between PIM1 overexpression and PLK1 inhibition. \textit{Clinical
Cancer Research}.



\hypertarget{ENREF104}{}van Steen K (2011) Travelling the world of
gene--gene interactions. \textit{Brief Bioinform}.



\hypertarget{ENREF105}{}van Steen M (2010) \textit{Graph Theory and
Complex Networks: An Introduction} Maarten van Steen, VU Amsterdam.



\hypertarget{ENREF106}{}Vapnik VN (1995) \textit{The nature of
statistical learning theory} Springer-Verlag New York, Inc.



Vargas JJ, Gusella GL, Najfeld V, Klotman ME, Cara A (2004) Novel
integrase-defective lentiviral episomal vectors for gene transfer.
\textit{Hum Gene Ther} \textbf{15}, 361-372.



\hypertarget{ENREF108}{}Vizeacoumar FJ, Arnold R, Vizeacoumar
FS\textit{, et al.} (2013) A negative genetic interaction map in
isogenic cancer cell lines reveals cancer cell vulnerabilities.
\textit{Mol Syst Biol} \textbf{9}, 696.



\hypertarget{ENREF109}{}Wang X, Simon R (2013) Identification of
potential synthetic lethal genes to p53 using a computational biology
approach. \textit{BMC Medical Genomics} \textbf{6}, 30.



\hypertarget{ENREF110}{}Wappett M (2014) BiSEp: Toolkit To Identify
Candidate Synthetic Lethality. R package version 2.0.



Warburg O (1956) On the Origin of Cancer Cells. \textit{Science}
\textbf{123}, 390-314.



\hypertarget{ENREF112}{}Watts DJ, Strogatz SH (1998) Collective dynamics
of {\textquotesingle}small-world{\textquotesingle} networks.
\textit{Nature} \textbf{393}, 440-442.



\hypertarget{ENREF113}{}Wong SL, Zhang LV, Tong AHY\textit{, et al.}
(2004) Combining biological networks to predict genetic interactions.
\textit{Proc Natl Acad Sci U S A} \textbf{101}, 15682-15687.



\hypertarget{ENREF114}{}Wu M, Li X, Zhang F\textit{, et al.} (2014) In
silico prediction of synthetic lethality by meta-analysis of genetic
interactions, functions, and pathways in yeast and human cancer.
\textit{Cancer Inform} \textbf{13}, 71-80.



\hypertarget{ENREF115}{}Yang H, Higgins B, Kolinsky K\textit{, et al.}
(2012) Antitumor activity of BRAF inhibitor vemurafenib in preclinical
models of BRAF-mutant colorectal cancer. \textit{Cancer Res}
\textbf{72}, 779-789.



Yuan Z-X, Wang X-Y, Qin Q-Y\textit{, et al.} (2013) The Prognostic Role
of \textit{BRAF} Mutation in Metastatic Colorectal Cancer Receiving
Anti-\textit{EGFR} Monoclonal Antibodies: A Meta-Analysis. \textit{PLoS
One} \textbf{8}, e65995.



\hypertarget{ENREF117}{}Zhang F, Wu M, Li XJ\textit{, et al.} (2015)
Predicting essential genes and synthetic lethality via influence
propagation in signalling pathways of cancer cell fates. \textit{J
Bioinform Comput Biol}, 1541002.



\hypertarget{ENREF118}{}Zhong W, Sternberg PW (2006) Genome-Wide
Prediction of \textit{C. elegans} Genetic Interactions.
\textit{Science} \textbf{311}, 1481-1484.



%\bigskip
%\end{document}
