\chapter{Simulation and Modeling of Synthetic Lethal Pathways}
\label{chap:simulation}


\paragraph{Aims}

  \begin{itemize}
   \item A Model of Synthetic Lethal Genes in Gene Expression Data
   
   \bigskip
   
   \item Simulations of Known Synthetic Lethal Genes within Pathway Networks
   
   \bigskip
   
   \item Comparison of SLIPT to Alternative Approaches
  \end{itemize}

\paragraph{Summary}

    \begin{itemize}
      \item We have designed a straight-forward rational query-based synthetic lethal detection method with the example of application to \textit{CDH1} in cancer gene expression
      
      \bigskip
      
      \item We have developed a simulation pipeline to generate continuous gene expression with pathway structure including a procedure to simulate synthetic lethality 
      
      \bigskip
      
      \item Our simulation procedure is robust across pathway structures and has desirable performance compared to other statistical techniques 
      \end{itemize}

%\section{Background}

Synthetic lethality (SL) is the death of a cell or organism with the combined loss of two non-essential genes.   This phenomenon was originally used to study genetic interactions and functional redundancy in models organisms (Boone et al. 2007).   While synthetic lethal experiments have been performed in Drosophila melanogaster (Dobzhansky 1946), Caenorhabditis elegans (Lehner et al. 2006), Escherichia coli (Butland et al. 2008), Schizosaccharomyces pombe (Roguev et al. 2007), and various mammalian cell lines (Kaelin 2005), the most extensive synthetic lethal screens have been performed with the synthetic gene array (SGA) technique in Saccharomyces cerevisiae (Boone et al. 2007; Costanzo et al. 2011; Tong et al. 2004).  

Originally defined by double mutants, a range of mechanisms for gene inactivation of synthetic lethal partners can induce cell death including RNA interference and drug treatment where it is sometimes called ‘induced essentiality’ or ‘non-oncogene addiction’ in cancer research (Fece de la Cruz et al. 2015).  Cellular viability is the main means to measure synthetic lethal effects experimentally because it is quantified and measured consistently (as shown in Figure 1), whereas qualitative measures of impaired organism viability are ambiguous and less relevant to yeast or cancer research.

The cancer genetics laboratory are currently working on developing a synthetic lethal approach to target the tumour suppressor gene CDH1 which has been found to cause predispose early-onset breast and stomach cancers in mutation carriers, including families of New Zealand M\={a}ori (Berx et al. 1995; Guilford et al. 1998).  These families are currently closely monitored and offered drastic preventative surgery.  If it were developed, a drug selective against CDH1 mutant tumours would serve not only as a chemopreventative alternative for these families but also benefit the wider community as a treatment for sporadic cases of CDH1 mutant cancer.  To augment experimental work on CDH1 with isogenic cell lines (Telford et al. 2015), a computational methodology is explored here to exploit public cancer genomic databases.

Microarray and massively-parallel sequencing technologies are driving a revolution in molecular biological research, particularly with regard to cancer where the premise of ‘genomic medicine’ is rapidly becoming feasible with the use of genomics to identify cancer genes, diagnose patients with actionable mutations, and use gene expression as a prognostic marker.  Genomic data could also be used to identify novel drug targets and synthetic lethal partners of known cancer genes in particular.  The Cancer Genome Atlas database (TCGA) and the overarching International Cancer Genome Consortium (ICGC) provide a valuable public cancer genome data resource because they support many different data types for the same samples, for many different cancer types, and for high sample sizes (Cancer Genome Atlas Research Network 2014; Cancer Genome Atlas Research Network et al. 2013; International Cancer Genome Consortium 2014).  They host data of patient clinical factors, gene expression, somatic mutation, DNA copy number, and DNA methylation which could all serve to predict synthetic lethality from frequency of mutually exclusive gene inactivation and its impact on patient survival.  A number of other databases are given in the Table 6 which may be used to explore gene function, drug target feasibility, or replicate analyses but TCGA and ICGC datasets will be the focus of this project.

There is a growing need for a robust approach to cost-effective prediction of candidate synthetic lethal interaction, particularly in cancer research.  Exploiting existing public genomic databases is an ideal way to utilise existing resources with suitable sample sizes, data types, and different limitations to those of laboratory experiments.  A number of computation approaches to synthetic lethality have been developed but many of these rely on data not available to cancer researchers, methods that are difficult to replicate, over-fitted to a particular dataset, having mixed validation results, or do not have a software tool accessible to the research community.  These methodologies are reviewed in detail in the accompanying literature review.  They will still be considered to develop an improved synthetic lethal interaction prediction tool (SLIPT).  

A bioinformatics approach has distinct limitations to experimental methods and would work well combined with genetic screen data and conventional molecular biology laboratory validation techniques to answer biological research questions.  Compared with an experimental screen, a bioinformatics approach has the benefits of reduced costs, with the potential for automation, scaling up, and replication of the same gene across populations and cell types.  Analysis of public genomic data accounts for real tumour variation showing detection with tumour heterogeneity and genomic instability.  Compared with a cell line or xenograft experimental model we are limited by difficulties in establishing validity of a novel method, lack of mechanism, or potential for testing drug activity in the same system.  However, computational methods may further miss useful therapeutic candidates from variable genetic background and be limited by the population sampled.  This research builds on previous work in an Honours project and similar approaches in the literature (Jerby-Arnon et al. 2014; Kelly 2013; Lu et al. 2015).

\section{Simulations and Modelling Synthetic Lethality in Expression Data}

Synthetic lethality was modelled for effects on expression levels and whether these are detectable in known interacting and non-interacting genes in simulated data. These were conducted for expression data but the nature of these simulations would be relevant to how synthetic lethality would manifest in other factors, particularly DNA copy number variation and DNA methylation. These simulations were discussed at length in the previous meeting and showed that synthetic lethality was detectable with our approach in simple cases. While it was less effective, the methods were able to detect synthetic lethal genes in expression data with correlation structure (generated with the multi-variate normal distribution) and were distinguishable from correlated genes. Therefore the strongest (most significant) synthetic lethal genes are more likely to be true synthetic lethal partners and a high number of hits are expected from correlated genes and co-regulated pathways.

The power of the method to detect interactions depleted with increasing multiple tests, interactions, and cryptic (third party) interacting partners. Increased sample size counteracted these effects as expected. This led the idea that pathways would be more suitable as the focus of this project. Biological pathways led to fewer multiple tests, more relevant to understanding cancer biology, and are often drug targets in practice.

\section{Simulations over simple graph structures}
\subsection{Performance}
\subsection{Synthetic lethality across graph stuctures}
\subsection{Performance with inhibition links}
\subsection{Performance with 20,000 genes}

\section{Simulations over pathway-based graphs}

\section{Comparing methods}
\subsection{SLIPT and Chi-Squared}
\subsubsection{Correlated query genes}
\subsection{Correlation}
\subsection{Bimodality with BiSEp}
%\subsection{Linear models}

\iffalse
\section{Developing a linear model predictor of synthetic lethality}
\subsection{Linear models}
\subsection{Polynomial models}
\subsection{Conditioning}
\subsection{SLIPTv2}
\fi
