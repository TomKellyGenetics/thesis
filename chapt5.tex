\chapter{Simulation and Modeling of Synthetic Lethal Pathways}
\label{chap:simulation}

\section{Abstract}

Synthetic lethal interactions occur between genes when their combined loss is deleterious to a cell due to their shard essential function.   These interactions are the basis of emerging anti-cancer drug design strategies against specific loss of genes in cancers, such as CDH1 in breast cancers.   As discussed in previous meetings, we have developed a bioinformatics gene-expression analysis approach complementary to high-throughput RNAi screening in pre-clinical models.   This approach successfully scaled up computationally, adapted to a range of microarray and RNA-Seq datasets, and applied to DNA copy number and somatic mutation data.   However, there are difficulties replicating between datasets and RNAi candidates, such as the synthetic lethal screen for CDH1 partners in MCF10A breast cells (Telford et al. 2015).   It is unclear whether this stems from different sources of error between methodologies, tissue specificity of synthetic lethal interactions, or the approaches detect different classes of genetic interactions.   Therefore, we construct a statistical model of synthetic lethality to understand the sources of error in our approach and analyse simulated data to test how many synthetic lethal partner genes can be detected from gene expression data.   We have developed a model using multivariate normal distributions of expression levels to test the effects of correlation structure, number of genes, number of samples, and underlying number of true synthetic lethal partners on the error rate and distribution of chi-squared test statistics.   There is structural and functional complex in gene expression profiles of predicted CDH1 synthetic lethal partners.   We intend to develop correlation structure simulations to model biological pathways and comparing simulations with real gene expression data.   Analysis and prediction of gene networks, feasible or robust drug targets, and biomarkers of drug response are further directions for this project.

\section{Background}

Synthetic lethality (SL) is the death of a cell or organism with the combined loss of two non-essential genes.   This phenomenon was originally used to study genetic interactions and functional redundancy in models organisms (Boone et al. 2007).   While synthetic lethal experiments have been performed in Drosophila melanogaster (Dobzhansky 1946), Caenorhabditis elegans (Lehner et al. 2006), Escherichia coli (Butland et al. 2008), Schizosaccharomyces pombe (Roguev et al. 2007), and various mammalian cell lines (Kaelin 2005), the most extensive synthetic lethal screens have been performed with the synthetic gene array (SGA) technique in Saccharomyces cerevisiae (Boone et al. 2007; Costanzo et al. 2011; Tong et al. 2004).  

Originally defined by double mutants, a range of mechanisms for gene inactivation of synthetic lethal partners can induce cell death including RNA interference and drug treatment where it is sometimes called ‘induced essentiality’ or ‘non-oncogene addiction’ in cancer research (Fece de la Cruz et al. 2015).  Cellular viability is the main means to measure synthetic lethal effects experimentally because it is quantified and measured consistently (as shown in Figure 1), whereas qualitative measures of impaired organism viability are ambiguous and less relevant to yeast or cancer research.

The cancer genetics laboratory are currently working on developing a synthetic lethal approach to target the tumour suppressor gene CDH1 which has been found to cause predispose early-onset breast and stomach cancers in mutation carriers, including families of New Zealand M\={a}ori (Berx et al. 1995; Guilford et al. 1998).  These families are currently closely monitored and offered drastic preventative surgery.  If it were developed, a drug selective against CDH1 mutant tumours would serve not only as a chemopreventative alternative for these families but also benefit the wider community as a treatment for sporadic cases of CDH1 mutant cancer.  To augment experimental work on CDH1 with isogenic cell lines (Telford et al. 2015), a computational methodology is explored here to exploit public cancer genomic databases.

Microarray and massively-parallel sequencing technologies are driving a revolution in molecular biological research, particularly with regard to cancer where the premise of ‘genomic medicine’ is rapidly becoming feasible with the use of genomics to identify cancer genes, diagnose patients with actionable mutations, and use gene expression as a prognostic marker.  Genomic data could also be used to identify novel drug targets and synthetic lethal partners of known cancer genes in particular.  The Cancer Genome Atlas database (TCGA) and the overarching International Cancer Genome Consortium (ICGC) provide a valuable public cancer genome data resource because they support many different data types for the same samples, for many different cancer types, and for high sample sizes (Cancer Genome Atlas Research Network 2014; Cancer Genome Atlas Research Network et al. 2013; International Cancer Genome Consortium 2014).  They host data of patient clinical factors, gene expression, somatic mutation, DNA copy number, and DNA methylation which could all serve to predict synthetic lethality from frequency of mutually exclusive gene inactivation and its impact on patient survival.  A number of other databases are given in the Table 6 which may be used to explore gene function, drug target feasibility, or replicate analyses but TCGA and ICGC datasets will be the focus of this project.

Figure 1.  Impact of various negative (a) and positive (b, c) synthetic genetic interactions on growth viability fitness in yeast.  Adapted from Costanzo et al. (2011).   

There is a growing need for a robust approach to cost-effective prediction of candidate synthetic lethal interaction, particularly in cancer research.  Exploiting existing public genomic databases is an ideal way to utilise existing resources with suitable sample sizes, data types, and different limitations to those of laboratory experiments.  A number of computation approaches to synthetic lethality have been developed but many of these rely on data not available to cancer researchers, methods that are difficult to replicate, over-fitted to a particular dataset, having mixed validation results, or do not have a software tool accessible to the research community.  These methodologies are reviewed in detail in the accompanying literature review.  They will still be considered to develop an improved synthetic lethal interaction prediction tool (SLIPT).  

Therefore the data types considered to be predictive of synthetic lethality and the biological questions that could be addressed by them are summarised in the Figure 2.  
 
Figure 2.  Mindmap of synthetic lethal predictors and biological areas of relevance.  Underlined points have been investigated with preliminary data, italicised points are being considered in the immediate future of this project.

A bioinformatics approach has distinct limitations to experimental methods and would work well combined with genetic screen data and conventional molecular biology laboratory validation techniques to answer biological research questions.  Compared with an experimental screen, a bioinformatics approach has the benefits of reduced costs, with the potential for automation, scaling up, and replication of the same gene across populations and cell types.  Analysis of public genomic data accounts for real tumour variation showing detection with tumour heterogeneity and genomic instability.  Compared with a cell line or xenograft experimental model we are limited by difficulties in establishing validity of a novel method, lack of mechanism, or potential for testing drug activity in the same system.  However, computational methods may further miss useful therapeutic candidates from variable genetic background and be limited by the population sampled.  This research builds on previous work in an Honours project and similar approaches in the literature (Jerby-Arnon et al. 2014; Kelly 2013; Lu et al. 2015).

\section{Simulations and Modelling Synthetic Lethality in Expression Data}

Synthetic lethality was modelled for effects on expression levels and whether these are detectable in known interacting and non-interacting genes in simulated data. These were conducted for expression data but the nature of these simulations would be relevant to how synthetic lethality would manifest in other factors, particularly DNA copy number variation and DNA methylation. These simulations were discussed at length in the previous meeting and showed that synthetic lethality was detectable with our approach in simple cases. While it was less effective, the methods were able to detect synthetic lethal genes in expression data with correlation structure (generated with the multi-variate normal distribution) and were distinguishable from correlated genes. Therefore the strongest (most significant) synthetic lethal genes are more likely to be true synthetic lethal partners and a high number of hits are expected from correlated genes and co-regulated pathways.

The power of the method to detect interactions depleted with increasing multiple tests, interactions, and cryptic (third party) interacting partners. Increased sample size counteracted these effects as expected. This led the idea that pathways would be more suitable as the focus of this project. Biological pathways led to fewer multiple tests, more relevant to understanding cancer biology, and are often drug targets in practice.

\section{Developing a Synthetic Lethal detection methodology}

\subsection{Testing Multivariate Normal Simulation of Synthetic lethality}

We have developed a model of synthetic lethality in gene expression data based on sampling a Multivariate Normal distribution.  This enables simulation of statistically testing for synthetic lethal genes where the true and false positives are known, discovery of the expected test statistic distributions for different conditions, educated thresholds for public data analysis, and building a complex model with known correlation structure between genes.  Sampling a small number of genes from this model shows, in Figure 4, that synthetic lethality is detectable with in a simple model.

Figure 4.  Chi-Square (upper) and p-values (lower) distributions show that synthetic lethal partners (red) are distinguishable from correlated (blue) and other genes (black) in an example simulation of sampling 1000 samples and 100 genes, from a multivariate normal distribution with 1 (left), 2 (centre), and 3 (right) synthetic lethal partners respectively, showing that synthetic lethal genes become more difficult to detect if there are more true partners.

Figure 5.  Chi-Square (upper), FDR adjusted p-values (centre), and Holm adjusted p-values (lower) show that show that synthetic lethal partners (red) are distinguishable from correlated (blue) and other genes (black) are distinguishable replicated across 1000 replicate simulated sampling of 1000 samples and 100 genes, from a multivariate normal distribution with 1 (left), 2 (centre), and 3 (right) synthetic lethal partners respectively, showing synthetic lethal genes become more difficult to detect in with more true partners but adjusting p-values may be too stringent an approach to this.

Having shown that the Chi-Square test is capable of detecting synthetic lethality, Figure 5 shows that detecting synthetic lethality in a simple case is largely robust and reproducible across many replicates with synthetic lethal and correlated genes clearly having higher test statistic scores and lower adjusted p-values than the null distribution of non-synthetic lethal genes when there are only 1 or 2 synthetic lethal partners.  While it is promising that correlated genes and synthetic lethal partners could be distinguished from other genes in a simple case, there is also indication that true synthetic lethal partners (candidates as robust drug targets) and their correlated genes (or pathways) could be distinguished by test statistic.

However, such clear evidence of synthetic lethality by co-loss under-representation is rarely detected in public data analyses, indicating cryptic additional synthetic lethal genes compensating for the loss of both the query and putative synthetic partner.  Therefore higher-order synthetic lethal is potentially very common, difficult to detect, and confounding attempts to identify synthetic lethal pairs from gene expression data.  In Figure 5, more than 3 synthetic lethal partners will be difficult to identify directly with a Chi-Square test.  Although deeper understanding of the system could still enable use of the procedure to prioritise small numbers of candidate genes, estimate the number of underlying true synthetic partners, and identify the biological pathways interacting with a gene to focus complementary experimental approaches.

With higher number of true synthetic lethal genes there is no clear threshold for Chi-Square values (or associated p-values) to detect synthetic lethality and choosing any threshold is a trade-off between sensitivity (ensuring all true positives are detected) and specificity (reducing the number of false positives detected).  Receiver operating characteristic (ROC) curves, as shown in Figures 6 and 7, summarise this trade-off to show the statistical performance of a test where the true synthetic lethal genes are known in the simulated data.  Performance of a statistical test is measured as the area under the ROC (AUROC) curves, as shown in Figures 8 and 9, to compare performance across simulations for different parameters such as type of model, correlation structure, the total number of genes, sample size and number of true synthetic lethal genes.  A random predictor has an AUROC of 0.5, whereas an ideal predictor has an AUROC of 1.0, so intermediate values are expected.

\subsection{Receiver Operating Characteristic Curves}

Figure 6.   ROC curves showing statistical performance (by area under the curve) of a synthetic lethal simulation based on sampling a Binomial distribution, with 20,000 genes, averaged over 1000 replicates, sample size (1000, 2000, 5000, or 10,000) and number of synthetic lethal genes (up to 100) varies by panel and colour showing better performance with fewer synthetic lethal genes or higher sample size.    

Figure 7.   ROC curves of a synthetic lethal simulation based on sampling a Multivariate Normal distribution, with 20,000 genes, averaged over 1000 replicates, sample size and number of synthetic lethal genes varies by panel and colour showing better performance than a Binomial model and similar performance with correlation structure (upper panes).

Figure 8.  Comparison of Binomial (red) and Multivariate Normal models with (blue) and without (green) correlation structure by simulation with 1000 samples, 20,000 genes, sample size varied by pane, and number of synthetic lethal partners on the x axis where performance on the y axis is measured as the AUROC showing better performance in the Multivariate Normal model than the Binomial model and similar performance in the Multivariate Normal model with correlation structure added for all simulation parameters.  There was better performance with fewer synthetic lethal partners or higher sample size with both Multivariate Normal models.   

Figure 6 shows performance of an earlier model based on the Binomial distribution for gene function calls, based on similar a Normally distributed model of gene expression which called gene function from an arbitrary expression cut-off.  This model is shown for comparison with Multivariate model we have chosen to develop since the Multivariate model, as shown in Figure 7, has better performance, allows the inclusion of correlation structure expected in gene expression data for biological pathways, and could have variable gene function cut-offs.  The Binomial model defines the synthetic lethal condition in a way that, while ensuring at least on synthetic lethal partner is active in query deficient samples, disrupts the number of samples with functional synthetic lethal genes compared to other genes affecting the expected proportions in the Chi-Square test.

Figures 7 and 8, show that the Multivariate model which corrects this effect by specifying synthetic lethal genes differently performs better in simulations, even with correlation structure expected to disrupt the synthetic lethal detection.  There is indication in Figure 8 that correlation structure even improves the performance of simulations.  Although replicated across parameters, the difference in performance of simulations with correlated genes (with each synthetic lethal partner) is marginal and the number of correlated genes is still vastly outnumbered by the total number of genes (20,000 modelling a complete mammalian genome).  Simulations with fewer total genes may show the impact of correlated genes more clearly, which is biologically plausible since some co-regulated pathways do involve a substantial proportion of the genome.

As indicated, the models behave as expected when performing better when simulated with higher sample size and fewer true synthetic lethal genes.  As summarised in Figure 9, this behaviour occurs in simulation with all of the models discussed above.  The number of synthetic lethal partners impacts performance with a sigmoidal decay where­­ higher sample size (albeit approaching the limit of feasible genomic-scale projects) markedly delay decay of AUROC towards random 0.5.  Therefore a large sample size is crucial for bioinformatics synthetic lethal discovery.  Only a small number of synthetic lethal partners will be detectable with a gene-centric approach motivating pathway-centric approaches and accounting for pathway structure, which has shown be more reproducible between model organism experiments (Dixon et al. 2009).  However, whether potential false positives are more likely to be correlated genes or occur due to the sheer number of true negatives (and multiple tests) is unclear.  The impact of correlation structure on the simulated data is explored in detail below in Figures 10-12 and the results of these simulations repeated is shown in Figure 13.    Figure 9.  Summary of effect of sample size and number of synthetic lethal partners on performance of simulations for prediction of synthetic lethality by AUROC on continuous scale (left) and as a barplot (right) showing that sample size (by colour) and number of synthetic lethal partners (x axis) affects performance as expected in which was replicated across all 3 models discussed above.

\subsection{Simulated Expression Heatmaps}

In Figures 10-12 below, simulations are summarised with expected (Sigma) and generated (Correlation) structure of gene expression patterns in the top figures.  The following line shows how the expression and gene function calls have been distributed with correlation structure and ordering samples (columns) to ensure a synthetic lethal partner or query gene is active in each sample.

Figure 10.  Simulation for 1 SL partner (100 genes, 1000 samples)

Figure 11.  Simulation for 2 SL partners (100 genes, 1000 samples)

Figure 12.  Simulation for 3 SL partners (100 genes, 1000 samples)

As shown in the Figures 10-12, the correlation structure of the simulated gene expression data (upper right) largely reflects the expected sigma matrix (upper left) used to specify the variation in the Multivariate Normal distribution with some variation due to low sampling error.  The Sigma and correlation matrices show blocks of correlated genes with each synthetic lethal partner where there are 1, 2, or 3 synthetic lethal partners in Figures 10, 11, and 12 respectively.  In the gene expression heatmap (lower right) and associated discrete gene function calls based on a threshold of the 30\% quantile (lower left), the sample (column) ordering shows how samples were ordered so at least one synthetic lethal gene is active in all query deficient samples.  The row (gene) ordering is based on a Chi-Square test statistic value and odds-ratio sign (with negative genes at the top), apart from Query gene at the top (with positive odds-ratio).  The Chi-Square values are shown on the outer colour bar on a log scale and the inner colour bar annotates the known gene class in the simulation: query (blue), synthetic lethal (red), correlated (orange), and other (green).

With 1 synthetic lethal partner, in Figure 10, the relationship between synthetic lethal (and correlated genes with the Query gene is clear and detectable with Chi-Square test (as shown with the colour bars) as expected.  The relationship is clearer in the true synthetic lethal partner showing that it should be distinguishable from confounding correlated genes.  With multiple synthetic lethal genes, as shown in Figures 11 and 12, the true synthetic lethal partner is less related to the expression profile of the Query gene and the co-loss under-representation is more difficult to detect since the number of co-occuring loss of synthetic lethal genes expected (even in Query functional samples is low).  In these examples, the Chi-Square test still correctly identifies synthetic lethal genes with the highest test statistic, although with a less well defined cut-off and it may not be reproducible (as discussed above).  This is consistent with more synthetic lethal partners being able to recover function and ensure cell survival which is plausible given the evolutionary robustness of molecular networks, difficulty detecting individual gene pairs in gene expression data, and rates of recurrence or drug resistance in cancer patients.  Therefore we have to consider cryptic synthetic lethal genes compensating for Query and candidate synthetic lethal partners due to higher-order genetic redundancy, cancer genomic evolution and cellular heterogeneity.

\subsection{Replication Simulation Heatmap}

The declining performance in ROC curves with more synthetic lethal genes shows that the ability to robustly distinguish synthetic lethal genes from other genes (including their correlated genes) declines as the synthetic lethal genes do not consistently have a higher Chi-Square test statistic across replicate sampling simulations.  Although it is noted that increased sample size can compensate for this decline, increasing the number of expected co-loss events and sensitivity of the procedure.  The effect of total gene number, impact of correlation structure, and reproducibility of Chi-Square tests across replicate sampling simulations is explored below.

Figure 13 is composed of columns of side colour bars ordered by Chi-Square and odds-ratio sign (with Query in the corrected position at the bottom) as shown in Figures 10-12 with separate columns for repeated sampling with different parameters.  Figure 13 is an example of this visualisation of simulations for a small number of genes (100) and replicates (10 each for 1 to 10 synthetic lethal partners).  Even in this small simulation, we see many of the processes discussed above summarised: the effect of number of synthetic lethal genes on Chi-Square tests, power to detect synthetic lethal and other correlated genes, decaying reproducibility and variation across replicates, lack of a clear threshold, and importance of directional conditions (e.g., odds-ratio sign) to distinguish synthetic lethal and co-expressed genes.  This visualisation is an effective way to capture the simulation process and compare conditions which will be valuable for more complex correlation structure and comparison to public data Chi-Square distributions.
    
Figure 13.  Comparison of simulation across various parameters for sampling a Multivariate Normal model for 100 genes and 1000 samples with correlation structure with 10 replicates (columns) for each number of synthetic lethal partners (1 to 10 in the top colour bar) with genes sorted by chi-squared value (and odds-ratio negative at the top) this shows preferential sorting of synthetic lethal partners (red) and correlated (orange) genes near the top (on the left) for lower numbers of synthetic lethal partners which becomes less clear or consistent across replicates for higher numbers of synthetic lethal partners, reflected in less variation in chi-square values (shown in log-scale on the right) and lack of a clear prediction threshold, however positive odds-ratio genes show no preference except for the query gene associated itself as expected.

This framework may also be useful to compare different analyses of public data and infer the true number of synthetic lethal partners from the distribution of test statistic scores.  With an effective visualisation, we can further explore more complex correlation structures (as shown in the supplementary Figures S1 and S2).  This will be important to develop simulated data as similar to empirical data as possible, to test whether synthetic lethal and correlated genes are robustly detectable, and discover effective drug targets (which are repeatable across a cohort, tissues or species).  The impact of high-order synthetic lethality, genetic background and variation between replicates indicates that more care has to be taken interpreting experimental model systems and genomics analysis will be valuable to ensure candidate drug targets are suitable for clinical application.  We show below that this visualisation scales up and shows similar effects for number of synthetic lethal genes in more replicates (Figure 14), more total genes (Figure 15), and both (Figure 16).
    
Figure 14.  Comparison of simulation across various parameters for sampling a Multivariate Normal model for 100 genes and 1000 samples with correlation structure with 100 replicates (columns) for each number of synthetic lethal partners (1 to 10 in the top colour bar) with genes sorted by chi-squared value (and odds-ratio negative at the top) this shows preferential sorting of synthetic lethal partners (red) and correlated (orange) genes near the top (on the left) for lower numbers of synthetic lethal partners which becomes less clear or consistent across replicates for higher numbers of synthetic lethal partners, reflected in less variation in chi-square values (shown in log-scale on the right) and lack of a clear prediction threshold, however positive odds-ratio genes show no preference except for the query gene associated itself as expected.  
   
Figure 15.  Comparison of simulation across various parameters for sampling a Multivariate Normal model for 1000 genes and 1000 samples with correlation structure with 10 replicates (columns) for each number of synthetic lethal partners (1 to 10 in the top colour bar) with genes sorted by chi-squared value (and odds-ratio negative at the top) this shows preferential sorting of synthetic lethal partners (red) and correlated (orange) genes near the top (on the left) for lower numbers of synthetic lethal partners which becomes less clear or consistent across replicates for higher numbers of synthetic lethal partners, reflected in less variation in chi-square values (shown in log-scale on the right) and lack of a clear prediction threshold, however positive odds-ratio genes show no preference except for the query gene associated itself as expected.
    
Figure 16.  Comparison of simulation across various parameters for sampling a Multivariate Normal model for 1000 genes and 1000 samples with correlation structure with 100 replicates (columns) for each number of synthetic lethal partners (1 to 10 in the top colour bar) with genes sorted by chi-squared value (and odds-ratio negative at the top) this shows preferential sorting of synthetic lethal partners (red) and correlated (orange) genes near the top (on the left) for lower numbers of synthetic lethal partners which becomes less clear or consistent across replicates for higher numbers of synthetic lethal partners, reflected in less variation in chi-square values (shown in log-scale on the right) and lack of a clear prediction threshold, however positive odds-ratio genes show no preference except for the query gene associated itself as expected.

\section{Simulation of synthetic lethality in graph structures}

\subsection{Developing a multivariate normal expression from graph structures}

\subsection{Simulations over simple graph structures}
\subsubsection{Performance}
\subsubsection{Synthetic lethality across graph stuctures}
\subsubsection{Performance with inhibition links}
\subsubsection{Performance with 20,000 genes}

\subsection{Simulations over pathway-based graphs}

\subsection{Comparing methods}
\subsubsection{SLIPT and Chi-Squared}
\paragraph{Correlated query genes}
\subsubsection{Correlation}
\subsubsection{Linear models}

\subsection{Developing a linear model predictor of synthetic lethality}
\subsubsection{Linear models}
\subsubsection{Polynomial models}
\subsubsection{Conditioning}
\subsubsection{SLIPTv2}

\section{Significance}

Development of an effective synthetic lethal discovery tool for bioinformatics analysis has a wide range of applications in genetics research including functional genomics, medical and agricultural applications.   Of particular interest is a complementary approach to discovery of synthetic lethal drug targets for cancer therapy to aid the cancer research community which currently relies on cell line and mouse models for screening and validation experiments (Fece de la Cruz et al. 2015).  The potential for synthetic lethal drug design against cancer mutations including gene loss or overexpression could lead to a revolution in cancer therapy and chemoprevention with personalised treatment of cancers and high risk individuals.  Examples of the synthetic lethal strategy to cancer treatment have been shown to be clinically effective with many large-scale RNAi screens underway to discover more cancer gene function and drug targets for similar application.

However, there are limitations to both experimental screens and computational approaches, both known to be prone to false-positives.  Modelling and simulation of synthetic lethal discovery in genomic data has been explored to address these concerns and ensure the validity of candidate synthetic lethal interactions, particularly given the recent emergence of a number of conflicting synthetic lethal screening and prediction approaches.  Exploring synthetic lethality in simulated data will ensure the optimal performance of our prediction method with comparison to the distribution of test statistic distribution in empirical gene expression data, informed selection of thresholds for prediction, and estimated error rates.  The model of gene expression with known synthetic lethal genes is limited by the assumption that it represents the distribution of gene expression when it may not.  Having shown synthetic lethality is detectable in simple models and added correlation structure, the model still needs to be developed to better represent real data.  However, the behaviour of synthetic lethal genes and effects of parameters explored so far remains important to inform future model design and interpretation of empirical data analysis.
The synthetic lethal discovery strategy could be adapted to any form of gene inactivation or disruption such as such as changes to gene expression, regulation, epigenetics, DNA sequence, or copy number which could plausibly induce cell death due to SL interactions.  Further applications of synthetic lethal interactions such as analysis of gene networks, tissue specificity, evolutionary conservation, or drug target feasibility are possible with synthetic lethal candidates predicted with confidence on a large scale.

Network analysis enables properties of the network and it’s connectivity to be measured and compared across datasets (Barabási \& Oltvai 2004).  Tissue specificity is an important consideration, largely unexplored with synthetic lethal studies, since it has clinical importance to ensure targeted drug treatments are effective, predict adverse effects in other tissues, determine whether targeted treatments could be repurposed for other cancer types or diseases, and whether drug resistance mechanisms could emerge.  Comparison of tissues, populations, and species can all ensure that synthetic lethal predictions are robust, that experimental candidates are clinically relevant, and treatments designed to exploit them would be specific to the disease in large patient cohort (with known biomarkers).

Drug targets must be feasible to have effective anti-cancer interventions designed against them, which raises the need for targets with existing drugs in the clinic, trials, or feasible to development with structural analysis or screening.  Druggable targets could be selected by gene functions known to be amendable to drugs, with a structure amenable with development, with conserved specific sites without homology to other genes, or with known approval or developing drugs which could be repurposed from other disease applications.

\section{Future Directions}

Further development of the synthetic lethal model and simulation is needed to explore the parameters, ensure relevance to empirical data analysis, and understanding the implications of findings so far.  An example of more complex correlation structure is shown in supplementary Figures S1 and S2 with genes correlated to the Query genes (showing need for directional synthetic lethal condition) and correlated with other non-synthetic lethal genes (showing the predictions are robust to other correlation structure).  The impact of these modifications on model performance in a large number of genes or simulation replicates is yet to be seen or whether such correlation structure reflects the correlation structure of empirical data (as shown in Figure 3 with the row dendrogram for correlation distance between genes), known biological pathways, or known synthetic lethal interactions. Correlation between synthetic lethal genes could also be considered.

Comparing the findings of modelling and simulation with public gene expression analysis and experimental screen targets is still needed to identify putative synthetic lethal interactions.  This application will be tested with the example of CDH1 as a query gene in breast cancer for follow up to earlier results, relevance to ongoing research in the Cancer genetics Laboratory, and comparison to the experimental screen data of MCF10A cells by Telford et al. (2015).  While this methodology is intended to be widely applicable, particularly to other cancer genes and will be made available to the research community (manuscript and code release in preparation).

As outlined in the accompanying timeline document, there are several avenues for further research on synthetic lethality in breast cancer. The main alternative themes are network analysis with a focus on tissue specificity or drug feasibility with an emphasis on pharmacogenomics, biological pathways, and whether candidate targets could be inactivated by compounds with favourable pharmacokinetic properties. Either approach remains within the scope of the project, although each will require adoption of new computational tools, which is important topic for consideration in the meeting and changes to the project direction later in the year.

\section{Conclusion}

Synthetic lethal interactions are important for understanding gene function and development of targeted anti-cancer treatments.  Synthetic lethal discovery with experimental screening is error prone and limited by the model systems in which it is performed.  A bioinformatics tool to predict synthetic lethal interactions from genomics data would greatly benefit the cancer research community (and wider genetics research community).  Several such tools exist, including one we have developed, but they have conflicting design and results are often inconsistent with experimental screen data. Therefore, modelling and simulation of synthetic lethality in gene expression data is needed to ensure the statistical validity of predictions.  We have developed a model with correlation structure based on a Multivariate Normal distribution for which simulations detect synthetic lethality with high performance in simple cases and which has the potential to be developed to model complex correlation structure, biological pathways, or patterns observed in empirical gene expression data.  The modelling, public data analysis, and experimental screen data approaches will be combined to further examine the example of CDH1 in breast cancer.  Analysis of gene networks, tissue specificity, biological pathways, or drug targets remain options to explore tool development and implications for synthetic lethal cancer research in the future. 
