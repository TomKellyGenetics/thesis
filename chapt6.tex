\chapter{Discussion}
\label{chap:discussion}

This thesis combines analysis of gene expression data from \gls{TCGA} with experimental screening results \citep{Telford2015} to demonstrate synthetic lethal discovery for \textit{CDH1} in expression data generated by genomics technologies with comparisons to existing experimental candidates. Together these findings further elucidate the functions for \textit{CDH1} in the cell, functional redundancy in breast cancer, and potential targets against cancers with loss of \textit{CDH1} function. These candidate synthetic lethal genes were further investigated for relationships within synthetic lethal pathways, developing a network-based approach to comparing genes identified in genomics experiments and analyses in the process.

The synthetic lethal detection methodology, \gls{SLIPT}, that was applied to gene expression data throughout this thesis was evaluated with simulated data. A simulation procedure was developed to stringently generate gene expression data from known synthetic lethal partners in simulated data, including simple and complex correlation structures and modelling synthetic lethal genes within pathways. Together, these results demonstrate \gls{SLIPT} as a robust widely applicable gene expression analysis procedure (for which an R package has been released) for discovery of synthetic lethal partner genes. Performance of \gls{SLIPT} on simulated data also highlights the strengths of the procedure and future directions to improve upon it.

\section{Synthetic Lethality and \textit{CDH1} Biology}
\label{chapt6:implications}

The \textit{CDH1} gene was selected to identify synthetic lethal partners to demonstrate the novel \gls{SLIPT} methodology as an important tumour suppressor gene in cancers. These include sporadic breast and stomach cancers and the familial syndromes such as \gls{HDGC}. The analysis of synthetic lethal partners of \textit{CDH1} in breast and stomach cancers was also enabled by the availability of molecular data \citep{TCGA2012, TCGA2014GC} and a synthetic lethal screen conducted in MCF10A breast cells \citep{Chen2014, Telford2015}.

Synthetic lethal interactions are generally regarded to arise due to functional redundancy \citep{Boone2007, Kaelin2005, Fece2015} and as such the synthetic lethal partners of \textit{CDH1} indicates the wide-ranging biological functions that E-cadherin is involved in. The diverse synthetic lethal pathways identified supports the known pleiotropic nature of the \textit{CDH1} gene by detecting established functions of \textit{CDH1}, replicating candidates from an experimental screen \citep{Telford2015}, and identifying novel interactions with candidate genes and pathways for further investigation. The highly pleiotropic functions of E-cadherin as also consistent with \textit{CDH1} being a tumour suppressor gene for which epithelial cells are significantly disrupted at the molecular level and prone to becoming cancerous.

\subsection{Established Functions of \textit{CDH1}}
\label{chapt6:function}

The \textit{CDH1} has established functions in cell-cell communication and maintaining the cytoskeletion, specifically with cell-cell adhesion by forming tight junctions and the adherens complex. More recently, additional functions of \textit{CDH1} in the extracellular matrix and fibrin clotting have also been identified. Synthetic lethal interactions within biological pathways (i.e., partners in the same pathway as the query gene) are expected according to previous synthetic lethal experiments and  \citep{Kelley2005, Boone2007}. Synthetic lethal interactions identified in these pathways are consistent with these being functions of \textit{CDH1}, in addition to potentially actionable targets against cancers.

%extracellular / fibrin clotting

\subsection{The Molecular Role of \textit{CDH1} in Cancer}
\label{chapt6:cancer}

The involvement of \textit{CDH1} in the extracellular matrix is also important in cancers as it indicates a mechanism by which \textit{CDH1} loss may affect the tumour microenvironment, contributing to it's role as a tumour and invasion suppressor. Furthermore, perturbations in the extracellular matrix and tumour microenvironment present an potential means by which to specifically inhibit (cancerous) \textit{CDH1}-deficient cells in addition to those currently being considered. Few genes in extracellular pathways were detected in an experimental screen \citep{Telford2015} conducted in an isolated cell model \citep{Chen2014} but these are not expected to be detected in such as system. These may be further supported in further investigations with 3D cell culture, ``organoid'', or mouse xenograft cancer models.

In contrast, many of the pathways involved in cell signalling, including \glspl{GPCR}, were identified by \gls{SLIPT} in addition to the experimental screen \citep{Telford2015}. These support the previous results in cell line models, that these pathways are essential to growth of \textit{CDH1}-deficient cancers and present a potential vulnerability specific to these (cancerous) cells. Furthermore, the replication of synthetic lethality of \textit{CDH1} with cell signalling pathways in \gls{TCGA} data across cancer types and genetic backgrounds robustly supports these pathways being clinically applicable beyond the genetic background of the model system of \textit{CDH1}\textsuperscript{-/-} MCF10A cells \citep{Chen2014}. While the specific synthetic lethal genes were not as consistently detected between the \gls{SLIPT} analyses and \gls{siRNA} screen \citep{Telford2015}, the was sufficient to identify synthetic lethal pathways for further experimental investigation which are more likely to be replicated between genetic backgrounds \citep{Dixon2008}. Together these results demonstrate how \gls{SLIPT} can be integrated with an experimental screen to triage potential therapeutic targets  for further pre-clinical investigation.

The analysis of expression data with \gls{SLIPT} is also indicative of additional biological mechanisms of synthetic lethal in pathways beyond those identified in screening experiments \citep{Telford2015}. In particular, translation and regulatory pathways, involving 3$^\prime$ \glspl{UTR} and \gls{NMD}, were identified as candidate synthetic lethal pathways with \textit{CDH1} by \gls{SLIPT}. These present downstream target regulated by the putative synthetic lethal signalling pathways which cancer cells are dependent on for sustained protein expression \citep{Gao2015} to proliferate and evade host defense processes such as apoptosis and immune responses. 

%translation

%\section{Synthetic Lethal Discovery in Expression Data}
%\label{chapt6:SLIPT_applications}

\section{Significance}
\label{chapt6:significance}

\subsection{Synthetic Lethality in the Genomic Era}
\label{chapt6:significance_genetics}

Development of an effective synthetic lethal discovery tool for bioinformatics analysis has a wide range of applications in genetics research including functional genomics, medical and agricultural applications.  The \gls{SLIPT} approach demonstrated in this thesis is widely applicable to other genes and biological questions. In addition to further query of cancer genes, including other tissues, synthetic lethal gene functions are also of wider interest for their implications for genetic redundancy. Highly redundant genes and the genetically robust systems they give rise to are of further relevance to evolutionary, developmental, and systems biology to understand how these change over time and play a role in fundamental development of cell types, in addition to cancers.

Developmental genes in particular, are highly evolutionary conserved and subject to high rates of redundancy. These are often difficult to study with conventional functional genetics since individual knockouts of redundant genes do not necessarily have a mutant phenotype. Identifying genes with a common function is therefore also important to the study of developmental genes with unknown functions. Synthetic lethal discovery methods such as \gls{SLIPT} provide a genomic approach to further systematic characterisation of gene function including such highly redundant developmental genes.

Similarly, variants of unknown significance and modifier loci are a major concerns in human genetics, including ``monogenic'' and ``rare'' diseases. Many of these could potentially be difficult to characterise individually due to synthetic lethal interactions where additional loci contribute to the disease (or only compensate for some variants). As such systematic identification of synthetic lethal interactions also has applications in the study of such ``oligogenic'' diseases along with similar applications in the study of heritability for traits including agricultural genome-based selection.

Genetic redundancy is also a concern in pharmacology. Polypharmacology and network medicine are rationales to account for this by using drugs with multiple (known and specific) targets \citep{Hopkins2008, Barabasi2011}. Further characterisation of synthetic lethal genes will be valuable to the design of effective multi-target drugs or combination therapies in a range of therapeutic applications including molecular targeted therapies against cancer for which combination therapies are a popular solution for acquired resistance against individual targeted therapies. Characterisation of genetic interactions and combination therapies also has the potential to expand pharmacogenomics investigations to understanding the impact of genotypes at multiple loci leading to adverse effects in a subset of the population or accounting for why the rest of the population does not experience this adverse effects since their synthetic lethal partner genes do not share the same variants.

Furthermore, redundant functions and synthetic lethal interactions also present a means to expand upon the concept of the ``minimal'' genome by accounting for essential gene functions that are performed by redundant genes (or in combination with pleiotropic) genes rather than simply those that are perturbed by individual genes as an essential gene approach is likely an underestimate that does not account for synthetic lethal interactions. 

Therefore synthetic lethal interactions are a fundamentally important part of genetics and further understanding of them in a genomics context, facilitated by methods such as \gls{SLIPT}, shows great potential to contribute a deeper understanding of gene functions and their role in traits or diseases in the post-genomic era. Genes do not function in isolation and so understanding them in the context of the complexity of a cell and across genetic backgrounds (such as the data provided by \gls{TCGA}) is essential to further characterise their functions and ensure that further applications are reproducible beyond experimental systems.


\subsection{Clinical Interventions based on Synthetic Lethality}
\label{chapt6:significance_clinic}

Synthetic lethal discovery with \gls{SLIPT} is of particular interest in cancer research as a complementary approach to discovery of synthetic lethal drug targets. The cancer research community relies on cell line and mouse models for screening and validation experiments \citep{Fece2015} which would benefit from integration with gene expression analysis as demonstrated for \textit{CDH1} and the screen conducted by \citet{Telford2015}. The potential for synthetic lethal drug design against cancer mutations including gene loss or overexpression could lead to a revolution in cancer therapy and chemoprevention with personalised treatment of cancers and high risk individuals.  Examples of the synthetic lethal strategy \citep{Farmer2005, Bryant2005} for cancer treatment have been shown to be clinically effective with many large-scale RNAi screens recently conduced to aiming discover gene function and drug targets for similar application with other cancer genes, including cancers in other tissues.

%However, there are limitations to both experimental screens and computational approaches, both known to be prone to false-positives.  Modelling and simulation of synthetic lethal discovery in genomic data has been explored to address these concerns and ensure the validity of candidate synthetic lethal interactions, particularly given the recent emergence of a number of conflicting synthetic lethal screening and prediction approaches.  Exploring synthetic lethality in simulated data will ensure the optimal performance of our prediction method with comparison to the distribution of test statistic distribution in empirical gene expression data, informed selection of thresholds for prediction, and estimated error rates.  The model of gene expression with known synthetic lethal genes is limited by the assumption that it represents the distribution of gene expression when it may not.  Having shown synthetic lethality is detectable in simple models and added correlation structure, the model still needs to be developed to better represent real data.  However, the behaviour of synthetic lethal genes and effects of parameters explored so far remains important to inform future model design and interpretation of empirical data analysis.


%Network analysis enables properties of the network and it’s connectivity to be measured and compared across datasets \citep{Barabasi2004}.  Tissue specificity is an important consideration, largely unexplored with synthetic lethal studies, since it has clinical importance to ensure targeted drug treatments are effective, predict adverse effects in other tissues, determine whether targeted treatments could be repurposed for other cancer types or diseases, and whether drug resistance mechanisms could emerge.  Comparison of tissues, populations, and species can all ensure that synthetic lethal predictions are robust, that experimental candidates are clinically relevant, and treatments designed to exploit them would be specific to the disease in large patient cohort (with known biomarkers).

While \gls{SLIPT} analysis and \gls{RNAi} screens represent a significant step towards anti-cancer medicines, further validation is required to ensure that the synthetic lethal candidate genes and pathways identified for \textit{CDH1} in breast and stomach cancer are applicable against \textit{CDH1}-deficient cancers in the clinic.  Validation with \gls{RNAi} or pharmacological inhibitors is needed since both the \gls{SLIPT} analysis and \gls{siRNA} screen are susceptible to false positives. These candidates will need to be tested in pre-clinical models (cell lines and mouse xenografts) before proceeding to clinical trials. A therapeutic intervention will also require a targeted therapeutic against the synthetic lethal partner if one has not been developed against another disease (for which it van be re-purposed). Drug targets must be feasible to have effective anti-cancer interventions designed against them, which raises the need for targets with existing drugs in the clinic, trials, or feasible to development with structural analysis or screening.  Druggable targets could be selected by gene functions known to be amendable to drugs, with a structure amenable with development, with conserved specific sites without homology to other genes, or with known approval or developing drugs which could be repurposed from other disease applications.

Targeted therapeutics designed based on synthetic lethal interactions have potential to vastly expand the applications of ``precision medicine'' against molecular targets, particularly in cancer where many have been cancer genes have been identified. Synthetic lethality expands the range of cancer genes which can be (indirectly) targeted to include tumour suppressor genes with loss of function (such as \textit{CDH1}) and oncogenes with disrupted functions that are dysregulated or highly homologous to non-cancerous proto-oncogenes (such as \textit{MYC}, \textit{EGFR} or \textit{KRAS}). Applications against tumour suppressor genes is a particularly important application as these cannot be approached by careful dosing. Synthetic lethal drug design also has the added benefit of being highly specific against a particular genotype (such as \text{CDH1}\textsuperscript{-/-}) with the potential for target therapies with a wide therapeutic index and few adverse effects, in contrast to many current anti-cancer drug regimens \citep{Hopkins2008, Kaelin2009}. These properties are highly desirable for chemoprevention applications such as treatment against \textit{CDH1}-deficient early cancers in \gls{HDGC} patients before they are detectable during screening.

\iffalse
\section{Evaluating the Synthetic Lethality Prediction Tool}
\label{chapt6:slipt}


\subsection{Strength of the Synthetic Lethality Prediction Tool}
\label{chapt6:slipt_strengths}

\subsection{Limitations of the Synthetic Lethality Prediction Tool}
\label{chapt6:slipt_limitations}

\subsection{Comparisons to Alternative Methods}
\label{chapt6:slipt_compare}

\subsubsection{Combined with Experimental Screening}
\label{chapt6:slipt_compare_experimental}

\subsubsection{Differences to Computational Methods}
\label{chapt6:slipt_compare_computational}
\fi

\section{Future Directions}
\label{chapt6:future}
%%paper

\iffalse
Such a bioinformatically-informed synthetic lethal screening and validation strategy could be integrated into existing and future screens for synthetic lethality in cancer. 

Possible improvements to the SLIPT method include developing a Bayesian inference method or simulations and modelling to account for pathway structure among synthetic lethal genes. Another extension would be to test for higher order synthetic lethal interactions, where 3 or more genes perform a redundant function. 

%%committee
The synthetic lethal discovery strategy could be adapted to any form of gene inactivation or disruption such as such as changes to gene expression, regulation, epigenetics, DNA sequence, or copy number which could plausibly induce cell death due to SL interactions.  Further applications of synthetic lethal interactions such as analysis of gene networks, tissue specificity, evolutionary conservation, or drug target feasibility are possible with synthetic lethal candidates predicted with confidence on a large scale.


%%committtee
Further development of the synthetic lethal model and simulation is needed to explore the parameters, ensure relevance to empirical data analysis, and understanding the implications of findings so far.  An example of more complex correlation structure is shown in supplementary Figures S1 and S2 with genes correlated to the Query genes (showing need for directional synthetic lethal condition) and correlated with other non-synthetic lethal genes (showing the predictions are robust to other correlation structure).  The impact of these modifications on model performance in a large number of genes or simulation replicates is yet to be seen or whether such correlation structure reflects the correlation structure of empirical data (as shown in Figure 3 with the row dendrogram for correlation distance between genes), known biological pathways, or known synthetic lethal interactions. Correlation between synthetic lethal genes could also be considered.

Comparing the findings of modelling and simulation with public gene expression analysis and experimental screen targets is still needed to identify putative synthetic lethal interactions.  This application will be tested with the example of CDH1 as a query gene in breast cancer for follow up to earlier results, relevance to ongoing research in the Cancer genetics Laboratory, and comparison to the experimental screen data of MCF10A cells by Telford et al. (2015).  While this methodology is intended to be widely applicable, particularly to other cancer genes and will be made available to the research community (manuscript and code release in preparation).

There are several avenues for further research on synthetic lethality in breast cancer. The main alternative themes are network analysis with a focus on tissue specificity or drug feasibility with an emphasis on pharmacogenomics, biological pathways, and whether candidate targets could be inactivated by compounds with favourable pharmacokinetic properties. Either approach remains within the scope of the project, although each will require adoption of new computational tools, which is important topic for consideration in the meeting and changes to the project direction later in the year.
\fi

\subsection{Refinements Synthetic Lethality Prediction Methods}
\label{chapt6:future_slipt_method}

%lm
%bayes

\subsubsection{Wider Use of Synthetic Lethality Prediction}
\label{chapt6:future_slipt_data}

%package
%cnv
%mt
%methyl
%cancer
%gene
%tissue

\subsection{Validation of Synthetic Lethal Genes and Pathways}
\label{chapt6:future_cdh1}

\subsubsection{Pre-clinical and Clinical Testing}
\label{chapt6:future_clinic}

\subsection{Application to Further Genes and Pathways}
\label{chapt6:future_slipt}

\clearpage
\section{Conclusion}
\label{chap:conclusion}

Synthetic lethal interactions are important for understanding gene function and development of highly specific targeted anti-cancer treatments. Synthetic lethality potential expanding the repertoire of applications for precision cancer medicine to indirectly targeting loss of function in tumour suppressor genes.  Synthetic lethal discovery with experimental screening is error prone and limited by the model systems in which it is performed.  There is a need for bioinformatics tool to predict synthetic lethal interactions from gene expression data facilitates rapid identification of synthetic lethal candidates to augment functional genetic screens and cancer drug target triage. I present the original \acrfull{SLIPT} methodology as a statically robust procedure which performs this analysis.

The \gls{SLIPT} methodology has been demonstrated to identify biologically relevant genes and pathways. An comprehensive analysis of synthetic lethal partners of the \textit{CDH1} was performed in \gls{TCGA} breast cancer data \citep{TCGA2012} with many of these findings replicated in stomach cancer data \citep{TCGA2014GC}. These genes clustered into several distinct groups, with distinct biological functions and elevated expression in different clinical subtypes.  These analyses identified of synthetic lethal candidates in the $G_{\alpha i}$ signalling, cytoplasmic microfibres, and extracellular fibrin clotting pathways which were validated in an \gls{siRNA} screen performed by \citet{Telford2015} and consistent with the known cytoskeletal and cell signalling roles of E-cadherin. These findings support interventions against these pathways being applicable to specific cancer therapeutics beyond the pre-clinical cell line models in which they were validated. \gls{SLIPT} has also identified synthetic lethal partners in novel pathways for \textit{CDH1} including the regulation of immune signalling and translational elongation which extend the range of pleiotropic functions of \textit{CDH1} and present further biological mechanisms to investigate the malignancy and vulnerabilities of \textit{CDH1}-deficient cancers.

While some of these pathways are not expected to be detected in an isolated experimental cell line model, pathway structure may have accounted for this disparity. Thus synthetic lethal candidates detected by \gls{SLIPT} and \gls{siRNA} were compared within graph structures of the candidate synthetic lethal pathways. However, this did not generally account for differences between detection by these approaches. Neither synthetic lethal detection methodology preferentially detected genes of more importance or connectivity in pathway structures using established network metrics. Nor could it be generally established that \gls{SLIPT} gene candidates were upstream or downstream of \gls{siRNA} gene candidates in pathway structures across biological pathways.

Pathway graph structures were also included in investigations with simulated data to ascertain whether the \gls{SLIPT} procedure performed desirably in data with complex correlation structures derived based on biological pathways. A simulation procedure was developed based on a statistical model of synthetic lethality which generates multivariate normal data with known synthetic lethal partners and correlation structures. The \gls{SLIPT} methodology had high statistical performance, particularly when detecting few synthetic lethal genes, with large sample sizes, and a background of many non synthetic lethal genes to distinguish true partners from. This method had high specificity, performed better than Pearson's correlation or the $\chi^2$-test, and had had optimal performance across simulation parameter combinations for the thresholds used throughout this thesis. These findings were robust across correlation structures, including those derived from complex pathway structures containing strong positive and negative correlations between genes. 
Together these findings support the release of the \gls{SLIPT} software R packages and the application of the method to identify synthetic lethal genes within pathways and use candidate synthetic lethal genes to identify synthetic lethal pathways as demonstrated in this thesis.

Therefore, I present a widely applicable synthetic lethal procedure using gene expression data for wider use in genomics research, including the development of precision cancer medicine. This methodology is supported by the release of a software package in R, simulation results based on a statistical model of synthetic lethality, the demonstration of bioinformatics and network biology investigations into interactions with the \textit{CDH1} gene in breast and stomach cancers. 


\iffalse
\clearpage
\paragraph{Aims}

  \begin{itemize}
   \item To develop a statistical approach to detect synthetic lethal gene pairs in cancer from expression data

   \bigskip
   
   \item To apply this methodology to public cancer gene expression data against \textit{CDH1} and analyse pathway structure with comparisons to experimental screen data

   \bigskip
   
   \item To construct a statistical model of synthetic lethality in multivariate normal expression data
 
   \bigskip
   
   \item To develop a simulation pipeline of expression with pathway structure on a high-performance computing cluster 

   \bigskip
   
   \item To examine the statistical performance of the methodology with simulated expression including pathways and compare it to other approaches

   \bigskip
   
   \item To release the synthetic lethal detection methodology and pathway simulation procedure as R software packages
   
  \end{itemize}
  

\clearpage
  
 \paragraph{Summary}
 
   \begin{itemize}
   \item We have developed a Synthetic Lethal detection method that generates a high number of synthetic lethal candidates
   
   \bigskip
   
   \item Pathways in cell signalling, extracellular matrix, and cytoskeletal functions were supported with experimental candidates and the known functions of E-cadherin
   
   \bigskip
   
   \item Several candidate pathways were supported by mutation analysis and replicated across breast and stomach cancer
   
   \bigskip
   
   \item Translation and immune functions were uniquely detected by the computational approach which may be explained by differences between patient samples and cell line models
   
   \bigskip
   
   \item There remains the need to identify actionable genes within these pathways, relationships with experimental candidates, and how these pathways may affect viability when lost
  \end{itemize}
  
    \begin{itemize}
   \item Synthetic Lethal genes were explored within a graph structures for key pathways identified previously 
   
   \bigskip
   
   \item In some cases these graph structures appeared to have relationships between synthetic lethal genes  
   
   \bigskip
   
   \item However, no existing network metrics of importance and connectivity with the networks were elevated significantly for Synthetic Lethal genes
   
   \bigskip
   
   \item Nor was there significant evidence of upstream and downstream relationships between SLIPT and siRNA Candidates in a shortest path permutation analysis
  \end{itemize}
  
      \begin{itemize}
      \item We have designed a straight-forward rational query-based synthetic lethal detection method with the example of application to \textit{CDH1} in cancer gene expression
      
      \bigskip
      
      \item We have developed a simulation pipeline to generate continuous gene expression with pathway structure including a procedure to simulate synthetic lethality 
      
      \bigskip
      
      \item Our simulation procedure is robust across pathway structures and has desirable performance compared to other statistical techniques 
      \end{itemize}
 \fi