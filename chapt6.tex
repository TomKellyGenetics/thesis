\chapter{Discussion}
\label{chap:discussion}

This thesis combines analysis of \gls{gene expression} data from \gls{TCGA} with experimental screening results \citep{Telford2015} to demonstrate \gls{synthetic lethal} discovery for partners of \textit{CDH1}. % in \glslink{gene expression}{expression} data generated by \glspl{genomic} technologies with comparisons to existing experimental candidates.
Together these findings further elucidate the functions of \textit{CDH1} in the cell, \gls{functional redundancy} in cancer, and represent potential \glslink{targeted therapy}{therapeutic targets} against loss of \textit{CDH1} function. These candidate \gls{synthetic lethal} genes were further investigated for relationships within \gls{synthetic lethal} pathways, and in the process a network-based approach to compare genes identified in \glspl{genomic} experiments was developed.

The \gls{synthetic lethal} detection methodology, \gls{SLIPT}, was applied to \gls{gene expression} data throughout this thesis and was evaluated with simulated data. A procedure was developed to stringently generate \gls{gene expression} data from known \gls{synthetic lethal} partners in simulated data. These simulations included simple and complex correlation structures, and modelling \gls{synthetic lethal} genes within pathways. Together, these results demonstrate \gls{SLIPT} as a robust widely applicable \gls{gene expression} analysis procedure (for which an R package has been made available) for discovery of \gls{synthetic lethal} partner genes. Performance of \gls{SLIPT} on simulated data also highlights the strengths of the procedure and future directions to improve upon it.

\section{Synthetic Lethality and \textit{CDH1} Biology}
\label{chapt6:implications}

The \textit{CDH1} \gls{tumour suppressor} gene was the focus of identifying \gls{synthetic lethal} partners to demonstrate the novel \gls{SLIPT} methodology. This gene is important in \glslink{sporadic}{sporadic} breast and stomach cancers, in addition to \gls{familial} syndromes, such as \gls{HDGC}. The analysis of \gls{synthetic lethal} partners of \textit{CDH1} in breast and stomach cancers was enabled by the availability of molecular data \citep{TCGA2012, TCGA2014GC} and a \gls{synthetic lethal} screen conducted in MCF10A breast cells \citep{Chen2014, Telford2015}.

Synthetic lethal interactions arise due to \gls{functional redundancy} \citep{Boone2007, Kaelin2005, Fece2015} and as such the \gls{synthetic lethal} partners of \textit{CDH1} indicates the wide-ranging biological functions that \gls{E-cadherin} is involved in. The diverse \gls{synthetic lethal} pathways identified supports the known pleiotropic nature of the \textit{CDH1} gene by detecting established functions of \textit{CDH1}, replicating candidates from an experimental screen \citep{Telford2015}, and identifying novel interactions with candidate genes and pathways for further investigation. The highly pleiotropic functions of \gls{E-cadherin} was also consistent with \textit{CDH1} being a \gls{tumour suppressor} gene. % for which epithelial cells are significantly disrupted at the molecular level and prone to becoming cancerous.

\subsection{Established Functions of \textit{CDH1}}
\label{chapt6:function}

The \textit{CDH1} has established functions in cell-cell communication and maintaining the cytoskeleton, specifically with cell-cell adhesion by forming tight junctions and the adherens complex. More recently, additional functions of \textit{CDH1} in the extracellular matrix and fibrin clotting have also been identified. \Gls{synthetic lethal} interactions within biological pathways (i.e., partners in the same pathway as the query gene) are expected according to previous \gls{synthetic lethal} experiments \citep{Kelley2005, Boone2007}. \Gls{synthetic lethal} interactions identified in these pathways are consistent with these being functions of \textit{CDH1}, in addition to potentially actionable targets against cancers.

%extracellular / fibrin clotting

\subsection{The Molecular Role of \textit{CDH1} in Cancer}
\label{chapt6:cancer}

The involvement of \textit{CDH1} in the extracellular matrix is important in cancers as it indicates a mechanism by which \textit{CDH1} loss may affect the tumour microenvironment, contributing to its role as a tumour and invasion suppressor. Furthermore, perturbations in the extracellular matrix and tumour microenvironment present a means by which to specifically inhibit (cancerous) \textit{CDH1}-deficient cells, in addition to those currently being considered. 
%Few genes in extracellular pathways were detected in an experimental screen \citep{Telford2015} conducted in an isolated cell model \citep{Chen2014} but these are not expected to be detected in such as system. 
These may be further supported in further investigations with 3D cell culture, ``organoid'', or mouse xenograft cancer models.

In contrast, many of the pathways involved in cell signalling, including \glspl{GPCR}, were identified by \gls{SLIPT} in addition to the experimental screen \citep{Telford2015}. These support the previous results in cell line models, that these pathways are \gls{essential} to the growth of \textit{CDH1}-deficient cancers and present a potential vulnerability specific to these (cancerous) cells. Furthermore, the replication of \glspl{synthetic lethal} of \textit{CDH1} with cell signalling pathways in \gls{TCGA} data across cancer types and genetic backgrounds robustly supports these pathways being clinically applicable beyond the genetic background of the model system of \textit{CDH1}\textsuperscript{-/-} MCF10A cells \citep{Chen2014}. While the specific \gls{synthetic lethal} genes were not as consistently detected between the \gls{SLIPT} analyses and \gls{siRNA} screen \citep{Telford2015}, they were sufficient to identify \gls{synthetic lethal} pathways for further experimental investigation, which are more likely to be replicated between genetic backgrounds \citep{Dixon2008}. Together these results demonstrate how \gls{SLIPT} can be integrated with an experimental screen to triage potential therapeutic targets  for further pre-clinical investigation.

The analysis of \glslink{gene expression}{expression} data with \gls{SLIPT} is also indicative of additional biological mechanisms of \glspl{synthetic lethal} in pathways beyond those identified in screening experiments \citep{Telford2015}. In particular, translation and regulatory pathways, involving 3$^\prime$ \glspl{UTR} and \gls{NMD}, were identified as candidate \gls{synthetic lethal} pathways with \textit{CDH1} by \gls{SLIPT}. These pathways represent downstream targets regulated by the putative \gls{synthetic lethal} signalling pathways which cancer cells are dependent on for sustained protein expression to proliferate and evade host defense processes such as apoptosis and immune responses \citep{Gao2015} . 

%translation

%\section{Synthetic Lethal Discovery in Expression Data}
%\label{chapt6:SLIPT_applications}

\section{Significance}
\label{chapt6:significance}

\subsection{Synthetic Lethality in the Genomic Era}
\label{chapt6:significance_genetics}

Development of an effective \gls{synthetic lethal} discovery tool for \glspl{bioinformatics} analysis has a wide range of applications in genetics research including functional \glspl{genomic}, medical and agricultural applications.  The \gls{SLIPT} approach demonstrated in this thesis is widely applicable to other genes and biological questions. In addition to further query of cancer genes, including other tissues, \gls{synthetic lethal} gene functions are also of wider interest for their implications for \glslink{functional redundancy}{genetic redundancy}. Highly redundant genes, and the genetically robust systems they give rise to, are of further relevance to evolutionary, developmental, and systems biology to understand how these change over time and play a role in fundamental development of cell types, in addition to cancers \citep{Nowak1997, Boone2007, Tischler2008}.

Developmental genes in particular, are highly evolutionary conserved and subject to high rates of \glslink{functional redundancy}{redundancy} \citep{Nowak1997, Kockel1997,Fromental-Ramain1996}. These are often difficult to study with conventional functional genetics since individual knockouts of redundant genes do not necessarily have a \gls{mutant} phenotype. Identifying genes with a common function is therefore also important to the study of developmental genes with unknown functions. \Gls{synthetic lethal} discovery methods such as \gls{SLIPT} provide a \gls{genomic} approach to further systematic characterisation of gene function including such highly redundant developmental genes.

Similarly, variants of unknown significance and modifier loci are a major concerns in human genetics, including ``monogenic'' and ``rare'' diseases. Many of these could potentially be difficult to characterise individually due to \gls{synthetic lethal} interactions where additional loci contribute to the disease (or only compensate for some variants). As such systematic identification of \gls{synthetic lethal} interactions also has applications in the study of such ``oligogenic'' diseases along with similar applications in the study of heritability for traits including agricultural \glslink{genome}{genomic} selection.

\glslink{functional redundancy}{Genetic redundancy} is also a concern in pharmacology. Polypharmacology and network medicine are rationales to account for this by using drugs with multiple (known and specific) targets \citep{Hopkins2008, Barabasi2011}. Further characterisation of \gls{synthetic lethal} genes will be valuable to the design of effective multi-target drugs or combination therapies in a range of therapeutic applications including molecular targeted therapies against cancer for which combination therapies are a popular solution for acquired resistance against individual targeted therapies. Characterisation of genetic interactions and combination therapies also has the potential to expand pharmacogenomic investigations. These may elucidate the impact of genotypes at multiple loci, which lead to adverse effects in a subset of the population due to variants in \gls{synthetic lethal} genes.

Furthermore, redundant functions and \gls{synthetic lethal} interactions also present a means to expand upon the concept of the ``minimal'' \gls{genome} \citep{Hutchison2016}. It is important to account for \gls{essential} gene functions that are performed by redundant genes (or in combination with \glslink{pleiotropy}{pleiotropic} genes), rather than simply those that are perturbed by individual genes. An \gls{essential} gene approach is likely to produce an underestimate that does not account for \gls{synthetic lethal} interactions. 

\Gls{synthetic lethal} interactions are fundamentally important throughout genetics. Further understanding of them in a \gls{genomic} context, facilitated by methods such as \gls{SLIPT}, would contribute towards deeper understanding of gene functions and their role in traits or diseases in the post-genomic era. Genes do not function in isolation and understanding them in the context of the complexity of a cell and across genetic backgrounds is \gls{essential} to further characterise their functions and ensure that findings can be validated or applied beyond experimental systems.


\subsection{Clinical Interventions based on Synthetic Lethality}
\label{chapt6:significance_clinic}

Synthetic lethal discovery with \gls{SLIPT} is of particular interest in cancer research as a complementary approach to discovery of \gls{synthetic lethal} drug targets. The cancer research community relies on cell line and mouse models for screening and validation experiments \citep{Fece2015} which would benefit from integration with \gls{gene expression} analysis as demonstrated for \textit{CDH1} and the screen conducted by \citet{Telford2015}. \Gls{synthetic lethal} drug design against cancer \glspl{mutation}, including gene loss or over-expression, could lead to a revolution in cancer \glslink{treatment}{therapy} and \gls{chemoprevention}. Such \glslink{targeted therapy}{therapeutics} would enable personalised treatment for cancer patients and high risk individuals.  Examples of the \gls{synthetic lethal} strategy \citep{Farmer2005, Bryant2005} for cancer treatment have been shown to be clinically effective \cite{McLachlan2016}. Many large-scale \gls{RNAi} screens have been conducted recently, aiming to discover gene function and drug targets for similar application with other cancer genes, including cancers in other tissues \citep{Fece2015}.

%However, there are limitations to both experimental screens and computational approaches, both known to be prone to false-positives.  Modelling and simulation of \gls{synthetic lethal} discovery in \gls{genomic} data has been explored to address these concerns and ensure the validity of candidate \gls{synthetic lethal} interactions, particularly given the recent emergence of a number of conflicting \gls{synthetic lethal} screening and prediction approaches.  Exploring \glspl{synthetic lethal} in simulated data will ensure the optimal performance of our prediction method with comparison to the distribution of test statistic distribution in empirical \gls{gene expression} data, informed selection of thresholds for prediction, and estimated error rates.  The model of \gls{gene expression} with known \gls{synthetic lethal} genes is limited by the assumption that it represents the distribution of \gls{gene expression} when it may not.  Having shown \glspl{synthetic lethal} is detectable in simple models and added correlation structure, the model still needs to be developed to better represent real data.  However, the behaviour of \gls{synthetic lethal} genes and effects of parameters explored so far remains important to inform future model design and interpretation of empirical data analysis.


%Network analysis enables properties of the network and it’s connectivity to be measured and compared across datasets \citep{Barabasi2004}.  Tissue specificity is an important consideration, largely unexplored with \gls{synthetic lethal} studies, since it has clinical importance to ensure targeted drug \glspl{treatment} are effective, predict adverse effects in other tissues, determine whether targeted \glspl{treatment} could be repurposed for other cancer types or diseases, and whether drug resistance mechanisms could emerge.  Comparison of tissues, populations, and species can all ensure that \gls{synthetic lethal} predictions are robust, that experimental candidates are clinically relevant, and \glspl{treatment} designed to exploit them would be specific to the disease in large patient cohort (with known biomarkers).

While \gls{SLIPT} analysis and \gls{RNAi} screens represent a significant step towards anti-cancer medicines, further validation is required to ensure that the \gls{synthetic lethal} candidate genes and pathways identified for \textit{CDH1} in breast and stomach cancer are applicable against \textit{CDH1}-deficient cancers in the clinic.  Validation with \gls{RNAi} or pharmacological inhibitors is needed since false positives may occur in \gls{SLIPT} analysis or \gls{siRNA} screens. These candidates will need to be tested in pre-clinical models (cell lines and mouse xenografts) before proceeding to clinical trials. A therapeutic intervention will also require a \glspl{targeted therapy}  to develop developed or repurposed against the \gls{synthetic lethal} partner. 
%Drug targets must be feasible to have effective anti-cancer interventions designed against them, which raises the need for targets with existing drugs in the clinic, trials, or feasible to development with structural analysis or screening. 
Drug targets could be triaged from \gls{synthetic lethal} genes by functions known to be amenable to drugs or structure with conserved specific sites that are not homologous to other genes, or those with existing drugs approved in trial for other applications. Both structure-aided drug design and compound screening are viable ways to target \gls{synthetic lethal} partners. % accompany genetic screens and computational analysis with pharmacological investigations.

\glslink{targeted therapy}{Targeted therapeutics} designed based on \gls{synthetic lethal} interactions could expand the applications of ``precision medicine'' against molecular targets. %, particularly in cancer where many have been cancer genes have been identified.
\Glspl{synthetic lethal} expands the range of cancer genes which can be (indirectly) targeted to include \gls{tumour suppressor} genes with loss of function, such as \textit{CDH1}. \Glspl{oncogene} with disrupted functions that are over-expressed or highly homologous to non-cancerous proto-oncogenes, such as \textit{MYC}, \textit{EGFR} or \textit{KRAS}, may also be targeted by \glspl{synthetic lethal}. Applications against \gls{tumour suppressor} genes is particularly important, as these cannot be approached by careful dosing. \Gls{synthetic lethal} drug design has the benefit of being highly specific against a particular genotype (such as \text{CDH1}\textsuperscript{-/-}) with the potential for \glslink{targeted therapy}{targeted therapies} with a wide therapeutic index and few adverse effects, in contrast to many current anti-cancer drug regimens \citep{Hopkins2008, Kaelin2009}. These properties are highly desirable for \gls{chemoprevention} applications, such as treatment against \textit{CDH1}-deficient in \gls{HDGC} patients \citep{Guilford2010}, as an alternative to monitoring or surgery.

\iffalse
\section{Evaluating the Synthetic Lethality Prediction Tool}
\label{chapt6:slipt}


\subsection{Strength of the Synthetic Lethality Prediction Tool}
\label{chapt6:slipt_strengths}

\subsection{Limitations of the Synthetic Lethality Prediction Tool}
\label{chapt6:slipt_limitations}

\subsection{Comparisons to Alternative Methods}
\label{chapt6:slipt_compare}

\subsubsection{Combined with Experimental Screening}
\label{chapt6:slipt_compare_experimental}

\subsubsection{Differences to Computational Methods}
\label{chapt6:slipt_compare_computational}
\fi

\section{Future Directions}
\label{chapt6:future}

While further validation and pre-clinical testing is required to translate the findings for \textit{CDH1} to cancer therapy or prevention, there are also further avenues for research into the detection of \glspl{synthetic lethal} in \gls{gene expression} and other \glspl{genomic} data. The \gls{SLIPT} methodology is amenable to wider application against a range of genes for which loss of function is deleterious, including other cancer genes in breast cancer or other tissues. \Gls{synthetic lethal} interactions are functionally informative, particularly for mode-of-action of known drug targets, and are also relevant for identifying functions of newly characterised genes in \glspl{genomic} studies and designing specific interventions against cells with loss of function in cancer and other diseases. Thus \gls{synthetic lethal} detection using \gls{SLIPT} in \glslink{gene expression}{expression} data could be further used for many other genes, including others relevant to human health and disease.

These investigations do not need to be limited to \glslink{gene expression}{expression} data. While \glslink{gene expression}{expression} as a measure of gene function has been the focus of this thesis, other \glspl{genomic} data could be used for a similar purpose for \gls{SLIPT} analysis. These include \acrshort{DNA} copy number, \acrshort{DNA} methylation, histone activation, \gls{mutation} status, protein abundance, and protein activation state. For some applications or genes, these molecular profiles may be more informative of gene function and \gls{synthetic lethal} relationships. However, \glslink{gene expression}{expression} was the focus of the investigations thus far as a widely accepted measure of gene function which has widely available \glspl{genomic} data.  \gls{SLIPT} is compatible with each of these data types (if the thresholds are selected appropriately) and may perform better for some applications with these molecular profiles or a weighted combination of these. As demonstrated, \gls{SLIPT} is also suitable for future investigations with pathway \glspl{metagene} and other summary data as well.

It may also be possible to improve the performance of \gls{SLIPT} with refinements to the statistical or computational approach. This thesis has focused on rational query-based approach which computes relatively quickly, even in R \citep{R_core}, and is relatively intuitive to interpret. These computations are compatible with parallel computing and the computational resources may be further reduced by using a different computing language. The \texttt{slipt} R package has been documented and released open-source (as described in Section~\ref{methods:r_packages}) to facilitate further development, wider adoption, or comparison with other scientific software for similar purposes. 

Alternative methods may be also improve on the statistical performance of \gls{SLIPT}. In particular, the sensitivity was generally as issue with higher numbers of \gls{synthetic lethal} partners in simulated data. While approaches using continuous data such as Pearson correlation and linear regression did not perform as well as \gls{SLIPT}, they could be improved. A least squares regression approach in particular, enables multiple measures of relationships such as the coefficients of the fitted curve and significance of the fit (computed from the residuals). A linear modelling approach using regression is also amenable to refinement such as extending from fitting a linear relationship to a polynomial or logistic regression. Another benefit to fitting linear models is that these would enable the conditioning of known \gls{synthetic lethal} partners to identify subtle signatures of further interacting partners.

This approach could also be applied iteratively on the strongest candidates from previous \gls{synthetic lethal} analyses in further rounds of prediction conditioned upon them. Similarly, \gls{synthetic lethal} prediction could also be approached with a Bayesian framework which is also amenable to Bayesian priors on known or previously predicted \gls{synthetic lethal} partners. Either of these approaches has the potential to improve upon the \gls{synthetic lethal} predictions which have been demonstrated as possible and biologically relevant by \gls{SLIPT}. 