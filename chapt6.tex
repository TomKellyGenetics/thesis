\chapter{Discussion}
\label{chap:discussion}

\paragraph{Aims}

  \begin{itemize}
   \item To develop a statistical approach to detect synthetic lethal gene pairs in cancer from expression data

   \bigskip
   
   \item To apply this methodology to public cancer gene expression data against \textit{CDH1} and analyse pathway structure with comparisons to experimental screen data

   \bigskip
   
   \item To construct a statistical model of synthetic lethality in multivariate normal expression data
 
   \bigskip
   
   \item To develop a simulation pipeline of expression with pathway structure on a high-performance computing cluster 

   \bigskip
   
   \item To examine the statistical performance of the methodology with simulated expression including pathways and compare it to other approaches

   \bigskip
   
   \item To release the synthetic lethal detection methodology and pathway simulation procedure as R software packages
   
  \end{itemize}
  
 \paragraph{Summary}
 
   \begin{itemize}
   \item We have developed a Synthetic Lethal detection method that generates a high number of synthetic lethal candidates
   
   \bigskip
   
   \item Pathways in cell signalling, extracellular matrix, and cytoskeletal functions were supported with experimental candidates and the known functions of E-cadherin
   
   \bigskip
   
   \item Several candidate pathways were supported by mutation analysis and replicated across breast and stomach cancer
   
   \bigskip
   
   \item Translation and immune functions were uniquely detected by the computational approach which may be explained by differences between patient samples and cell line models
   
   \bigskip
   
   \item There remains the need to identify actionable genes within these pathways, relationships with experimental candidates, and how these pathways may affect viability when lost
  \end{itemize}
  
    \begin{itemize}
   \item Synthetic Lethal genes were explored within a graph structures for key pathways identified previously 
   
   \bigskip
   
   \item In some cases these graph structures appeared to have relationships between synthetic lethal genes  
   
   \bigskip
   
   \item However, no existing network metrics of importance and connectivity with the networks were elevated significantly for Synthetic Lethal genes
   
   \bigskip
   
   \item Nor was there significant evidence of upstream and downstream relationships between SLIPT and siRNA Candidates in a shortest path permutation analysis
  \end{itemize}
  
      \begin{itemize}
      \item We have designed a straight-forward rational query-based synthetic lethal detection method with the example of application to \textit{CDH1} in cancer gene expression
      
      \bigskip
      
      \item We have developed a simulation pipeline to generate continuous gene expression with pathway structure including a procedure to simulate synthetic lethality 
      
      \bigskip
      
      \item Our simulation procedure is robust across pathway structures and has desirable performance compared to other statistical techniques 
      \end{itemize}

%committe meeting (sim 2015)

\section{Significance}

Development of an effective synthetic lethal discovery tool for bioinformatics analysis has a wide range of applications in genetics research including functional genomics, medical and agricultural applications.   Of particular interest is a complementary approach to discovery of synthetic lethal drug targets for cancer therapy to aid the cancer research community which currently relies on cell line and mouse models for screening and validation experiments (Fece de la Cruz et al. 2015).  The potential for synthetic lethal drug design against cancer mutations including gene loss or overexpression could lead to a revolution in cancer therapy and chemoprevention with personalised treatment of cancers and high risk individuals.  Examples of the synthetic lethal strategy to cancer treatment have been shown to be clinically effective with many large-scale RNAi screens underway to discover more cancer gene function and drug targets for similar application.

However, there are limitations to both experimental screens and computational approaches, both known to be prone to false-positives.  Modelling and simulation of synthetic lethal discovery in genomic data has been explored to address these concerns and ensure the validity of candidate synthetic lethal interactions, particularly given the recent emergence of a number of conflicting synthetic lethal screening and prediction approaches.  Exploring synthetic lethality in simulated data will ensure the optimal performance of our prediction method with comparison to the distribution of test statistic distribution in empirical gene expression data, informed selection of thresholds for prediction, and estimated error rates.  The model of gene expression with known synthetic lethal genes is limited by the assumption that it represents the distribution of gene expression when it may not.  Having shown synthetic lethality is detectable in simple models and added correlation structure, the model still needs to be developed to better represent real data.  However, the behaviour of synthetic lethal genes and effects of parameters explored so far remains important to inform future model design and interpretation of empirical data analysis.
The synthetic lethal discovery strategy could be adapted to any form of gene inactivation or disruption such as such as changes to gene expression, regulation, epigenetics, DNA sequence, or copy number which could plausibly induce cell death due to SL interactions.  Further applications of synthetic lethal interactions such as analysis of gene networks, tissue specificity, evolutionary conservation, or drug target feasibility are possible with synthetic lethal candidates predicted with confidence on a large scale.

Network analysis enables properties of the network and it’s connectivity to be measured and compared across datasets (Barabási \& Oltvai 2004).  Tissue specificity is an important consideration, largely unexplored with synthetic lethal studies, since it has clinical importance to ensure targeted drug treatments are effective, predict adverse effects in other tissues, determine whether targeted treatments could be repurposed for other cancer types or diseases, and whether drug resistance mechanisms could emerge.  Comparison of tissues, populations, and species can all ensure that synthetic lethal predictions are robust, that experimental candidates are clinically relevant, and treatments designed to exploit them would be specific to the disease in large patient cohort (with known biomarkers).

Drug targets must be feasible to have effective anti-cancer interventions designed against them, which raises the need for targets with existing drugs in the clinic, trials, or feasible to development with structural analysis or screening.  Druggable targets could be selected by gene functions known to be amendable to drugs, with a structure amenable with development, with conserved specific sites without homology to other genes, or with known approval or developing drugs which could be repurposed from other disease applications.

\section{Future Directions}

%%paper

Such a bioinformatically-informed synthetic lethal screening and validation strategy could be integrated into existing and future screens for synthetic lethality in cancer. 

Possible improvements to the SLIPT method include developing a Bayesian inference method or simulations and modelling to account for pathway structure among synthetic lethal genes. Another extension would be to test for higher order synthetic lethal interactions, where 3 or more genes perform a redundant function. 



%%committtee
Further development of the synthetic lethal model and simulation is needed to explore the parameters, ensure relevance to empirical data analysis, and understanding the implications of findings so far.  An example of more complex correlation structure is shown in supplementary Figures S1 and S2 with genes correlated to the Query genes (showing need for directional synthetic lethal condition) and correlated with other non-synthetic lethal genes (showing the predictions are robust to other correlation structure).  The impact of these modifications on model performance in a large number of genes or simulation replicates is yet to be seen or whether such correlation structure reflects the correlation structure of empirical data (as shown in Figure 3 with the row dendrogram for correlation distance between genes), known biological pathways, or known synthetic lethal interactions. Correlation between synthetic lethal genes could also be considered.

Comparing the findings of modelling and simulation with public gene expression analysis and experimental screen targets is still needed to identify putative synthetic lethal interactions.  This application will be tested with the example of CDH1 as a query gene in breast cancer for follow up to earlier results, relevance to ongoing research in the Cancer genetics Laboratory, and comparison to the experimental screen data of MCF10A cells by Telford et al. (2015).  While this methodology is intended to be widely applicable, particularly to other cancer genes and will be made available to the research community (manuscript and code release in preparation).

There are several avenues for further research on synthetic lethality in breast cancer. The main alternative themes are network analysis with a focus on tissue specificity or drug feasibility with an emphasis on pharmacogenomics, biological pathways, and whether candidate targets could be inactivated by compounds with favourable pharmacokinetic properties. Either approach remains within the scope of the project, although each will require adoption of new computational tools, which is important topic for consideration in the meeting and changes to the project direction later in the year.

\section{Conclusion}

Synthetic lethal interactions are important for understanding gene function and development of targeted anti-cancer treatments.  Synthetic lethal discovery with experimental screening is error prone and limited by the model systems in which it is performed.  A bioinformatics tool to predict synthetic lethal interactions from genomics data would greatly benefit the cancer research community (and wider genetics research community).  Several such tools exist, including one we have developed, but they have conflicting design and results are often inconsistent with experimental screen data. Therefore, modelling and simulation of synthetic lethality in gene expression data is needed to ensure the statistical validity of predictions.  We have developed a model with correlation structure based on a Multivariate Normal distribution for which simulations detect synthetic lethality with high performance in simple cases and which has the potential to be developed to model complex correlation structure, biological pathways, or patterns observed in empirical gene expression data.  The modelling, public data analysis, and experimental screen data approaches will be combined to further examine the example of CDH1 in breast cancer.  Analysis of gene networks, tissue specificity, biological pathways, or drug targets remain options to explore tool development and implications for synthetic lethal cancer research in the future. 
