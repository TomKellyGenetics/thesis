%Background
\textbf{\textcolor{red}{Introduction}}

Synthetic lethal genetic interactions are re-emerging as an important concept in the post-genomics era due to their potential for use in precision medicine against cancers. Synthetic lethal drug design exploits the functional redundancy of genes disrupted in cancers (including tumour suppressors) to develop specific treatments against them. \textit{CDH1}, which encodes \gls{E-cadherin}, is a tumour supressor gene with loss of function in breast and stomach cancers. Experimental screens have identified candidate synthetic lethal interactions with \textit{CDH1}, which can be further supported with bioinformatics analysis. Furthermore, gene expression data enables investigation of synthetic lethal pathways and \textcolor{red}{the structure of synthetic lethal genes}. 

%Methods \ Approach
\textbf{\textcolor{red}{Methods}}

\textcolor{red}{A computational methodology}, the Synthetic Lethal Prediction Tool (\acrshort{SLIPT}) \textcolor{red}{was} developed to detect synthetic lethal interactions in gene expression data. The application of this methodology is demonstrated on interactions with \textit{CDH1} in breast and stomach cancer data from The Cancer Genome Atlas (\acrshort{TCGA}) project. Synthetic lethal genes and pathways were further investigated with unsupervised clustering, gene set over-representation analysis, metagenes, and permutation resampling. In particular, analyses focused on comparing \acrshort{SLIPT} gene candidates to an experimental \acrfull{siRNA} screen \citep{Telford2015}. Network analysis methods were applied to the most supported pathways to test for pathway structure between synthetic lethal candidates. Simulation and modelling was used to assess the statistical performance of \acrshort{SLIPT}, including simulated data with correlation structures \textcolor{red}{from} graph structures.

%Results \ Findings
\textbf{\textcolor{red}{Results}}

%chapt 3
Many candidate synthetic lethal partners of \textit{CDH1} were detected in \acrshort{TCGA} breast cancer. These genes clustered into several distinct groups, with distinct biological functions and elevated expression in different clinical subtypes. While the number of genes detected by both \acrshort{SLIPT} and \acrshort{siRNA} was not significant, these contained significantly enriched pathways. In particular, $G_{\alpha i}$ signalling, cytoplasmic microfibres, and extracellular fibrin clotting were robustly supported by both approaches, which is consistent with the known cytoskeletal and cell signalling roles of \gls{E-cadherin}. Many of these pathways were replicated in stomach cancer data. The pathways supported only by \acrshort{SLIPT} included regulation of immune signalling and translation, which were not expected to be detected in an isogenic cell line model but are still candidates for further investigation. 

%chapt 4
Synthetic lethal candidates detected by \acrshort{SLIPT} and \acrshort{siRNA} were compared within the graph structures of the candidate synthetic lethal pathways.
%These genes did not differ with respect to network metrics of importance or connectivity in the pathway.
%There was little support, across pathways, that \acrshort{SLIPT} gene candidates were consistently upstream or downstream of \acrshort{siRNA} gene candidates within pathways. 
%However, \acrshort{SLIPT}
\textcolor{red}{SLIPT genes} had lower centrality and were \textcolor{red}{consistently} upstream of \acrshort{siRNA} candidates, specifically in the $G_{\alpha i}$ signalling pathway.
%chapt 5

A statistical model of synthetic lethality was used to simulate gene expression data with known synthetic lethal partners for a gene. The \acrshort{SLIPT} methodology had high statistical performance when detecting few synthetic lethal partners, which diminished with more synthetic lethal partners or lower sample size. The \acrshort{SLIPT} methodology performed better than Pearson correlation or the $\chi^2$-test. In particular, it performed well with high specificity for datasets containing thousands of genes, or genes positively correlated with the query gene (as expected to occur in gene expression data). \acrshort{SLIPT} was robust across correlation structures, including those derived from complex pathway structures, and often distinguished synthetic lethal genes from those positively or negatively correlated with them. %Therefore \acrshort{SLIPT} is appropriate to identify synthetic lethal genes within pathways and use candidate synthetic lethal genes (and their correlates) to identify synthetic lethal pathways.

%Conclusions
%\clearpage
%\textbf{Summary}

Thus this thesis has developed, evaluated, and applied a bioinformatics approach for the discovery of synthetic lethal genes from gene expression data. This approach has been demonstrated to detect biologically informative and clinically relevant candidate synthetic lethal partners for \textit{CDH1} in breast and stomach cancers. %These investigations have also involved the development of network analysis and simulation procedures which are relevant in a range of genomic data analysis contexts.

\iffalse
\clearpage

\begin{center}
 \textbf{Research Contributions During Candidature}
\end{center}

\textbf{Publications}

\begin{small} \begin{flushleft} Kelly, S. T. and Spencer, H. G. (2017) Population-Genetics Models of Sex-Limited \Gls{genomic} Imprinting. \textit{Theoretical Population Biology} \textbf{115}:35-44 \doi{10.1016/j.tpb.2017.03.004} \end{flushleft} \end{small}

\textbf{Manuscripts Submitted}

%Several publications have been prepared during the candidature of this thesis, including some findings related to the thesis topic on which I elaborate in more depth here. Please see the original article for the results which have been accepted for publication by peer-review:
\begin{small}
Kelly, S. T., Single, A. B., Telford, B. J., Beetham, H. G, Godwin, T. D., Chen, A., Black, M., A., and Guilford, P. J. (2017) Towards HDGC chemoprevention: vulnerabilities in \gls{E-cadherin}-negative cells identified by \gls{genomic} interrogation of isogenic cell lines and whole tumors.  Submitted to \textit{Cancer Prevention Research}.

Kelly, S. T., Chen, A., Guilford, P. J., and Black, M. A. (2017) Synthetic lethal interaction prediction of target pathways in \gls{E-cadherin} deficient breast cancers. Submitted to \textit{BMC \Glspl{genomic}}.
\end{small}
\fi

\iffalse
\textbf{Community Blog Posts}

Black, M. A., Kelly, S. T., and Cadzow, M.
Posted on the \textit{Software Carpentry} website 2016 July 4\textsuperscript{th}: 
``Software Carpentry workshop at the University of Otago, New Zealand''
\url{https://software-carpentry.org/blog/2016/07/otago-workshop.html}

Kelly, S. T., Black, M., A., Bae, S., Hayek, W., and Pawlik, A. Posted on the \textit{Software Carpentry} website 2016 September 28\textsuperscript{th}:  ``Software Carpentry Workshop Attendance: a New Zealand Perspective``
\url{https://software-carpentry.org/blog/2016/09/attendance-nz.html}
\fi

\iffalse
\textbf{Conference Presentations}

\begin{small}
Consortium of Biological Sciences 2017 (Kobe) December TBC

eResearch 2017 (Queenstown) February 20\textsuperscript{th}-22\textsuperscript{nd}
%``Detecting Synthetic Lethality from Cancer Gene Expression: A PhD project on genetic interactions with CDH1 inactivation in \gls{TCGA} data''

Research Bazaar 2016 (Dunedin) February 2\textsuperscript{nd}-4\textsuperscript{th}

eResearch 2016 (Queenstown) February 9\textsuperscript{th}-11\textsuperscript{th}
%``Sifting the Needles in the Haystack: Permutation Resampling Biological Pathways in Cancer \Gls{genomic} Interaction Data'' (Supported by REANNZ)

Genetics Otago Symposium 2016 (Dunedin) March 7\textsuperscript{th}-8\textsuperscript{th} 
%``A Bioinformatics approach to Genetic Interactions: Synthetic Lethal Pathways with \gls{E-cadherin} in Breast Cancer \Glspl{genomic} Data''

DunDead: Zombie Science and Culture Festival 2014 (Dunedin) %Ignite Speaker
August 16\textsuperscript{th}-17\textsuperscript{th}
%``Hidden in Plain Sight - The Genetics of Zombies''

eResearch 2014 (Hamilton) %Ignite Speaker
June 30\textsuperscript{th}-July 2\textsuperscript{nd}
%``Bioinformatic analysis of synthetic lethal genetic interactions in breast cancer''
(Supported by Google)

\end{small}

\textbf{Poster Presentations}

\begin{small}
Next Generation Sequencing Asia 2016 (Singapore) October 11\textsuperscript{th}-12\textsuperscript{th}
%``Bioinformatic Investigations of Synthetic Lethal Interactions with \gls{E-cadherin} in Breast Cancer''
(Supported by the University of Otago Division of Health Sciences; Maurice and Phyllis Paykel Trust)

Research Bazaar 2015 (Melbourne) February 16\textsuperscript{th}-18\textsuperscript{th}
%``My digital research toolkit''
(Supported by the New Zealand eScience Infrastructure)

Otago School of Medical Sciences Postgraduate Symposium 2015 (Dunedin) April 28\textsuperscript{th}-29\textsuperscript{th}

QMB Cancer Drugs Satellite 2014 (Queenstown) August 24\textsuperscript{th}-25\textsuperscript{th}
%“Bioinformatics Prioritisation of Synthetic Lethal Targets for Drug Activity Against E-Cadherin Deficient Cancers”

\end{small}

\clearpage
\textbf{Seminar Presentations}

\begin{small}

University of Otago Department of Biochemistry 2017 (Dunedin) November TBC

T\={o}hoku University 2016 (Sendai) November 11\textsuperscript{th}

Okinawa Institute of Science and Technology 2016 (Onna) November 1\textsuperscript{st}

S\={o}kendai Graduate University 2016 (Hayama) October 25\textsuperscript{th}

T\={o}ky\={o} University Institute of Medical Science 2016 (Shirokanedai) October 24\textsuperscript{th} 

National Institute of Genetics 2016 (Mishima) October 21\textsuperscript{st}

RIKEN Division of \Gls{genomic} Technologies 2016 (Yokohama) October 20\textsuperscript{th}
%``Analysis of Synthetic Lethal Pathways in Breast Cancer: A PhD project on genetic interactions with CDH1 inactivation in \gls{TCGA} data''

\end{small}
\fi



\iffalse
Presentations within University of Otago Departments
Department of Biochemistry 2017 PhD Seminar
Upcoming Seminar (to schedule later upon completion of thesis)
``Synthetic Lethal Interactions in Cancer \Glspl{genomic} Data: Prediction and Simulation''

Department of Biochemistry 2016 Colloquia July 22nd
``Analysis of Synthetic Lethal Pathways in Breast Cancer: A PhD project on genetic interactions with CDH1 inactivation in \gls{TCGA} data''
``Demonstration of R data visualization packages: vioplotx and heatmap.2x''

Otago Mozilla Study Group 2016 June 9th
``Taking Advantage of Online Communities: Getting the most out of StackExchange and StackOverflow Q&A Sites''

Genetics Otago Journal Club 2016 March 23rd 
Day T, Read AF (2016) Does High-Dose Antimicrobial Chemotherapy Prevent the Evolution of Resistance? PLoS Comp Biol 12(1): e1004689

Otago Mozilla Study Group 2015 October 22nd
``Data is Beautiful: An Introduction to Data Visualization in R''

Department of Biochemistry 2015 Colloquia August 27th
``A Demonstration of Statistical Analysis in R: Antibiotic Resistance in Biofilms'' (collaborator data)

Otago Mozilla Study Group 2015 June 9th
``Literature Tools: Bibliometrics and Reference Management''

Biochemistry Department Journal Club 2015 April 25th
``Quelling the Flames: Order Matters in Cancer''
Ortmann CA, Kent DG, Nangalia J, et al. (2015) Effect of mutation order on myeloproliferative neoplasms. N Engl J Med 372, 601-612.

Genetics Otago Journal Club 2014 September 3rd
Peris et al. (2014) Population structure and reticulate evolution of Saccharomyces eubayanus and its lager-brewing hybrids. Molecular Ecology 23(8):2031-2045

\fi



\iffalse
\textbf{Software Packages}

Software packages in the R language have been released. Please refer to the appropriate GitHub repository for more information (including documentation, vignettes, and installation instructions), on the following account: \href{https://github.com/TomKellyGenetics}{https://github.com/TomKellyGenetics}

\begin{small}

\begin{itemize}
 \item \href{https://github.com/TomKellyGenetics/slipt}{\texttt{slipt}} to accompany the synthetic lethal publication above and release SLIPT (Synthetic Lethal Interaction Prediction Tool)
 \item \href{https://github.com/TomKellyGenetics/vioplotx}{\texttt{vioplotx}} to provide enhanced violin plots
 \item \href{https://github.com/TomKellyGenetics/heatmap.2x}{\texttt{heatmap.2x}} to provide annotated heatmaps
 \item \href{https://github.com/TomKellyGenetics/igraph.extensions}{\texttt{igraph.extensions}} metapackage for the packages for iGraph objects:
 \begin{itemize}
  \item \href{https://github.com/TomKellyGenetics/plot.igraph}{\texttt{plot.igraph}} to provide plotting for directed graphs
  \item \href{https://github.com/TomKellyGenetics/info.centrality}{\texttt{info.centrality}} to compute network analysis metrics
  \item \href{https://github.com/TomKellyGenetics/pathway.structure.permutation}{\texttt{pathway\-.structure\-.permutation}} for resampling within pathways
  \item \href{https://github.com/TomKellyGenetics/graphsim}{\texttt{graphsim}} to simulate expression (\texttt{mvtnorm}) from pathway structures
 \end{itemize}
\end{itemize}

\end{small}

The \href{https://github.com/TomKellyGenetics/slipt-app}{\texttt{slipt-app}} GitHub repository also hosts an application for \acrshort{SLIPT} developed in the R \texttt{shiny} environment as part of a related project. There is a digital copy of this thesis, including high resolution full-colour figures, hosted at:

\href{https://github.com/TomKellyGenetics/thesis/blob/master/thesis.pdf}{https://github.com/TomKellyGenetics/thesis/blob/master/thesis.pdf}
\fi

\iffalse
\href{https://github.com/TomKellyGenetics/slipt}{\texttt{slipt}} is the Synthetic Lethal interaction Prediction Tool, released to accompany the synthetic lethal publication above. \href{https://github.com/TomKellyGenetics/slipt-app}{\texttt{slipt-app}} contains an application developed in the R \texttt{shiny} environment as part of a related project.

Several plotting functions were customised for the Figures in this thesis (and the above publications), notably \href{https://github.com/TomKellyGenetics/heatmap.2x}{\texttt{heatmap.2x}} and \href{https://github.com/TomKellyGenetics/vioplotx}{\texttt{vioplotx}} have been prepared largely for use during this project but are also documented and available to other R users. These are enhancements to the \gls{CRAN} \texttt{gplots} and \texttt{vioplot} packages respectively and are intended be user-friendly for those familiar with \texttt{heatmap.2} or \texttt{vioplot} and \texttt{boxplot} (base R) functions. These are backwards compatible with these functions, taking similar imputs as demonstrated in the appropriate vignettes.

The use of iGraph (the R \texttt{igraph} package) operations of graph-network structure in the analysis and simulations of pathways involved several original or customised functions to manipuate or plot \texttt{igraph} objects and adjacency matrices. These can be install separately from their respective repositories of with the metapackage: \href{https://github.com/TomKellyGenetics/igraph.extensions}{\texttt{igraph.extensions}}. \href{https://github.com/TomKellyGenetics/plot.igraph}{\texttt{plot.igraph}} enables plotting graph networks with customised inhibitor arrow and node colours.\href{https://github.com/TomKellyGenetics/info.centrality}{\texttt{info.centrality}} enables the calculation of additional node and network centrality metrics not available in the \texttt{igraph} package. \href{https://github.com/TomKellyGenetics/pathway.structure.permutation}{\texttt{pathway\-.structure\-.permutation}} enables testing of related states or node groups in a network by directionality of shortest paths. The \href{https://github.com/TomKellyGenetics/graphsim}{\texttt{graphsim}} package has been set up to simulate a multi-variate normal gene expression dataset with \texttt{mvtnorm} while deriving the correlation structure, $\Sigma$, from a graph structure. Note that these require various packages for graph theory, statistics and matrix operations and these will be installed as dependencies.
\fi

