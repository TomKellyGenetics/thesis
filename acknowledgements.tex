I thank my supervisors A/Prof. Mik Black and Prof. Parry Guilford for their support and guidance throughout my postgraduate studies. It has been a great experience, I look forward to seeing what your research groups produce in the future, may this not be the end for us.

I am also thankful for the guidance and mentorship of Prof. Hamish Spencer for career advice throughout my studies and time in his research group.

I am also grateful to the past and current members of these research groups, and my peers at the laboratory benches and computers across campus. The peer support, camaraderie, and guidance of newer students has been an incredible part of my time at Otago and has made my thesis studies not just easier but possible at all. The postgraduate community is very special here in Dunedin and I have truly made some lifelong friends. You are talented researchers and amazing people. May we meet again some day. Where-ever you may end up, its small world and there's always time to catch up. I'd be delighted to host visitors while working abroad.

I cannot thank my friends, flatmates, family, and diligent proofreaders enough for their patience and support during such as massive, challenging, and (I'm sure you've heard too often) stressful undertaking during both my PhD and the study leading up to it. There are too many of you to name everyone here without leaving someone out, so thank you all for everything you've done, both the good times and the tough. Thank you for at least pretending to understand complex math brought up at the wrong moment. Thank you for checking my writing or slides, even when sprung on you last minute. Thank you for your time when what I really needed was a chat, a walk, a drink with ``the guys'', or a moment to think clearly.

I thank the various organisations that supported this research project:

\begin{small}
 
\begin{itemize}
\item
This thesis was supported by the Postgraduate Tassell Scholarship in Cancer Research, a University of Otago Doctoral Scholarship.

\item
The New Zealand eScience Infrastucture (NeSI) provided access to the Intel Pan high-performance computing cluster, support, and training to use it effectively. Various aspects of this thesis would not have been possible without access to such an incredible national resource. 

\item
The Health Research Council (HRC) of New Zealand provided funding for experimental research in the Cancer Genetics Laboratory. Some aspects of this project would not have been possible without access to the data and findings funded by this grant.

\item
The Allan Wilson Centre and Otago School of Biomedical Sciences provided funding for summer research placements which was a valuable opportunity to gain experience and training used in this thesis project.

\end{itemize}

\end{small}


I thank the following organisations for support towards presenting findings in this thesis at conference and seminars:

\begin{small}

\begin{itemize}

\item
Google (eResearch 2014, Hamilton)

\item
NeSI (Software Carpentry training and Research Bazaar 2015, Melbourne)

\item
REANNZ, NZGL, and NeSI (eResearch 2016, Queenstown)

\item
Otago Division of Health Sciences, Oxford Global, and Maurice and Phyllis Paykel Trust (NGS Asia 2016, Singapore)

\item
RIKEN Division of Genomic Technologies and the Okinawa Institute of Science and Technology (seminar visits in Japan)

\end{itemize}

\end{small}

Thanks most of all to my fiance\'{e}, Dr Yui Kawagishi, you've been an inspiration. Thank you for your support and encouragement, every day, even from afar: it has always made a difference. It’s been incredible to see you flourish in your career and I look forward to joining you again soon. May the next chapter of our adventures involve a bit less Skype across timezones.

\begin{CJK}{UTF8}{goth}どうもありがとう由ちゃん。頑張った!もうすぐ行きます。また来月!\end{CJK}