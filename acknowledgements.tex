I thank my supervisors A/Prof. Mik Black and Prof. Parry Guilford for their support and guidance throughout this my postgraduate studies. It has been a great experience, I look forward to seeing what your research groups produce in the future, may this not be the end.

I am also thankful for the guidance and mentorship of Prof. Hamish Spencer for career advice throughout my studies and time in his research group.

I am also grateful to the past and current members of these research groups, and my peers at the laboratory benches and computers across campus. The peer support, comraderie, and guidance to newer students has been an incredible part of my time at Otago and has made my thesis studies not just easier but possible at all. The postgraduate community is very special here and have truly made some lifelong friends from all over the world, you are talented researchers and amazing people. May we meet again some day. Whereever you may end up, there's always time to catch up and I'd be delighted to host some visits while working abroad.

I cannot thank my friends, flatmates and family enough for their patience and support during such as massive, challenging, and (I'm sure you've heard too many times) stressful undertaking during both my PhD and the study leading up to it. There are too many of you to name everyone here without leaving someone out, so thank you all for everything you've done, both the good times and the tough. Thank you for pretending to understand when I try to discuss complex math at the wrong moment. Thank you for checking my writing or slides, even if I should have given you more time. Thank for your time when all I really needed was a chat over a walk or a pint and a moment to think clearly.

I must also thank various organisations supported this research project:

\begin{itemize}
\item
This thesis was supported by the Postgraduate Tassell Scholarship in Cancer Research, a University of Otago Doctoral Scholarship.

\item
The New Zealand eScience Infrastucture (NeSI) provided access to the Intel Pan high-performance computing cluster, support, and training to use it effectively. Various aspects of this thesis would not have been possible without access to such a resource. 

\item
The Health Research Council (HRC) of New Zealand provided funding for experimental research in the Cancer Genetics Laboratory. Again some aspects of this project would not have been possible without access to the data and findings funded by this grant.

\item
The Allan Wilson Centre and Otago School of Biomedical Sciences provided funding for summer research placements which was a valuable opportunity to gain experience and training used in this thesis project.

\end{itemize}

I thank the following organisations for support towards presenting findings in this thesis at conference and seminars:

\begin{itemize}

\item
Google (towards eResearch 2014 conference, Hamilton)

\item
NeSI (towards Software Carpentry training and Research Bazaar 2015, Melbourne)

\item
REANNZ, NZGL, and NeSI (towards eResearch 2016 conference, Queenstown)

\item
Otago Division of Health Sciences, Department of Biochemistry, Oxford Global, and Maurice and Phyllis Paykel Trust (towards NGS Asia 2016, Singapore)

\item
RIKEN Division of Genomics Technologies and the Okinawa Institue of Science and Technology (for hosting seminars in Japan)

\end{itemize}

Thanks most of all to my fiance\'{e}, Dr Yui Kawagishi, you've been an inspiration. Thank you for your support, help, and encouragement, even from afar times, it has always made a difference. It's been incredible to see you flourish in your career and I look forward to joining you again soon. May the next chapter of our adventures involve a bit less Skype across timezones.
