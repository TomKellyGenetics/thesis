\chapter{Methods Developed During Thesis}
\label{chap:methods_dev}
%\section{Overview/meta-text}

In this Chapter, I will outline the rationale and development of various methods used throughout this thesis to examine synthetic lethality in gene expression data, graph structures, models and simulations. First by describing the \acrfull{SLIPT}, a bioinformatics approach to triage of synthetic lethal candidate genes. This is considered one of the main research outputs of the thesis, which is supported by comparisons to an experimental screen from a related project and performance on simulated data. These supporting data will be covered in further Chapters but preliminary data to support the use and design of \gls{SLIPT} are provided alongside description of the method. This includes the construction of a statistical model of synthetic lethality in (continuous multivariate Gaussian) gene expression data, which enables testing \gls{SLIPT} upon simulated data with known synthetic lethal partners. Another key component of the simulation pipeline used later is the generation of simulated data from a known graph structure or simulated biological pathway. The development of this simulation procedure and other statistical treatment of graph and network structures will also be covered. Various R packages have been developed to support this project, most notably the \texttt{slipt} package to implement the \gls{SLIPT} methodology. The additional R packages for handling graph structures, simulations, and custom plotting features will also be described as research outputs of this thesis, methods applied throughout, and contributions to the open-source software community that made this project feasible.

\section{A Synthetic Lethal Detection Methodology} \label{methods:SLIPT}
%\subsection{Rationale and Design of Test}
%\subsection{Synthetic Lethal Detection Method}

The \gls{SLIPT} methodology identifies gene expression patterns consistent with synthetic lethal interactions between a query gene and a panel of candidate interacting partners. Gene expression is called low, medium, or high by separating samples into tertiles (3-quantiles) for each gene. Genes with insufficient expression across all samples were excluded by requiring that the first tertile of raw counts is above zero. Then a $\chi^2$ test is performed between the query gene and each candidate partner, with the p-values for the $\chi^2$ test being corrected for multiple testing using false discovery rate (FDR) error control to reduce false positives for large candidate gene panels \citep{fdr1995}. Significance was called only if FDR adjusted p-values were below the threshold $p < 0.05$. A synthetic lethal interaction is predicted  (as shown in Figure~\ref{fig:SLIPT_Method}) when (i) the $\chi^2$ test is significant; (ii) observed low-query, low-candidate samples are less frequent than expected; and (iii) observed low-query, high-candidate and high-query, low-candidate samples are more frequent than expected.
%The query and candidate genes are swapped to replicate the directional condition. %redundant
%Where synthetic lethality is scored SL-Q if it is predicted in query-low samples and SL-C if it is predicted in candidate-low samples (as shown in Figure~\ref{fig:SLIPT_Method}). Synthetic lethality is only reported in this text if it meets both of these conditions and a significant p-value where it is scored SL-2. %too detailed

\begin{figure*}[!b]
\begin{mdframed}
\begin{center}
  \resizebox{0.8 \textwidth}{!}{
    \fbox{\input{SL_Method.pdf_tex}}
   }
   \end{center}
   \caption[Framework for synthetic lethal prediction]{\small \textbf{Framework for synthetic lethal prediction.} \gls{SLIPT} was designed to identify candidate interacting genes from gene expression data using the $\chi^2$ test against a query gene. Samples are sorted into low, medium, and high expression quantiles for each gene to test for a directional shift. A sample being low in both genes of a synthetic lethal pair is unlikely, since loss of both genes will be deleterious, and is expected to be statistically under-represented in a gene expression dataset. We expect a corresponding (symmetric) increase in frequency of sample with low-high gene pairs. Synthetic lethal candidate (exprSL) partners of a gene are identified by running this procedure on all possible partner genes, selecting those with an FDR-adjusted $\chi^2$ p-value of $p < 0.05$, and meeting the directional criteria. Since synthetic lethal genes are partners of each other commutatively, the symmetric direction criteria are all required such that synthetic lethal genes will predicted to be partners of each other.
}
\label{fig:SLIPT_Method}
\end{mdframed}
\end{figure*}


The synthetic lethal prediction procedure has also been adapted to utilise somatic mutation data for the query gene. This is intended to utilise a query gene known to be recurrently mutated in the disease (and dataset), with the majority of mutations inactivating gene function (such as null or frameshift mutations). A synthetic lethal interaction is predicted  (as shown in Figure~\ref{fig:SLIPT_Method_mtSL}) when (i) the $\chi^2$ test is significant; (ii) observed mutant-query, low-candidate samples are less frequent than expected; and (iii) observed mutant-query, high-candidate and wild-type-query, low-candidate samples are more frequent than expected. Unless otherwise specified, computationally predicted synthetic lethal gene candidates from \gls{SLIPT} used expression data (exprSL) for both genes (as shown in Figure~\ref{fig:SLIPT_Method}) rather than mutation data (mtSL) for the query gene (as shown in Figure~\ref{fig:SLIPT_Method_mtSL}).

\begin{figure*}[!tb]
\begin{mdframed}
  \begin{center}
  \resizebox{0.8 \textwidth}{!}{
    \fbox{\input{SL_Method_mtSL.pdf_tex}}
   }
   \end{center}
   \caption[Synthetic lethal prediction adapted for mutation]{\small \textbf{Synthetic lethal prediction adapted for mutation.} \gls{SLIPT} was also adapted to identify candidate interacting genes using (somatic) mutation data of the query gene in the $\chi^2$ test. Samples are sorted into low, medium, and high expression quantiles for each candidate gene and tested for a directional shift against mutation status of the query gene. A sample having low expression or mutation for the synthetic lethal pair is expected to be unlikely with a corresponding increase in frequency of sample with mutant-high or wild-type-low gene pairs. Synthetic lethal candidate (mtSL) partners of a gene are identified by running this procedure on all possible partner genes, selecting those with an FDR-adjusted $\chi^2$ p-value of $p < 0.05$, and meeting the directional criteria. %Synthetic lethal genes are partners of each other commutatively with synthetic lethal genes will predicted to be partners of each other.
}
\label{fig:SLIPT_Method_mtSL}
\end{mdframed}
\end{figure*}

The \gls{SLIPT} methodology is amenable for use on expression data including pathway metagenes (as generated in Section~\ref{methods:metagene}). The suitability of the \gls{SLIPT} methodology to application on public gene expression data will further be supported by simulation results in Section~\ref{chapt2:simulation_2015} and Chapter~\ref{chap:simulation}, including comparison to other statistical methods. \gls{SLIPT} results for \textit{CDH1} will also compared experimental screen results in a breast cell line \citep{Telford2015}, primary screen results are discussed in Section~\ref{chapt3:compare_SL_genes} and secondary (validation) screen results are presented in Appendix~\ref{appendix:secondary_screen}.

%This methodology was adapted to used pathway metagene quantiles rather than gene expression as an input for pathway synthetic lethality testing. The p-values for $\chi^2$ tests were also corrected for multiple testing with the false discovery rate \citep{fdr1995} across all pathways tested from the same database and with significance defined as a FDR adjusted p-values $p < 0.05$ as above.

%mtSLIPT method
%A similar methodology was developed in both cases to test for synthetic lethality where the query gene has an inactivating mutation in some patients. Since most mutations, particularly in tumour suppressor genes, are deleterious all somatic non-synonymous mutations were counted as mutant and synthetic lethality was tested with the query gene changed accordingly (as shown in Figure~\ref{fig:mtSLIPT_Method}. To distinguish these methods they are abbreviated to exprSLIPT and mtSLIPT respectively depending on the molecular property used to define low gene activity of the query gene.

\FloatBarrier

\section{Synthetic Lethal Simulation and Modelling} \label{methods:simulation_SL_expression} 

A statistical model of Synthetic Lethality was developed to generate simulated data to test the \gls{SLIPT} procedure. This section will describe the synthetic lethal model and the simulation procedure for generating gene expression data with known synthetic lethal partners. Some preliminary results to support usage of the \gls{SLIPT} methodology throughout this thesis will be presented here. The simulation procedure will be applied in more depth in Chapter~\ref{chap:simulation}, including in combination with simulations from graph structures.

\subsection{A Model of Synthetic Lethality in Expression Data} \label{methods:SL_Model}

A conceptual model of synthetic lethality was constructed (see Figure~\ref{fig:SL_Model}), which will be used to build a statistical model of synthetic lethal gene expression from which to simulate expression data to on which test SLIPT and various potential synthetic lethal prediction methods. In the model, synthetic lethality arises between genes with related functions as a cell death phenotype when these functions are removed.

\begin{figure*}[!p]
\begin{mdframed}
\begin{center}
  \resizebox{0.95 \textwidth}{!}{
    %\input{{{"SL_Model.pdf_tex"}}
    \fbox{
    \includegraphics{{"SL_Model"}}
   }
   }
   \end{center}
   \caption[A model of synthetic lethal gene expression]{\small \textbf{A model of synthetic lethal gene expresion.} A conceptual model of synthetic lethal interactions between a Query gene and partner gene ($G_X$). Genes that are synthetic lethal may not both be non-functional in the same sample without another gene compensating for the loss of function. This is most likely to be detectable as low gene expression, whether they are lost by mutation, deletion, DNA methylation, or suppressing regulatory signals. This could manifest as coexpression, mutual exclusivity, or directional shifts in sample frequency. Thus the alternative hypothesis ($H_{A}$) is that synthetic lethal genes will have a reduced frequency of co-loss samples while the null hypothesis ($H_{0}$) is that non-synthetic lethal gene pairs would show no such relationship, even if they may be correlated for other means such as pathway relationships. In this model synthetic lethal genes may compensate for the loss of each other but this is not assumed, only that loss of both is unfavourable to cell viability and probability of detecting samples with combined gene loss.
}
\label{fig:SL_Model}
\end{mdframed}
\end{figure*}


This model suggests that synthetic lethality is detectable in measures of gene inactivation across a sample population, namely mutation, DNA copy number, DNA methylation, and suppression of expression. While any of these mechanisms of gene inactivation could lead to synthetic lethality, expression data is readily available and changes in these alternative mechanisms are likely to impact on the amount of expressed (functional) RNA or protein detectable. There are several ways that functional relationships between genes could manifest in expression data, including coexpression, mutual exclusivity and directional shifts. Co-expression is overly simplistic and has previously performed poorly as a predictor of synthetic lethality \citep{Jerby2014}, although this will still be tested with correlation measures in later simulations. Here the alternative hypothesis is that synthetic lethality will lead to a detectable directional shift in the number of samples exhibiting low or high expression of either gene. This model does not preclude mutual exclusivity \citep{Wappett2016}, compensating expression or co-loss under-representation \citep{Lu2015} as previously postulated to occur between synthetic lethal genes. 

The first condition of the synthetic lethal model is that if there are only two synthetic lethal genes (e.g., \textit{CDH1} and one SL partner), then they will not both be non-functional in the same sample (in an ideal model). Gene function is thus determined for each sample in a model of synthetic lethal with the proportion of samples with a functional or non-functional gene being arbitrary. Whether a gene is functional can similarly be modelled by an arbitrary threshold of continuous and normally distributed gene expression data to define gene function (as shown in Figure~\ref{fig:SL_Model_Expression}). For the purposes of modeling synthetic lethality in breast cancer expression data, a threshold of the 30\textsuperscript{th} percentile of the expression levels was used because approximately 30\% of samples analysed had \textit{CDH1} inactivation. This was generalised for a model of the proportion of samples inactivated for each gene. In this ideal case, no samples lowly expressing both of these genes are expected to be observed. While this is not observed, that is to be expected as it is unlikely that only 2 genes will have an exclusive synthetic lethal partnership. The threshold of the 0.3 quantile was used in simulations dervied from this model throughout this thesis.

\begin{figure*}[!tb]
\begin{mdframed}
  \begin{center}
  \resizebox{0.7125 \textwidth}{!}{
  \fbox{
    \includegraphics{{"SL_Model_Expression"}}
   }
   }
   \end{center}
   \caption[Modeling synthetic lethal gene expression]{\small \textbf{Modeling synthetic lethal gene expression.} When modeling synthetic lethal interactions between a Query gene with partner genes ($G_X$ and $G_Y$) above,  cellular viability requires that at least of genes is not inactivated. Expression below a threshold is used as a model of loss of function, where genes are regarded as non-functional for the purposes of modelling synthetic lethality. Tumour suppressor genes with loss of function also have cancer specific phenotypes (although these thresholds are not necessarily the same). Expression is modeled by a normally (Gaussian) distributed continuous data such as (log-scale) data from RNA (microarray or RNA-Seq), protein, or pathway metagenes. This rationale generalises for several genes on a multivariate normal distribution.
}
\label{fig:SL_Model_Expression} 
\end{mdframed}
\end{figure*}


\begin{figure*}[!p]
\begin{mdframed}
  \begin{center}
  \resizebox{0.95 \textwidth}{!}{
    %\input{{{"SL_Model.pdf_tex"}}
    \fbox{
    \includegraphics{{"SL_Model_Higher"}}
   }
   }
   \end{center}
   \caption[Synthetic lethality with multiple genes]{\small \textbf{Synthetic lethality with multiple genes.} Higher order synthetic lethal interactions may occur between 3 or more genes, affecting the simulated expression (or synthetic lethal predictions) even if undetected when observed pairwise. Consider interactions between a Query gene and two partner genes ($G_X$ and $G_Y$). They may interact with the Query pairwise (inviable when either gene pair is lost) or form a higher-order interaction such as the ``synthetic lethal triplet''  if any of the genes provide an essential function (inviable only when all are lost). Either is plausible with the potential pathway structures. A synthetic lethal triple has 8 potential combinations of gene functional but one is not expected to be observed (due to inviability) but pairwise inactivation may be observed if additional partner genes are functional. The proportion of these combinations vary depending on the functional threshold.
}
\label{fig:SL_Model_Higher}
\end{mdframed}
\end{figure*}



A synthetic lethal pair of genes is unlikely to act in isolation, therefore higher-order synthetic lethal interactions (i.e., 3 or more genes) must be considered in the model as shown in Figure~\ref{fig:SL_Model_Higher}. Even when testing pairwise interactions, modelling higher level interactions that may interfere is important. If there are additional synthetic lethal partners, there are two possibilities for adding these: 1) that they are independent partners of the query genes interacting pairwise (and not with each other) or 2) that an addition partner gene interacts with both of the synthetic lethal genes already in the system and any of the three (or more) are required to be functional for the cell to survive.

The signal (in terms of gene expression data) will be weaker for this latter case and this model has the more stringent assumption that all synthetic lethal partner genes interact with each other: that only one of these must be expressed to satisfy the model of synthetic lethality. In this model any of the synthetic lethal genes in a higher-order interaction is able to provide the missing function of the others, allowing for higher-level synthetic lethal partners to compensate for loss a synthetic lethal gene pair. While samples expressing low levels of the synthetic lethal gene pairs will be under-represented, they may not be completely absent from the dataset due to these higher-level interactions.

In the example of 3 synthetic lethal genes~\ref{fig:SL_Model_Higher}, only one of genes involved in the higher-order synthetic lethal interaction is required for cell viability. For synthetic lethal pairs, only a subset of these samples will be inviable (i.e., removed from simulated data), leading to an under-representation.

In practice, samples are not removed from a simulated dataset, rather the expression and function of the query gene is generated across samples separately from the pool of potential partner genes. The query gene data is matched to simulated samples (as shown in Figure~\ref{fig:simulate_add_query}), satisfying the synthetic lethal condition with the procedure described in Section~\ref{methods:simulating_SL}. This is performed to maintain a comparable samples size across simulations and the preserve the assumed (multivariate) normal distribution of the data. 

\FloatBarrier

\subsection{Simulation Procedure} \label{methods:simulating_SL}

Simulations were developed to simulate normal distributions of expression data and define function with a threshold cut-off. This is the reverse to the procedure of \gls{SLIPT} to predict synthetic lethal partners (although the threshold is assumed to be unknown when testing upon simulated data). While gene function is used as an intermediary step in modelling synthetic lethal genes in expression data, the normal distribution is sampled for simulated data to represent normalised empirical gene expression data for which \gls{SLIPT} (and other methods) will be applicable.

\begin{figure*}[!htbp]
\begin{mdframed}
%  \resizebox{\textwidth}{!}{
         \begin{center}
%
        \subfigure[Simulated expression matrix]{%
            \label{fig:simulate_function:first}
            %\includegraphics[width=0.5\textwidth]{{"/home/tomkelly/Documents/PhD Otago Uni/SL_Model/graph_sim_method/expr_mat_inhibiting".png}}
            \includegraphics[width=0.45\textwidth]{{"/home/tomkelly/Documents/PhD Otago Uni/SL_Model/graph_sim_method/expr_mat".png}}
        }%
        \subfigure[Corresponding gene function calls]{%
           \label{fig:simulate_function:second}
           %\includegraphics[width=0.5\textwidth]{{"/home/tomkelly/Documents/PhD Otago Uni/SL_Model/graph_sim_method/expr_inhib_disc_mat".png}} %%check if same tree order (sample) as expression
           \includegraphics[width=0.45\textwidth]{{"/home/tomkelly/Documents/PhD Otago Uni/SL_Model/graph_sim_method/expr_disc_mat".png}}
        }%
%
    \end{center}
   %\caption[Simulating gene expression and function]{\small \textbf{\textbf{Simulating gene function.}} Simulated data with samples (columns) and genes A--H (rows) shows how a simulated dataset is transformed from a continuous dataset (on a blue to red colour scale) to a discrete matrix of gene function (samples with functional gene levels are shaded in black and non-functional in grey).}
   \caption[Simulating gene function]{\small \textbf{\textbf{Simulating gene function.}} A simulated dataset with samples (columns) and genes A--H (rows) is transformed from a continuous (coloured blue--red) scale to a discrete matrix of gene function (black for functional levels and grey for non-functional).}
%}
\label{fig:simulate_function}
\end{mdframed}

\iffalse
\begin{mdframed}
%  \resizebox{\textwidth}{!}{
         \begin{center}
%
	\subfigure[Simulated gene function with SL genes]{%
            \label{fig:simulate_add_query:first}
            %\includegraphics[width=0.5\textwidth]{{"/home/tomkelly/Documents/PhD Otago Uni/SL_Model/graph_sim_method/expr_inhib_SL_disc_mat".png}}
            %\includegraphics[width=0.5\textwidth,trim=4cm 2cm 0cm 0cm,clip]{{"/home/tomkelly/Documents/PhD Otago Uni/SL_Model/graph_sim_method/expr_SL_disc_mat".png}}
            \includegraphics[width=0.45\textwidth]{{"/home/tomkelly/Documents/PhD Otago Uni/SL_Model/graph_sim_method/expr_SL_disc_mat".png}}
        }%
        \subfigure[Query gene added with SL condition]{%
           \label{fig:simulate_add_query:second}
           %\includegraphics[width=0.5\textwidth]{{"/home/tomkelly/Documents/PhD Otago Uni/SL_Model/graph_sim_method/expr_inhib_disc_query_mat_graph".png}} %%check if same tree order (sample) as expression
           %\includegraphics[width=0.5\textwidth,trim=4cm 2cm 0cm 0cm,clip]{{"/home/tomkelly/Documents/PhD Otago Uni/SL_Model/graph_sim_method/expr_disc_query_mat_graph".png}}
           \includegraphics[width=0.45\textwidth]{{"/home/tomkelly/Documents/PhD Otago Uni/SL_Model/graph_sim_method/expr_disc_query_mat_graph".png}}
        }%
%
    \end{center}
   %\caption[Simulating synthetic lethal gene function]{\small \textbf{\textbf{Simulating synthetic lethal gene function.}} Simulated data with samples (columns) and genes (rows) in a discrete matrix of gene function (shaded in black for sample with functional gene levels). Genes A and I are selected to be synthetic lethal partners of a ``Query'' gene, which of these genes will be the true partner in each sample is selected randomly and indicated in green which samples are considered for the purposes of simulating synthetic lethality (shaded in forest green for samples with functional gene levels). Note that samples are ordered such that either the query gene or selected partner are functional in any particular sample.}
   \caption[Simulating synthetic lethal gene function]{\small \textbf{\textbf{Simulating synthetic lethal gene function.}} In a discrete simulated gene function dataset (black for functional levels and grey for non-functional) with samples (columns) and genes (rows), genes A and I are the SL partners of a ``Query'' gene. A partner  gene is selected randomly (shown in green) in each sample for simulating synthetic lethality (forest green for functional genes). %Note that samples are ordered such that either the query gene or selected partner are functional in any particular sample.
   }
%}
\label{fig:simulate_add_query}
\end{mdframed}
\fi
%\end{figure*}
%\begin{figure*}[!ht]
\begin{mdframed}
%  \resizebox{\textwidth}{!}{
         \begin{center}
%
	\subfigure[Simulated gene function with SL genes]{%
            \label{fig:simulate_add_query:first}
            %\includegraphics[width=0.5\textwidth]{{"/home/tomkelly/Documents/PhD Otago Uni/SL_Model/graph_sim_method/expr_inhib_SL_disc_mat".png}}
            %\includegraphics[width=0.5\textwidth,trim=4cm 2cm 0cm 0cm,clip]{{"/home/tomkelly/Documents/PhD Otago Uni/SL_Model/graph_sim_method/expr_SL_disc_mat".png}}
            \includegraphics[width=0.45\textwidth]{{"/home/tomkelly/Documents/PhD Otago Uni/SL_Model/graph_sim_method/expr_SL_disc_mat".png}}
        }%
        \subfigure[Query gene added with SL condition]{%
           \label{fig:simulate_add_query:second}
           %\includegraphics[width=0.5\textwidth]{{"/home/tomkelly/Documents/PhD Otago Uni/SL_Model/graph_sim_method/expr_inhib_disc_query_mat_graph".png}} %%check if same tree order (sample) as expression
           %\includegraphics[width=0.5\textwidth,trim=4cm 2cm 0cm 0cm,clip]{{"/home/tomkelly/Documents/PhD Otago Uni/SL_Model/graph_sim_method/expr_disc_query_mat_graph".png}}
           \includegraphics[width=0.45\textwidth]{{"/home/tomkelly/Documents/PhD Otago Uni/SL_Model/graph_sim_method/expr_disc_query_mat_graph".png}}
        }%
%
    \end{center}
   %\caption[Simulating synthetic lethal gene function]{\small \textbf{\textbf{Simulating synthetic lethal gene function.}} Simulated data with samples (columns) and genes (rows) in a discrete matrix of gene function (shaded in black for sample with functional gene levels). Genes A and I are selected to be synthetic lethal partners of a ``Query'' gene, which of these genes will be the true partner in each sample is selected randomly and indicated in green which samples are considered for the purposes of simulating synthetic lethality (shaded in forest green for samples with functional gene levels). Note that samples are ordered such that either the query gene or selected partner are functional in any particular sample.}
   \caption[Simulating synthetic lethal gene function]{\small \textbf{\textbf{Simulating synthetic lethal gene function.}} In a discrete simulated gene function dataset (shaded for functional levels and pale otherwise) with samples (columns) and genes (rows), genes A and I are SL partners of a ``Query'' gene. A partner is selected (highlighted in green) randomly in each sample for simulating synthetic lethality, then ordered such that the query gene or an SL partner are functional in each sample.
   }
%}
\label{fig:simulate_add_query}
\end{mdframed}
\end{figure*}


Sampling a distribution for expression profiles has the added advantage of being amenable to simulating correlation structures with the multivariate normal distribution (using the \texttt{mvtnorm} R package \citep{Genz2009, mvtnorm}). The parameter $\Sigma$ is a covariance matrix defines the correlation structure between simulated genes being sampled. With a diagonal of one, this $\Sigma$\ matrix simulates genes with a standard deviation of one and the covariance parameters between them are the correlations between each gene. In Figure~\ref{fig:simulate_function}, an example of such a simulated multivariate normal dataset is shown with the functional threshold applied.

Once we have generated a simulated dataset, the samples are compared by gene function (as derived from a functional threshold). Known underlying synthetic lethal partners are selected within the dataset and a query gene is generated by sampling from the normal distribution. These are matched (as shown for 2 synthetic lethal partners in Figure~\ref{fig:simulate_add_query}) such that the synthetic lethal condition is met: that at least one of the synthetic partner genes and the query gene are functional in any particular cell. The samples are ordered by functional data (without assuming correlation of underyling expression values) with the query gene in one direction and the remaining dataset ordered by the selected synthetic lethal partner.


\begin{figure*}[!htb]
\begin{mdframed}
%  \resizebox{\textwidth}{!}{
         \begin{center}
%
        \subfigure[Initial expression matrix]{%
            \label{fig:simulate_SL:first}
            %\includegraphics[width=0.5\textwidth]{{"/home/tomkelly/Documents/PhD Otago Uni/SL_Model/graph_sim_method/expr_mat_inhibiting".png}}
            \includegraphics[width=0.45\textwidth]{{"/home/tomkelly/Documents/PhD Otago Uni/SL_Model/graph_sim_method/expr_mat".png}}
        }%
        \subfigure[Simulated synthetic lethal dataset]{%
           \label{fig:simulate_SL:second}
           %\includegraphics[width=0.5\textwidth]{{"/home/tomkelly/Documents/PhD Otago Uni/SL_Model/graph_sim_method/expr_inhib_query_mat_graph".png}} %%check if same tree order (sample) as expression
           \includegraphics[width=0.45\textwidth]{{"/home/tomkelly/Documents/PhD Otago Uni/SL_Model/graph_sim_method/expr_query_mat_graph".png}}
        }%
%
    \end{center}
   %\caption[Simulating synthetic lethal gene expression]{\small \textbf{\textbf{Simulating synthetic lethal gene expression.}} Simulated data with samples (columns) and genes (rows) showing how a simulated continuous dataset (on a blue to red colour scale) is matched to a query gene such that at least one synthetic lethal partner is above a functional threshold when the query gene is below it satisfying the synthetic lethal model.}
   \caption[Simulating synthetic lethal gene expression]{\small \textbf{\textbf{Simulating synthetic lethal gene expression.}} A simluated continuous expression dataset (blue--red scale) with samples (columns) and genes (rows) is matched to a query gene such that at least one synthetic lethal partner is above a functional threshold when the query gene is below it satisfying the synthetic lethal model.}
%}
\label{fig:simulate_SL}
\end{mdframed}
\end{figure*}

This results a simulated dataset where samples with non-functional query gene have at least one functional partner gene. Similarly, the query gene is functional in all samples where all of the synthetic lethal partner genes are non-functional. Therefore a dataset has been generated with known synthetic lethal partners (see Figure~\ref{fig:simulate_SL}) by as few assumptions about the relationships between the each synthetic lethal pair as possible (and allowing compensating functions from higher-order interactions). This has been designed to have the most stringent (least detectable) synthetic lethal relationships where higher-order interactions are possible for the purposes of testing pairwise detection procedures such as \gls{SLIPT}.  


\FloatBarrier

\section{Detecting Simulated Synthetic Lethal Partners} \label{chapt2:simulation_2015}

The synthetic lethal detection methodology (\gls{SLIPT}), as described in Section~\ref{methods:SLIPT}, was tested on simulated data with known synthetic lethal partners, generated using the procedure described in Section~\ref{methods:simulating_SL}. This section will present basic simulations to demonstrate the methodology and support it's use throughout this thesis. These will be performed with sampling from basic statistical distributions as described, including multivariate normal distribution with correlated blocks of genes, with the $\Sigma$ matrix show in the plots where relevant. A more complex multivariate normal sampling procedure based on pathway graph structures, as described in section ~\ref{methods:graphsim}, will be applied in Chapter~\ref{chap:simulation}. 

\subsection{Binomial Simulation of Synthetic lethality} \label{chapt2:simulation_binom}
%[relevant?]


A previous version of the synthetic lethal simulation procedure (described in Section~\ref{methods:simulating_SL}), used gene function sampled directly from a binomial distribution using the binomial probability of observing functional gene levels ($p = 0.3$) in one observation ($n = 1$) for each samples: $$X\sim Bin(n,p)$$  Once a query gene consistent with synthetic lethality has been added, these functional levels were passed directly into \gls{SLIPT} as ``low'' and ``high'' categories.

The simulation procedure was performed with 20,000 total genes (as feasible in the human genome and expression datasets) with a variable number of true synthetic lethal partners and sample sizes of 500, 1000, 2000, and 5000. Each ROC curve was derived from the results of 10,000 replicate simulations. The statistical performance (as shown in Figure~\ref{fig:Binomial_AUC}) of such an approach based on the $\chi^2$ p-value declines towards random predictions (an AUROC of 0.5) with an increasing number of underlying true synthetic lethal partners to detect. However, increased sample size mitigates this decline to some extent, as expected with a statistical predictor, particularly for moderate numbers of synthetic lethal partners. 

\begin{figure*}[!hp]
\begin{mdframed}
  \begin{center}
  \resizebox{0.6 \textwidth}{!}{
  \fbox{
    \includegraphics{{"SL_Model_Binomial_1K_AUC_samples_prop"}}
   }
   }
   \end{center}
   \caption[Performance of binomial simulations]{\small \textbf{Performance of binomial simulations.} Gene function was simulated by binomial sampling and tested for synthetic lethal genes. Statistical performance declines with additional known synthetic partners but this is mitigated by increased sample sizes.}
\label{fig:Binomial_AUC}
\end{mdframed}

\begin{mdframed}
  \begin{center}
  \resizebox{0.6 \textwidth}{!}{
  \fbox{
    \includegraphics{{"SL_Model_Test_Graph_10K_Graph1_ROC_Compare_Binom(Feb)_v_Mvtn(Aprxy)_Full"}}
   }
   }
   \end{center}
   \caption[Comparison of statistical performance]{\small \textbf{Comparison of statistical performance.} Binomial simulation of synthetic lethality (in colour) is compared (in greyscale) to multivariate normal simulations (detailed below) which consistently outperforms binomial simulation across parameters.}
\label{fig:Binomial_Compare}
\end{mdframed}
\end{figure*}

Simulations based on a simple binomial model of synthetic lethality are limited but form a basis for builing a more complex model including expression and correlation structures. While this does not represent the data that \gls{SLIPT} will be applied to, binomial simulations do demonstrate that \gls{SLIPT} is able to distinguish small numbers of synthetic lethal partners in a simplistic simulated system with behavious expected with resepect to sample size. This supported further development of the synthetic lethal model and simulation pipeline (as described in Section~\ref{methods:simulation_SL_expression}) using the multivariate normal distribution.

The multivariate normal simulation procedure is more representative of the (normalised) expression data \gls{SLIPT} is intended for and enables the prediction procedure to be tested without changes to the methodology (presented in more detail in Section~\ref{chapt2:simulation_mvtnorm}). Sampling continuous expression values from a normal distribution allows the expression threshold for gene function to differ from the categorical ``low'' and ``high'' expression binning performed by \gls{SLIPT} (as discussed in Section~\ref{methods:SL_Model}) which represents that the \gls{SLIPT} procedure does not assume a known threshold for expression but rather uses expression as an estimate of gene function. This functionality can be included in the multivariate normal simulation without compromising the statistical performance of the \gls{SLIPT}, rather the performance estimates (shown in Figure~\ref{fig:Binomial_Compare}) were a marked improvement over the binomial simulation procedure across simulation parameters in an equivalent simulation (without correlation structure). This imprivement may be due to binomial model defining the synthetic lethal condition in a way that, while ensuring at least on synthetic lethal partner is active in query deficient samples, disrupts the number of samples with functional synthetic lethal genes compared to other genes affecting the expected sample proportions of $\chi^2$ test.

\FloatBarrier

\subsection{Multivariate Normal Simulation of Synthetic lethality} \label{chapt2:simulation_mvtnorm}

The multivariate normal simulation procedure was initially performed using the \texttt{mvtnorm} R package \citep{Genz2009, mvtnorm} (as described in Section~\ref{methods:simulation_SL_expression}) without correlation structure.

Expression is sampled from multivariate normal distribution with a mean ($\mu = 0$), standard deviation ($\sigma = 1$), and no correlation between genes ($r = 0$): $$X\sim N(\bar{\mu},\Sigma)$$  Once a query gene consistent with synthetic lethality has been added, the simulated expression values are tested by \gls{SLIPT} exactly as described in Section~\ref{methods:SLIPT}.

\begin{figure*}[!hp]
\begin{mdframed}
%  \resizebox{\textwidth}{!}{
         \begin{center}
%
        \subfigure[Statistical evaluation]{%
            \label{fig:simulation_Apr15ROC:Perf}
            \includegraphics[width=0.475\textwidth]{{SL_Model_Apr15mvnormCor_1K_ROC1_samplesx_prop".png}}
        }%
        \subfigure[Receiver operating characteristic]{%
            \label{fig:simulation_Apr15ROC:ROC}
            \includegraphics[width=0.475\textwidth]{{"SL_Model_Apr15mvnormCor_1K_ROC2_samplesx_prop".png}}
        }%
        
        \subfigure[Statistical performance]{%
           \label{fig:simulation_Apr15ROC:AUC}
           \includegraphics[width=0.65\textwidth]{{"SL_Model_Apr15mvnormCor_1K_AUC_samplesx_prop".png}}
        }%
    \end{center}
   \caption[Performance of multivariate normal simulations]{\small \textbf{Performance of multivariate normal simulations.} Simulation of synthetic lethality was performed sampling from a multivariate normal distribution (without correlation structure). Performance of \gls{SLIPT} declines for more synthetic partners but this is mitigated by increased sample sizes (in darker colours). This generally occurs as the sensitivity decreases for a greater number of true positives to detect, leading to a trade off in accuracy as seen in a trough for false discovery rate and the ROC curves.}
%}
\label{fig:simulation_Apr15ROC}
\end{mdframed}
\end{figure*}

As shown in Figure~\ref{fig:simulation_Apr15ROC:Perf}, the statistical accuracy of \gls{SLIPT} as a binary classifier is considerably high across simulations of a full human dataset of 20,000 genes. However, with the $\chi^2$ p-value as a threshold for prediction, this is largely to desirable specificity: the majority of non-SL genes are distinguished from the few underlying synthetic lethal genes. In this regard, the \gls{SLIPT} methodology generally performs better with larger datasets with more expected negatives and thus the results of simulations of smaller numbers of genes (such as the graph structures analysed in Chapter~\ref{chap:simulation}) can be applied to larger datasets where they are expected to perform comparably or better with a lower false negative rate. Accordingly, key results will be supported by replication with larger numbers of non-SL genes added to the simulations. 

However, with higher numbers of synthetic lethal genes to detect, the sensitivity (in Figure~\ref{fig:simulation_Apr15ROC:Perf}) of \gls{SLIPT} as a binary classifier of synthetic lethality declines, although this is somewhat mitigated by higher sample sizes (shown in darker colours). Thus the minority of true synthetic lethal partners are more difficult to distinguish when there are more of them (and a weaker expression signal from each). While a reasonable reduction of the false discovery rate can be achieved for moderate numbers of underlying synthetic lethal partners, we can not be sure how many partners are expected to be detected in analyses of expression data. However this simulation procedure is amenable to assessing the performance of \gls{SLIPT} across simulation parameters, graph structures and comparisons to other approaches (presented in more detail in Chapter~\ref{chap:simulation}).

Not all of the genes detected by \gls{SLIPT} will be true synthetic lethals but these will be among the strongest candidates and it performs better with fewer underlying synthetic lethals to detect. This supports a focus on pathway analyses, in particular detecting pathways for further investigation. Since individually gene candidates are not necessarily gene synthetic lethal themselves, pathway over-representation analysis will be performed to detect functional groups recurrently detected by \gls{SLIPT} as these detection of functionally related genes further support their role in synthetic lethal relationships in addition to being biologically informative. Alternatively, pathway metagenes will reduce the number of underlying synthetic lethals to identify synthetic lethal pathways. Both of these approaches will be applied in Chapter~\ref{chap:SLIPT} to identify and replicate synthetic pathways of \textit{CDH1}. Pathways are also more likely to replicate across experimental models as demonstrated by \citet{Dixon2008}.
 
The receiver operating characteristic curves (in Figure~\ref{fig:simulation_Apr15ROC:ROC}) demonstrate that \gls{SLIPT} is subject to near equal trade-off between sensitivity and specificity across threshold values. The lower sensitivity and higher specificity with a binary classifiation (in Figure~\ref{fig:simulation_Apr15ROC:Perf}) stems from stringent testing by \gls{SLIPT} with (FDR) p-values adjusted for multiple tests. The area under these curves is also used to compare statistical performce (in Figure~\ref{fig:simulation_Apr15ROC:AUC}), with declining performance across increased underlying synthetic lethal partners and increased performance with sample size in multivariate normal simulations.

\FloatBarrier

\subsubsection{Multivariate Normal Simulation with Correlated Genes} \label{chapt2:simulation_mvtnorm_cor}
%\subsubsection{Simulated Expression Heatmaps}

Correlation structures can be added to the simulation procedure (as discussed in Section~\ref{methods:simulation_SL_expression}), starting with simple correlated blocks of genes as the $\Sigma$ parameter depicted in Figure~\ref{fig:simulation_May4SL:first}. These correlated blocks represent genes with correlated expression such as that expected by coregulation or biological pathways. Figure~\ref{fig:simulation_May4SL} gives an example of 4 synthetic lethal genes (out of 100), each with 5 correlated genes that are not themselves synthetic lethal partners of the query gene. This serves to test whether synthetic lethal genes are distinguishable from correlated partners. This $\Sigma$ matrix produces a similar correlation structure (Figure~\ref{fig:simulation_May4SL:second}) in the resulting expression profiles (Figure~\ref{fig:simulation_May4SL:third}) where apart from correlated blocks of genes ($r = 0.8$), the remaining genes have only slight variations due to random sampling. The structure of the dataset, particularly between synthetic lethal genes and the query, is shown at the gene expression (Figure~\ref{fig:simulation_May4SL:third}) and function (Figure~\ref{fig:simulation_May4SL:fifth}). These are ordered by the \gls{SLIPT} results and the synthetic lethal genes are ranked high, with the majority of them being distinguishable from highly correlated genes.

\begin{figure*}[!hp]
\begin{mdframed}
%  \resizebox{\textwidth}{!}{
         \begin{center}
%
        \subfigure[Input $\Sigma$ matrix parameter]{%
            \label{fig:simulation_May4SL:first}
            \includegraphics[width=0.35\textwidth]{{"SL_Model_May15mvnorm_heatmap_4SL_cor_comp_top(1)".pdf}}
        }%
        \subfigure[Simulated correlation matrix]{%
            \label{fig:simulation_May4SL:second}
            \includegraphics[width=0.35\textwidth]{{"SL_Model_May15mvnorm_heatmap_4SL_cor_comp_top(2)".pdf}}
        }%
        
        \subfigure[Simulated gene expression]{%
           \label{fig:simulation_May4SL:third}
           \includegraphics[width=0.35\textwidth]{{"SL_Model_May15mvnorm_heatmap_4SL_cor_comp_top(4)".pdf}}
        }%
	\subfigure[Simulated gene function]{%
           \label{fig:simulation_May4SL:fifth}
           \includegraphics[width=0.35\textwidth]{{"SL_Model_May15mvnorm_heatmap_4SL_cor_comp_top(3)".pdf}}
        }%
    \end{center}
   \caption[Simulating expression with correlated gene blocks]{\small \textbf{Simulating expression with correlated gene blocks.} A $\Sigma$ matrix (a) is used generate a multivariate normal distribution with of 100 genes correlated blocks of genes (correlated by 0.8) with a comparable structure (b) to the input $\Sigma$, as shown by correlation on a red--green scale. The annotation bars for genes give the $\chi^2$ (in blue if the direction of \gls{SLIPT} is met or red otherwise) and the gene category (blue for query, cyan for query-correlated, red for SL, orange for SL-correlated, forest green for non-SL-correlated, and green for non-SL). The simulated gene expression (c) and function (d) generated are ordered by $\chi^2$ showing the functional structure of synthetic lethal genes and that they are among the strongest \gls{SLIPT} results.}
%}
\label{fig:simulation_May4SL}
\end{mdframed}
\end{figure*}

The use of correlation structures generalises to larger datasets, such as 1000 genes shown in Figure~\ref{fig:simulation_May4SL1K}. Synthetic lethal genes are highly ranked by \gls{SLIPT} and still largely distinguishable from correlated genes. As previously discussed in Section~\ref{chapt2:simulation_mvtnorm}, these synthetic lethal genes are still detectable among a larger number of true negatives and the \gls{SLIPT} methodology performs better on such datasets.

These plots (Figures~\ref{fig:simulation_May4SL} and~\ref{fig:simulation_May4SL1K}) also show similar correlated blocks with a non-synthetic lethal gene (true negative) and the query gene (which is not synthetic lethal with itself). Neither of these should be synthetic lethal (or detected to be) but they may impact upon the performance of the model, particularly the specificity as correlated negative genes may be distinguishable from true synthetic lethals. The non-synthetic lethal correlated block has no impact on synthetic lethal detection but the impact of query correlated genes will be discussed in Section~\ref{chapt2:simulation_mvtnorm_query_cor} and Chapter~\ref{chap:simulation}.

\begin{figure*}[!htp]
\begin{mdframed}
%  \resizebox{\textwidth}{!}{
         \begin{center}
%
        \subfigure[Input $\Sigma$ matrix parameter]{%
            \label{fig:simulation_May4SL1K:first}
            \includegraphics[width=0.35\textwidth]{{"SL_Model_May15mvnorm_heatmap_4SL_cor_comp_1K_top(1)".pdf}}
        }%
        \subfigure[Simulated correlation matrix]{%
            \label{fig:simulation_May4SL1K:second}
            \includegraphics[width=0.35\textwidth]{{"SL_Model_May15mvnorm_heatmap_4SL_cor_comp_1K_top(2)".pdf}}
        }%
        
        \subfigure[Simulated gene expression]{%
           \label{fig:simulation_May4SL1K:third}
           \includegraphics[width=0.35\textwidth]{{"SL_Model_May15mvnorm_heatmap_4SL_cor_comp_1K_top(4)".pdf}}
        }%
	\subfigure[Simulated gene function]{%
           \label{fig:simulation_May4SL1K:fifth}
           \includegraphics[width=0.35\textwidth]{{"SL_Model_May15mvnorm_heatmap_4SL_cor_comp_1K_top(3)".pdf}}
        }%
    \end{center}
   \caption[Simulating expression with correlated gene blocks]{\small \textbf{Simulating expression with correlated gene blocks.} Using the (a) $\Sigma$ matrix, sampling from a multivariate normal distribution with of 1000 genes produced (b) correlated blocks of genes (correlated by 0.8) on a red--green scale. The simulated gene expression (c) and function (d) generated are ordered by $\chi^2$ and \gls{SLIPT} direction show that synthetic lethal genes are among the strongest \gls{SLIPT} results with high specficity against many potential false positives. These are annotated for $\chi^2$ (on a red--green scale) and category (blue for query, cyan for query-correlated, red for SL, orange for SL-correlated, forest green for non-SL-correlated, and green for non-SL) for each gene.}
%}
\label{fig:simulation_May4SL1K}
\end{mdframed}
\end{figure*}

%\subsection{Replication Simulation Heatmap}
These simulations (on 100 genes) were repeated to examine the variation between detection on different samples and varying the number of underlying synthetic lethal partners, in simulated gene expression data with correlations structure. A small nuber (10 for each) simulations are shown in Figure~\ref{fig:simulation_May4SLreps} to demonstrate the variation between replicate simulations, with iterative sampling from the same multivariate normal distribution. These simulations show synthetic lethal genes are not only highly ranked by \gls{SLIPT} when there are few of them but also that they are fairly consistent across replicate simulations. Whereas they become less consistent for increasing numbers of true synthetic lethal partners to detect and thus more difficult to distinguish from other geners, particularly those correlated with them. Similarly, the $\chi^2$ values show a marked stepwise increase with clear thresholds for SL and correlated genes in simple simulations, whereas these become less evident for higher numbers of SL partners.

\begin{figure*}[!t]
\begin{mdframed}
%  \resizebox{\textwidth}{!}{
         \begin{center}
%
        \subfigure[Gene category in simulations]{%
            \label{fig:simulation_May4SLreps:first}
            \includegraphics[width=0.35\textwidth]{{"SL_Model_May15mvnorm_heatmap_10XSL_cor_comp2(1)".png}}
        }%
        \subfigure[Corresponding $\chi^2$ values]{%
            \label{fig:simulation_May4SLreps:second}
            \includegraphics[width=0.35\textwidth]{{"SL_Model_May15mvnorm_heatmap_10XSL_cor_comp2(2)".png}}
        }%

    \end{center}
   \caption[Synthetic lethal prediction across simulations]{\small \textbf{Synthetic lethal prediction across simulations.} The gene category (blue for query, cyan for query-correlated, red for SL, orange for SL-correlated, forest green for non-SL-correlated, and green for non-SL) ordered by $\chi^2$ signed by the \gls{SLIPT} directional condition is shown across simulations. For each of 1--10 SL partners, 10 simulations demonstrate that the increasing numbers of SL partners become harder detect. The $\chi^2$ values show a clear threshold for SL and correlated genes when there are fewer of them, distinguishable from correlated genes in this case.}
%}
\label{fig:simulation_May4SLreps}
\end{mdframed}
\end{figure*}

%5 correlated genes
 \begin{figure*}[!hp]
\begin{mdframed}
%  \resizebox{\textwidth}{!}{
         \begin{center}
%
        \subfigure[Statistical evaluation]{%
            \label{fig:simulation_Apr15ROC2:Perf}
            \includegraphics[width=0.475\textwidth]{{"SL_Model_Apr15mvnorm_1K_ROC1_samplesx_prop".png}}
        }%
        \subfigure[Receiver operating characteristic]{%
            \label{fig:simulation_Apr15ROC2:ROC}
            \includegraphics[width=0.475\textwidth]{{"SL_Model_Apr15mvnorm_1K_ROC2_samplesx_prop".png}}
        }%
        
        \subfigure[Statistical performance]{%
           \label{fig:simulation_Apr15ROC2:AUC}
           \includegraphics[width=0.65\textwidth]{{"SL_Model_Apr15mvnorm_1K_AUC_samplesx_prop".png}}
        }%
    \end{center}
   \caption[Performance with correlations]{\small \textbf{Performance with correlations.} Simulation of synthetic lethality was performed sampling from a multivariate normal distribution (with correlation structure). Performance of \gls{SLIPT} declines for more synthetic partners but this is mitigated by increased sample sizes (darker colours). This generally occurs as the sensitivity decreases for a greater number of true positives to detect, leading to a trade off in accuracy as seen in a trough for false discovery rate and the ROC curves.}
%}
\label{fig:simulation_Apr15ROC2}
\end{mdframed}
\end{figure*}

\begin{figure*}[!ht]
\begin{mdframed}
  \begin{center}
  \resizebox{0.5 \textwidth}{!}{
  \fbox{
    \includegraphics{{"SL_Model_Test_Graph_10K_Graph1_ROC_Compare_Mvtn(Apry)_v_Cor(Aprxy)_Full"}}
   }
   }
   \end{center}
   \caption[Comparison of statistical performance with correlation structure]{\small \textbf{Comparison of statistical performance with correlation structure.} Multivariate simulation of synthetic lethality with correlation structure (in colour) has comparable performance to simulation without correlations (in greyscale)  with known synthetic partners across parameters.}
\label{fig:mvtnorm_cor_compare}
\end{mdframed}
\end{figure*}


Whether the synthetic lethal genes detected in simple simulations (in Figure~\ref{fig:simulation_May4SLreps}) are robustly detectable across greater number of simulations, in addition to further comparions, was tested with a supporting ROC analysis. These results (in Figure~\ref{fig:simulation_Apr15ROC2}) are very similar to simulations without correlation structure, with \gls{SLIPT} as a binary classifier having a poor sensitivity with increasing numbers of synthetic lethal partners to detect but high specificity in a total of 20,000 genes with the vast majority being true negatives. This is reflected in a similar decline in statistical performance for increasing numbers of synthetic lethal partners and a compensating increase in performance with higher sample size. Overall, the statistical performance is very similar to simulations without correlation structure (as shown in Figure~\ref{fig:mvtnorm_cor_compare}).

Thus \gls{SLIPT} is robust across correlation structures and applicable to real gene expresssion data where pathway structures and correlations are a consideration. These correlation structures are not intended to model specific biological pathways or represent them, rather they serve to test the impact of correlation structure on the performance of \gls{SLIPT} with an extreme example of closely correlated ($r = 0.8$) gene blocks. More complex correlation structures, such as genes postively correlated with the query gene and derived from pathway graph structures (as described in~\ref{methods:graphsim}) will be examined below (in Section~\ref{chapt2:simulation_mvtnorm_query_cor}) and in Chapter~\ref{chap:simulation} respectively.

%query correlated
%5 correlated genes
 \begin{figure*}[!hp]
\begin{mdframed}
%  \resizebox{\textwidth}{!}{
         \begin{center}
%
        \subfigure[Statistical evaluation]{%
            \label{fig:simulation_May15ROC:Perf}
            \includegraphics[width=0.475\textwidth]{{"SL_Model_May15mvnorm_dir_1K_ROC1_samplesx_prop".png}}
        }%
        \subfigure[Receiver operating characteristic]{%
            \label{fig:simulation_May15ROC:ROC}
            \includegraphics[width=0.475\textwidth]{{"SL_Model_May15mvnorm_dir_1K_ROC2_samplesx_prop".png}}
        }%
        
        \subfigure[Statistical performance]{%
           \label{fig:simulation_May15ROC:AUC}
           \includegraphics[width=0.65\textwidth]{{"SL_Model_May15mvnorm_dir_1K_AUC_samplesx_prop".png}}
        }%
    \end{center}
   \caption[Performance with query correlations]{\small \textbf{Performance with query correlations.} Simulation of synthetic lethality was performed sampling from a multivariate normal distribution (with correlation structure including correlated genes with non-SL and query genes). As before, performance of \gls{SLIPT} declines for more synthetic partners and is mitigated by increased sample sizes (darker colours)but the sensitivity remains higher for a greater number of true positives with corresponding improvements in ROC curves.}
%}
\label{fig:simulation_May15ROC2}
\end{mdframed}
\end{figure*}

In particular, genes correlated with true synthetic lethal genes have little impact on the performance of \gls{SLIPT} detection: synthetic lethal genes are as distinguishable from true negative genes as without correlated genes. Synthetic lethal correlated genes will not interfere detect of true synthetic lethals, although they may be ranked next below them and be biologically informative with related gene functions.

\FloatBarrier

\subsubsection{Specificity with Query-Correlated Pathways}  \label{chapt2:simulation_mvtnorm_query_cor}

Another consideration for correlation structures is postively correlated genes with the query that are not synthetic lethal. As described in Section~\ref{chapt2:simulation_mvtnorm_cor}, 5 highly correlated ($r = 0.8$) with the query gene were added. These simulations perform similarly to before (in Figure~\ref{fig:simulation_May15ROC2}) with a higher specficity and a lower false discovery rate being feasible (as shown in~\ref{fig:simulation_May15ROC:Perf}).


%both have 5x5 cor genes (dir tested for May)
\iffalse
\begin{figure*}[!htbp]
\begin{mdframed}
  \begin{center}
  \resizebox{0.5 \textwidth}{!}{
  \fbox{
    \includegraphics{{"SL_Model_Test_Graph_10K_Graph1_ROC_Compare_Mvtn(Apry)_v_Cor(Mayxy)_Full"}}
   }
   }
   \end{center}
   \caption[Comparison of performance for query correlations]{\small \textbf{Comparison of performance for query correlations.} Multivariate simulation of synthetic lethality with correlation structure (in colour) clearly has lower performance than simulation including query correlations (in greyscale) across parameters. The query correlation simulation shows predictive potential for \gls{SLIPT} even with many underlying synthetic lethal partners as postively correlated genes are distinguished robustly.}
\label{fig:mvtnorm_query_cor_compare}
\end{mdframed}
\end{figure*}
\fi

\FloatBarrier

\subsubsubsection{Importance of Directional Testing}
It is important to notice here that the directional criteria of the \gls{SLIPT} procedure is enhancing it's performance, particularly in distinguishing positively correlated true negatives. The multivariate normal simulation results, with 20,000 genes including all of the correlation structures discussed (SL, non-SL, and query correlated genes), are compared here for \gls{SLIPT} with and without ($\chi^2$) directional testing. There is a marked improvement in statistical performance with directional criteria, particularly with increased sensitivity and lower false discovery rate (as shown in Figure~\ref{fig:mvtnorm_dir_compare}).

\begin{figure*}[!hb]
\begin{mdframed}
%  \resizebox{\textwidth}{!}{
         \begin{center}
%
        \subfigure[$\chi^2$ testing without direction]{%
            \label{fig:mvtnorm_dir_compare:perf_null}
            \fbox{
            \includegraphics[width=0.475\textwidth]{{"SL_Model_May15mvnorm_1K_ROC1_samplesx_prop".png}}
            }
        }%
        \subfigure[\gls{SLIPT} with directional criteria]{%
           \label{fig:mvtnorm_dir_compare:perf_dir}
            \fbox{
           \includegraphics[width=0.475\textwidth]{{"SL_Model_May15mvnorm_dir_1K_ROC1_samplesx_prop".png}}
           }
        }%

    \end{center}
   \caption[Statistical evaluation of directional criteria]{\small \textbf{Statistical evaluation of directional criteria.} A simulated multivariate normal dataset of 20,000 genes with correlation structures was tested by \gls{SLIPT} with the directional condition and and the equivalent $\chi^2$ test without. \gls{SLIPT} exhibited a consistently higher sensitivity and lower false discovery rate.}
%}
\label{fig:mvtnorm_dir_compare}
\end{mdframed}
\end{figure*}

\begin{figure*}[!htbp]
\begin{mdframed}
%  \resizebox{\textwidth}{!}{
         \begin{center}
%
        \subfigure[$\chi^2$ testing without direction]{%
            \label{fig:mvtnorm_dir_compare:ROC_null}
            \fbox{
            \includegraphics[width=0.475\textwidth]{{"SL_Model_May15mvnorm_1K_ROC2_samplesx_prop".png}}
            }
        }%
        \subfigure[\gls{SLIPT} with directional criteria]{%
           \label{fig:mvtnorm_dir_compare:ROC_dir}
            \fbox{
           \includegraphics[width=0.475\textwidth]{{"SL_Model_May15mvnorm_dir_1K_ROC2_samplesx_prop".png}}
           }
        }%


        \subfigure[Statistical performance]{%
           \label{fig:mvtnorm_dir_compare:AUC}
           \includegraphics[width=0.6\textwidth]{{"SL_Model_Test_Graph_10K_Graph1_ROC_Compare_Mvtn(Mayy)_v_Cor(Mayxy)_Full".png}}
        }%
%
    \end{center}
   \caption[Performance of directional criteria]{\small \textbf{Performance with directional criteria.} A simulated multivariate normal dataset of 20,000 genes with correlation structures was tested by \gls{SLIPT} with the directional condition and and the equivalent $\chi^2$ test without. \gls{SLIPT} has higher performance across simulation parameters, clearly differing from random (grey diagonal) in ROC curves up to 100 SL genes (b). The performance (c) of \gls{SLIPT} (in greyscale) was consistently higher than the $\chi^2$ test (in color).}
%}
\label{fig:mvtnorm_dir_compare2}
\end{mdframed}
\end{figure*}

This is encouraging for the application of \gls{SLIPT} to empirical expression datasets as postively correlated genes are likely to occur and the directional condition robustly improves the performance of \gls{SLIPT} across simulation parameters. Without assuming the underlying number of synthetic lethal genes, \gls{SLIPT} will perform better than the $\chi^2$ test alone at detecting them. This is further supported irrespective of significance threshold for the $\chi^2$ test by the ROC analysis in Figure~\ref{fig:mvtnorm_dir_compare2}. The directional \gls{SLIPT} methodology outperforms the ordinary $\chi^2$ test at detecting synthetic lethal partners with some predictive power (above random and AUROC of 0.5) even up to 100 synthetic lethal genes.

Together these simulation results support the application of the \gls{SLIPT} methodology as it has been performed throughout Chapter~\ref{chap:SLIPT} and~\ref{chap:Pathways}. However, the methodology and simulation procedure will explored in more detail in Chapter~\ref{chap:simulation}, with the inclusion of graph structures and comparison to other synthetic lethal detection approaches.

\FloatBarrier

\section{Graph Structure Methods}
Graph structures have been used in several ways in this project with novel approaches to analysis and simulations. Procedures were developed for statistical and network analysis of gene states in pathway structures. Specifically, the relationships between \gls{siRNA} and \gls{SLIPT} genes were tested within biological pathways in Chapter~\ref{chap:Pathways}. These graph structures were also used in Chapter~\ref{chap:simulation} for the simulation of synthetic lethality to derive correlation structure between simulated gene expression profiles in manner that resembles biological pathways.


\subsection{Upstream and Downstream Gene Detection} \label{methods:pathway_str} 
Comparison of experimental and computational candidate synthetic lethal partner genes within pathway structures arose from the hypothesis that these sets of genes were related by pathway structure. Due to differences in how these candidates were generated, it should not be expected that they detect the identical genes within the candidate biological pathways, rather they may be related by being upstream or downstream of each other. 

Using the Reactome version 52 data \citep{Reactome} as described in Section~\ref{methods:graph_data}, genes identified by each synthetic lethal discovery approach were mapped to the graph structure for the candidate pathways identified in Chapter~\ref{chap:SLIPT} (with subgraphs defined as described in Section~\ref{methods:subgraphs}). To test whether \gls{siRNA} candidate genes were upstream of \gls{SLIPT} candidate genes, shortest paths were traced between each potential pair of these genes in a directed network. The number of genes where the \gls{siRNA} candidate was upstream were scored ``up'' and where the \gls{siRNA} candidate was downstream were scored ``down''.  This procedure enabled counting the total number of shortest paths which supported \gls{siRNA} genes being upstream or downstream of the \gls{SLIPT} genes and measuring the difference between these to determine if there is an imbalance in a particular direction. While this difference is indicative of the number of paths between the gene candidate groups in either direction, alone it is not sufficient to statistically support structure or relationships between \gls{siRNA} and \gls{SLIPT} genes. However, it may be combined with a permutation resampling procedure (as described in Section~\ref{methods:network_permutation}) to test for directional relationships in either direction.

The original version of this procedure excluded gene detected by both approaches since they would count in both directions. Upon further consideration, the intersection genes were restored to being accounted for by the shortest paths counts since they may count unequally to being upstream or downstream of each gene set if there are unequal numbers above or below them in the pathway structure.

\subsubsection{Permutation Analysis for Statistical Significance} \label{methods:network_permutation}
A permutation procedure was developed to randomly assign members of the pathway to \gls{siRNA} and/or \gls{SLIPT} groups, with the same number of each candidate partner gene set as observed in the pathway. These permuted genes are measured for pathway structure between the permuted gene groups as performed for the observed candidates (as performed in Section~\ref{methods:pathway_str}). A distribution of pathway structure relationships expected by chance is generated by permuting iteratively over these pathways. This null distribution can be compared to the observed counts of relationships (in either direction), which yields a permutation p-value as the proportion of permutations in which had value or greater or more extreme magnitude than the observed value.

The null hypothesis is that there is no relationship between these gene groups that would not have occurred had the genes been selected at random. Thus we can test both the alternate hypothesis that the \gls{siRNA} genes were upstream of the \gls{SLIPT} genes or that they are downstream of them.

The permutation procedure does not assume the underlying distribution of the data under the null hypothesis and accounts for the total number of nodes, edges, \gls{siRNA}, and \gls{SLIPT} genes in each pathway network structure. The intersection size of the \gls{siRNA} and \gls{SLIPT} genes was originally not accounted for under the shortest path counts procedure that excluded them. A refined version of this procedure ensured that the number of intersecting genes was equal to the number observed to test for pathway structure without changing the intersection size, the subject of prior analyses.

\subsubsection{Hierarchy Based on Biological Context} \label{methods:pathway_rank}
An alternative approach to pathway structure was performed based on the biological context that genes at the upstream and downstream ends of a pathway perform different functions, such as a kinase signalling cascade recieving signals from external stimuli and passes these on ribosomes or the nucleus. The genes were assigned to a hierarchy to determine if genes of either candidate group (or those with stronger support for either group) performed upstream or downstream functions disproportionately.

A network-based approach was used to determine the pathway hierarchy of genes in a computationally rational way when applied to different biological pathways with a directed graph structure, $G$ (without loops). The diameter of the network (i.e., the length of longest possible shortest path between the most distant genes) was used to identify a gene ($z$) at the downstream end of the pathway (at the end of a diameter spanning shortest path), assigned a hierarchy of: $$hierarchy(z) = 1 + diameter(G)$$ Having identified the downstream end of the pathway, genes upstream (e.g., gene $i$) of this are assigned a hierarchy by the length of their shortest path to this gene, $z$. $$hierarchy(i) = hierarchy(z) - d_{iz}$$ The remaining unassigned genes (e.g., gene $j$) gain the hierarchy of the length of the shortest path downstream from the nearest assigned gene if possible. $$hierarchy(j) = hierarchy(i) + d_{ij}$$ This process may be performed iteratively to fill in pathway hierarchy but it was not necessary to perform further iterations for the candidate synthetic lethal pathways investigated (amenable to this procedure) which exhibited clear directional structure and the small world property (with a low diameter). Thus genes in a pathway graph structure were assigned integer valued hierarchy upstream to downstream by this procedure: $$hierarchy \in \{1, 2, 3, ..., 1 + diameter(G)\}$$
%new paragraph?
This hierarchy of pathway directionality (such as that shown in Figure~\ref{fig:SL_Pathway_PI3K_Ranking}) can be used for comparison with measures of the number of genes of each candidate group and the support for being synthetic lethal partners with either approach.

\subsection{Simulating Gene Expression from Graph Structures} \label{methods:graphsim}
A further refinement of the simulation procedure generated expression data with correlation structure, derived from a known graph structure. %rather than correlated blocks for $\Sigma$ in \texttt{mvtnorm} \citet{mvtnorm}
This enables modelling of synthetic lethal partners within a biological pathway and the investigation of impact of pathway structure on synthetic lethal prediction. A simulated pathway is first constructed as a graph structure, with the \texttt{igraph} R package \citet{igraph}, with the added annotation of the state of the edges (i.e, whether they activate or inhibit downstream pathway members). This simulation procedure was intended for biological pathway members with correlated gene expression (higher than the background of genes in other pathways) but it may also be applicable to modelling protein levels (in a kinase regulation cascade) or substrates and products (in a metabolic pathway).% as it is to gene regulation as it has been applied here.

\begin{figure*}[!htb]
\begin{mdframed}
%  \resizebox{\textwidth}{!}{
         \begin{center}
%
        \subfigure[Activating pathway structure]{%
            \label{fig:simple_graph:first}
            \fbox{
            \includegraphics[width=0.45\textwidth]{{"/home/tomkelly/Documents/PhD Otago Uni/SL_Model/graph_sim_method/simple_graph".png}}
            }
        }%
        \subfigure[Pathway structure including inhibitions]{%
           \label{fig:simple_graph:second}
            \fbox{
           \includegraphics[width=0.45\textwidth]{{"/home/tomkelly/Documents/PhD Otago Uni/SL_Model/graph_sim_method/simple_graph_inhibiting".png}}
           }
        }%
%
    \end{center}
   \caption[Simulated graph structures]{\small \textbf{\textbf{Simulated graph structures.}} A constructed graph structure used as an example to demonstrate the simulation procedure. Activating links are denoted by blue arrows and inhibiting links by red edges.}
%}
\label{fig:simple_graph}
\end{mdframed}
\end{figure*}

First, the graph structure is constructed for simulated data to be generated from (by sampling from a multivariate normal distribution using the \texttt{mvtnorm} R package \citep{Genz2009, mvtnorm}). Throughout this section, the simulation procedure will be demonstrated with the relatively simple constructed graph structure shown in Figure~\ref{fig:simple_graph}. This graph structure visualisation was specifically developed for (directed) iGraph objects in R and has been released in the \texttt{plot.igraph} package and \texttt{igraph.extensions} library (see Table~\ref{tab:computers_r_packages_dev} and Section~\ref{methods:igraph_extensions}). The \texttt{plot\_directed} function allows customisation of plot parameters for each node or edge and mixed (directed) edge types for indicating activation or inhibition. These inhibition links (which often occur in biological pathways) are demonstrated in Figure~\ref{fig:simple_graph:second}.

\begin{figure*}[!hp]
\begin{mdframed}
%  \resizebox{\textwidth}{!}{
         \begin{center}
%
        \subfigure[Activating pathway structure]{%
            \label{fig:simulation_activating:first}
            \includegraphics[width=0.35\textwidth]{{"/home/tomkelly/Documents/PhD Otago Uni/SL_Model/graph_sim_method/simple_graph".png}}
        }%
        \subfigure[Distance matrix ]{%
            \label{fig:simulation_activating:second}
            \includegraphics[width=0.35\textwidth]{{"/home/tomkelly/Documents/PhD Otago Uni/SL_Model/graph_sim_method/dist_mat".png}}
        }%
        
        \subfigure[Sigma, $\Sigma$ (expected correlation)]{%
           \label{fig:simulation_activating:third}
           \includegraphics[width=0.35\textwidth]{{"/home/tomkelly/Documents/PhD Otago Uni/SL_Model/graph_sim_method/sigma_mat".png}}
        }%
	\subfigure[Simulated correlation structure]{%
           \label{fig:simulation_activating:fifth}
           \includegraphics[width=0.35\textwidth]{{"/home/tomkelly/Documents/PhD Otago Uni/SL_Model/graph_sim_method/expr_cor_mat".png}}
        }%
        	
	\subfigure[Simulated expression data]{%
            \label{fig:simulation_activating:fourth}
            \includegraphics[width=0.35\textwidth]{{"/home/tomkelly/Documents/PhD Otago Uni/SL_Model/graph_sim_method/expr_mat".png}}
        }%
        \subfigure[Simulated gene function calls]{%
           \label{fig:simulation_activating:sixth}
           \includegraphics[width=0.35\textwidth]{{"/home/tomkelly/Documents/PhD Otago Uni/SL_Model/graph_sim_method/expr_disc_mat".png}}
        }%
    \end{center}
   \caption[Simulating expression from a graph structure]{\small \textbf{\textbf{Simulating expression from a graph structure.}} An example graph structure is used to derive a correlation structure from the relative distances between nodes and simulate continuous gene expression with sampling from the multivariate normal distribution.}
%}
\label{fig:simulation_activating}
\end{mdframed}
\end{figure*}

The simulation procedure is designed to use such graph structures to inform development of a ``Sigma'' variance-covariance matrix ($\Sigma$) for sampling from a multivariate normal distribution (using the \texttt{mvtnorm} R package). Given a graph structure (or adjacency matrix), such as Figure~\ref{fig:simulation_activating:first}, a relation matrix is calculated based on distance such that nearer nodes are given higher weight than farther nodes. For the purposes of this thesis a geometrically decreasing (relative) distance weighting is used, with each more distant node being related by $\sfrac{1}{2}$ compared to the next nearest as shown in Figure~\ref{fig:simulation_activating:second}. However, an arithmetically decreasing (absolute) distance weighting is also available in the \texttt{graphsim} R package release of this procedure.

A $\Sigma$ matrix is derived from this distance weighting matrix, creating a matrix (with a diagonal of $1$) where each node has a variance and standard deviation of 1. Thus covariances between adjacent nodes are assigned by a correlation parameter and the remaining matrix based on weighting these correlations with by the distance matrix (or the nearest ``positive definite'' matrix). For the purposes of this thesis, the correlation parameter is $0.8$ unless otherwise specified (as used for the example in Figure~\ref{fig:simulation_activating:third}). This $\Sigma$ matrix is used to sample from a multivariate normal distribution with each gene having a mean of $0$, standard deviation $1$, and covariance within the range $[0,1]$ such that they are correlations. This procedure generates a simulated (continuous normally distributed) expression profile for each node (as shown in Figure~\ref{fig:simulation_activating:fourth}) with corresponding correlation structure (Figure~\ref{fig:simulation_activating:fifth}). The simulated correlation structure closely resembles the expected correlation structure (Sigma in~\ref{fig:simulation_activating:third}) even for the relatively modest sample size ($N=100$) illustrated in~\ref{fig:simulation_activating}. Once a simulated gene expression dataset has been generated (as in Figure~\ref{fig:simulation_activating:fourth}), then a discrete matrix of gene function can be constructed with a functional threshold quantile to simulate functional relationships of synthetic lethality (as shown in Figure~\ref{fig:SL_Model_Expression}). For the purposes of this thesis, this threshold is the 0.3 quantile (as discussed in Section~\ref{methods:SL_Model}) which generates functional discrete matrices such as those used for synthetic lethal simulation in Section~\ref{methods:simulating_SL} (as shown Figure~\ref{fig:simulation_activating:sixth})

\begin{figure*}[!hp]
\begin{mdframed}
%  \resizebox{\textwidth}{!}{
         \begin{center}
%
        \subfigure[Pathway structure with inhibition]{%
            \label{fig:simulation_inhibiting:first}
            \includegraphics[width=0.35\textwidth]{{"/home/tomkelly/Documents/PhD Otago Uni/SL_Model/graph_sim_method/simple_graph_inhibiting".png}}
        }%
        \subfigure[Distance matrix ]{%
            \label{fig:simulation_inhibiting:second}
            \includegraphics[width=0.35\textwidth]{{"/home/tomkelly/Documents/PhD Otago Uni/SL_Model/graph_sim_method/dist_mat".png}}
        }%
        
        \subfigure[Sigma, $\Sigma$ (expected correlation)]{%
           \label{fig:simulation_inhibiting:third}
           \includegraphics[width=0.35\textwidth]{{"/home/tomkelly/Documents/PhD Otago Uni/SL_Model/graph_sim_method/sigma_mat_inhibiting".png}}
        }%
	\subfigure[Simulated correlation structure]{%
           \label{fig:simulation_inhibiting:fifth}
           \includegraphics[width=0.35\textwidth]{{"/home/tomkelly/Documents/PhD Otago Uni/SL_Model/graph_sim_method/expr_cor_mat_inhibiting".png}}
        }%
        
        \subfigure[Simulated expression data]{%
            \label{fig:simulation_inhibiting:fourth}
            \includegraphics[width=0.35\textwidth]{{"/home/tomkelly/Documents/PhD Otago Uni/SL_Model/graph_sim_method/expr_mat_inhibiting".png}}
        }%
        \subfigure[Simulated gene function calls]{%
           \label{fig:simulation_inhibiting:sixth}
           \includegraphics[width=0.35\textwidth]{{"/home/tomkelly/Documents/PhD Otago Uni/SL_Model/graph_sim_method/expr_inhib_disc_mat".png}}
        }%
    \end{center}
   \caption[Simulating expression from graph structure with inhibitions]{\small \textbf{\textbf{Simulating expression from graph structure with inhibitions.}} An example graph structure is used to derive a correlation structure from the relative distances between nodes and simulate continuous gene expression with sampling from the multivariate normal distribution.}
%}
\label{fig:simulation_inhibiting}
\end{mdframed}
\end{figure*}

The simulation procedure (depicted in Figure~\ref{fig:simulation_activating}) is amenable to pathways containing inhibition links (as shown in Figure~\ref{fig:simulation_inhibiting}) with several refinements. With the inhibition links (as shown in Figure~\ref{fig:simulation_inhibiting:first}), distances are calculated in the same manner as before (Figure~\ref{fig:simulation_inhibiting:second}) with inhibitions accounted for by iteratively multiplying downstream nodes by $-1$ to form blocks of negative correlations (as shown in Figures~\ref{fig:simulation_inhibiting:third} and~\ref{fig:simulation_inhibiting:fifth}). As before, a multivariate normal distribution with these negative correlations can be sampled to generate simulated data (as shown in Figures~\ref{fig:simulation_inhibiting:fourth} and~\ref{fig:simulation_inhibiting:sixth}).  

These simulated datasets are amenable to simulating synthetic lethal partners of a query gene within a graph network. The query gene is assumed to be separate from the graph network pathway and is added to the dataset using the procedure in Section~\ref{methods:simulating_SL}. Thus we can simulate known synthetic lethal partner genes within a synthetic lethal partner pathway structure.

\iffalse

	\includegraphics{{"/home/tomkelly/Documents/PhD Otago Uni/SL_Model/graph_sim_method/dist_mat".png}}
	\includegraphics{{"/home/tomkelly/Documents/PhD Otago Uni/SL_Model/graph_sim_method/sigma_mat".png}}
		\includegraphics{{"/home/tomkelly/Documents/PhD Otago Uni/SL_Model/graph_sim_method/expr_mat".png}}
			\includegraphics{{"/home/tomkelly/Documents/PhD Otago Uni/SL_Model/graph_sim_method/expr_cor_mat".png}}
		\includegraphics{{"/home/tomkelly/Documents/PhD Otago Uni/SL_Model/graph_sim_method/expr_disc_mat".png}}
		
	\includegraphics{{"/home/tomkelly/Documents/PhD Otago Uni/SL_Model/graph_sim_method/state_matrix_inhibiting".png}}
	\includegraphics{{"/home/tomkelly/Documents/PhD Otago Uni/SL_Model/graph_sim_method/dist_mat".png}}
		\includegraphics{{"/home/tomkelly/Documents/PhD Otago Uni/SL_Model/graph_sim_method/sigma_mat_inhibiting".png}}
		\includegraphics{{"/home/tomkelly/Documents/PhD Otago Uni/SL_Model/graph_sim_method/expr_inhib_mat".png}}
			\includegraphics{{"/home/tomkelly/Documents/PhD Otago Uni/SL_Model/graph_sim_method/expr_inhib_cor_mat".png}}
		\includegraphics{{"/home/tomkelly/Documents/PhD Otago Uni/SL_Model/graph_sim_method/expr_inhib_disc_mat".png}}
	
\fi	

\FloatBarrier

\section{Customised Functions and Packages Developed} \label{methods:r_packages}

[Move to Appendix?]

Various R packages have been developed throughout this thesis using \texttt{devtools} \citep{devtools} and \texttt{roxygen} \citep{roxygen} to enable reproducibility of customised analysis and visualisation. Many of these have the added benefit of the functions being documented, demonstrated in example vignettes, and released on GitHub to enable the research community to access utilise them in their own analysis. These are summarised in Table~\ref{tab:computers_r_packages_dev}, along with the corresponding urls for their GitHub repository which contains a README file with instructions for installation with the \texttt{devtools} R package \citep{devtools} and links to the relevant vignette(s) where available.

\subsection{Synthetic Lethal Interaction Prediction Tool}
The statistical methodology for detection of synthetic lethality in gene expression data (\gls{SLIPT}) is one of the main novel procedures developed in this thesis, as described in Section~\ref{methods:SLIPT}. The \texttt{slipt} R package has been prepared for release to accompany a publication demonstrating the applications of the methodology for identifying candidate interacting genes and pathways with \textit{CDH1} in breast cancer \citep{TCGA2012}.

\gls{SLIPT} is amenable to analysis of any effectively continuous measure of gene activity (e.g., microarray, RNA-Seq, protein abundance, or pathway metagenes). Executing \texttt{slipt} is straightforward: the \texttt{prep\_data\_for\_SL} function scores samples as ``low'', ``medium'', or ``high'' for each gene, then the \texttt{detect\_SL} function tests a given query gene against all potential partners by performing the chi-squared test and directional conditions. This function returns a table summarising the observed and expected sample numbers used for the directional criteria, the $\chi^2$ values, and corresponding p-values including adjusting for multiple comparisons. The \texttt{count\_of\_SL} and \texttt{table\_of\_SL} functions serve to facilitate summary and extraction of the positive \gls{SLIPT} hits, respectively, from the table of predictions of synthetic lethal partners.

The \gls{SLIPT} methodology in this package release has been used in later analyses rather than the corresponding source R code, including use on remote machines and upon simulated data. In particular, the functions in the package facilitate alterations to parameters, such as the proportion of samples called as exhibiting low or high gene activity. This release support reproducibility and enables wider use of \gls{SLIPT} in future investigations into other disease genes.

\subsection{Data Visualisation}
Customisations to existing data visualisations in R have been developed to present data throughout this thesis. The \texttt{vioplotx} and \texttt{heatmap.2x} packages are enhancements of the \texttt{vioplot} package \citep{vioplot} and \texttt{heatmap.2} provided by the \texttt{gplots} package \citep{gplots}. 

\begin{figure*}[!p]
\begin{mdframed}
%  \resizebox{\textwidth}{!}{
         \begin{center}
%
        \subfigure[Customised violin plot]{%
            \label{fig:vioplot:first}
            \includegraphics[width=0.45\textwidth]{{"vioplot1".png}}
        }%
        \subfigure[Split violin plot]{%
            \label{fig:vioplot:second}
            \includegraphics[width=0.45\textwidth]{{"vioplot2".png}}
        }%
        \end{center}
   \caption[Demonstration of violin plots with custom features]{\small \textbf{Demonstration of violin plots with custom features.} An example of the \texttt{iris} dataset is plotted to show the custom features of the \texttt{vioplotx} package including a) individual colour, shape and size parameters of each violin, scaling violin widths by area, and b) splitting violins to compare subsets of data.}
%}
\label{fig:vioplot}
\end{mdframed}
\end{figure*}

The \texttt{vioplotx} package provides an alternative visualisation (of continuous variables against categories) to the more familiar boxplot, showing variability of the data by the width of the plots. As demonstrated in Figure~\ref{fig:vioplot}, the customised version enables separate plotting parameters for each violin with vector inputs for colour, shape, and size of various elements of the median point, central boxplot, borders, and fill colour for the violin. Scaling violin width to adjust violin area and splitting data by a second categorical variable is also enabled. This function is intended to be backwards compatible with the inputs of \texttt{vioplot} (applying scalar inputs across all violins) and \texttt{boxplot} (by enabling formula inputs as an S3 method). Each of these features is demonstrated with examples in respective vignettes on the package \href{https://github.com/TomKellyGenetics/vioplotx}{GitHub repository}.

\begin{figure*}[!p]
\begin{mdframed}
%  \resizebox{\textwidth}{!}{
         \begin{center}
            \includegraphics[width=0.9 \textwidth]{{"heatmap2x".png}}
        \end{center}
   \caption[Demonstration of annotated heatmap]{\small \textbf{Demonstration of annotated heatmap}. The example heatmap depicts the additional row and column annotation bars enabled by \texttt{heatmap.2x}, extending the features of \texttt{gplots} with backwards compatible inputs.}
%}
\label{fig:heatmap.2x} 
\end{mdframed}
\end{figure*}

The \texttt{heatmap.2x} provides extensions for annotation colour bars for both the rows and columns (as shown in Figure~\ref{fig:heatmap.2x}). Multiple bars are enabled on both axes with matrix inputs (rather than single vector for \texttt{heatmap.2}) which facilitates additional plotting of gene and sample characteristics for comparison with correlation matrices, expression profiles, or pathway metagenes. Annotation bar inputs correspond to their orientation on the plot, each colour bar is provided as a column for the row annotation on the left of the heatmap and as a row for the column annotation on top of the heatmap. Row and column annotation bars are labelled with the column or row names respectively. Additional parameters enable resizing of these annotation bar labels and control of reordering columns for if samples are ordered in advance (e.g., ranked by a metagene or split into groups clustered separately).  These features were used through this thesis and are provided in a package \href{https://github.com/TomKellyGenetics/heatmap.2x}{GitHub repository}.


\FloatBarrier

\subsection{Extensions to the iGraph Package} \label{methods:igraph_extensions}
The following features were developed during this thesis using ``iGraph'' data objects, building upon the \texttt{igraph} package \citep{igraph}. These have been released as separate packages for each respective procedure and can be installed together as a collection of extensions to the \texttt{igraph} package.

\subsubsection{Sampling Simulated Data from Graph Structures}
The \texttt{graphsim} package implements the procedure for simulating gene expression from graph structures (as described in Section~\ref{methods:graphsim}). By default, this derives a matrix with a geometrically decreasing weighting by distance (by shortest paths) between each pair of nodes with. An absolute decreasing weighting is also available with the option of to derive correlation structures from adjacency matrices or the number of links common partners (i.e., size of the shared ``neighbourhood'' \citep{Hell1976}) between each pair of nodes. Functions to compute these are called directly by passing parameters to them when running the \texttt{generate\_expression} or \texttt{make\_sigma\_mat} commands. This package enables simulating expression data directly from a graph structure (with the intermediate steps automated) or generating $\Sigma$ parameters for \texttt{mvtnorm} from graph structures or matrices derived from them. These functions support assigning activating or inhibiting to each edge (with a \texttt{state} parameter).

\subsubsection{Plotting Directed Graph Structures}
The \texttt{plot.igraph} package provides the \texttt{plot\_directed} function specifically developed for directed graph structures and to plot activating or inhibiting for each edge (as described in Section~\ref{methods:graphsim}). As shown in Figure~\ref{fig:simple_graph2}, this function supports separate plotting parameters for each node, node label, and edge. This includes colours of node fill, border, label text, and edges and size of nodes, edge widths, arrowhead lengths, and font size of labels. The  \texttt{state} parameter for assigning activating or inhibiting to each edge determines whether edges are dipected with 30\textdegree\ or 90\textdegree\ arrowheads. Colours are assigned separately and so they may be customised. Vectorised parameters are applied across each node or edge, whereas scalar parameters apply the same plotting parameters across them. The default layout function is \texttt{layout.fruchterman.reingold} but any layout function supported by \texttt{plot} function in \texttt{igraph} \citep{igraph} is compatible such as \texttt{layout.kamada.kawai} used to implement the Kamada--Kawai algorithm \citep{Kamada1989} for graph plots throughout this thesis.
 

 \begin{figure*}[!htp]
\begin{mdframed}
%  \resizebox{\textwidth}{!}{
         \begin{center}
         \fbox{
          \includegraphics[width=0.5\textwidth]{{"/home/tomkelly/Documents/PhD Otago Uni/SL_Model/graph_sim_method/simple_graph_inhibiting".png}}
	  }
	  \end{center}
   \caption[Simulating graph structures]{\small \textbf{\textbf{Simulating graph structures.}} An example graph structure which will be used throughout demonstrating the simulation procedure from graph structures. Here activating links are denoted by blue arrows and inhibiting links by red edges.}
%}
\label{fig:simple_graph2}
\end{mdframed}
\end{figure*}

\FloatBarrier
 
\subsubsection{Computing Information Centrality} 
The shortest paths of a network are computed by the \texttt{igraph} package \citet{igraph} which can be extended to calculate the network efficiency but is not provided by the package itself (ss described in Section~\ref{methods:network_metrics}). The ``information centrality'' of a vertex is computed as the relative change in the network efficiency when the vertex is removed. Information centrality is calculated iteratively for each node and the sum of information centrality for each vertex is the information centrality for the network. These metrics are provided by the \texttt{info.centrality} package.

\subsubsection{Testing Pathway Structure with Permutation Testing}
A network-based procedure developed was used for comparison of \gls{siRNA} and \gls{SLIPT} candidate genes in a pathway structure. Such pathway structure relationships were tested by computing the number of shortest paths between two different groups of nodes in either direction within a graph . This pathway relationship metric was implemented in the \texttt{pathway.structure.permutation} package with permutation testing (as described in sections~\ref{methods:pathway_str} and~\ref{methods:network_permutation}). 

\subsubsection{Metapackage to Install iGraph Functions}
These features may be installed together with \texttt{igraph.extensions} which can be accessed from a \href{https://github.com/TomKellyGenetics/igraph.extensions}{GitHub repository}. This meta-package installs \texttt{igraph} \citep{igraph} and the packages described in Section~\ref{methods:igraph_extensions} including their dependencies for matrix operations and statistical procedures: \texttt{Matrix}, \texttt{matrixcalc}, and \texttt{mvtnorm} \citep{Matrix, matrixcalc, Genz2009, mvtnorm}.


%simulation 2015 committee meeting
\iffalse
\section{Developing a Synthetic Lethal detection methodology}

\subsection{Testing Multivariate Normal Simulation of Synthetic lethality}

We have developed a model of synthetic lethality in gene expression data based on sampling a Multivariate Normal distribution.  This enables simulation of statistically testing for synthetic lethal genes where the true and false positives are known, discovery of the expected test statistic distributions for different conditions, educated thresholds for public data analysis, and building a complex model with known correlation structure between genes.  Sampling a small number of genes from this model shows, in Figure 4, that synthetic lethality is detectable with in a simple model.

Figure 4.  Chi-Square (upper) and p-values (lower) distributions show that synthetic lethal partners (red) are distinguishable from correlated (blue) and other genes (black) in an example simulation of sampling 1000 samples and 100 genes, from a multivariate normal distribution with 1 (left), 2 (centre), and 3 (right) synthetic lethal partners respectively, showing that synthetic lethal genes become more difficult to detect if there are more true partners.

Figure 5.  Chi-Square (upper), FDR adjusted p-values (centre), and Holm adjusted p-values (lower) show that show that synthetic lethal partners (red) are distinguishable from correlated (blue) and other genes (black) are distinguishable replicated across 1000 replicate simulated sampling of 1000 samples and 100 genes, from a multivariate normal distribution with 1 (left), 2 (centre), and 3 (right) synthetic lethal partners respectively, showing synthetic lethal genes become more difficult to detect in with more true partners but adjusting p-values may be too stringent an approach to this.

Having shown that the Chi-Square test is capable of detecting synthetic lethality, Figure 5 shows that detecting synthetic lethality in a simple case is largely robust and reproducible across many replicates with synthetic lethal and correlated genes clearly having higher test statistic scores and lower adjusted p-values than the null distribution of non-synthetic lethal genes when there are only 1 or 2 synthetic lethal partners.  While it is promising that correlated genes and synthetic lethal partners could be distinguished from other genes in a simple case, there is also indication that true synthetic lethal partners (candidates as robust drug targets) and their correlated genes (or pathways) could be distinguished by test statistic.

However, such clear evidence of synthetic lethality by co-loss under-representation is rarely detected in public data analyses, indicating cryptic additional synthetic lethal genes compensating for the loss of both the query and putative synthetic partner.  Therefore higher-order synthetic lethal is potentially very common, difficult to detect, and confounding attempts to identify synthetic lethal pairs from gene expression data.  In Figure 5, more than 3 synthetic lethal partners will be difficult to identify directly with a Chi-Square test.  Although deeper understanding of the system could still enable use of the procedure to prioritise small numbers of candidate genes, estimate the number of underlying true synthetic partners, and identify the biological pathways interacting with a gene to focus complementary experimental approaches.

With higher number of true synthetic lethal genes there is no clear threshold for Chi-Square values (or associated p-values) to detect synthetic lethality and choosing any threshold is a trade-off between sensitivity (ensuring all true positives are detected) and specificity (reducing the number of false positives detected).  Receiver operating characteristic (ROC) curves, as shown in Figures 6 and 7, summarise this trade-off to show the statistical performance of a test where the true synthetic lethal genes are known in the simulated data.  Performance of a statistical test is measured as the area under the ROC (AUROC) curves, as shown in Figures 8 and 9, to compare performance across simulations for different parameters such as type of model, correlation structure, the total number of genes, sample size and number of true synthetic lethal genes.  A random predictor has an AUROC of 0.5, whereas an ideal predictor has an AUROC of 1.0, so intermediate values are expected.

\subsection{Receiver Operating Characteristic Curves}

Figure 6.   ROC curves showing statistical performance (by area under the curve) of a synthetic lethal simulation based on sampling a Binomial distribution, with 20,000 genes, averaged over 1000 replicates, sample size (1000, 2000, 5000, or 10,000) and number of synthetic lethal genes (up to 100) varies by panel and colour showing better performance with fewer synthetic lethal genes or higher sample size.    

Figure 7.   ROC curves of a synthetic lethal simulation based on sampling a Multivariate Normal distribution, with 20,000 genes, averaged over 1000 replicates, sample size and number of synthetic lethal genes varies by panel and colour showing better performance than a Binomial model and similar performance with correlation structure (upper panes).

Figure 8.  Comparison of Binomial (red) and Multivariate Normal models with (blue) and without (green) correlation structure by simulation with 1000 samples, 20,000 genes, sample size varied by pane, and number of synthetic lethal partners on the x axis where performance on the y axis is measured as the AUROC showing better performance in the Multivariate Normal model than the Binomial model and similar performance in the Multivariate Normal model with correlation structure added for all simulation parameters.  There was better performance with fewer synthetic lethal partners or higher sample size with both Multivariate Normal models.   

Figure 6 shows performance of an earlier model based on the Binomial distribution for gene function calls, based on similar a Normally distributed model of gene expression which called gene function from an arbitrary expression cut-off.  This model is shown for comparison with Multivariate model we have chosen to develop since the Multivariate model, as shown in Figure 7, has better performance, allows the inclusion of correlation structure expected in gene expression data for biological pathways, and could have variable gene function cut-offs.  

Figures 7 and 8, show that the Multivariate model which corrects this effect by specifying synthetic lethal genes differently performs better in simulations, even with correlation structure expected to disrupt the synthetic lethal detection.  There is indication in Figure 8 that correlation structure even improves the performance of simulations.  Although replicated across parameters, the difference in performance of simulations with correlated genes (with each synthetic lethal partner) is marginal and the number of correlated genes is still vastly outnumbered by the total number of genes (20,000 modelling a complete mammalian genome).  Simulations with fewer total genes may show the impact of correlated genes more clearly, which is biologically plausible since some co-regulated pathways do involve a substantial proportion of the genome.

As indicated, the models behave as expected when performing better when simulated with higher sample size and fewer true synthetic lethal genes.  As summarised in Figure 9, this behaviour occurs in simulation with all of the models discussed above.  The number of synthetic lethal partners impacts performance with a sigmoidal decay where­­ higher sample size (albeit approaching the limit of feasible genomic-scale projects) markedly delay decay of AUROC towards random 0.5.  Therefore a large sample size is crucial for bioinformatics synthetic lethal discovery.  Only a small number of synthetic lethal partners will be detectable with a gene-centric approach motivating pathway-centric approaches and accounting for pathway structure, which has shown be more reproducible between model organism experiments (Dixon et al. 2009).  However, whether potential false positives are more likely to be correlated genes or occur due to the sheer number of true negatives (and multiple tests) is unclear.  The impact of correlation structure on the simulated data is explored in detail below in Figures 10-12 and the results of these simulations repeated is shown in Figure 13.    Figure 9.  Summary of effect of sample size and number of synthetic lethal partners on performance of simulations for prediction of synthetic lethality by AUROC on continuous scale (left) and as a barplot (right) showing that sample size (by colour) and number of synthetic lethal partners (x axis) affects performance as expected in which was replicated across all 3 models discussed above.

\subsection{Simulated Expression Heatmaps}

In Figures 10-12 below, simulations are summarised with expected (Sigma) and generated (Correlation) structure of gene expression patterns in the top figures.  The following line shows how the expression and gene function calls have been distributed with correlation structure and ordering samples (columns) to ensure a synthetic lethal partner or query gene is active in each sample.

Figure 10.  Simulation for 1 SL partner (100 genes, 1000 samples)

Figure 11.  Simulation for 2 SL partners (100 genes, 1000 samples)

Figure 12.  Simulation for 3 SL partners (100 genes, 1000 samples)

As shown in the Figures 10-12, the correlation structure of the simulated gene expression data (upper right) largely reflects the expected sigma matrix (upper left) used to specify the variation in the Multivariate Normal distribution with some variation due to low sampling error.  The Sigma and correlation matrices show blocks of correlated genes with each synthetic lethal partner where there are 1, 2, or 3 synthetic lethal partners in Figures 10, 11, and 12 respectively.  In the gene expression heatmap (lower right) and associated discrete gene function calls based on a threshold of the 30\% quantile (lower left), the sample (column) ordering shows how samples were ordered so at least one synthetic lethal gene is active in all query deficient samples.  The row (gene) ordering is based on a Chi-Square test statistic value and odds-ratio sign (with negative genes at the top), apart from Query gene at the top (with positive odds-ratio).  The Chi-Square values are shown on the outer colour bar on a log scale and the inner colour bar annotates the known gene class in the simulation: query (blue), synthetic lethal (red), correlated (orange), and other (green).

With 1 synthetic lethal partner, in Figure 10, the relationship between synthetic lethal (and correlated genes with the Query gene is clear and detectable with Chi-Square test (as shown with the colour bars) as expected.  The relationship is clearer in the true synthetic lethal partner showing that it should be distinguishable from confounding correlated genes.  With multiple synthetic lethal genes, as shown in Figures 11 and 12, the true synthetic lethal partner is less related to the expression profile of the Query gene and the co-loss under-representation is more difficult to detect since the number of co-occuring loss of synthetic lethal genes expected (even in Query functional samples is low).  In these examples, the Chi-Square test still correctly identifies synthetic lethal genes with the highest test statistic, although with a less well defined cut-off and it may not be reproducible (as discussed above).  This is consistent with more synthetic lethal partners being able to recover function and ensure cell survival which is plausible given the evolutionary robustness of molecular networks, difficulty detecting individual gene pairs in gene expression data, and rates of recurrence or drug resistance in cancer patients.  Therefore we have to consider cryptic synthetic lethal genes compensating for Query and candidate synthetic lethal partners due to higher-order genetic redundancy, cancer genomic evolution and cellular heterogeneity.

\subsection{Replication Simulation Heatmap}

The declining performance in ROC curves with more synthetic lethal genes shows that the ability to robustly distinguish synthetic lethal genes from other genes (including their correlated genes) declines as the synthetic lethal genes do not consistently have a higher Chi-Square test statistic across replicate sampling simulations.  Although it is noted that increased sample size can compensate for this decline, increasing the number of expected co-loss events and sensitivity of the procedure.  The effect of total gene number, impact of correlation structure, and reproducibility of Chi-Square tests across replicate sampling simulations is explored below.

Figure 13 is composed of columns of side colour bars ordered by Chi-Square and odds-ratio sign (with Query in the corrected position at the bottom) as shown in Figures 10-12 with separate columns for repeated sampling with different parameters.  Figure 13 is an example of this visualisation of simulations for a small number of genes (100) and replicates (10 each for 1 to 10 synthetic lethal partners).  Even in this small simulation, we see many of the processes discussed above summarised: the effect of number of synthetic lethal genes on Chi-Square tests, power to detect synthetic lethal and other correlated genes, decaying reproducibility and variation across replicates, lack of a clear threshold, and importance of directional conditions (e.g., odds-ratio sign) to distinguish synthetic lethal and co-expressed genes.  This visualisation is an effective way to capture the simulation process and compare conditions which will be valuable for more complex correlation structure and comparison to public data Chi-Square distributions.
    
Figure 13.  Comparison of simulation across various parameters for sampling a Multivariate Normal model for 100 genes and 1000 samples with correlation structure with 10 replicates (columns) for each number of synthetic lethal partners (1 to 10 in the top colour bar) with genes sorted by chi-squared value (and odds-ratio negative at the top) this shows preferential sorting of synthetic lethal partners (red) and correlated (orange) genes near the top (on the left) for lower numbers of synthetic lethal partners which becomes less clear or consistent across replicates for higher numbers of synthetic lethal partners, reflected in less variation in chi-square values (shown in log-scale on the right) and lack of a clear prediction threshold, however positive odds-ratio genes show no preference except for the query gene associated itself as expected.

This framework may also be useful to compare different analyses of public data and infer the true number of synthetic lethal partners from the distribution of test statistic scores.  With an effective visualisation, we can further explore more complex correlation structures (as shown in the supplementary Figures S1 and S2).  This will be important to develop simulated data as similar to empirical data as possible, to test whether synthetic lethal and correlated genes are robustly detectable, and discover effective drug targets (which are repeatable across a cohort, tissues or species).  The impact of high-order synthetic lethality, genetic background and variation between replicates indicates that more care has to be taken interpreting experimental model systems and genomics analysis will be valuable to ensure candidate drug targets are suitable for clinical application.  We show below that this visualisation scales up and shows similar effects for number of synthetic lethal genes in more replicates (Figure 14), more total genes (Figure 15), and both (Figure 16).
    
Figure 14.  Comparison of simulation across various parameters for sampling a Multivariate Normal model for 100 genes and 1000 samples with correlation structure with 100 replicates (columns) for each number of synthetic lethal partners (1 to 10 in the top colour bar) with genes sorted by chi-squared value (and odds-ratio negative at the top) this shows preferential sorting of synthetic lethal partners (red) and correlated (orange) genes near the top (on the left) for lower numbers of synthetic lethal partners which becomes less clear or consistent across replicates for higher numbers of synthetic lethal partners, reflected in less variation in chi-square values (shown in log-scale on the right) and lack of a clear prediction threshold, however positive odds-ratio genes show no preference except for the query gene associated itself as expected.  
   
Figure 15.  Comparison of simulation across various parameters for sampling a Multivariate Normal model for 1000 genes and 1000 samples with correlation structure with 10 replicates (columns) for each number of synthetic lethal partners (1 to 10 in the top colour bar) with genes sorted by chi-squared value (and odds-ratio negative at the top) this shows preferential sorting of synthetic lethal partners (red) and correlated (orange) genes near the top (on the left) for lower numbers of synthetic lethal partners which becomes less clear or consistent across replicates for higher numbers of synthetic lethal partners, reflected in less variation in chi-square values (shown in log-scale on the right) and lack of a clear prediction threshold, however positive odds-ratio genes show no preference except for the query gene associated itself as expected.
    
Figure 16.  Comparison of simulation across various parameters for sampling a Multivariate Normal model for 1000 genes and 1000 samples with correlation structure with 100 replicates (columns) for each number of synthetic lethal partners (1 to 10 in the top colour bar) with genes sorted by chi-squared value (and odds-ratio negative at the top) this shows preferential sorting of synthetic lethal partners (red) and correlated (orange) genes near the top (on the left) for lower numbers of synthetic lethal partners which becomes less clear or consistent across replicates for higher numbers of synthetic lethal partners, reflected in less variation in chi-square values (shown in log-scale on the right) and lack of a clear prediction threshold, however positive odds-ratio genes show no preference except for the query gene associated itself as expected.
\fi