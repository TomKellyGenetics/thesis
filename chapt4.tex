\chapter{Pathway Structure of Synthetic Lethal Genes}
\label{chap:Pathways}

\section{Abstract}

Effective screening, prediction, and analysis of synthetic lethal interactions are a crucial part of developing next generation anti-cancer strategies. Therefore, we propose developing a computational statistical procedure to identify synthetic lethal interactions and construct gene networks. This will enable the development of personalised medicine targeted to particular molecular aberrations. Genetic tests and genomics have the potential to revolutionise cancer screening, diagnosis, and prognostics; targeted therapeutics, similarly, have applications in prevention and therapy of sporadic or hereditary cancers with known molecular properties.

Construction of genetic interaction networks is important to understand the functional complexity of cellular and molecular biology, particular how it relates to existing networks for protein binding, gene regulation, genetic interaction, and gene co-expression. Comparison of networks between species will enable use of known interactions in yeast and understanding of the evolutionary importance of genetic interactions. Comparing protein and gene networks is valuable to determine which are more effective for prediction of drug targets and development of biomarkers.

Comparison of networks between cells could lead to clinically significant findings and development of effective anti-cancer drugs: both comparison between normal and cancerous cells from the same tissue and comparison across tissues. Among the most exciting applications is the use to prioritise drug screening and repurpose existing drugs. Genetic interaction discovery and gene network analysis also have the potential to develop predictions for a drug's therapeutic index, tumour specificity, tissue independence, and synergistic interactions based on known targets.

\section{Background}

\section{Reactome Network structure and Information Centrality as a measure of gene essentiality}

Network structure is another useful strategy to analyse gene function and this has been used to investigate network properties of a network constructed from of Reactome pathways imported with the paxtoolsr R package (Demir et al. 2010). Most notably, information centrality which has been proposed as a measure of gene essentiality was calculated as performed by Kranthi et al. (2013) using the efficiency and shortest path between each pair or nodes in the network before and after a node of interest is removed to test the importance of a node to network connectivity. Reactome contains substrates and cofactors in addition to genes or proteins, supporting the idea of centrality as a measure of essentiality, a number nodes with the highest centrality were essential nutrients including Mg$2^+$, Ca$2^+$, Zn$2^+$,  and Fe$3^+$.

\section{Synthetic lethal genes in synthetic lethal pathways}

\section{Methods}
\subsection{Sourcing graph structure data}
\subsection{Constructing pathway subgraphs}
\subsection{Centrality Measures}
\subsection{upstream and downstream gene detection}
\subsection{permutation analysis}

\section{Centrality and connectivity of synthetic lethal genes}

\section{Upstream or downstream synthetic lethal candidates}

\section{Hierachical approach}
%closer to membrane or nucleus

\section{Discussion}

\section{Conclusion}