\chapter{Synthetic Lethal Pathway Structure}
\label{chap:Pathways}


\paragraph{Aims}

  \begin{itemize}
   \item Synthetic Lethal Genes within a Biological Pathway Structure
   
   \bigskip
   
   \item Importance and Connectivity of Synthetic Lethal Genes within Pathway Networks
   
   \bigskip
   
   \item Upstream and Downstream Relationships between SLIPT and siRNA Candidates
  \end{itemize}

\paragraph{Summary}

  \begin{itemize}
   \item Synthetic Lethal genes were explored within a graph structures for key pathways identified previously 
   
   \bigskip
   
   \item In some cases these graph structures appeared to have relationships between synthetic lethal genes  
   
   \bigskip
   
   \item However, no existing network metrics of importance and connectivity with the networks were elevated significantly for Synthetic Lethal genes
   
   \bigskip
   
   \item Nor was there significant evidence of upstream and downstream relationships between SLIPT and siRNA Candidates in a shortest path permutation analysis
  \end{itemize}

\section{Reactome Network structure and Information Centrality as a measure of gene essentiality}

Network structure is another useful strategy to analyse gene function and this has been used to investigate network properties of a network constructed from of Reactome pathways imported with the paxtoolsr R package (Demir et al. 2010). Most notably, information centrality which has been proposed as a measure of gene essentiality was calculated as performed by Kranthi et al. (2013) using the efficiency and shortest path between each pair or nodes in the network before and after a node of interest is removed to test the importance of a node to network connectivity. Reactome contains substrates and cofactors in addition to genes or proteins, supporting the idea of centrality as a measure of essentiality, a number nodes with the highest centrality were essential nutrients including Mg\textsuperscript{2$+$}, Ca\textsuperscript{2$+$}, Zn\textsuperscript{2$+$},  and Fe\textsuperscript{3$+$}.

\section{Synthetic lethal genes in synthetic lethal pathways}

\section{Centrality and connectivity of synthetic lethal genes}

\section{Upstream or downstream synthetic lethal candidates}

\section{Hierachical approach}
%closer to membrane or nucleus

\section{Discussion}

\section{Conclusion}